\documentclass[xetex,mathserif,serif]{beamer}
\usepackage{polyglossia}
\setdefaultlanguage[babelshorthands=true]{russian}
\usepackage{minted}
\usepackage{tabu}

\useoutertheme{infolines}

\usepackage{fontspec}
\setmainfont{FreeSans}
\newfontfamily{\russianfonttt}{FreeSans}

\tabulinesep=0.8mm

\title{Аннотации: MyJUnit}
\author[Юрий Литвинов]{Юрий Литвинов \newline \textcolor{gray}{\small\texttt{yurii.litvinov@gmail.com}}}

\date{04.05.2017г}

\begin{document}
	
	\frame{\titlepage}
	
	\begin{frame}
		\frametitle{Небольшое напоминание про аннотации}
		\begin{itemize}
			\item Нужны для добавления метаданных в классы
			\begin{itemize}
				\item Могут потом использоваться во время компиляции или во время выполнения
				\item Активно используются самим компилятором (\mintinline{java}|@Override|, \mintinline{java}|@SuppressWarnings|)
			\end{itemize}
			\item Могут использоваться для генерации кода во время компиляции (например, Lombok)
			\item Полезны во время выполнения вместе с рефлексией (например, JUnit)
		\end{itemize}
	\end{frame}

	\begin{frame}[fragile]
		\frametitle{Объявление своей аннотации}
		\begin{minted}{java}
@Target(ElementType.TYPE)
@Retention(RetentionPolicy.RUNTIME)
public @interface Mammal {
    String sound();
    int color();
}
		\end{minted}
\end{frame}

	\begin{frame}[fragile]
		\frametitle{Использование}
		\begin{minted}{java}
@Mammal(color = 0xFFA844, sound = "uuuu")
class Giraffe {
}
		\end{minted}
\end{frame}

	\begin{frame}[fragile]
		\frametitle{Работа с аннотациями в рантайме}
		\begin{minted}{java}
@PermissionRequired(User.Permission.USER_MANAGEMENT)
class UserDeleteAction {
    public void invoke(User user) { /* */ }
}

User user = ...;
Class<?> actionClass = ...;
PermissionRequired permissionRequired =
        actionClass.getAnnotation(PermissionRequired.class);
if (permissionRequired != null)
    if (user != null && user.getPermissions().contains(
        permissionRequired.value())) {
            // выполнить действие
        }
    }
}
		\end{minted}
\end{frame}

	\begin{frame}
		\frametitle{Домашка, MyJUnit}
		\framesubtitle{Последняя!}
		Реализовать command-line приложение, принимающее на вход путь и выполняющее запуск тестов, расположенных по этому пути
		\begin{itemize}
			\item Тестом считается метод, помеченный аннотацией Test
			\begin{itemize}
				\item У аннотации может быть два аргумента --- expected для исключения, ignore --- для отмены запуска и указания причины
			\end{itemize}
			\item Перед и после запуска каждого теста в классе должны запускаться методы, помеченные аннотациями Before и After
			\item Перед и после запуска тестов в классе должны запускаться методы, помеченные аннотациями BeforeClass и AfterClass
		\end{itemize}
	\end{frame}

	\begin{frame}
		\frametitle{Домашка, MyJUnit (2)}
		Приложение должно выводить в стандартный поток вывода отчет:
		\begin{itemize}
			\item о результате и времени выполнения прошедших и упавших тестов
			\item о причине отключенных тестов
		\end{itemize}
		Дедлайн: \textbf{17.05.2017, 23:59}
	\end{frame}


	\begin{frame}
		\frametitle{На паре}
		Реализовать command-line приложение, принимающее на вход имя файла и запускающее все тесты в нём
		\begin{itemize}
			\item Тестом считается метод, помеченный аннотацией Test
			\begin{itemize}
				\item У аннотации может аргумент ignore --- для отмены запуска и указания причины
			\end{itemize}
		\end{itemize}
		Приложение должно выводить в стандартный поток вывода отчет:
		\begin{itemize}
			\item о результате и времени выполнения прошедших и упавших тестов
			\item о причине отключенных тестов
		\end{itemize}
	\end{frame}

\end{document}
