\documentclass[xetex,mathserif,serif]{beamer}
\usepackage{polyglossia}
\setdefaultlanguage[babelshorthands=true]{russian}
\usepackage{minted}
\usepackage{tabu}

\useoutertheme{infolines}

\usepackage{fontspec}
\setmainfont{FreeSans}
\newfontfamily{\russianfonttt}{FreeSans}

\tabulinesep=0.8mm

\title{Контрольная-2}
\author[Юрий Литвинов]{Юрий Литвинов \newline \textcolor{gray}{\small\texttt{yurii.litvinov@gmail.com}}}

\date{11.05.2017г}

\begin{document}
	
	\frame{\titlepage}
	
	\begin{frame}
		\frametitle{Задача, "Найди пару"}
		Реализовать игру "Найди пару":
		\begin{itemize}
			\item Поле с кнопками размера N x N, кнопки без надписей
			\item Кнопке ставится в соответствие случайное число от 0 до $\frac{N ^{\ 2}}{2}$
			\item Игрок нажимает на две произвольные (разные) кнопки, на них показывается соответствующие им числа 
			\item Если числа совпали, кнопки делаются неактивными
			\item Если числа не совпали, кнопки возвращаются в изначальное положение
			\item Игра заканчивается, когда игрок открыл все пары чисел
			\begin{itemize}
				\item Программа должна генерировать числа так, чтобы это было возможно
			\end{itemize}
		\end{itemize}
	\end{frame}

\end{document}
