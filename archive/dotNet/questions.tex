\documentclass[a5paper]{article}
\usepackage[a5paper, top=8mm, bottom=8mm, left=8mm, right=8mm]{geometry}

\usepackage{polyglossia}
\setdefaultlanguage[babelshorthands=true]{russian}

\usepackage{fontspec}
\setmainfont{FreeSerif}
\newfontfamily{\russianfonttt}[Scale=0.7]{DejaVuSansMono}

\usepackage[font=scriptsize]{caption}

\usepackage{amsmath}
\usepackage{amssymb,amsfonts,textcomp}
\usepackage{color}
\usepackage{array}
\usepackage{hhline}
\usepackage{cite}
\usepackage{textcomp}

\usepackage[hang,multiple]{footmisc}
\renewcommand{\footnotelayout}{\raggedright}

\PassOptionsToPackage{hyphens}{url}\usepackage[xetex,linktocpage=true,plainpages=false,pdfpagelabels=false]{hyperref}
\hypersetup{colorlinks=true, linkcolor=blue, citecolor=blue, filecolor=blue, urlcolor=blue, pdftitle=1, pdfauthor=, pdfsubject=, pdfkeywords=}

\newlength\Colsep
\setlength\Colsep{10pt}

\usepackage{tabu}

\usepackage{graphicx}
\usepackage{indentfirst}
\usepackage{multirow}
\usepackage{subfig}
\usepackage{footnote}
\usepackage{minted}

\newcommand{\todo}[1] {
\begin{center}\textcolor{red}{TODO: #1}\end{center}
}

\sloppy
\pagestyle{plain}

\title{Вопросы к экзамену "Программирование на .NET}
\author{Юрий Литвинов\\\small{yurii.litvinov@gmail.com}}

\begin{document}

\thispagestyle{empty}

\section*{Вопросы к экзамену ``Программирование на .NET''}

\begin{flushright}\begin{small}Юрий Литвинов\\\small{yurii.litvinov@gmail.com}\end{small}\end{flushright}

\begin{enumerate}
	\item Язык C\#, CLI, основы синтаксиса языка, ссылочные типы и типы-значения, преобразования типов, представление объектов в памяти
	\item Методы: способы передачи параметров, абстрактные, виртуальные и статические методы, модификаторы видимости.
	\item Платформа .NET: общее описание, CLR, IL, CTS
	\item Сборки: понятие сборки, сильные и слабые имена, загрузка сборки, GAC, Binding redirect, MsBuild
	\item NuGet, JIT, Ngen, понятие Managed Heap, AppDomain. Понятие целевой платформы, реализации: .NET Framework, Mono, .NET Core
	\item Исключения: бросание, перебрасывание и обработка, библиотечные исключения, свойства исключений, хорошие практики
	\item Рефлексия: загрузка сборки, создание экземпляра объекта, работа с полями и методами, dynamic
	\item Модульное тестирование: популярные библиотеки, хорошие практики, mock-объекты
	\item Контейнеры и генерики в .NET, энумераторы, открытые и закрытые типы, особенности статических полей в генериках, генерики и вложенные классы
	\item Генерики и наследование, ограничения на параметры-типы, ковариантность и контравариантность
	\item LINQ: основные методы, синтаксис, основные реализации, свои провайдеры
	\item Делегаты, их внутреннее устройство, delegate chaining, Invoke, шаблонные типы делегатов из стандартной библиотеки
	\item События, анонимные методы, лямбда-выражения, замыкания, каноничное объявление события, ручное управление подпиской
	\item Rx.NET, интерфейсы IObservable и IObserver, холодные и горячие последовательности, Rx.NET и LINQ, Subject
	\item WinForms: назначение, класс Control, обработка и валидация ввода, Data Binding, хорошие практики
	\item WPF: назначение и родственные технологии. XAML: атрибуты, конвертеры типов, расширения, коллекции. Структура классов WPF, логическое и визуальное дерево.
	\item WPF: зависимые свойства, routed events, команды. Data binding: конвертеры, направления привязки, валидация.
	\item WPF: стили, триггеры, шаблоны, ресурсы. Геометрия контрола, задание положения контрола и преобразования системы координат
	\item Потоки в .NET: классы Thread и ThreadPool, примитивы синхронизации уровня ядра: ключевое слово lock, мониторы, семафоры, WaitHandle, ManualResetEvent/AutoResetEvent, гибридные конструкции (*Slim)
	\item Lock-free-программирование: основные понятия, атомарные чтения/записи, volatile, Interlocked, Compare-And-Swap
	\item Класс Task, исполнение и отмена асинхронных операций. Async/await.
	\item Сборка мусора, mark and sweep, поколения, Large Object Heap, когда происходит сборка мусора
	\item Режимы сборки мусора: Workstation/Server, многопоточная сборка. Динамическая настройка GC, ручное управление, мониторинг
	\item Финализаторы, IDisposable, using, реализация финализации, ручное управление жизнью объекта, fixed, WeakReference
	\item Веб-сервисы, веб-приложения, архитектура ASP.NET MVC. Работа с БД: понятие ORM, библиотека Entity Framework
	\item Язык F\#: основные особенности, let-определения, кортежи, лямбды, списки, Option, взаимная рекурсия, pipe, композиция
	\item Каррирование, match, виды шаблонов, последовательности, записи, размеченные объединения
	\item Хвостовая рекурсия, паттерн ``Аккумулятор'', Continuation Passing Style
	\item Генерики в F\#, автоматическое обобщение, словари операций, касты, гибкие ограничения
	\item Методы отладки проблем типизации, value restriction, point-free, особенности арифметических операторов
	\item ООП в F\#: методы, каррирование и кортежи при передаче параметров, конструкторы, свойства, мутабельность
	\item Модификаторы видимости, наследование, абстрактные классы и интерфейсы, реализация по умолчанию, объектные выражения, модули и пространства имён
\end{enumerate}

\end{document}
