\documentclass[xetex,mathserif,serif]{beamer}
\usepackage{polyglossia}
\setdefaultlanguage[babelshorthands=true]{russian}
\usepackage{minted}
\usepackage{tabu}
\usepackage{moresize}

\useoutertheme{infolines}

\usepackage{fontspec}
\setmainfont{FreeSans}
\newfontfamily{\russianfonttt}{FreeSans}

\setbeamertemplate{blocks}[rounded][shadow=false]

\setbeamercolor*{block title alerted}{fg=red!50!black,bg=red!20}
\setbeamercolor*{block body alerted}{fg=black,bg=red!10}

\tabulinesep=1.2mm

\title{Пользовательский интерфейс, WinForms}
\subtitle{Практика}
\author[Юрий Литвинов]{Юрий Литвинов\\\small{\textcolor{gray}{yurii.litvinov@gmail.com}}}
\date{12.10.2017г}

\newcommand{\todo}[1] {
	\begin{center}\textcolor{red}{TODO: #1}\end{center}
}

\begin{document}

	\frame{\titlepage}

	\begin{frame}
		\frametitle{Задача, "Найди пару"}
		Реализовать игру "Найди пару":
		\begin{itemize}
			\item Поле с кнопками размера N x N, кнопки без надписей
			\item Кнопке ставится в соответствие случайное число от 0 до $\frac{N ^{\ 2}}{2}$
			\item Игрок нажимает на две произвольные (разные) кнопки, на них показываются соответствующие им числа 
			\item Если числа совпали, кнопки делаются неактивными
			\item Если числа не совпали, кнопки возвращаются в изначальное положение
			\item Игра заканчивается, когда игрок открыл все пары чисел
		\end{itemize}
	\end{frame}

\end{document}
