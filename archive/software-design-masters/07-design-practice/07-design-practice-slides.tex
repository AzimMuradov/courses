\documentclass[xetex,mathserif,serif]{beamer}
\usepackage{polyglossia}
\setdefaultlanguage[babelshorthands=true]{russian}
\usepackage{minted}
\usepackage{tabu}

\useoutertheme{infolines}

\usepackage{fontspec}
\setmainfont{FreeSans}
\newfontfamily{\russianfonttt}{FreeSans}

\definecolor{links}{HTML}{2A1B81}
\hypersetup{colorlinks,linkcolor=,urlcolor=links}

\tabulinesep=0.7mm

\newcommand{\attribution}[1] {
	\vspace{-5mm}\begin{flushright}\begin{scriptsize}\textcolor{gray}{\textcopyright\, #1}\end{scriptsize}\end{flushright}
}

\title{Практика по проектированию}
\author[Юрий Литвинов]{Юрий Литвинов \newline \textcolor{gray}{\small\texttt{yurii.litvinov@gmail.com}}}

\date{25.03.2020г}

\begin{document}
	
	\frame{\titlepage}

	\begin{frame}
		\frametitle{Задача, система контроля версий}
		Требуется спроектировать систему контроля версий, представляющую из себя консольное приложение и умеющую:
		\begin{itemize}
			\item commit с commit message (сообщение обязательно и принимается как параметр, система должна сама добавлять ещё дату коммита и автора)
			\item работу с ветками: создание и удаление
			\item checkout по имени ревизии или ветки
			\item log --- список ревизий вместе с commit message в текущей ветке
			\item merge --- сливает указанную ветку с текущей
			\begin{itemize}
				\item Разрешение конфликтов
			\end{itemize}
		\end{itemize}
	\end{frame}

	\begin{frame}
		\frametitle{Нефункциональные требования}
		\begin{itemize}
			\item Вывод в консоль --- только в клиентском коде типа main(), основной код должен позволять себя использовать как библиотеку
			\item Развитый программный интерфейс, должно быть можно без проблем потом прикрутить GUI
			\item Сопровождаемость и расширяемость
		\end{itemize}
	\end{frame}

	\begin{frame}
		\frametitle{Что надо сделать за пару}
		\begin{itemize}
			\item Разбиться на команды по два-три человека, наладить совместную работу
			\item Нарисовать диаграмму компонентов + диаграмму классов
			\begin{itemize}
				\item Возможно, на одной диаграмме сразу
				\item Возможно наоборот, по одной диаграмме для каждого компонента
			\end{itemize}
			\item Успеть показать и обсудить пару решений
			\item Выложить все решения на HwProj
			\item Нельзя подсматривать в Git Book
		\end{itemize}
	\end{frame}

	\begin{frame}
		\frametitle{Соображения}
		\begin{itemize}
			\item Как представляются файлы, коммиты, ветки, репозиторий?
			\item Как выполняется компрессия и выполняется ли вообще? Насколько просто получить текущую, предыдущую, произвольную версии?
			\item Каков жизненный цикл файла?
			\item Как выполняется работа с файловой системой?
			\item Как выполняется работа с пользователем? Как представляются команды?
		\end{itemize}
	\end{frame}

\end{document}
