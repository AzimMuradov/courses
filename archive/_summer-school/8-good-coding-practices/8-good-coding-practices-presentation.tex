\documentclass[xetex,mathserif,serif]{beamer}
\usepackage{polyglossia}
\setdefaultlanguage[babelshorthands=true]{russian}
\usepackage{minted}

\useoutertheme{infolines}

\setmainfont{FreeSans}
\newfontfamily{\russianfonttt}{FreeSans}

\title{Хорошие практики кодирования}
\author[Юрий Литвинов]{Юрий Литвинов \newline \textcolor{gray}{\small\texttt{yurii.litvinov@gmail.com}}}
\date{13.07.2017г}

\begin{document}
	
	\frame{\titlepage}

	\begin{frame}
		\frametitle{Абстрактные типы данных}
		\begin{itemize}
			\item \mintinline{csharp}|currentFont.size = 16| --- плохо
			\item \mintinline{csharp}|currentFont.size = PointsToPixels(12)| --- чуть лучше
			\item \mintinline{csharp}|currentFont.sizeInPixels = PointsToPixels(12)| --- ещё чуть лучше
			\item \mintinline{csharp}|currentFont.SetSizeInPoints(sizeInPoints)| \newline
					\mintinline{csharp}|currentFont.SetSizeInPixels(sizeInPixels)| --- совсем хорошо
		\end{itemize}
	\end{frame}

	\begin{frame}[fragile]
		\frametitle{Пример плохой абстракции}
		\begin{minted}{csharp}
public class Program {
   public void InitializeCommandStack() { ... }
   public void PushCommand(Command command) { ... }
   public Command PopCommand() { ... }
   public void ShutdownCommandStack() { ... }
   public void InitializeReportFormatting() { ... }
   public void FormatReport(Report report) { ... }
   public void PrintReport(Report report) { ... }
   public void InitializeGlobalData() { ... }
   public void ShutdownGlobalData() { ... }
}
		\end{minted}
\end{frame}

	\begin{frame}[fragile]
		\frametitle{Пример хорошей абстракции}
		\begin{footnotesize}
			\begin{minted}{csharp}
public class Employee {
   public Employee(
           FullName name,
           string address,
           string workPhone,
           string homePhone,
           TaxId taxIdNumber,
           JobClassification jobClass
   ) { ... }

   public FullName Name { get {...}; set {...} }
   public string Address { get {...}; set {...} }
   public string WorkPhone { get {...}; set {...} }
   public string HomePhone { get {...}; set {...} }
   public TaxId TaxIdNumber { get {...}; set {...} }
   public JobClassification JobClassification { get {...}; set {...} }
}
			\end{minted}
		\end{footnotesize}
\end{frame}

	\begin{frame}[fragile]
		\frametitle{Уровень абстракции (плохо)}
		\begin{minted}{csharp}
public class EmployeeRoster : MyList<Employee> {
   public void AddEmployee(Employee employee) { ... }
   public void RemoveEmployee(Employee employee) { ... }
   public Employee NextItemInList() { ... }
   public Employee FirstItem() { ... }
   public Employee LastItem() { ... }
}
		\end{minted}
\end{frame}

	\begin{frame}[fragile]
		\frametitle{Уровень абстракции (хорошо)}
		\begin{minted}{csharp}
public class EmployeeRoster {
   public void AddEmployee(Employee employee) { ... }
   public void RemoveEmployee(Employee employee) { ... }
   public Employee NextEmployee() { ... }
   public Employee FirstEmployee() { ... }
   public Employee LastEmployee() { ... }
}
		\end{minted}
\end{frame}

	\begin{frame}
		\frametitle{Общие рекомендации}
		\begin{itemize}
			\item Про каждый класс знайте, реализацией какой абстракции он является
			\item Учитывайте противоположные методы (Add/Remove, On/Off, ...)
			\item Соблюдайте принцип единственности ответственности
			\begin{itemize}
				\item Может потребоваться разделить класс на несколько разных классов просто потому, что методы по смыслу слабо связаны
			\end{itemize}
			\item По возможности делайте некорректные состояния невыразимыми в системе типов
			\begin{itemize}
				\item Комментарии в духе ``не пользуйтесь объектом, не вызвав  init()'' можно заменить конструктором
			\end{itemize}
			\item При рефакторинге надо следить, чтобы интерфейсы не деградировали
		\end{itemize}
	\end{frame}

	\begin{frame}
		\frametitle{Ещё рекомендации}
		\begin{itemize}
			\item Класс не должен ничего знать о своих клиентах
			\item Лёгкость чтения кода важнее, чем удобство его написания
			\item Опасайтесь семантических нарушений инкапсуляции
			\begin{itemize}
				\item ``Не будем вызывать ConnectToDB(), потому что GetRow() сам его вызовет, если соединение не установлено'' --- это программирование \textit{сквозь} интерфейс
			\end{itemize}
			\item Protected- и internal- полей тоже не бывает
			\begin{itemize}
				\item На самом деле, у класса два интерфейса --- для внешних объектов и для потомков (может быть отдельно третий, для классов внутри сборки, но это может быть плохо)
			\end{itemize}
		\end{itemize}
	\end{frame}

	\begin{frame}
		\frametitle{Наследование}
		\begin{itemize}
			\item Включение лучше
			\begin{itemize}
				\item Переконфигурируемо во время выполнения
				\item Более гибко
				\item Иногда более естественно
			\end{itemize}
			\item Наследование --- отношение ``является'', закрытого наследования не бывает
			\begin{itemize}
				\item Наследование --- это наследование интерфейса (полиморфизм подтипов, subtyping)
			\end{itemize}
			\item Хороший тон --- явно запрещать наследование (sealed-классы)
			\item Не вводите новых методов с такими же именами, как у родителя
			\item Code smells:
			\begin{itemize}
				\item Базовый класс, у которого только один потомок
				\item Пустые переопределения
				\item Очень много уровней в иерархии наследования
			\end{itemize}
		\end{itemize}
	\end{frame}

	\begin{frame}[fragile]
		\frametitle{Пример}
		\begin{footnotesize}
			\begin{columns}
				\begin{column}{0.35\textwidth}
					\begin{minted}{csharp}
class Operation {
   private char sign = '+';
   private int left;
   private int right;
   public int Eval()
   {
       switch (sign) {
           case '+': return left + right;
       }
       throw new Exception();
   }
}
					\end{minted}
				\end{column}
				\begin{column}{0.1\textwidth}
					vs
				\end{column}
				\begin{column}{0.45\textwidth}
					\begin{minted}{csharp}
abstract class Operation {
   private int left;
   private int right;
   protected int Left { get { return left; } }
   protected int Right { get { return right; } }
   abstract public int Eval();
}

class Plus : Operation {
   override public int Eval() { 
        return this.Left + this.Right; 
   }
}
					\end{minted}
				\end{column}
			\end{columns}
		\end{footnotesize}
\end{frame}

	\begin{frame}
		\frametitle{Конструкторы}
		\begin{itemize}
			\item Инициализируйте все поля, которые надо инициализировать
			\begin{itemize}
				\item После конструктора должны выполняться все инварианты
			\end{itemize}
			\item НЕ вызывайте виртуальные методы из конструктора
			\item private-конструкторы для объектов, которые не должны быть созданы (или одиночек)
			\item Deep copy предпочтительнее Shallow copy
			\begin{itemize}
				\item Хотя второе может быть эффективнее
			\end{itemize}
			\item Не создавайте ненужных объектов, не храните ссылки, если они не нужны
		\end{itemize}
	\end{frame}

	\begin{frame}
		\frametitle{О дизайне классов и интерфейсов}
		\begin{itemize}
			\item Минимизируйте видимость классов и методов
			\item Минимизируйте мутабельность
			\begin{itemize}
				\item Лучше разделять методы на те, которые возвращают состояние и те, которые его меняют
			\end{itemize}
			\item Не бойтесь принимать и возвращать функции
			\item Предпочитайте интерфейсы абстрактным классам
			\item Не используйте ``тэги типов''
			\item Не используйте без нужды вложенные классы
			\item Проверяйте параметры public-методов на валидность
			\begin{itemize}
				\item Помните о null-ах
			\end{itemize}
			\item Fail fast
		\end{itemize}
	\end{frame}

	\begin{frame}
		\frametitle{Общепрограммистские рекомендации}
		\begin{itemize}
			\item Минимизируйте области видимости
			\item Не используйте числа с плавающей точкой там, где нужны точные значения
			\begin{itemize}
				\item Сравнивать числа с плавающей точкой через == нельзя
			\end{itemize}
			\item Знайте и используйте библиотеки
			\item Используйте исключения только в исключительных ситуациях
			\item Не игнорируйте исключения
		\end{itemize}
	\end{frame}

\end{document}