\documentclass[xetex,mathserif,serif]{beamer}
\usepackage{polyglossia}
\setdefaultlanguage[babelshorthands=true]{russian}
\usepackage{minted}
\usepackage{tabu}

\useoutertheme{infolines}

\usepackage{fontspec}
\setmainfont{FreeSans}
\newfontfamily{\russianfonttt}{FreeSans}

\usepackage{textpos}
\setlength{\TPHorizModule}{1cm}
\setlength{\TPVertModule}{1cm}

\definecolor{links}{HTML}{2A1B81}
\hypersetup{colorlinks,linkcolor=,urlcolor=links}

\tabulinesep=1.2mm

\title{Правда про enum-ы}
\subtitle{(c) Effective Java}
\author[Юрий Литвинов]{Юрий Литвинов\\\small{\textcolor{gray}{yurii.litvinov@gmail.com}}}
\date{27.02.2019г}

\newcommand{\attribution}[1] {
\vspace{-5mm}\begin{flushright}\begin{scriptsize}\textcolor{gray}{\textcopyright\, #1}\end{scriptsize}\end{flushright}
}

\begin{document}

	\frame{\titlepage}

	\begin{frame}[fragile]
		\frametitle{Enum-ы}
		\begin{itemize}
			\item Типобезопасны
			\item Автоматически получают пространство имен
			\item Возможны конструкторы и методы
		\end{itemize}
		\begin{minted}{java}
public enum Apple { FUJI, PIPPIN, GRANNY_SMITH }
public enum Orange { NAVEL, TEMPLE, BLOOD }

enum Season {
    WINTER, 
    SPRING,
    SUMMER, 
    AUTUMN
}
		\end{minted}
	\end{frame}

	\begin{frame}
		\frametitle{Обещанная правда про Enum-ы}
		\begin{itemize}
			\item Enum-ы --- это классы
			\item Каждый элемент перечисления --- это public static final поле
			\item Нет доступных конструкторов
			\begin{itemize}
				\item Невозможно создать экземпляр, кроме элемента перечисления
				\item Обобщение паттерна ``Одиночка''
			\end{itemize}
			\item Реализуют Comparable и Serializable
		\end{itemize}
	\end{frame}

	\begin{frame}[fragile]
		\frametitle{Методы}
		\begin{itemize}
			\item Можно сравнивать с помощью \mintinline{java}|==|
			\item \mintinline{java}|name(), ordinal(), toString()|
		\end{itemize}
		\begin{minted}{java}
var season = Season.WINTER; 
System.out.println(
    season.name() + ", " +
    season.toString() + ", " +
    season.ordinal()
);

WINTER, WINTER, 0
		\end{minted}
	\end{frame}

	\begin{frame}[fragile]
		\frametitle{Статические методы}
		\begin{itemize}
			\item \mintinline{java}|valueOf()|
				\begin{minted}{java}
String name = "WINTER"; 
Season season = Season.valueOf(name); 

Season.valueOf(null); // NullPointerException
Season.valueOf("HOLIDAYS"); // IllegalArgumentException
				\end{minted}
			\item \mintinline{java}|values()|
				\begin{minted}{java}
System.out.println(Arrays.toString(Season.values()));
				\end{minted}
				\vspace{3mm}
				\begin{minted}{text}
[WINTER, SPRING, SUMMER, AUTUMN]
				\end{minted}
			\item Автоматически добавляются компилятором
		\end{itemize}
	\end{frame}

	\begin{frame}[fragile]
		\frametitle{Поля}
		\begin{minted}{java}
enum Type {
    INT(true), 
    INTEGER(false),
    STRING(false); 
 
    private final boolean primitive;
     
    Type(boolean primitive) { this.primitive = primitive; } 
 
    public boolean isPrimitive() { return primitive; } 
} 
		\end{minted}
	\end{frame}

	\begin{frame}[fragile]
		\frametitle{Методы}
		\begin{minted}{java}
enum Direction { 
   UP, DOWN; 
 
   public Direction opposite() {
       switch (this) {
           case UP:
               return DOWN;
           case DOWN:
               return UP;
           throw new AssertionError("Unknown op: " + this);
       }
   } 
}
		\end{minted}
	\end{frame}

	\begin{frame}[fragile]
		\frametitle{Методы (constant-specific)}
		\begin{minted}{java}
enum Direction { 
   UP { 
        public Direction opposite() { return DOWN; } 
   }, 
   
   DOWN { 
        public Direction opposite() { return UP; } 
   }; 
 
   public abstract Direction opposite(); 
}
		\end{minted}
	\end{frame}

	\begin{frame}
		\frametitle{Почему?}
		\begin{itemize}
			\item Enum-ы --- это иерархия классов
			\item Каждый элемент перечисления --- это экземпляр своего наследника от базового класса-перечисления
		\end{itemize}
	\end{frame}

	\begin{frame}[fragile]
		\frametitle{Пример}
		\begin{footnotesize}
			\begin{minted}{java}
enum Type {
    INT(true) {
        public Object parse(String string) { return Integer.valueOf(string); }  
    }, 
    INTEGER(false) {
        public Object parse(String string) { return Integer.valueOf(string); }  
    },
    STRING(false) {
        public Object parse(String string) { return string; }
    }; 
 
    private final boolean primitive;
     
    Type(boolean primitive) { this.primitive = primitive; } 
 
    public boolean isPrimitive() { return primitive; } 
    public abstract Object parse(String string); 
}
			\end{minted}
		\end{footnotesize}
	\end{frame}

	\begin{frame}[fragile]
		\frametitle{fromString}
		\begin{minted}{java}
// Implementing a fromString method on an enum type
private static final Map<String, Type> stringToEnum =
    Stream.of(values()).collect(
        toMap(Object::toString, e -> e));

// Returns Type for string, if any
public static Optional<Type> fromString(String symbol) {
    return Optional.ofNullable(stringToEnum.get(symbol));
}
		\end{minted}
		\begin{itemize}
			\item Сначала инициализируются элементы enum-а, затем остальные статические поля
			\begin{itemize}
				\item Поэтому доступ к статическим полям из конструкторов элементов запрещён
			\end{itemize}
		\end{itemize}
	\end{frame}

	\begin{frame}[fragile]
		\frametitle{Ещё пример}
		\begin{scriptsize}
			\begin{minted}{java}
enum PayrollDay {
    MONDAY, TUESDAY, WEDNESDAY, THURSDAY, FRIDAY, SATURDAY, SUNDAY;

    private static final int HOURS_PER_SHIFT = 8;

    double pay(double hoursWorked, double payRate) {
        double basePay = hoursWorked * payRate;
        double overtimePay;
        switch (this) {
            case SATURDAY: case SUNDAY:
                overtimePay = hoursWorked * payRate / 2;
                break;
            default:
                overtimePay = hoursWorked <= HOURS_PER_SHIFT 
                        ? 0 : (hoursWorked - HOURS_PER_SHIFT) * payRate / 2;
                break;
        }
        return basePay + overtimePay;
    }
}
			\end{minted}
		\end{scriptsize}
	\end{frame}

	\begin{frame}[fragile]
		\frametitle{Паттерн ``Стратегия'' на enum-ах}
		\begin{footnotesize}
			\begin{minted}{java}
enum PayrollDay {
    MONDAY(PayType.WEEKDAY), TUESDAY(PayType.WEEKDAY),
    WEDNESDAY(PayType.WEEKDAY), THURSDAY(PayType.WEEKDAY),
    FRIDAY(PayType.WEEKDAY),
    SATURDAY(PayType.WEEKEND), SUNDAY(PayType.WEEKEND);
    
    private final PayType payType;

    PayrollDay(PayType payType) { this.payType = payType; }

    double pay(double hoursWorked, double payRate) {
        return payType.pay(hoursWorked, payRate);
    }
    ...
}
			\end{minted}
		\end{footnotesize}
	\end{frame}

	\begin{frame}[fragile]
		\frametitle{PayType}
		\begin{scriptsize}
			\begin{minted}{java}
private enum PayType {
    WEEKDAY {
        double overtimePay(double hours, double payRate) {
            return hours <= HOURS_PER_SHIFT ? 0 :
                    (hours - HOURS_PER_SHIFT) * payRate / 2;
        }
    },

    WEEKEND {
        double overtimePay(double hours, double payRate) {
            return hours * payRate / 2;
        }
    };

    private static final int HOURS_PER_SHIFT = 8;
    
    abstract double overtimePay(double hrs, double payRate);
    
    double pay(double hoursWorked, double payRate) {
        double basePay = hoursWorked * payRate;
        return basePay + overtimePay(hoursWorked, payRate);
    }
}
			\end{minted}
		\end{scriptsize}
	\end{frame}

	\begin{frame}[fragile]
		\frametitle{Опасайтесь ordinal()}
		\begin{small}
			Неправильно:
			\begin{minted}{java}
public enum Ensemble {
    SOLO, DUET, TRIO, QUARTET, QUINTET,
    SEXTET, SEPTET, OCTET, NONET, DECTET;
    public int numberOfMusicians() { return ordinal() + 1; }
}
			\end{minted}
			\vspace{3mm}
			Правильно:
			\begin{minted}{java}
public enum Ensemble {
    SOLO(1), DUET(2), TRIO(3), QUARTET(4), QUINTET(5),
    SEXTET(6), SEPTET(7), OCTET(8), DOUBLE_QUARTET(8),
    NONET(9), DECTET(10), TRIPLE_QUARTET(12);

    private final int numberOfMusicians;
    Ensemble(int size) { this.numberOfMusicians = size; }
    public int numberOfMusicians() { return numberOfMusicians; }
}
			\end{minted}
		\end{small}
	\end{frame}

	\begin{frame}[fragile]
		\frametitle{Битовые поля}
		Как делали без enum-ов:

		\begin{minted}{java}
public class Text {
    public static final int STYLE_BOLD = 1 << 0;
    public static final int STYLE_ITALIC = 1 << 1;
    public static final int STYLE_UNDERLINE = 1 << 2;
   
    public void applyStyles(int styles) {
        // styles — побитовое "или"
        // text.applyStyles(STYLE_BOLD | STYLE_ITALIC);
    }
}
		\end{minted}
	\end{frame}

	\begin{frame}[fragile]
		\frametitle{EnumSet}
		\begin{itemize}
			\item Все возможности Set
			\item Внутри --- long или long[]
			\begin{itemize}
				\item Производительность сравнима с битовыми масками
			\end{itemize}
		\end{itemize}
		\vspace{4mm}
		\begin{minted}{java}
public class Text {
    public enum Style { BOLD, ITALIC, UNDERLINE }
    
    public void applyStyles(Set<Style> styles) {
        // text.applyStyles(EnumSet.of(Style.BOLD, Style.ITALIC));
    }
}
		\end{minted}
	\end{frame}

	\begin{frame}[fragile]
		\frametitle{EnumMap, пример}
		\begin{minted}{java}
public class Plant {
    public enum LifeCycle { ANNUAL, PERENNIAL, BIENNIAL }
    
    private final String name;
    private final LifeCycle lifeCycle;
    
    public Plant(String name, LifeCycle lifeCycle) {
        this.name = name;
        this.lifeCycle = lifeCycle;
    }
}
		\end{minted}
	\end{frame}

	\begin{frame}[fragile]
		\frametitle{EnumMap, пример без}
		\begin{minted}{java}
var plantsByLifeCycle = 
    (Set<Plant>[]) new Set[Plant.LifeCycle.values().length];
    // Indexed by Plant.LifeCycle.ordinal()

for (int i = 0; i < plantsByLifeCycle.length; i++) {
    plantsByLifeCycle[i] = new HashSet<>();
}

for (Plant p : garden) {
    plantsByLifeCycle[p.lifeCycle.ordinal()].add(p);
}
		\end{minted}
	\end{frame}

	\begin{frame}[fragile]
		\frametitle{EnumMap, пример с}
		\begin{minted}{java}
var plantsByLifeCycle =
    new EnumMap<Plant.LifeCycle, Set<Plant>>(Plant.LifeCycle.class);

for (Plant.LifeCycle lc : Plant.LifeCycle.values()) {
    plantsByLifeCycle.put(lc, new HashSet<>());
}

for (Plant p : garden) {
    plantsByLifeCycle.get(p.lifeCycle).add(p);
}
		\end{minted}
	\end{frame}

	\begin{frame}[fragile]
		\frametitle{Или, через Stream API}
		\begin{minted}{java}
System.out.println(Arrays.stream(garden)
    .collect(groupingBy(p -> p.lifeCycle,
        () -> new EnumMap<>(LifeCycle.class), toSet())));
		\end{minted}
	\end{frame}

	\begin{frame}[fragile]
		\frametitle{EnumMap EnumMap-ов, пример без}
		\begin{small}
			\begin{minted}{java}
public enum Phase {
    SOLID, LIQUID, GAS;
    
    public enum Transition {
        MELT, FREEZE, BOIL, CONDENSE, SUBLIME, DEPOSIT;
        
        private static final Transition[][] TRANSITIONS = {
            { null, MELT, SUBLIME },
            { FREEZE, null, BOIL },
            { DEPOSIT, CONDENSE, null }
        };

        public static Transition from(Phase src, Phase dst) { 
            return TRANSITIONS[src.ordinal()][dst.ordinal()]; 
        }
    }
}
			\end{minted}
		\end{small}
	\end{frame}

	\begin{frame}[fragile]
		\frametitle{EnumMap EnumMap-ов, пример с}
		\begin{small}
			\begin{minted}{java}
public enum Phase {
    SOLID, LIQUID, GAS;
    
    public enum Transition {
        MELT(SOLID, LIQUID), FREEZE(LIQUID, SOLID),
        BOIL(LIQUID, GAS), CONDENSE(GAS, LIQUID),
        SUBLIME(SOLID, GAS), DEPOSIT(GAS, SOLID);

        private final Phase from;
        private final Phase to;
        
        Transition(Phase from, Phase to) {
            this.from = from;
            this.to = to;
        }
        ...
    }
}
			\end{minted}
		\end{small}
	\end{frame}

	\begin{frame}[fragile]
		\frametitle{EnumMap EnumMap-ов, пример с (2)}
		\begin{footnotesize}
			\begin{minted}{java}
public enum Transition {
        ...
        private static final Map<Phase, Map<Phase, Transition>> m =
            Stream.of(values()).collect(groupingBy(t -> t.from,
                () -> new EnumMap<>(Phase.class),
                toMap(t -> t.to, t -> t,
                    (x, y) -> y, () -> new EnumMap<>(Phase.class))));
        
        public static Transition from(Phase from, Phase to) { 
            return m.get(from).get(to); 
        }
    }
}
			\end{minted}
		\end{footnotesize}
	\end{frame}

	\begin{frame}[fragile]
		\frametitle{Добавим новое состояние}
		\begin{small}
			\begin{minted}{java}
public enum Phase {
    SOLID, LIQUID, GAS, PLASMA;
    
    public enum Transition {
        MELT(SOLID, LIQUID), FREEZE(LIQUID, SOLID),
        BOIL(LIQUID, GAS), CONDENSE(GAS, LIQUID),
        SUBLIME(SOLID, GAS), DEPOSIT(GAS, SOLID),
        IONIZE(GAS, PLASMA), DEIONIZE(PLASMA, GAS);

        // Дальше ничего не поменялось
        ...
    }
}
			\end{minted}
		\end{small}
	\end{frame}

\end{document}
