\documentclass[xetex,mathserif,serif]{beamer}
\usepackage{polyglossia}
\setdefaultlanguage[babelshorthands=true]{russian}
\usepackage{minted}
\usepackage{tabu}

\useoutertheme{infolines}

\usepackage{fontspec}
\setmainfont{FreeSans}
\newfontfamily{\russianfonttt}{FreeSans}

\usepackage{textpos}
\setlength{\TPHorizModule}{1cm}
\setlength{\TPVertModule}{1cm}

\definecolor{links}{HTML}{2A1B81}
\hypersetup{colorlinks,linkcolor=,urlcolor=links}

\tabulinesep=1.2mm

\title{Практика по рефлексии}
\author[Юрий Литвинов]{Юрий Литвинов\\\small{\textcolor{gray}{yurii.litvinov@gmail.com}}}
\date{20.02.2019г}

\newcommand{\attribution}[1] {
\vspace{-5mm}\begin{flushright}\begin{scriptsize}\textcolor{gray}{\textcopyright\, #1}\end{scriptsize}\end{flushright}
}

\begin{document}

	\frame{\titlepage}

	\begin{frame}
		\frametitle{Задача}
		\begin{itemize}
			\item С помощью рефлексии реализовать класс Serialization с двумя статическими функциями:
			\begin{itemize}
				\item \mintinline{java}|void serialize(Object, OutputStream)|
				\begin{itemize}
					\item Записывает состояние полей переданного объекта в поток
				\end{itemize}
				\item \mintinline{java}|T deserialize(InputStream, Class<T>)|
				\begin{itemize}
					\item Создаёт экземпляр класса и инициализирует его поля данными из потока
				\end{itemize}
			\end{itemize}
			\item Считается, что в сериализации/десериализации используются объекты:
			\begin{itemize}
				\item классы которых имеют публичный конструктор без параметров
				\item типы полей ограничены примитивными типами и строками
			\end{itemize}
		\end{itemize}
	\end{frame}

\end{document}
