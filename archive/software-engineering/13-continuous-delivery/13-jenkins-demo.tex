\documentclass[a5paper]{article}
\usepackage[a5paper, top=8mm, bottom=8mm, left=8mm, right=8mm]{geometry}

\usepackage{polyglossia}
\setdefaultlanguage[babelshorthands=true]{russian}

\usepackage{fontspec}
\setmainfont{FreeSerif}
\newfontfamily{\russianfonttt}[Scale=0.7]{DejaVuSansMono}

\usepackage[font=scriptsize]{caption}

\usepackage{amsmath}
\usepackage{amssymb,amsfonts,textcomp}
\usepackage{color}
\usepackage{array}
\usepackage{hhline}
\usepackage{cite}

\usepackage[hang,multiple]{footmisc}
\renewcommand{\footnotelayout}{\raggedright}

\PassOptionsToPackage{hyphens}{url}\usepackage[xetex,linktocpage=true,plainpages=false,pdfpagelabels=false]{hyperref}
\hypersetup{colorlinks=true, linkcolor=blue, citecolor=blue, filecolor=blue, urlcolor=blue, pdftitle=1, pdfauthor=, pdfsubject=, pdfkeywords=}

\usepackage{tabu}

\usepackage{graphicx}
\usepackage{indentfirst}
\usepackage{multirow}
\usepackage{subfig}
\usepackage{footnote}
\usepackage{minted}

\sloppy
\pagestyle{plain}

\title{Jenkins, демонстрация}
\author{Юрий Литвинов\\\small{yurii.litvinov@gmail.com}}

\date{16.10.2018}

\begin{document}

\maketitle
\thispagestyle{empty}

\section{Требования}

Jenkins можно скачать с \url{https://jenkins.io/}, в разных вариантах: либо как устанавливаемую программу, либо как Java .war-файл. Если выбрано последнее, то чтобы его запустить, надо выполнить \verb|java -jar jenkins.war --httpPort=8080| и, когда он проинициализируется, зайти на \url{http://localhost:8080}. Там он предложит поставить плагины, можно выбрать ``Поставить самые популярные'', а можно и повыбирать плагины руками. Из их названий более-менее понятно, для чего они. Ставится оно под виндой в C:/Users/<user>/.jenkins.

Если Jenkins поставлен инсталлятором, то он сам запустится, откроет \url{http://localhost:8080} и предложит поставить плагины. В этом случае всё под виндой оказывается обычно в C:/Program Files (x86)/Jenkins.

\section{Jenkins}

Jenkins, как и AppVeyor,  управляется с помощью конфигурационного файла, который должен лежать в корне репозитория и называться Jenkinsfile. Для того, чтобы что-то сделать, нам потребуется 
гит-репозиторий, потому как Jenkins первое, что делает --- получает из указанного репозитория исходники. Достаточно любого репозитория на гитхабе (локальный тоже пойдёт, но только если гит запущен как сервер).

Создадим Jenkinsfile, примерно такого содержания:

\begin{minted}{groovy}
pipeline {
    agent any
    stages {
        stage('build') {
            steps {
                bat 'mvn --version'
            }
        }
    }
}
\end{minted}

Да, Jenkins (а точнее, его плагин Pipeline, который, собственно, и будет управлять сборкой) использует синтаксис Groovy, знакомый по Gradle. Попрбуем создать билд на основе этого файла. Для этого надо запустить Jenkins (командой \verb|java -jar jenkins.war --httpPort=8080|), пойти на \url{localhost:8080} и авторизоваться тем паролем, который вы указали при настройке Jenkins-а.

Дальше жмём на New Item, вводим название пайплайна (любое), выбираем Multibranch Pipeline (чтобы Jenkins следил за ветками и пуллреквестами в репозиторий и собирал их), жмём Ок. Заполняем сведения про билд, например, ссылку на репозиторий, жмём save --- и он уже пытается получить репозиторий и собрать. 

Так, как было выше, нам требуется установленный Maven в путях. Более модно не настраивать окружение сборки всякий раз на каждой машине, на которой мы поднимаем Jenkins, а использовать Docker. Например, тот же самый билд, но качающий образ с maven-ом прямо в процессе сборки:

\begin{minted}{groovy}
pipeline {
    agent { docker 'maven:3.5.2-jdk-8-slim' }
    stages {
        stage('build') {
            steps {
                bat 'mvn --version'
            }
        }
    }
}
\end{minted}

Под виндой, правда, прямо из коробки не заработает, потому что оно хочет sh и nohup в путях, но оно есть в Cygwin и даже в Git For Windows, так что по идее можно просто подредактировать PATH, указав на папки с нужными бинарниками. Но с большой вероятностью не заработает всё равно (потому как большинство плагинов для Jenkins ожидают линуксового окружения), поэтому идеологически правильнее сам Jenkins запускать в Docker-контейнере, командой (для винды)

\begin{minted}{sh}
docker run ^
  --rm ^
  -u root ^
  -p 8080:8080 ^
  -v jenkins-data:/var/jenkins_home ^
  -v /var/run/docker.sock:/var/run/docker.sock ^
  -v "%HOMEPATH%":/home ^
  jenkinsci/blueocean
\end{minted}

После этого Jenkins придётся активировать ещё раз (обратите внимание, пароль для активации пишется в консоли, если его потерять, придётся освоить исполнение команд в контейнере через docker exec). И придётся снова скачать и обновить плагины. Зато теперь он будет работать даже под виндой, думая, что он запущен под линуксом, всякие проблемы с sh и nohup будет решать уже Docker.

Кроме того, докер-образ поставляется с плагином Blue Ocean, это новый суперудобный фронтэнд для Jenkins, который вообще всё сделает за вас. Доступен, например, по ссылке \url{http://localhost:8080/blue}, там всё интуитивно понятно, очень рекомендую попробовать настроить билд с ним.

\end{document}
