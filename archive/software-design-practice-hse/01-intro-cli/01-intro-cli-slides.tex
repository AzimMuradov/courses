\documentclass[xetex,mathserif,serif]{beamer}
\usepackage{polyglossia}
\setdefaultlanguage[babelshorthands=true]{russian}
\usepackage{minted}
\usepackage{tabu}

\useoutertheme{infolines}

\usepackage{fontspec}
\setmainfont{FreeSans}
\newfontfamily{\russianfonttt}{FreeSans}

\definecolor{links}{HTML}{2A1B81}
\hypersetup{colorlinks,linkcolor=,urlcolor=links}

\tabulinesep=0.7mm

\title{Проектирование программного обеспечения (практика)}
\subtitle{Практика 1: Введение, задача про CLI}
\author[Юрий Литвинов]{Юрий Литвинов \newline \textcolor{gray}{\small\texttt{yurii.litvinov@gmail.com}}}

\date{17.01.2022}

\begin{document}
    
    \frame{\titlepage}

    \section{Введение}
    
    \begin{frame}
        \frametitle{Формальности}
        \begin{itemize}
            \item В конце курса оценка
            \begin{itemize}
                \item Домашние работы
                \item Дедлайны (-50\% баллов за пропуск)
                \item Работа в аудитории
                \item Индивидуальные и групповые задачи
                \item Связанные цепочки заданий
            \end{itemize}
            \item Материалы курса и условия заданий будут на вики (рано или  поздно)
            \item Коммуникации --- в чате курса в Telegram
            \begin{itemize}
                \item Ссылка на пуллреквест в собственный репозиторий на GitHub
                \item Репозиторий написать мне в личку в Telegram (\url{https://t.me/yurii_litvinov})
                \item Задач будет несколько, так что выкладывать лучше так, чтобы их все было можно смерджить
            \end{itemize}
        \end{itemize}
    \end{frame}

    \begin{frame}
        \frametitle{Критерии оценивания}
        \begin{itemize}
            \item Меньше 60\% заданий -- 0
            \item От 60\% до 100\% --- линейная шкала от 2 до 10 баллов
            \begin{itemize}
                \item то есть, ровно 60\% заданий --- 2 балла
                \item 80\% заданий --- 6 баллов
                \item 100\% заданий --- 10 баллов
            \end{itemize}
            \item Задачи оцениваются по 10 баллов
            \item Оценки за работу на паре с небольшим весом (ближайшая такая пара через две недели)
            \begin{itemize}
                \item Оценка входит в процент выполненных заданий!
            \end{itemize}
            \item В итоговой оценке практика учитывается с весом 0.4, экзамен --- 0.6
            \item Округление арифметическое
            \item ``Принцип мажорирующей двойки''
            \item Две оценки --- в конце этого и следующего модулей
            \begin{itemize}
                \item Итоговая --- оценка за 3-й модуль с весом 0.6 и оценка за 4-й модуль с весом 0.4
            \end{itemize}
        \end{itemize}
    \end{frame}

    \begin{frame}
        \frametitle{Краткое содержание курса по практике}
        \begin{itemize}
            \item Снова лекции
            \begin{itemize}
                \item Про практические аспекты архитектуры
                \item Про архитектурную документацию
                \item Про UML и другие языки проектирования
                \item Про антипаттерны
                \item Про распределённые приложения и технологии, с ними связанные
                \item Про деплой и облачные сервисы
                \item Различные примеры архитектур
            \end{itemize}
            \item Небольшие задачи прямо на паре
            \begin{itemize}
                \item Проектировать разные приложения
                \item Технические вещи, типа рисования диаграмм
            \end{itemize}
            \item Относительно большие задачи на дом, как на проектирование, так и на реализацию
        \end{itemize}
    \end{frame}
    
    \begin{frame}
        \frametitle{Что ожидается от кода}
        \begin{itemize}
            \item Работоспособность и соответствие требованиям условия
            \item Наличие архитектурной документации
            \begin{itemize}
                \item Комментарии к каждому классу, интерфейсу и public-методу
                \item Краткое описание деталей реализации в README
            \end{itemize}
            \item Следование стайлгайдам и правилам здравого смысла 
            \item Язык программирования --- любой
            \item Наличие юнит-тестов
            \item Применение индустриальных практик: логирование, Continuous Integration, обработка исключений
        \end{itemize}
    \end{frame}

    \begin{frame}
        \frametitle{Ещё комментарии}
        \begin{itemize}
            \item Овердизайн и активное использование знаний с лекций приветствуются
            \item Обоснованность принятых решений важнее, чем техника кодирования
            \item Некоторые требования могут показаться ненужными --- это нормально
            \begin{itemize}
                \item Мы учимся не написанию кода, а инструментам и техникам проектирования
            \end{itemize}
            \item Комментарии вида ``у вас неправильная архитектура'' будут очень редки, как ни странно
            \item Списывать нельзя
        \end{itemize}
    \end{frame}

    \section{Задача про CLI}
    
    \begin{frame}
        \frametitle{Задача про CLI}
        Реализовать простой интерпретатор командной строки, поддерживающий команды:
        \begin{itemize}
            \item \textbf{cat [FILE]} --- вывести на экран содержимое файла
            \item \textbf{echo} --- вывести на экран свой аргумент (или аргументы)
            \item \textbf{wc [FILE]} --- вывести количество строк, слов и байт в файле
            \item \textbf{pwd} --- распечатать текущую директорию
            \item \textbf{exit} --- выйти из интерпретатора
        \end{itemize}
    \end{frame}
    
    \begin{frame}
        \frametitle{Задача про CLI (продолжение)}
        \begin{itemize}
            \item Должны поддерживаться одинарные и двойные кавычки (full and weak quoting)
            \item Окружение (команды вида ``имя=значение''), оператор \$
            \item Вызов внешней программы через Process (или его аналоги)
            \begin{itemize}
                \item если введено что-то, чего интерпретатор не знает
            \end{itemize}
            \item Пайплайны (оператор ``|'')
        \end{itemize}
    \end{frame}
    
    \begin{frame}[fragile]
        \frametitle{Примеры}
        \begin{small}
            \begin{minted}{sh}
>echo "Hello, world!"
Hello, world!

> FILE=example.txt
> cat $FILE
Some example text

> cat example.txt | wc
1 3 18

> echo 123 | wc
1 1 3

> x=ex
> y=it
> $x$y
            \end{minted}
        \end{small}
    \end{frame}

    \begin{frame}
        \frametitle{Что ожидается в качестве решения}
        \begin{itemize}
            \item Архитектурная документация, как умеете
            \begin{itemize}
                \item Структурная диаграмма (классов, компонентов, квадратиков со стрелочками)
                \item Словесное описание работы системы
                \item Достаточно подробно, чтобы не требовалось принимать важные решения при кодировании
                \begin{itemize}
                    \item Не должно быть <<ну тут мы парсим строку>>
                \end{itemize}
            \end{itemize}
            \item Реализовывать проект пока не нужно
            \begin{itemize}
                \item Через неделю будет задание это реализовать
            \end{itemize}
        \end{itemize}
    \end{frame}

    \begin{frame}
        \frametitle{Что делать дома}
        \begin{itemize}
            \item Сделать для этого курса репозиторий
            \item Прислать мне ссылку в Telegram (\url{https://t.me/yurii_litvinov})
            \item Одному из членов команды выложить решение в виде .md или .pdf-файла в отдельную ветку
            \item Сделать пуллреквест к себе в основную ветку
            \begin{itemize}
                \item Назвать его как-то разумно, чтобы было понятно, о какой задаче речь
            \end{itemize}
            \item Написать в чат курса, что задача готова к проверке
            \item Смерджить пуллреквест, когда задача зачтена
            \item \textbf{Дедлайн: 10:00 24.01.2022}
        \end{itemize}
    \end{frame}

    \begin{frame}
        \frametitle{Что делать сейчас}
        Первые фазы жизненного цикла
        \begin{itemize}
            \item Разбиться на команды по примерно три человека
            \item Выполнить анализ и определить подходы к решению
            \item Выявить подводные камни и способы их преодоления
            \item Декомпозировать задачу на подсистемы, классы и методы
            \item Нарисовать первое приближение структурной диаграммы
            \item Быть готовыми в конце пары выйти и рассказать предлагаемое решение
            \item Дома это надо будет уточнить, расширить и оформить
        \end{itemize}
    \end{frame}

    \begin{frame}
        \frametitle{Соображения}
        \begin{itemize}
            \item Проектирование сверху вниз
            \begin{itemize}
                \item Определитесь с общей структурой системы
                \item Определитесь с компонентами, их ответственностью и связями между ними
                \item Только после этого переходите к проектированию компонентов
                \begin{itemize}
                    \item По такой же схеме
                \end{itemize}
                \item Возможно, придётся возвращаться на уровень выше
            \end{itemize}
            \item Опасайтесь архитектурной жадности, надо вовремя остановиться
        \end{itemize}
    \end{frame}

    \begin{frame}
        \frametitle{На что обратить внимание}
        \begin{itemize}
            \item Как представляются команды и пайплайны?
            \item Как создаются команды?
            \item Как они исполняются? Как взаимодействуют потоки в пайплайне?
            \item Кто и как выполняет разбор входной строки?
            \begin{itemize}
                \item Кто, как и когда выполняет подстановки?
            \end{itemize}
            \item Как представляются переменные окружения?
            \item Что с многопоточностью?
        \end{itemize}
    \end{frame}

\end{document}
