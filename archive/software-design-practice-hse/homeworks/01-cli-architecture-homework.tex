\documentclass[a5paper]{homework}

\begin{document}

\makeHomeworkHeading{
    title = {Домашняя работа 1. Архитектура CLI},
    publicationDate = {17.01.2022},
    deadline = {24.01.2022},
    score = {10}
}

В командах по три человека спроектировать простой интерпретатор командной строки, поддерживающий команды:

\begin{itemize}
    \item cat [FILE] --- вывести на экран содержимое файла;
    \item echo --- вывести на экран свой аргумент (или аргументы);
    \item wc [FILE] --- вывести количество строк, слов и байт в файле;
    \item pwd --- распечатать текущую директорию;
    \item exit --- выйти из интерпретатора.
\end{itemize}

При этом должны поддерживаться (и, соответственно, явно отражены в архитектуре):

\begin{itemize}
    \item одинарные и двойные кавычки (full and weak quoting);
    \item окружение (команды вида <<имя=значение>>), оператор \$;
    \item вызов внешней программы, если введено что-то, чего интерпретатор не знает;
    \item пайплайны (оператор <<|>>).
\end{itemize}

Примеры:
\begin{minted}{bash}
>echo "Hello, world!"
Hello, world!

> FILE=example.txt
> cat $FILE
Some example text

> cat example.txt | wc
1 3 18

> echo 123 | wc
1 1 3

> x=ex
> y=it
> $x$y
\end{minted}

Решение должно удовлетворять следующим нефункциональным требованиям:

\begin{itemize}
    \item легко добавлять новые команды;
    \item чёткое разграничение ответственности между элементами архитектуры;
    \begin{itemize}
        \item это не должен быть просто клубок классов, требуется некая компонентная структура;
    \end{itemize} 
    \item наличие словесного архитектурного описания.
\end{itemize}

В следующих заданиях надо будет реализовать эту архитектуру, причём в два этапа --- сначала без подстановок и пайпов, затем всё. Соответственно, решение должно отражать этапность разработки. Как --- на ваше усмотрение (либо сделать две отдельные диаграммы и разбить текст на разделы <<фаза 1>> и <<фаза 2>>, либо выделить на диаграмме цветом функциональность первого и второго этапа). Пока код писать не надо.

Результатом должна являться структурная диаграмма (например, диаграмма классов UML --- пока как умеете), описывающая систему, и текстовое описание того, как спроектированное приложение должно работать. Решение сдаётся в виде .md или .pdf-файла в репозитории. При этом обязательно либо выложить исходник диаграммы, либо указать ссылку на исходник в каком-то из облачных сервисов.

В чём рисовать:

\begin{itemize}
    \item \url{https://www.diagrams.net/} --- бесплатный и относительно неплохой веб-редактор диаграмм;
    \item \url{https://www.genmymodel.com/} --- тоже неплохой веб-редактор;
    \item \url{https://www.visual-paradigm.com/download/community.jsp} --- десктопный бесплатный UML-редактор на Java.
\end{itemize}

Обязательно укажите, с кем вы в команде.

\end{document}
