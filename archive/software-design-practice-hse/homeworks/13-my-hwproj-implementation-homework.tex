\documentclass[a5paper]{homework}

\begin{document}

\makeHomeworkHeading{
    title = {Домашняя работа 13. MyHwProj, реализация},
    publicationDate = {25.04.2022},
    deadline = {16.05.2022},
    score = {10}
}

В командах реализовать систему для проверки домашних заданий студентов по требованиям из предыдущего задания. В дополнение к предыдущим требованиям, также нужно реализовать REST API для доступа к информации, хранимой в системе, по требованиям из соответствующего задания на практику (на сей раз реализовать по-настоящему, чтобы веб-UI и REST API использовали одну бизнес-логику\footnote{Вспомните про гексагональную архитектуру.}). При этом:

\begin{itemize}
    \item можно использовать любую технологию разработки веб-приложений, лишь бы она поддерживала работу с очередями сообщений и REST API;
    \item в качестве очереди сообщений, через которую фронтенд общается с раннерами, можно использовать любую очередь сообщений, особо рекомендуется RabbitMQ или Kafka;
    \item можно считать, что количество раннеров фиксировано: например, их всегда два.
\end{itemize}

Красивый пользовательский интерфейс не требуется (прямо вообще, убогая вёрстка без стилей вполне сойдёт), однако красивое оформление с использованием Bootstrap принесёт вам дополнительный балл.

Не забудьте описать в README, как это всё запускать.

\end{document}
