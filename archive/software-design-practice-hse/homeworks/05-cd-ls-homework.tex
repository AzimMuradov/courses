\documentclass[a5paper]{homework}

\begin{document}

\makeHomeworkHeading{
    title = {Домашняя работа 5. Cd, ls},
    publicationDate = {14.02.2022},
    deadline = {07.03.2022},
    score = {10}
}

Реализовать в Command-Line Interface \emph{другой команды} встроенные команды cd и ls. При этом:

\begin{itemize}
    \item обе команды могут принимать 0 или 1 аргумент, поведение их должно соответствовать поведению аналогичных команд традиционных шеллов (например, bash);
    \item может потребоваться изменение кода существующей реализации, поскольку cd меняет состояние шелла;
    \item текущая рабочая папка, которую меняет cd, должна меняться и для встроенных команд, и для внешних команд;
    \begin{itemize}
        \item например, cd ../../.. и потом git status должно показывать статус репозитория тремя папками выше начальной папки;
    \end{itemize}
    \item не забудьте про юнит-тесты;
    \item не забудьте также обновить архитектурную документацию. Диаграмму при этом можно не править.
\end{itemize}

Также надо написать краткое ревью на архитектуру и реализацию команды, в которой вам надо было реализовать задачу: насколько просто было реализовать новые команды и добавить на них юнит-тесты. Опишите также (вежливо), что было удобным, что показалось неудобным, что и как можно было бы сделать лучше. Ревью выложить как отдельный .md-файл в репозиторий.

Сдавать работу надо, создав форк репозитория жертв и сделать пуллреквест в исходный репозиторий (если стесняетесь, допускается также пуллреквест в свой форк).

Кому в каком репозитории делать, определяется случайно преподавателем, ссылки на репозитории <<жертв>> будут разосланы каждой команде отдельно. Вполне может быть, что доставшаяся вам реализация на незнакомом вам языке или технологии --- цель этой задачи состоит ещё и в расширении кругозора.

\end{document}
