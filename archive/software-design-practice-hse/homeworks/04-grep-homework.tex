\documentclass[a5paper]{homework}

\begin{document}

\makeHomeworkHeading{
    title = {Домашняя работа 4. Grep},
    publicationDate = {07.02.2022},
    deadline = {21.02.2022},
    score = {10}
}

В тех же командах, что и раньше, реализовать в своём Command-Line Interface из предыдущих домашних заданий поддержку встроенной команды grep.

Требуется поддержка:

\begin{itemize}
    \item регулярных выражений в запросе;
    \item ключа -w --- поиск только слова целиком;
    \begin{itemize}
        \item формально grep -w ищет подстроки, ограниченные <<non-word constituent character>>, а именно не буквы, цифры или символ подчёркивания; однако что такое буквы --- лучше спросить у вашей стандартной библиотеки, потому что бывает Unicode и его классы символов: впрочем, ваша реализация может отличаться от настоящего поведения grep;
    \end{itemize}
    \item ключа -i --- регистронезависимый (case-insensitive) поиск;
    \item ключа -A --- следующее за -A число говорит, сколько строк после совпадения надо распечатать: например, -A 0 печатает только строку, на которой найдено совпадение, а -A 10 --- ещё и 10 строк ниже;
    \begin{itemize}
        \item подумайте, что будет, если области печати пересекаются.
    \end{itemize}
\end{itemize}

Поскольку для grep требуется нетривиальный разбор ключей, его необходимо реализовать с использованием одной из библиотек для разбора аргументов командной строки (например, Apache Commons CLI). Выбрать библиотеку необходимо самостоятельно, и, что самое главное, \emph{описать в README.md, из чего выбирали и почему выбрали именно ту, что выбрали}.

Примеры:

\begin{minted}{bash}
> grep "Минимальный" README.md
Минимальный синтаксис grep

> grep "Минимальный$" README.md

> grep "^Минимальный" README.md
Минимальный синтаксис grep

> grep -i "минимальный" README.md
Минимальный синтаксис grep

> grep -w "Минимал" README.md

> grep -A 1 "II" README.md
\end{minted}

Нужно также:

\begin{itemize}
    \item обновить архитектурную документацию и заодно привести диаграмму к синтаксису UML, если она ему не соответствует;
    \item написать юнит-тесты на новую функциональность;
    \item написать комментарии.
\end{itemize}

Сдавать, как обычно, создав новую ветку для grep на базе предыдущей и сделав новый пуллреквест.

\end{document}
