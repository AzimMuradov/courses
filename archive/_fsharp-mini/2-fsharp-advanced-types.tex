\documentclass[xetex,mathserif,serif]{beamer}
\usepackage{polyglossia}
\setdefaultlanguage[babelshorthands=true]{russian}
\usepackage{listings}
\usepackage{tabu}

\useoutertheme{infolines}

\usepackage{fontspec}
\setmainfont{FreeSans}
\newfontfamily{\russianfonttt}{FreeSans}

\setbeamertemplate{blocks}[rounded][shadow=false]
\setbeamercolor*{block title example}{fg=green!50!black,bg=green!20}
\setbeamercolor*{block body example}{fg=black,bg=green!10}

\setbeamercolor*{block title alerted}{fg=red!50!black,bg=red!20}
\setbeamercolor*{block body alerted}{fg=black,bg=red!10}

\lstset{language=Caml,basicstyle=\small\normalfont,keywordstyle=\color{red},showstringspaces=false}

\lstset{emph={member,float,int,static,open,class,val,new,inherit,abstract,override,default,base,interface,int,bool,unit,string,module,struct,namespace,return,public,throw,use,finally,elif},emphstyle={\color{red}},deletekeywords={value}}

\DeclareMathSymbol{\mlq}{\mathord}{operators}{``}
\DeclareMathSymbol{\mrq}{\mathord}{operators}{`'}

\tabulinesep=1.2mm

\title{Базовые паттерны ФП. Генерики в F\#.}
\author{Юрий Литвинов}
\date{09.12.2016г}

\begin{document}
	
	\frame{\titlepage}
	
	\section{Паттерны ФП}

	\begin{frame}[fragile]
		\frametitle{Замена цикла рекурсией}
		\framesubtitle{Императивное разложение на множители}
		\begin{exampleblock}{F\#}
			\begin{lstlisting}
let factorizeImperative n =
    let mutable primefactor1 = 1
    let mutable primefactor2 = n
    let mutable i = 2
    let mutable fin = false
    while (i < n && not fin) do
        if (n % i = 0) then
            primefactor1 <- i
            primefactor2 <- n / i
            fin <- true
        i <- i + 1
    if (primefactor1 = 1) then None
    else Some (primefactor1, primefactor2)
\end{lstlisting}
\end{exampleblock}

\end{frame}

	\begin{frame}[fragile]
		\frametitle{Замена цикла рекурсией}
		\framesubtitle{Рекурсивное разложение на множители}
		\begin{exampleblock}{F\#}
			\begin{lstlisting}
let factorizeRecursive n =
    let rec find i =
        if i >= n then None
        elif (n % i = 0) then Some(i, n / i)
        else find (i + 1)
    find 2
\end{lstlisting}
\end{exampleblock}
		
\end{frame}

\begin{frame}[fragile]
	\frametitle{Хвостовая рекурсия, проблема}
	\framesubtitle{Императивный вариант}
	\begin{exampleblock}{F\#}
		\begin{lstlisting}
open System.Collections.Generic

let createMutableList() =
    let l = new List<int>()
    for i = 0 to 100000 do
        l.Add(i)
    l
\end{lstlisting}
\end{exampleblock}
	
\end{frame}

\begin{frame}[fragile]
	\frametitle{Хвостовая рекурсия, проблема}
	\framesubtitle{Рекурсивный вариант, казалось бы}
	\begin{exampleblock}{F\#}
		\begin{lstlisting}
let createImmutableList() =
    let rec createList i max =
        if i = max then
            []	
        else
            i :: createList (i + 1) max
    createList 0 100000
\end{lstlisting}
\end{exampleblock}
	
\end{frame}

\begin{frame}[fragile]
	\frametitle{Факториал без хвостовой рекурсии}
	\begin{exampleblock}{F\#}
		\begin{lstlisting}
let rec factorial x =
    if x <= 1
    then 1 
    else x * factorial (x - 1)
\end{lstlisting}
\end{exampleblock}
	
	\begin{exampleblock}{F\#}
		\begin{lstlisting}
let rec factorial x =
    if x <= 1
    then
        1
    else
        let resultOfRecusion = factorial (x - 1)
        let result = x * resultOfRecusion
        result
\end{lstlisting}
\end{exampleblock}
	
\end{frame}

\begin{frame}[fragile]
	\frametitle{Факториал с хвостовой рекурсией}
	\begin{exampleblock}{F\#}
		\begin{lstlisting}
let factorial x =
    let rec tailRecursiveFactorial x acc =
        if x <= 1 then
            acc
        else
            tailRecursiveFactorial (x - 1) (acc * x)
    tailRecursiveFactorial x 1
\end{lstlisting}
\end{exampleblock}

\end{frame}
	
\begin{frame}[fragile]
	\frametitle{После декомпиляции в C\#}
		\begin{alertblock}{C\#}
			\begin{lstlisting}[language=Java]
public static int tailRecursiveFactorial(int x, int acc)
{
    while (true)
    {
        if (x <= 1)
        {
            return acc;
        }
        acc *= x;
        x--;
    }
}
\end{lstlisting}
\end{alertblock}
	
\end{frame}

\begin{frame}[fragile]
	\frametitle{Паттерн ``Аккумулятор''}
	\begin{exampleblock}{F\#}
		\begin{lstlisting}
let rec map f list =
    match list with
    | [] -> []
    | hd :: tl -> (f hd) :: (map f tl)

let map f list =
    let rec mapTR f list acc =
        match list with
        | [] -> acc
        | hd :: tl -> mapTR f tl (f hd :: acc)
    mapTR f (List.rev list) []
\end{lstlisting}
\end{exampleblock}
	
\end{frame}

\begin{frame}[fragile]
	\frametitle{Continuation Passing Style}
	\frametitle{Аккумулятор --- функция}
	\begin{exampleblock}{F\#}
		\begin{lstlisting}
let printListRev list =
    let rec printListRevTR list cont =
        match list with
        | [] -> cont ()
        | hd :: tl ->
            printListRevTR tl (fun () -> 
                printf "%d " hd; cont () )
    printListRevTR list (fun () -> printfn "Done!")
\end{lstlisting}
\end{exampleblock}
	
\end{frame}

	\section{Шаблонные типы}
	
	\begin{frame}[fragile]
		\frametitle{Шаблонные типы}
		\begin{exampleblock}{F\#}
			\begin{lstlisting}
type 'a list = ...
type list<'a> = ...
\end{lstlisting}
\end{exampleblock}

		\begin{exampleblock}{F\#}
			\begin{lstlisting}
List.map : ('a -> 'b) -> 'a list -> 'b list
let map<'a,'b> : ('a -> 'b) -> 'a list -> 'b list = 
    List.map

let rec map (f : 'a -> 'b) (l : 'a list) =
    match l with
    | h :: t -> (f h) :: (map f t)
    | [] -> []
\end{lstlisting}
\end{exampleblock}
\end{frame}

	\begin{frame}[fragile]
		\frametitle{Автоматическое обобщение}
		\begin{exampleblock}{F\#}
			\begin{lstlisting}
let getFirst (a, b, c) = a
let mapPair f g (x, y) = (f x, g y)
\end{lstlisting}
\end{exampleblock}

		\begin{alertblock}{F\# Interactive}
			\begin{lstlisting}
val getFirst: 'a * 'b * 'c -> 'a
val mapPair : ('a -> 'b) -> ('c -> 'd) 
    -> ('a * 'c) -> ('b * 'd)
\end{lstlisting}
\end{alertblock}
\end{frame}
	
	\section{Встроенные шаблонные операции}

	\begin{frame}[fragile]
		\frametitle{Generic-сравнение}
		\begin{exampleblock}{F\#}
			\begin{lstlisting}
val compare : 'a -> 'a -> int
val (=) : 'a -> 'a -> bool
val (<) : 'a -> 'a -> bool
val (<=) : 'a -> 'a -> bool
val (>) : 'a -> 'a -> bool
val (>=) : 'a -> 'a -> bool
val (min) : 'a -> 'a -> 'a
val (max) : 'a -> 'a -> 'a
\end{lstlisting}
\end{exampleblock}
\end{frame}

	\begin{frame}[fragile]
		\frametitle{Сравнение сложных типов}
		\begin{alertblock}{F\# Interactive}
			\begin{lstlisting}[keywordstyle=\color{black}]
> ("abc", "def") < ("abc", "xyz");;
val it : bool = true
> compare (10, 30) (10, 20);;
val it : int = 1

> compare [10; 30] [10; 20];;
val it : int = 1
> compare [| 10; 30 |] [| 10; 20 |];;
val it : int = 1
> compare [| 10; 20 |] [| 10; 30 |];;
val it : int = -1
\end{lstlisting}
\end{alertblock}
\end{frame}

	\begin{frame}[fragile]
		\frametitle{Generic-печать}
		\begin{alertblock}{F\# Interactive}
			\begin{lstlisting}[keywordstyle=\color{black}]
> any_to_string (Some(100, [1.0; 2.0; 3.1415]));;
val it : string = "Some (100, [1.0; 2.0; 3.1415])"

> sprintf "result = %A" ([1], [true]);;
val it : string = "result = ([1], [true])"
val it : int = -1
\end{lstlisting}
\end{alertblock}
\end{frame}

	\begin{frame}[fragile]
		\frametitle{Boxing/unboxing}
		\begin{alertblock}{F\# Interactive}
			\begin{lstlisting}[keywordstyle=\color{black}]
> box 1;;
val it : obj = 1

> box "abc";;
val it : obj = "abc"

> let sobj = box "abc";;
val sobj : obj = "abc"

> (unbox<string> sobj);;
val it : string = "abc"

> (unbox sobj : string);;
val it : string = "abc"
\end{lstlisting}
\end{alertblock}
\end{frame}

	\begin{frame}[fragile]
		\frametitle{Сериализация}
		\begin{exampleblock}{F\#}
			\begin{lstlisting}
open System.IO
open System.Runtime.Serialization.Formatters.Binary

let writeValue outputStream (x: 'a) =
    let formatter = new BinaryFormatter()
    formatter.Serialize(outputStream, box x)

let readValue inputStream =
    let formatter = new BinaryFormatter()
    let res = formatter.Deserialize(inputStream)
    unbox res
\end{lstlisting}
\end{exampleblock}
\end{frame}

	\begin{frame}[fragile]
		\frametitle{Сериализация, пример использования}
		\begin{exampleblock}{F\#}
			\begin{lstlisting}
let addresses = Map.of_list [ 
    "Jeff", "123 Main Street, Redmond, WA 98052";
    "Fred", "987 Pine Road, Phila., PA 19116";
    "Mary", "PO Box 112233, Palo Alto, CA 94301" ]

let fsOut = new FileStream("Data.dat", FileMode.Create)
writeValue fsOut addresses
fsOut.Close()

let fsIn = new FileStream("Data.dat", FileMode.Open)
let res : Map<string,string> = readValue fsIn
fsIn.Close()
\end{lstlisting}
\end{exampleblock}
\end{frame}

	\section{Обобщение кода}

	\begin{frame}[fragile]
		\frametitle{Алгоритм Евклида, не генерик}
		\begin{exampleblock}{F\#}
			\begin{lstlisting}
let rec hcf a b =
    if a = 0 then b
    elif a < b then hcf a (b - a)
    else hcf (a - b) b
\end{lstlisting}
\end{exampleblock}

\begin{alertblock}{F\# Interactive}
\begin{lstlisting}[keywordstyle=\color{black}]
val hcf : int -> int -> int

> hcf 18 12;;
val it : int = 6

> hcf 33 24;;
val it : int = 3
\end{lstlisting}
\end{alertblock}
\end{frame}

	\begin{frame}[fragile]
		\frametitle{Алгоритм Евклида, генерик}
		\begin{exampleblock}{F\#}
			\begin{lstlisting}
let hcfGeneric (zero, sub, lessThan) =
    let rec hcf a b =
        if a = zero then b
        elif lessThan a b then hcf a (sub b a)
        else hcf (sub a b) b
    hcf    
    
let hcfInt = hcfGeneric (0, (-), (<))
let hcfInt64 = hcfGeneric (0L, (-), (<))
let hcfBigInt = hcfGeneric (0I, (-), (<))
\end{lstlisting}
\end{exampleblock}

\begin{alertblock}{F\# Interactive}
\begin{lstlisting}[keywordstyle=\color{black}]
val hcfGeneric: 'a * ('a -> 'a -> 'a) * ('a -> 'a -> bool) 
    -> ('a -> 'a -> 'a)
\end{lstlisting}
\end{alertblock}
\end{frame}

	\begin{frame}[fragile]
		\frametitle{Словари операций}
		\begin{exampleblock}{F\#}
			\begin{lstlisting}
type Numeric<'a> =
    { Zero: 'a;
      Subtract: ('a -> 'a -> 'a);
      LessThan: ('a -> 'a -> bool); }

let hcfGeneric (ops : Numeric<'a>) =
    let rec hcf a b =
        if a = ops.Zero then b
        elif ops.LessThan a b then hcf a 
            (ops.Subtract b a)
        else hcf (ops.Subtract a b) b
    hcf
\end{lstlisting}
\end{exampleblock}
\end{frame}

	\begin{frame}[fragile]
		\frametitle{Тип функции}
\begin{alertblock}{F\# Interactive}
\begin{lstlisting}[keywordstyle=\color{black}]
val hcfGeneric : Numeric<'a> -> ('a -> 'a -> 'a)
\end{lstlisting}
\end{alertblock}
\end{frame}

	\begin{frame}[fragile]
		\frametitle{Примеры использования}
		\begin{exampleblock}{F\#}
			\begin{lstlisting}
let intOps = { Zero = 0; 
    Subtract = (-); 
    LessThan = (<) }
    
let bigintOps = { Zero = 0I; 
    Subtract = (-); 
    LessThan = (<) }

let hcfInt = hcfGeneric intOps
let hcfBigInt = hcfGeneric bigintOps
\end{lstlisting}
\end{exampleblock}
\end{frame}

	\begin{frame}[fragile]
		\frametitle{Результат}
\begin{alertblock}{F\# Interactive}
\begin{lstlisting}[keywordstyle=\color{black}]
val hcfInt : (int -> int -> int)
val hcfBigInt : (bigint -> bigint -> bigint)

> hcfInt 18 12;;
val it : int = 6

> hcfBigInt 1810287116162232383039576I 
    1239028178293092830480239032I;;
val it : bigint = 33224I
\end{lstlisting}
\end{alertblock}
\end{frame}

	\section{Наследование и генерики}
	
	\begin{frame}[fragile]
		\frametitle{Повышающий каст}
\begin{alertblock}{F\# Interactive}
\begin{lstlisting}[keywordstyle=\color{black}]
> let xobj = (1 :> obj);;
val xobj : obj = 1

> let sobj = ("abc" :> obj);;
val sobj : obj = "abc"
\end{lstlisting}
\end{alertblock}
\end{frame}

	\begin{frame}[fragile]
		\frametitle{Понижающий каст}
\begin{alertblock}{F\# Interactive}
\begin{lstlisting}[keywordstyle=\color{black}]
> let boxedObject = box "abc";;
val boxedObject : obj

> let downcastString = (boxedObject :?> string);;
val downcastString : string = "abc"

> let xobj = box 1;;
val xobj : obj = 1

> let x = (xobj :?> string);;
error: InvalidCastException raised at or near stdin:(2,0)
\end{lstlisting}
\end{alertblock}
\end{frame}

	\begin{frame}[fragile]
		\frametitle{Каст и сопоставление шаблонов}
		\begin{exampleblock}{F\#}
			\begin{lstlisting}
let checkObject (x: obj) =
    match x with
    | :? string -> printfn "The object is a string"
    | :? int -> printfn "The object is an integer"
    | _ -> printfn "The input is something else"

let reportObject (x: obj) =
    match x with
    | :? string as s -> 
        printfn "The input is the string '%s'" s
    | :? int as d -> 
        printfn "The input is the integer '%d'" d
    | _ -> printfn "the input is something else"
\end{lstlisting}
\end{exampleblock}
\end{frame}

	\begin{frame}[fragile]
		\frametitle{Гибкие ограничения}
\begin{alertblock}{F\# Interactive}
\begin{lstlisting}[keywordstyle=\color{black}]
> open System.Windows.Forms
> let setTextOfControl (c : #Control) (s:string) = 
    c.Text <- s;;
val setTextOfControl: #Control -> string -> unit

> open System.Windows.Forms;;
> let setTextOfControl (c : 'a when 'a :> Control) 
    (s:string) = c.Text <- s;;
val setTextOfControl: #Control -> string -> unit
\end{lstlisting}
\end{alertblock}
\end{frame}

	\begin{frame}[fragile]
		\frametitle{Гибкие ограничения: пример}
		\begin{exampleblock}{F\#}
			\begin{lstlisting}
module Seq =
...
val append : #seq<'a> -> #seq<'a> -> seq<'a>
val concat : #seq<#seq<'a>> -> seq<'a>
...

Seq.append [1; 2; 3] [4; 5; 6]
Seq.append [| 1; 2; 3 |] [4; 5; 6]
Seq.append (seq { for x in 1 .. 3 -> x }) [4; 5; 6]
Seq.append [| 1; 2; 3 |] [| 4; 5; 6 |]
\end{lstlisting}
\end{exampleblock}
\end{frame}

	\begin{frame}[fragile]
		\frametitle{Повышающий каст: проблема}
		\begin{exampleblock}{F\#}
			\begin{lstlisting}
open System
open System.IO
let textReader =
    if DateTime.Today.DayOfWeek = DayOfWeek.Monday
    then Console.In
    else File.OpenText("input.txt")
\end{lstlisting}
\end{exampleblock}

\begin{alertblock}{F\# Interactive}
\begin{lstlisting}[keywordstyle=\color{black}]
else File.OpenText("input.txt")
^^^^^^^^^^^^^^^^^^^^^^^^^^^^^^^
error: FS0001: This expression has type StreamReader 
but is here used with type TextReader 
stopped due to error
\end{lstlisting}
\end{alertblock}
\end{frame}

	\begin{frame}[fragile]
		\frametitle{Повышающий каст: решение}
		\begin{exampleblock}{F\#}
			\begin{lstlisting}
let textReader =
    if DateTime.Today.DayOfWeek = DayOfWeek.Monday
    then Console.In
    else (File.OpenText("input.txt") :> TextReader)
\end{lstlisting}
\end{exampleblock}
\end{frame}

	\section{Отладка типов}

	\begin{frame}[fragile]
		\frametitle{Проблемы в выводе типов, методы и свойства}
\begin{alertblock}{F\# Interactive}
\begin{lstlisting}[keywordstyle=\color{black}]
> let transformData inp =
    inp |> Seq.map (fun (x, y) -> (x, y.Length));;

inp |> Seq.map (fun (x, y) -> (x, y.Length))
----------------------------------^^^^^^^^
stdin(11,36): error: Lookup on object of indeterminate 
type. A type annotation may be needed prior to this 
program point to constrain the type of the object. 
This may allow the lookup to be resolved.
\end{lstlisting}
\end{alertblock}
\end{frame}

	\begin{frame}[fragile]
		\frametitle{Решение}
		\begin{exampleblock}{F\#}
			\begin{lstlisting}
let transformData inp =
    inp |> Seq.map (fun (x, y:string) -> (x, y.Length))
\end{lstlisting}
\end{exampleblock}
\end{frame}

	\begin{frame}[fragile]
		\frametitle{Уменьшение общности}
		\begin{exampleblock}{F\#}
			\begin{lstlisting}
let printSecondElements (inp : #seq<'a * int>) =
    inp
    |> Seq.iter (fun (x, y) -> printfn "y = %d" x)
\end{lstlisting}
\end{exampleblock}

\begin{alertblock}{F\# Interactive}
\begin{lstlisting}[keywordstyle=\color{black}]
|> Seq.iter (fun (x, y) -> printfn "y = %d" x)
------------------------------------^^^^^^^^^
stdin(21,38): warning: FS0064: This construct causes 
code to be less generic than indicated by the type 
annotations. The type variable 'a has been 
constrained to the type 'int'.
\end{lstlisting}
\end{alertblock}
\end{frame}

	\begin{frame}[fragile]
		\frametitle{Уменьшение общности, отладка}
		\begin{exampleblock}{F\#}
			\begin{lstlisting}
type PingPong = Ping | Pong

let printSecondElements (inp : #seq<PingPong * int>) =
    inp |> Seq.iter (fun (x, y) -> printfn "y = %d" x)
\end{lstlisting}
\end{exampleblock}

\begin{alertblock}{F\# Interactive}
\begin{lstlisting}[keywordstyle=\color{black}]
|> Seq.iter (fun (x,y) -> printfn "y = %d" x)
-------------------------------------------^
stdin(27,47): error: FS0001: The type 'PingPong' is not 
compatible with any of the types byte, int16, int32, 
int64, sbyte, uint16, uint32, uint64, nativeint, 
unativeint, arising from the use of a printf-style 
format string
\end{lstlisting}
\end{alertblock}
\end{frame}

	\begin{frame}[fragile]
		\frametitle{Value Restriction}
\begin{alertblock}{F\# Interactive}
\begin{lstlisting}[keywordstyle=\color{black}]
> let empties = Array.create 100 [];;
------^^^^^^
error: FS0030: Value restriction. Type inference 
has inferred the signature 
val empties : '_a list []
but its definition is not a simple data constant. 
Either define 'empties' as a simple data expression, 
make it a function, or add a type constraint 
to instantiate the type parameters.
\end{lstlisting}
\end{alertblock}
\end{frame}

	\begin{frame}[fragile]
		\frametitle{Корректные определения}
		\begin{exampleblock}{F\#}
			\begin{lstlisting}
let emptyList = []
let initialLists = ([], [2])
let listOfEmptyLists = [[]; []]
let makeArray () = Array.create 100 []
\end{lstlisting}
\end{exampleblock}
		
\begin{alertblock}{F\# Interactive}
\begin{lstlisting}[keywordstyle=\color{black}]
val emptyList : 'a list
val initialLists : ('a list * int list)
val listOfEmptyLists : 'a list list
val makeArray : unit -> 'a list []
\end{lstlisting}
\end{alertblock}
\end{frame}

	\begin{frame}[fragile]
		\frametitle{Способы борьбы}
		\begin{exampleblock}{F\#}
			\begin{lstlisting}
let empties = Array.create 100 []
let empties : int list [] = Array.create 100 []


let mapFirst = List.map fst
('a * 'b) list -> 'a list

let mapFirst inp = List.map fst inp
let printFstElements = List.map fst
    >> List.iter (printf "res = %d")

let printFstElements inp = inp
    |> List.map fst
    |> List.iter (printf "res = %d")
\end{lstlisting}
\end{exampleblock}
\end{frame}

	\begin{frame}[fragile]
		\frametitle{Способы борьбы (2)}
		\begin{exampleblock}{F\#}
			\begin{lstlisting}
let empties = Array.create 100 []

let empties () = Array.create 100 []
let intEmpties : int list [] = empties()
let stringEmpties : string list [] = empties()

let emptyLists = Seq.init 100 (fun _ -> [])
let emptyLists<'a> : seq<'a list> = Seq.init 
    100 (fun _ -> [])
\end{lstlisting}
\end{exampleblock}
\end{frame}

	\begin{frame}[fragile]
		\frametitle{Способы борьбы, результат}
\begin{alertblock}{F\# Interactive}
\begin{lstlisting}[keywordstyle=\color{black}]
> Seq.length emptyLists;;
val it : int = 100

> emptyLists<int>;;
val it : seq<int list> = seq [[]; []; []; []; ...]

> emptyLists<string>;;
val it : seq<string list> = seq [[]; []; []; []; ...]
\end{lstlisting}
\end{alertblock}
\end{frame}

	\section{Point-free}
	
	\begin{frame}[fragile]
		\frametitle{Point-free}
		\begin{exampleblock}{F\#}
			\begin{lstlisting}
let fstGt0 xs = List.filter (fun (a, b) -> a > 0) xs

let fstGt0'1 : (int * int) list -> (int * int) list = 
    List.filter (fun (a, b) -> a > 0)

let fstGt0'2 : (int * int) list -> (int * int) list = 
    List.filter (fun x -> fst x > 0)

let fstGt0'3 : (int * int) list -> (int * int) list = 
    List.filter (fun x -> ((<) 0 << fst) x)

let fstGt0'4 : (int * int) list -> (int * int) list = 
    List.filter ((<=) 0 << fst)
\end{lstlisting}
\end{exampleblock}
\end{frame}

	\section{Арифметические операторы}

	\begin{frame}[fragile]
		\frametitle{Арифметические операторы}
		\begin{exampleblock}{F\#}
			\begin{lstlisting}
let twice x = (x + x)
let threeTimes x = (x + x + x)
let sixTimesInt64 (x:int64) = threeTimes x + threeTimes x
\end{lstlisting}
\end{exampleblock}

\begin{alertblock}{F\# Interactive}
\begin{lstlisting}[keywordstyle=\color{black}]
val twice : x:int -> int
val threeTimes : x:int64 -> int64
val sixTimesInt64 : x:int64 -> int64
\end{lstlisting}
\end{alertblock}
\end{frame}

\end{document}



