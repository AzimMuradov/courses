\documentclass[xetex,mathserif,serif]{beamer}
\usepackage{polyglossia}
\setdefaultlanguage[babelshorthands=true]{russian}
\usepackage{minted}

\useoutertheme{infolines}

\usepackage{fontspec}
\setmainfont{FreeSans}
\newfontfamily{\russianfonttt}{FreeSans}

\title{Практика, сетевая пушка}
\author[Юрий Литвинов]{Юрий Литвинов \newline \textcolor{gray}{\small\texttt{yurii.litvinov@gmail.com}}}

\date{29.05.2019г}

\begin{document}
	
	\frame{\titlepage}

	\begin{frame}
		\frametitle{Задача на пару}
		Переделать задачу про пушку из д/з 2 так, чтобы можно было играть по сети:
		\begin{itemize}
			\item должно быть можно запустить два приложения (возможно, на разных компьютерах) и подключить их друг к другу
			\begin{itemize}
				\item одно запускается в режиме сервера и ждёт подключения
				\item одно запускается в режиме клиента и подключается на указанный адрес и порт
			\end{itemize}
			\item пушки в реальном времени движутся и стреляют друг в друга
			\item когда одна из пушек уничтожена, обоим игрокам выводится сообщение о победе/поражении и игра начинается снова
		\end{itemize}
	\end{frame}

\end{document}
