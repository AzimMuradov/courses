\documentclass[a5paper]{article}
\usepackage[a5paper, top=8mm, bottom=8mm, left=8mm, right=8mm]{geometry}

\usepackage{polyglossia}
\setdefaultlanguage[babelshorthands=true]{russian}
\usepackage{minted}

\usepackage{fontspec}
\setmainfont{FreeSerif}
\newfontfamily{\russianfonttt}[Scale=0.7]{DejaVuSansMono}

\usepackage[font=scriptsize]{caption}

\usepackage{amsmath}
\usepackage{amssymb,amsfonts,textcomp}
\usepackage{color}
\usepackage{array}
\usepackage{hhline}
\usepackage{cite}
\usepackage{ulem}

\usepackage[xetex,linktocpage=true,plainpages=false,pdfpagelabels=false]{hyperref}
\hypersetup{colorlinks=true, linkcolor=blue, citecolor=blue, filecolor=blue, urlcolor=blue, pdftitle=1, pdfauthor=, pdfsubject=, pdfkeywords=}

\usepackage{tabu}

\usepackage{graphicx}
\usepackage{indentfirst}
\usepackage{multirow}
\usepackage{subfig}
\usepackage{footnote}
\usepackage{listings}

\newcommand{\attribution}[1] {
    \vspace{-4mm}\begin{flushright}\begin{scriptsize}%\textcolor{gray}
    {\textcopyright\, #1}\end{scriptsize}\end{flushright}
}

\sloppy
\pagestyle{plain}

\title{Практика 4: Моделирование требований}

\date{28.02.2022}

\begin{document}

\maketitle
\thispagestyle{empty}

\section{Задача практику}

Надо вспомнить запрос \url{https://bit.ly/defects-rfp} и построить по нему:

\begin{enumerate}
    \item диаграмму случаев использования, описывающую пользователей и случаи использования разрабатываемого приложения;
    \item диаграмму активностей для основного бизнес-процесса, поддерживаемого приложением --- регистрации и ремонта дефекта;
    \item BPMN-диаграмму для всего бизнес-процесса завода, включая внешних его участников.
\end{enumerate}

Как и в прошлый раз, работа ведётся в формате <<один человек на экране, остальные у себя повторяют>>, по одному человеку на диаграмму.

\end{document}
