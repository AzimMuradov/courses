\documentclass[xetex,mathserif,serif]{beamer}
\usepackage{polyglossia}
\setdefaultlanguage[babelshorthands=true]{russian}
\usepackage{listings}
\usepackage{tabu}

\useoutertheme{infolines}

\usepackage{fontspec}
\setmainfont{FreeSans}
\newfontfamily{\russianfonttt}{FreeSans}

\setbeamertemplate{blocks}[rounded][shadow=false]
\setbeamercolor*{block title example}{fg=green!50!black,bg=green!20}
\setbeamercolor*{block body example}{fg=black,bg=green!10}

\setbeamercolor*{block title alerted}{fg=red!50!black,bg=red!20}
\setbeamercolor*{block body alerted}{fg=black,bg=red!10}

\lstset{language=Caml,basicstyle=\small\normalfont,keywordstyle=\color{red},showstringspaces=false}

\lstset{emph={member,float,int,static,open,class,val,new,inherit,abstract,override,default,base,interface,int,bool,unit,string,module,struct,namespace,return,public,throw,use},emphstyle={\color{red}},deletekeywords={value}}

\DeclareMathSymbol{\mlq}{\mathord}{operators}{``}
\DeclareMathSymbol{\mrq}{\mathord}{operators}{`'}

\title{Async и многопоточное программирование}
\author{Юрий Литвинов}
\date{08.04.2016г}

\begin{document}
	
	\frame{\titlepage}
	
    \begin{frame}
        \frametitle{Многопоточное программирование}
        Зачем это нужно:
        \begin{itemize}
            \item Не ``вешать'' GUI
            \item Использовать ресурсы ``железа''
            \item Использовать асинхронные операции ввода-вывода
            \item Показывать прогресс
        \end{itemize}
    \end{frame}
    
    \begin{frame}
        \frametitle{Основные понятия}
        \begin{description}
            \item[Процесс] --- экземпляр выполняемой программы и все связанные с ней данные и ресурсы
            \item[Поток] --- поток исполняемых команд (стек, регистры)
            \item[Параллельная программа] --- программа с несколькими исполняемыми потоками или процессами
            \item[Асинхронная программа] выполняет запросы, которые выполняются не мгновенно, а через некоторое время
            \item[Реактивная программа] --- программа, нормальное состояние которой --- ожидать наступление какого-нибудь события
        \end{description}
    \end{frame}
    
    \begin{frame}[fragile]
        \frametitle{Простой способ: Async workflow}
   		\begin{exampleblock}{F\#}
   			\begin{lstlisting}[basicstyle=\scriptsize]
open System.Net
open System.IO
let sites = ["http://se.math.spbu.ru";
             "http://spisok.math.spbu.ru"]
let fetchAsync url =
    async { 
        do printfn "Creating request for %s..." url
        let request = WebRequest.Create(url)
        use! response = request.AsyncGetResponse()
        do printfn "Getting response stream for %s..." url
        use stream = response.GetResponseStream()
        do printfn "Reading response for %s..." url
        use reader = new StreamReader(stream)
        let html = reader.ReadToEnd()
        do printfn "Read %d characters for %s..." html.Length url 
    }

sites |> List.map (fun site -> site |> fetchAsync |> Async.Start) 
    |> ignore
\end{lstlisting}
\end{exampleblock}
\end{frame}

    \begin{frame}[fragile]
        \frametitle{Что получится}
		\begin{alertblock}{F\# Interactive}
			\begin{lstlisting}[basicstyle=\scriptsize,keywordstyle=\color{black},emphstyle=\color{black}]
Creating request for http://se.math.spbu.ru...
Creating request for http://spisok.math.spbu.ru...
val sites : string list =
  ["http://se.math.spbu.ru"; "http://spisok.math.spbu.ru"]
val fetchAsync : url:string -> Async<unit>
val it : unit = ()

> Getting response stream for http://spisok.math.spbu.ru...
Reading response for http://spisok.math.spbu.ru...
Read 4475 characters for http://spisok.math.spbu.ru...
Getting response stream for http://se.math.spbu.ru...
Reading response for http://se.math.spbu.ru...
Read 217 characters for http://se.math.spbu.ru...
\end{lstlisting}
\end{alertblock}
\end{frame}

    \begin{frame}[fragile]
        \frametitle{Переключение между потоками}
        Распечатаем Id потоков, в которых вызываются методы printfn:
        \begin{exampleblock}{F\#}
    		\begin{lstlisting}
open System.Threading

let tprintfn fmt =
    printf "[.NET Thread %d]"   
        Thread.CurrentThread.ManagedThreadId;
    printfn fmt
\end{lstlisting}
\end{exampleblock}
\end{frame}
    
    \begin{frame}[fragile]
        \frametitle{Что получилось теперь}
   		\begin{alertblock}{F\# Interactive}
   			\begin{lstlisting}[basicstyle=\scriptsize,keywordstyle=\color{black},emphstyle=\color{black}]
[.NET Thread 47][.NET Thread 49]Creating request 
        for http://se.math.spbu.ru...
Creating request for http://spisok.math.spbu.ru...
val sites : string list =
  ["http://se.math.spbu.ru"; "http://spisok.math.spbu.ru"]
val tprintfn : fmt:Printf.TextWriterFormat<'a> -> 'a
val fetchAsync : url:string -> Async<unit>
val it : unit = ()

> [.NET Thread 49]Getting response stream for 
        http://spisok.math.spbu.ru...
[.NET Thread 49]Reading response for http://spisok.math.spbu.ru...
[.NET Thread 50]Getting response stream for http://se.math.spbu.ru...
[.NET Thread 50]Reading response for http://se.math.spbu.ru...
[.NET Thread 50][.NET Thread 49]Read 217 characters 
        for http://se.math.spbu.ru...
Read 4475 characters for http://spisok.math.spbu.ru...
\end{lstlisting}
\end{alertblock}
\end{frame}
    
    \begin{frame}[fragile]
        \frametitle{Пул потоков}
        \framesubtitle{Почему такие результаты}
        Пул потоков --- набор заранее созданных потоков, управляемых рантаймом .NET автоматически.
        \begin{exampleblock}{F\#}
    		\begin{lstlisting}
open System.Threading

fun _ -> printfn "Hello from %d!" 
      <| Thread.CurrentThread.ManagedThreadId
|> List.replicate 10
|> List.map ThreadPool.QueueUserWorkItem 
|> ignore
\end{lstlisting}
\end{exampleblock}
\end{frame}

    \begin{frame}[fragile]
        \frametitle{Что получится}
   		\begin{alertblock}{F\# Interactive}
   			\begin{lstlisting}[keywordstyle=\color{black},emphstyle=\color{black}]
valHello from 73! 
Hello from 73! 
Hello from 73! 
Hello from 73! 
 it Hello from 77: Hello from Hello from 74! 
! 
unit = 65! 
Hello from Hello from 76! 
78! 
()

> Hello from 75! 
\end{lstlisting}
\end{alertblock}
\end{frame}

    \begin{frame}[fragile]
        \frametitle{Подробнее про Async}
        \framesubtitle{Async --- это Workflow}
        \begin{exampleblock}{F\#}
    		\begin{lstlisting}
type Async<'a> = Async of ('a -> unit) * (exn -> unit) 
        -> unit

type AsyncBuilder with
    member Return : 'a -> Async<'a>
    member Delay : (unit -> Async<'a>) -> Async<'a>
    member Using: 'a * ('a -> Async<'b>) -> 
            Async<'b> when 'a :> System.IDisposable
    member Let: 'a * ('a -> Async<'b>) -> Async<'b>
    member Bind: Async<'a> * ('a -> Async<'b>) 
            -> Async<'b>
\end{lstlisting}
\end{exampleblock}
\end{frame}

    \begin{frame}[fragile]
        \frametitle{Во что Async раскрывает компилятор}
        \framesubtitle{Если кто не помнит про Workflow-ы}
   		\begin{exampleblock}{F\#}
   			\begin{lstlisting}[basicstyle=\footnotesize]
async { 
    let request = WebRequest.Create(url)
    let! response = request.AsyncGetResponse()
    let stream = response.GetResponseStream()
    let reader = new StreamReader(stream)
    let html = reader.ReadToEnd()
    html
} 

async.Delay(fun () ->
    WebRequest.Create(url) |> (fun request ->
        async.Bind(request.AsyncGetResponse(), (fun response ->
            response.GetResponseStream() |> fun stream ->
                new StreamReader(stream) |> fun reader ->
                    reader.ReadToEnd() |> fun html ->
                        async.Return(html)))))
\end{lstlisting}
\end{exampleblock}
\end{frame}

	\begin{frame}
		\frametitle{Какие конструкции поддерживает Async}
		\begin{footnotesize}
			\begin{tabu} {| X[0.3 l p] | X[1 l p] |}
				\tabucline-
				Конструкция               & Описание           \\
				\tabucline-
				\everyrow{\tabucline-}
				let! pat = expr           & Выполняет асинхронное вычисление expr и присваивает результат pat, когда оно заканчивается   \\
				let pat = expr            & Выполняет синхронное вычисление expr и присваивает результат pat немедленно \\
				use! pat = expr           & Выполняет асинхронное вычисление expr и присваивает результат pat, когда оно заканчивается. Вызовет Dispose для каждого имени из pat, когда Async закончится.    \\
				use pat = expr            & Выполняет синхронное вычисление expr и присваивает результат pat немедленно. Вызовет Dispose для каждого имени из pat, когда Async закончится. \\
				do! expr                  & Выполняет асинхронную операцию expr, эквивалентно let! () = expr \\
				do expr                   & Выполняет синхронную операцию expr, эквивалентно let () = expr \\
				return expr               & Оборачивает expr в Async<'T> и возвращает его как результат Workflow \\
				return! expr              & Возвращает expr типа Async<'T> как результат Workflow \\
   			\end{tabu}
		\end{footnotesize}
	\end{frame}
	
	\begin{frame}
		\frametitle{Control.Async}
		\framesubtitle{Что можно делать со значением Async<'T>, сконструированным билдером}
		\begin{footnotesize}
			\begin{tabu} {| X[0.7 l p] | X[1 l p] | X[1 l p] |}
				\tabucline-
				Метод              & Тип                                         & Описание           \\
				\tabucline-
				\everyrow{\tabucline-}
				RunSynchronously   & Async<'T> * ?int * ?CancellationToken -> 'T & Выполняет вычисление синхронно, возвращает результат \\
				Start              & Async<unit> * ?CancellationToken -> unit    & Запускает вычисление асинхронно, тут же возвращает управление \\
				Parallel           & seq<Async<'T> > -> Async<'T []>             & По последовательности Async-ов делает новый Async, исполняющий все Async-и параллельно и возвращающий массив результатов \\
				Catch              & Async<'T> -> Async<Choice<'T,exn> >         & По Async-у делает новый Async, исполняющий Async и возвращающий либо результат, либо исключение \\
   			\end{tabu}
		\end{footnotesize}
	\end{frame}
	
	\begin{frame}[fragile]
		\frametitle{Пример}
   		\begin{exampleblock}{F\#}
   			\begin{lstlisting}
let writeFile fileName bufferData =
    async {
      use outputFile = System.IO.File.Create(fileName)
      do! outputFile.AsyncWrite(bufferData) 
    }

Seq.init 1000 (fun num -> createSomeData num)
|> Seq.mapi (fun num value -> 
      writeFile ("file" + num.ToString() + ".dat") value)
|> Async.Parallel
|> Async.RunSynchronously
|> ignore
\end{lstlisting}
\end{exampleblock}
\end{frame}	
	
	\begin{frame}[fragile]
		\frametitle{Подробнее про Async.Catch}
   		\begin{exampleblock}{F\#}
   			\begin{lstlisting}
asyncTaskX
    |> Async.Catch
    |> Async.RunSynchronously
    |> fun x ->
        match x with
        | Choice1Of2 result -> 
           printfn "Async operation completed: %A" result
        | Choice2Of2 (ex : exn) -> 
           printfn "Exception thrown: %s" ex.Message
\end{lstlisting}
\end{exampleblock}
\end{frame}
	
	\begin{frame}[fragile]
		\frametitle{Обработка исключений прямо внутри Async}
   		\begin{exampleblock}{F\#}
   			\begin{lstlisting}
async {
    try
        // ...
    with
    | :? IOException as ioe ->
        printfn "IOException: %s" ioe.Message
    | :? ArgumentException as ae ->
        printfn "ArgumentException: %s" ae.Message
}
\end{lstlisting}
\end{exampleblock}
\end{frame}

	\begin{frame}[fragile]
		\frametitle{Отмена операции}
		\framesubtitle{Задача, которую можно отменить}
   		\begin{exampleblock}{F\#}
   			\begin{lstlisting}
open System
open System.Threading

let cancelableTask =
    async {
        printfn "Waiting 10 seconds..."
        for i = 1 to 10 do
            printfn "%d..." i
            do! Async.Sleep(1000)
        printfn "Finished!"
    }
\end{lstlisting}
\end{exampleblock}
\end{frame}

	\begin{frame}[fragile]
		\frametitle{Отмена операции}
		\framesubtitle{Код, который её отменяет}
   		\begin{exampleblock}{F\#}
   			\begin{lstlisting}
let cancelHandler (ex : OperationCanceledException) =
    printfn "The task has been canceled."

Async.TryCancelled(cancelableTask, cancelHandler)
|> Async.Start

// ...

Async.CancelDefaultToken()
\end{lstlisting}
\end{exampleblock}
\end{frame}

	\begin{frame}[fragile]
		\frametitle{CancellationToken}
   		\begin{exampleblock}{F\#}
   			\begin{lstlisting}
open System.Threading

let computation = Async.TryCancelled(cancelableTask, 
        cancelHandler)
let cancellationSource = new CancellationTokenSource()

Async.Start(computation, cancellationSource.Token)

// ...

cancellationSource.Cancel()
\end{lstlisting}
\end{exampleblock}
\end{frame}

	\begin{frame}[fragile]
		\frametitle{Async.StartWithContinuations}
   		\begin{exampleblock}{F\#}
   			\begin{lstlisting}
Async.StartWithContinuations(
    someAsyncTask,
    (fun result -> printfn "Task completed with result %A" result),
    (fun exn -> 
        printfn "Task threw an exception with Message: 
                %s" exn.Message),
    (fun oce -> printfn "Task was cancelled. 
                Message: %s" oce.Message)
)
\end{lstlisting}
\end{exampleblock}
\end{frame}

	\begin{frame}[fragile]
		\frametitle{Async.FromContinuations}
   		\begin{exampleblock}{F\#}
   			\begin{lstlisting}[basicstyle=\scriptsize]
open System

let trylet f x = (try Choice1Of2 (f x) with exn -> Choice2Of2(exn))

let protect cont econt f x =
    match trylet f x with
    | Choice1Of2 v -> cont v
    | Choice2Of2 exn -> econt exn

type System.IO.Stream with
    member stream.ReadAsync (buffer, offset, count) =
        Async.FromContinuations (fun (cont, econt, cancel) ->
            stream.BeginRead
                (buffer = buffer,
                 offset = offset,
                 count = count,
                 state = null,
                 callback = 
                     AsyncCallback(protect cont econt stream.EndRead))
            |> ignore)
\end{lstlisting}
\end{exampleblock}
\end{frame}

	\begin{frame}[fragile]
		\frametitle{Пример}
		\framesubtitle{Реализация Async.Parallel}
   		\begin{exampleblock}{F\#}
   			\begin{lstlisting}[basicstyle=\footnotesize]
open System.Threading   			

let Parallel(taskSeq) =
    Async.FromContinuations (fun (cont, econt, cancel) ->
        let tasks = Seq.toArray taskSeq
        let count = ref tasks.Length
        let results = Array.zeroCreate tasks.Length
        tasks |> Array.iteri (fun i p ->
            Async.Start
                (async { let! res = p
                         do results.[i] <- res;
                         let n = Interlocked.Decrement(count)
                         do if n = 0 then cont results })))
\end{lstlisting}
\end{exampleblock}
\end{frame}

	\begin{frame}[fragile]
		\frametitle{Async.AwaitEvent}
		\framesubtitle{Для более простых случаев}
   		\begin{exampleblock}{F\#}
   			\begin{lstlisting}
open System

let timer = new Timers.Timer(2000.0)
let timerEvent = Async.AwaitEvent (timer.Elapsed) 
    |> Async.Ignore

printfn "Waiting for timer at %O" DateTime.Now.TimeOfDay
timer.Start()

printfn "Doing something useful while waiting for event"
Async.RunSynchronously timerEvent

printfn "Timer ticked at %O" DateTime.Now.TimeOfDay
\end{lstlisting}
\end{exampleblock}
\end{frame}

	\begin{frame}[fragile]
		\frametitle{Большой пример: обработка ``изображений''}
		\framesubtitle{Шаг 1: подготовка тестовых данных}
   		\begin{exampleblock}{F\#}
   			\begin{lstlisting}[basicstyle=\footnotesize]
open System.IO

let numImages = 200
let size = 512
let numPixels = size * size

let MakeImageFiles() =
    printfn "making %d %dx%d images... " numImages size size
    let pixels = Array.init numPixels (fun i -> byte i)
    for i = 1 to numImages do
        File.WriteAllBytes(sprintf "Image%d.tmp" i, pixels)
    printfn "done."
\end{lstlisting}
\end{exampleblock}
\end{frame}

	\begin{frame}[fragile]
		\frametitle{Большой пример: обработка ``изображений''}
		\framesubtitle{Шаг 2: вычислительно сложная функция над ``изображением''}
   		\begin{exampleblock}{F\#}
   			\begin{lstlisting}
let processImageRepeats = 20

let TransformImage(pixels, imageNum) =
    printfn "TransformImage %d" imageNum
    // Perform a CPU-intensive operation on the image.
    let mutable newPixels = pixels
    for i in 1..processImageRepeats do 
        newPixels <- Array.map (fun b -> b + 1uy)) pixels
    newPixels
\end{lstlisting}
\end{exampleblock}
\end{frame}

	\begin{frame}[fragile]
		\frametitle{Большой пример: обработка ``изображений''}
		\framesubtitle{Шаг 3: синхронная обработка}
   		\begin{exampleblock}{F\#}
   			\begin{lstlisting}
let ProcessImageSync(i) =
    use inStream = File.OpenRead(sprintf "Image%d.tmp" i)
    let pixels = Array.zeroCreate numPixels
    let nPixels = inStream.Read(pixels, 0, numPixels);
    let pixels' = TransformImage(pixels, i)
    use outStream 
            = File.OpenWrite(sprintf "Image%d.done" i)
    outStream.Write(pixels', 0, numPixels)

let ProcessImagesSync() =
    printfn "ProcessImagesSync...";
    for i in 1 .. numImages do
        ProcessImageSync(i)
\end{lstlisting}
\end{exampleblock}
\end{frame}

	\begin{frame}[fragile]
		\frametitle{Большой пример: обработка ``изображений''}
		\framesubtitle{Шаг 4: асинхронная обработка}
   		\begin{exampleblock}{F\#}
   			\begin{lstlisting}[basicstyle=\scriptsize]
open Microsoft.FSharp.Control
open Microsoft.FSharp.Control.CommonExtensions

let ProcessImageAsync i =
    async { use inStream = File.OpenRead(sprintf "Image%d.tmp" i)
            let! pixels = inStream.AsyncRead(numPixels)
            let pixels' = TransformImage(pixels, i)
            use outStream = File.OpenWrite(sprintf "Image%d.done" i)
            do! outStream.AsyncWrite(pixels') }

let ProcessImagesAsync() =
    printfn "ProcessImagesAsync..."
    let tasks = [ for i in 1 .. numImages -> ProcessImageAsync(i) ]
    Async.Parallel tasks |> Async.RunSynchronously |> ignore
    printfn "ProcessImagesAsync finished!"
\end{lstlisting}
\end{exampleblock}
\end{frame}

	\begin{frame}
		\frametitle{Доклады}
		Как обычно, успешный доклад --- -1 домашка. 2 недели на подготовку.
		\begin{itemize}
    		\item Юнит-тестирование в F\# (NUnit, FsUnit, FsCheck)
    		\item Единицы измерения в F\# 
    		\item Ленивые вычисления в F\# (lazy, seq)
    		\item Active patterns
    		\item Code Quotations
    		\item WebSharper    		
		\end{itemize}
    \end{frame}

\end{document}
