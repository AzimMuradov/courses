\documentclass[xetex,mathserif,serif]{beamer}
\usepackage{polyglossia}
\setdefaultlanguage[babelshorthands=true]{russian}
\usepackage{listings}

\useoutertheme{infolines}

\usepackage{fontspec}
\setmainfont{FreeSans}
\newfontfamily{\russianfonttt}{FreeSans}

\newtheorem{rustheorem}{Теорема}

\definecolor{links}{HTML}{2A1B81}
\hypersetup{colorlinks,linkcolor=,urlcolor=links}

\title{Нетипизированное $\lambda$-исчисление}
\author{Юрий Литвинов}
\date{28.02.2020г}

\begin{document}
	
	\frame{\titlepage}

	\section{Введение}
	
	\begin{frame}
		\frametitle{Лямбда-исчисление}
		\framesubtitle{Математическая основа функционального программирования}
		\begin{itemize}
			\item Формальная система, основанная на $\lambda$-нотации, ещё одна формализация
					понятия <<вычисление>>, помимо машин Тьюринга (и нормальных алгорифмов
					Маркова, если кто-то про них помнит)
			\item Введено Алонзо Чёрчем в 1930-х для исследований в теории вычислимости
			\item Имеет много разных модификаций, включая <<чистое>> $\lambda$-исчисление и
					разные типизированные $\lambda$-исчисления
			\item Реализовано в языке LISP, с тех пор прочно вошло в программистский обиход
					(даже анонимные делегаты в C\# называют лямбда-функциями, как вы помните)
		\end{itemize}
	\end{frame}
		
	\section{$\lambda$-нотация}
		
	\begin{frame}
		\frametitle{Лямбда-нотация}
		Способ вводить функции, не придумывая для них название каждый раз
		$$x \rightarrow t[x] \Longrightarrow \lambda x.t[x]$$
		Например,
		$$\lambda x.x$$
		$$\lambda x.x^2$$
	\end{frame}

	\begin{frame}
		\frametitle{Применение функции (или аппликация)}
		Математически привычно
		$$f(x)$$
		Но непонятно, о чём идёт речь --- о функции $f$, принимающей аргумент $x$, или о результате применения
		$f$ к $x$. 

		В лямбда-исчислении $f(x)$ обозначается как
		$$f \; x$$
		При этом принято, что
		$$\lambda x. x + y = \lambda x.(x + y), \;\;\; 
		\lambda x. x + y \neq (\lambda x.x) + y$$
		Примеры записи:
		$$(\lambda x.x^2) \; 5 = 25$$
		$$(\lambda x.\lambda y.x + y) \; 2 \; 5 = 7$$
	\end{frame}

	\begin{frame}
		\frametitle{Каррирование (Currying)}
		В $\lambda$-исчислении не нужны функции нескольких переменных:
		$$\lambda x.\lambda y.x + y \stackrel{def}{=} \lambda x \; y.x + y$$
		Можно понимать как функцию, которая возвращает функцию:
		$$\lambda x.\lambda y.x + y \equiv \lambda x.(\lambda y.x + y)$$
		$$\mathbb{R} \rightarrow (\mathbb{R} \rightarrow \mathbb{R})$$
		Частичное применение:
		$$(\lambda x.\lambda y.x + y) \; 5 \equiv \lambda x.(x + 5)$$
	\end{frame}
	
	\section{$\lambda$-исчисление как формальная система}
		
	\begin{frame}
		\frametitle{$\lambda$-исчисление как формальная система}
		\framesubtitle{Внезапно, математика на парах по проге}
		Всё, что было выше, хорошо, но неформально. Формализуем, чтобы иметь возможность
		применять математические методы.
		
		\textbf{Нетипизированное лямбда-исчисление:}
		\begin{itemize}
			\item Всё --- $\lambda$-термы (числа и операции вводятся через них)
			\begin{itemize}
				\item Не делается различий между данными и функциями, можно применять 
						функцию к функции (вообще говоря, есть только функции, они же являются данными)
			\end{itemize}
			\item Процесс вычисления вводится как набор формальных преобразований над
					$\lambda$-термами
			\begin{itemize}
				\item \textbf{Операционная} семантика
			\end{itemize}
		\end{itemize}
	\end{frame}

	\begin{frame}
		\frametitle{$\lambda$-термы}
		$\lambda$-терм --- это:
		\begin{itemize}
			\item Переменная: $v \in V$, где $V$ --- некоторое множество, называемое
					множеством переменных
			\item Аппликация: если $A$ и $B$ --- $\lambda$-термы, то $A \; B$ ---
					$\lambda$-терм.
			\item $\lambda$-абстракция: если $A$ --- $\lambda$-терм, а $v$ --- переменная,
					то $\lambda v. A$ --- $\lambda$-терм
			\item Других способов получить $\lambda$-терм нет
		\end{itemize}
	\end{frame}

	\begin{frame}
		\frametitle{Соглашения об ассоциативности}
		\framesubtitle{Чтобы не надо было писать кучу скобок}
		\begin{itemize}
			\item Аппликация левоассоциативна: $F \; X \; Y = (F \; X) \; Y$
			\item $\lambda$-абстракция правоассоциативна: 
					$\lambda x \; y.M  = \lambda x.(\lambda y.M)$
			\item $\lambda$-абстракция распространяется вправо настолько, 
					насколько возможно: $\lambda x.M \; N = (\lambda x.M \; N)$
		\end{itemize}
	\end{frame}

	\begin{frame}
		\frametitle{Свободные и связанные переменные}
		\begin{itemize}
			\item $\lambda$-абстракция $\lambda x.T[x]$ \textbf{связывает} переменную $x$ в терме $T[x]$
			\item Если значение выражения зависит от значения переменной, то говорят, что
					переменная \textbf{свободно} входит в выражение
		\end{itemize}
		Пример:
		$$\sum_{m = 1}^{n} m = \frac{n(n + 1)}{2}$$
		Здесь $n$ входит свободно, а $m$ связана.
		Имя связанной переменной можно менять:
		$$\int_{0}^{x}2y + a\ dy = x^2 + ax \longrightarrow \int_{0}^{x}2z + a\ dz = x^2 + ax$$
		но
		$$\int_{0}^{x}2a + a\ da \neq x^2 + ax$$				
	\end{frame}

	\begin{frame}
		\frametitle{Свободные и связанные переменные, формально}
		Как обычно, определение рекурсивно по структуре терма:
		\begin{itemize}
			\item $FV(x) = x$
			\item $FV(S \; T) = FV(S) \cup FV(T)$
			\item $FV(\lambda x.S) = FV(S) \setminus \{x\}$
		\end{itemize}

		\begin{itemize}
			\item $BV(x) = \emptyset$
			\item $BV(S \; T) = BV(S) \cup BV(T)$
			\item $BV(\lambda x.S) = BV(S) \cup \{x\}$
		\end{itemize}
		Примеры:
		$$S = (\lambda x\ y.x) (\lambda x.z\ x) \Rightarrow FV(S) = {z}, BV(S) = \{x,y\}$$
	\end{frame}
	
	\begin{frame}
		\frametitle{Подстановка}
		$T[x := S]$ - подстановка в терме $T$ терма $S$ вместо всех свободных вхождений 
		переменной $x$ (например, $x[x := T] = T$).
		
		Проблема:
		$$(\lambda y.x + y)[x := y] = \lambda y. y + y$$
		
		Решения:
		\begin{itemize}
			\item Запретить свободным переменным иметь одинаковые имена и называться так же, 
					как связанные (соглашение Барендрегта)
			\item Переименовывать связанные переменные <<на лету>> перед выполнением подстановки
		\end{itemize}		
	\end{frame}
	
	\begin{frame}
		\frametitle{Подстановка, формально}
		\begin{itemize}
			\item $x[x := T] = T$
			\item $y[x := T] = y$
			\item $(S_1\ S_2)[x := T] = S_1[x := T]\ S_2[x := T]$
			\item $(\lambda x.S)[x := z] = \lambda x.S$
			\item $(\lambda y.S)[x := T] = \lambda y.(S[x := T])$, если $y \notin FV(T)$ или $x \notin FV(S)$
			\item $(\lambda y.S)[x := T] = \lambda z.(S[y := z][x := T])$, иначе ($z$ при этом выбирается так, 
					что $z \notin FV(S) \cup FV(T)$
		\end{itemize}		
	\end{frame}
		
	\begin{frame}
		\frametitle{Зачем мы это делали}
		Можно ввести отношение \textbf{равенства} над термами, имеющее физический смысл 
		<<термы означают одно и то же>> и отношение \textbf{редукции}, означающее <<термы имеют 
		одинаковое \textbf{значение}>>, что нужно для определения \textbf{вычисления} (хотя заметьте, что пока в
		формальной системе даже понятия <<значение>> нет).
		
		Делать это мы будем, определив аксиомы и правила вывода над термами, через \textbf{преобразования}
		термов.
	\end{frame}
		
	\begin{frame}
		\frametitle{Преобразования}
		\begin{description}
			\item [$\alpha$-преобразование]: $\lambda x.S \rightarrow_\alpha 
					\lambda y.S[x := y]$ при условии, что $y \notin FV(S)$. 
					Даёт возможность переименовывать связанные переменные.
			\item [$\beta$-преобразование]: $(\lambda x.S) T \rightarrow_\beta S[x := T]$.
					Определяет процесс вычисления.
			\item [$\eta$-преобразование]: $\lambda x.T\ x \rightarrow_\eta T$, 
					если $x \notin FV(T)$. Обеспечивает	\textbf{экстенсиональность} 
					--- две функции экстенсионально эквивалентны, если на всех
					одинаковых входных данных дают одинаковый результат:
					$$\forall x\ F\ x = G\ x$$
		\end{description}
	\end{frame}

	\begin{frame}
		\frametitle{Аксиомы равенства $\lambda$-термов}
		$$\dfrac{S \rightarrow_\alpha T\ \ 
			\mbox{или}\ \ S \rightarrow_\beta T\ \ 
			\mbox{или}\ \  S \rightarrow_\eta T}{S = T}$$
		$$\dfrac{}{T = T}$$
		$$\dfrac{S = T}{T = S}$$
		$$\dfrac{S = T \wedge T = U}{S = U}$$
		$$\dfrac{S = T}{S\ U = T\ U}$$
		$$\dfrac{S = T}{U\ S = U\ T}$$
		$$\dfrac{S = T}{\lambda x.S = \lambda x.T}$$
	\end{frame}
	
	\section{$\beta$-редукция}

	\begin{frame}
		\frametitle{Вычисление, что мы хотим}
		Очевидно, что равенство --- это отношение эквивалентности. Оно <<не даёт терять
		информацию>>, потому что всегда можно вернуться к исходному терму. А мы хотим 
		вычислять значение терма, то есть всё-таки терять информацию о синтаксисе 
		терма, сохраняя	его <<смысл>>. Так что уберём симметричность, получив 
		отношение \textbf{$\beta$-редукции}, которое уже не эквивалентность и позволяет 
		делать с термом что-то осмысленное.
	\end{frame}

	\begin{frame}
		\frametitle{Аксиомы $\beta$-редукции}
		$$\dfrac{S \rightarrow_\alpha T\ \ 
			\mbox{или}\ \ S \rightarrow_\beta T\ \ 
			\mbox{или}\ \  S \rightarrow_\eta T}{S \rightarrow_\beta T}$$ 
		$$\dfrac{}{T \rightarrow_\beta T}$$
		$$\dfrac{S \rightarrow_\beta T \wedge T \rightarrow_\beta U}{S \rightarrow_\beta U}$$
		$$\dfrac{S \rightarrow_\beta T}{S\ U \rightarrow_\beta T\ U}$$
		$$\dfrac{S \rightarrow_\beta T}{U\ S \rightarrow_\beta U\ T}$$
		$$\dfrac{S \rightarrow_\beta T}{\lambda x.S \rightarrow_\beta \lambda x.T}$$
	\end{frame}

	\begin{frame}
		\frametitle{Пример}
		\framesubtitle{Редукция не всегда уменьшает размер терма}
		$$(\lambda x.x\ x\ x)\ (\lambda x.x\ x\ x) \rightarrow_\beta$$ 
		$$(\lambda x.x\ x\ x)\ (\lambda x.x\ x\ x)\ (\lambda x.x\ x\ x) \rightarrow_\beta$$
		$$(\lambda x.x\ x\ x)\ (\lambda x.x\ x\ x)\ (\lambda x.x\ x\ x)\ (\lambda x.x\ x\ x) \rightarrow_\beta ...$$
		так что 
		$$(\lambda x.y)\ ((\lambda x.x\ x\ x)\ (\lambda x.x\ x\ x)) \rightarrow_\beta$$ 
		$$(\lambda x.y)\ ((\lambda x.x\ x\ x)\ (\lambda x.x\ x\ x)\ (\lambda x.x\ x\ x)) \rightarrow_\beta$$
		$$(\lambda x.y)\ ((\lambda x.x\ x\ x)\ (\lambda x.x\ x\ x)\ (\lambda x.x\ x\ x)\ (\lambda x.x\ x\ x)) \rightarrow_\beta ...$$
		но
		$$(\lambda x.y)\ ((\lambda x.x\ x\ x)\ (\lambda x.x\ x\ x)) \rightarrow_\beta y$$		
	\end{frame}
	
	\begin{frame}
		\frametitle{Редексы}
		\framesubtitle{Reducible expressions}
		\textbf{Редэксом} называется пара термов, в которой можно выполнить подстановку, или выражение вида
		$$(\lambda x.S) T$$ 
		По правилу $\beta$-редукции 
		$$(\lambda x.S) T \rightarrow_\beta S[x := T]$$
		Например,
		$$(\lambda f.\lambda x.f\ x\ x) \textbf{+} \rightarrow_\beta \lambda x.\textbf{+}\ x\ x$$
		Терм без редэксов называется термом в \textbf{нормальной форме} (он вычислен, его нельзя дальше упростить)
	\end{frame}

	\begin{frame}
		\frametitle{Стратегии редукции}
		При выполнении редукции можно выбрать, какой редэкс заменять, это и есть стратегия редукции.
		\begin{description}
			\item[аппликативная] стратегия --- заменяем самый левый редэкс, не содержащий в себе других 
					редэксов (самое маленькое подвыражение)
			\item[нормальная] стратегия --- заменяем самый левый самый внешний редэкс
		\end{description}
		Аппликативная стратегия соответствует передаче параметра по значению (сначала вычисляем параметр, потом 
		передаём его в функцию), нормальная стратегия соответствует передаче параметра по имени (или ленивому вычислению),
		когда мы откладываем вычисление параметра до последнего, в надежде, что он нам не понадобится.
	\end{frame}

	\begin{frame}
		\frametitle{Какая стратегия лучше}
		\begin{rustheorem}[Карри о нормализации]
			Если у терма есть нормальная форма, то последовательное сокращение самого левого внешнего 
			редекса приводит к ней.
		\end{rustheorem}
		\begin{rustheorem}[Чёрча-Россера]
			Если терм $M$ $\beta$-редукцией редуцируется к термам $N$ и $K$, то существует терм $L$ такой, что
			к нему редуцируются и $N$, и $K$.
		\end{rustheorem}
		То есть нормальная форма не всегда есть (см. пример про $(\lambda x.x\ x\ x)\ (\lambda x.x\ x\ x)$), но
		если она есть, её можно получить нормальной стратегией, причём нормальная форма единственная.
	\end{frame}

	\section{Комбинаторы}

	\begin{frame}
		\frametitle{Комбинаторы}
		\textbf{Комбинатор} формально --- это $\lambda$-терм без свободных переменных. Неформально --- функция,
		которая позволяет комбинировать функции, без упоминания данных.
		
		Известные комбинаторы:
		$$\textbf{I} \equiv \lambda x.x \mbox{ --- тождественная функция}$$
		$$\omega \equiv \lambda s.s\ s \mbox{ --- комбинатор самоприменимости}$$
		$$\Omega \equiv \omega\omega \equiv (\lambda s.s\ s) (\lambda s.s\ s) \mbox{ --- расходящийся комбинатор}$$
		$$K \equiv \lambda x\ y.x \mbox{ --- канцеллятор (первый элемент пары)}$$
		$$K_\ast \equiv \lambda x\ y.y \mbox{ --- второй элемент пары}$$
		$$S \equiv \lambda x\ y\ z. x\ z\ (y\ z) \mbox{ --- коннектор}$$
		$$B \equiv \lambda f\ g\ x. f\ (g\ x) \mbox{ --- композиция}$$		
	\end{frame}
	
	\begin{frame}
		\frametitle{Комбинаторы, примеры}
		$$I\ I \equiv (\lambda x.x) (\lambda x.x) \rightarrow_\beta \lambda x.x \equiv I$$		
				
		$$K\ I \equiv (\lambda x.\lambda y.x) (\lambda x.x) \rightarrow_\beta $$		
		$$\rightarrow_\beta \lambda y.(\lambda x.x) \rightarrow_\alpha \lambda x.\lambda y.y \equiv K_\ast$$		
	\end{frame}

	\begin{frame}
		\frametitle{Комбинатор неподвижной точки}
		\begin{rustheorem}[О неподвижной точке]
			Для любого $\lambda$-терма $F$ существует неподвижная точка:
			$$\forall F\ \ \exists X : F\ X = X$$
			\vspace{-5mm}
		\end{rustheorem}
		\vspace{-5mm}
		\begin{rustheorem}[О комбинаторе неподвижной точки]
			Существует комбинатор неподвижной точки
			$$Y = \lambda f.(\lambda x.f\ (x\ x)) (\lambda x.f\ (x\ x))$$
			такой, что 
			$$\forall F\ \ \ F\ (Y\ F) = Y\ F$$
		\end{rustheorem}
		\vspace{-5mm}
		\begin{proof}
			\vspace{-7mm}
			$$Y\ F \equiv (\lambda x.F\ (x\ x))(\lambda x.F\ (x\ x)) 
				= F\ (\lambda x.F\ (x\ x))(\lambda x.F\ (x\ x)) = F(Y\ F)$$
			\vspace{-10mm}
		\end{proof}
	\end{frame}

	\begin{frame}
		\frametitle{Зачем это надо}
		Рекурсия. Проблема $\lambda$-исчисления в том, что у функций нет имён, поэтому они не могут 
		вызывать сами себя, вообще.
		
		Например,
		$$factorial = \lambda n. if\ (isZero\ n)\ 1\ (mult\ n\ (factorial\ (pred\ n)))$$
		Но так писать нельзя, $factorial$ в правой части. Перепишем, применив $\eta$-преобразование:
		$$factorial = (\lambda f.\lambda n.if\ (isZero\ n)\ 1\ (mult\ n\ (f\ (pred\ n)))) factorial$$
		Внезапно, 
		$$factorial = Y (\lambda f.\lambda n.if\ (isZero\ n)\ 1\ (mult\ n\ (f\ (pred\ n))))$$
		(ну, $F\ (Y\ F) = Y\ F$, тут $factorial$ выступает в роли неподвижной точки, а $F$ --- 
		штуки в скобках).
	\end{frame}

	\begin{frame}
		\frametitle{Пример}
		$$factorial\ 3 = (Y\ F)\ 3$$
		$$= F\ (Y\ F)\ 3$$
		$$= if\ (isZero\ 3)\ 1\ (mult\ 3\ ((Y\ F)\ (pred\ 3)))$$
		$$= mult\ 3\ ((Y\ F)\ 2)$$
		$$= mult\ 3\ (F\ (Y\ F)\ 2)$$
		$$= mult\ 3\ (mult\ 2\ ((Y\ F)\ 1))$$
		$$= mult\ 3\ (mult\ 2\ (mult\ 1\ ((Y\ F)\ 0)))$$
		$$= mult\ 3\ (mult\ 2\ (mult\ 1\ 1))$$
		$$= 6$$
		(очень рекомендую курс ``Системы типизации лямбда-исчисления'' Дениса Москвина на \url{https://www.lektorium.tv}, примеры и часть дальнейшего изложения --- оттуда)
	\end{frame}

	\section{$\lambda$-исчисление как язык программирования}

	\begin{frame}
		\frametitle{Булевые выражения}
		Пока что на $\lambda$-исчислении факториал не написать, нет чисел и \textit{if}-ов. Начнём с булевых выражений:
		$$TRUE \equiv \lambda x.\lambda y.x$$
		$$FALSE \equiv \lambda x.\lambda y.y$$
		Ну и оператор \textbf{IF}:
		$$IF \equiv \lambda b.\lambda t.\lambda f.b\ t\ f$$
		--- обратите внимание, булевые константы вводились так, чтобы \textit{IF} получился таким простым
	\end{frame}

	\begin{frame}
		\frametitle{Булевые операторы}
		Ввести булевые операторы очень просто через \textit{IF}:
		$$AND \equiv \lambda a\ b. IF\ a\ b\ FALSE$$
		$$OR \equiv \lambda a\ b. IF\ a\ TRUE\ b$$
		$$NOT \equiv \lambda b.IF\ b\ FALSE\ TRUE$$
		Какова нормальная форма терма \textit{NOT}?
	\end{frame}

	\begin{frame}
		\frametitle{Ответ}
		$$NOT = \lambda b.IF\ b\ FALSE\ TRUE$$
		$$= \lambda b.((\lambda b'.\lambda t.\lambda f.b'\ t\ f)\ b\ (\lambda x.\lambda y.y)\ (\lambda x.\lambda y.x))$$
		$$\rightarrow_\beta \lambda b.((\lambda t.\lambda f.b\ t\ f)\ (\lambda x.\lambda y.y)\ (\lambda x.\lambda y.x))$$
		$$\rightarrow_\beta \lambda b.((\lambda f.b\ (\lambda x.\lambda y.y)\ f)\ (\lambda x.\lambda y.x))$$
		$$\rightarrow_\beta \lambda b.(b\ (\lambda x.\lambda y.y)\ (\lambda x.\lambda y.x))$$
		А если $b$ может быть только $TRUE$ и $FALSE$, всё проще:
		$$NOT = \lambda b\ t\ f.b\ f\ t$$
		Легко убедиться подстановкой $TRUE$ и $FALSE$ в обе формулы
	\end{frame}

	\begin{frame}
		\frametitle{Нумералы Чёрча}
		Теория чисел может быть введена через $\lambda$-исчисление. Числа вводятся так (нумералы Чёрча):
		$$0 \equiv \lambda s\ z.z$$
		$$1 \equiv \lambda s\ z.s\ z$$
		$$2 \equiv \lambda s\ z.s\ (s\ z)$$
		$$3 \equiv \lambda s\ z.s\ (s\ (s\ z))$$
		$$4 \equiv \lambda s\ z.s\ (s\ (s\ (s\ z)))$$
	\end{frame}

	\begin{frame}
		\frametitle{Арифметические операции}
		$$S \equiv \lambda n.\lambda f.\lambda x.f\ ((n\ f)\ x)$$
		то есть
		$$S\ n = (\lambda n\ f\ x.f\ (n\ f\ x))\ n$$
		$$\rightarrow_\beta \lambda f\ x.f\ (n\ f\ x)$$
		$$= n + 1$$
		
		Сложение:
		$$ADD \equiv \lambda m\ n.\lambda f\ x.(m\ f)\ ((n\ f)\ x))$$
		или
		$$ADD \equiv \lambda m\ n.(m\ S)\ n$$
	\end{frame}

	\begin{frame}
		\frametitle{Умножение и степень, проверка на 0}
		$$MUL \equiv \lambda m\ n.m\ (ADD\ n)\ 0$$
		$$EXP \equiv \lambda m\ n.m\ (MUL\ n)\ 1$$
		$$ISZRO \equiv \lambda n.n\ (\lambda x.FALSE)\ TRUE$$
	\end{frame}

	\begin{frame}
		\frametitle{Пары}
		Конструктор пары:
		$$PAIR \equiv \lambda x\ y\ f.f\ x\ y$$
		идея такая же, как у булевых констант и \textit{IF}-а --- обернуть значения в аппликацию функции. Конкретная пара:
		$$PAIR\ a\ b = \lambda f.f\ a\ b$$
		
		Проекции:
		$$FST \equiv \lambda p.p\ TRUE$$
		$$SND \equiv \lambda p.p\ FALSE$$
	\end{frame}

	\begin{frame}
		\frametitle{Почему}
		Потому что мы так определили $TRUE$ и $FALSE$:
		$$FST\ (PAIR\ a\ b) = PAIR\ a\ b\ TRUE$$
		$$\equiv (\lambda x\ y\ f.f\ x\ y)\ a\ b\ TRUE$$
		$$= TRUE\ a\ b$$
		$$= (\lambda x.\lambda y.x)\ a\ b$$
		$$= a$$

		$$SND\ (PAIR\ a\ b) = PAIR\ a\ b\ FALSE$$
		$$\equiv (\lambda x\ y\ f.f\ x\ y)\ a\ b\ FALSE$$
		$$= FALSE\ a\ b$$
		$$= (\lambda x.\lambda y.y)\ a\ b$$
		$$= b$$
	\end{frame}

	\begin{frame}
		\frametitle{Функция предшествования}
		Вспомогательные функции:
		$$ZP \equiv PAIR\ 0\ 0$$
		$$SP \equiv \lambda p.PAIR\ (SND\ p)\ (SUCC\ (SND\ p))$$
		\begin{itemize}
			\item $ZP$ --- ZeroPair
			\item $SP\ (PAIR\ i\ j) = PAIR\ j\ (j + 1)$
		\end{itemize}

		\vspace{3mm}

		Функция предшествования:
		$$PRED \equiv \lambda m.FST\ (m\ SP\ ZP)$$
	\end{frame}

	\begin{frame}
		\frametitle{Примитивная рекурсия}
		$$XZ \equiv \lambda x.PAIR\ x\ 0$$
		$$FS \equiv \lambda f\ p.PAIR\ (f\ (FST\ p)\ (SND\ p))\ (SUCC\ (SND\ p))$$
		$$REC \equiv \lambda m\ f\ x.FST\ (m\ (FS\ f)\ (XZ\ x))$$
		Получаем:
		$$(x, 0) \rightarrow$$
		$$(f\ x\ 0, 1) \rightarrow$$
		$$(f\ (f\ x\ 0) 1, 2) \rightarrow$$
		$$(f\ (f\ (f\ x\ 0)\ 1)\ 2, 3) \rightarrow ...$$
		В частности, 
		$$PRED = \lambda m.REC\ m\ (\lambda x\ y.y)\ 0$$
	\end{frame}

	\begin{frame}
		\frametitle{Списки}
		$$NIL \equiv \lambda c\ n.n$$
		$$CONS \equiv \lambda e\ l\ c\ n.c\ e\ (l\ c\ n)$$
		Например,
		$$[] \equiv NIL \equiv \lambda c\ n.n$$
		$$[5; 3; 2] \equiv CONS\ 5\ (CONS\ 3\ (CONS\ 2\ NIL)) \equiv \lambda c\ n.c\ 5\ (c\ 3\ (c\ 2\ n))$$
		Стандартные функции:
		$$EMPTY \equiv \lambda l.l\ (\lambda h\ t.FALSE)\ TRUE$$
		$$HEAD \equiv \lambda l.l\ (\lambda h\ t.h)\ 0$$
		$$TAIL \equiv \lambda l.FST\ (l\ (\lambda a\ b.PAIR\ (SND\ b)\ (CONS\ a\ (SND\ b)))\ (PAIR\ NIL\ NIL))$$
	\end{frame}

\end{document}