\documentclass[xetex,mathserif,serif]{beamer}
\usepackage{polyglossia}
\setdefaultlanguage[babelshorthands=true]{russian}
\usepackage{minted}

\useoutertheme{infolines}

\usepackage{fontspec}
\setmainfont{FreeSans}
\newfontfamily{\russianfonttt}{FreeSans}

\title{Функциональное программирование на языке F\#}
\subtitle{Введение}
\author{Юрий Литвинов}
\date{13.02.2018г}

\begin{document}
	
	\frame{\titlepage}
	
	\section{Введение}
	
	\begin{frame}
		\frametitle{О чём этот курс}
		\begin{itemize}
			\item Теория и практика функционального программирования
			\begin{itemize}
				\item $\lambda$-исчисление
				\item Базовые принципы ФП (программирование без состояний, 
					функции высших порядков, каррирование~и~т.д.)
				\item Типы в функциональном программировании (немутабельные коллекции,
					генерики, автообобщение~и~т.д.)
				\item Паттерны функционального программирования (CPS, монады, point-free)
			\end{itemize}
			\item Программирование на F\# 
			\begin{itemize}
				\item ООП в F\#
				\item Асинхронное и многопоточное программирование
			\end{itemize}
		\end{itemize}
	\end{frame}

	\begin{frame}
		\frametitle{Отчётность}
		\begin{itemize}
			\item Домашка (относительно много)
			\item Одна контрольная в середине семестра
			\item Курсовая работа
			\item Доклад (-1 домашка)
		\end{itemize}
	\end{frame}
	
	\section{Курсовые}

	\begin{frame}
		\frametitle{Отступление про курсовые работы}
		\begin{itemize}
			\item Курсовая по дисциплине, отдельно в зачётку не идёт
			\item Семестровая + некоторая наука + текст
			\item Объём --- 5-7 страниц содержательного текста
			\item Конференции
			\begin{itemize}
				\item {}<<Современные технологии в теории и практике программирования>> --- 31 марта
			\end{itemize}
		\end{itemize}
	\end{frame}
	
	\begin{frame}
		\frametitle{Структура отчёта}
		\begin{itemize}
			\item Титульный лист (http://math.spbu.ru/rus/study/alumni\_info.html)
			\item Оглавление
			\item Введение в предметную область, постановка задачи
			\item Обзор литературы и существующих решений
			\item Описание предлагаемого решения, сравнение с существующими
			\item Заключение
			\item Список источников (ГОСТ Р 7.0.5--2008)
			\item Приложения (если есть)
		\end{itemize}
	\end{frame}
	
	\begin{frame}
		\frametitle{Где брать темы}
		\begin{itemize}
			\item Продолжать начатое
			\item Стажировки JetBrains
			\item Студпроекты Теркома
			\item Придумать самим 
			\begin{itemize}
				\item Политически немудро, но может быть интересно
			\end{itemize}
			\item Взять что-нибудь у кого-нибудь поблизости
			\begin{itemize}
				\item Робототехника
				\item Формальные методы
				\item Machine Learning
			\end{itemize}
		\end{itemize}
	\end{frame}
	
	\section{Введение в функциональное программирование}
	
	\begin{frame}
		\frametitle{Императивное программирование}
		Программа как последовательность \textbf{операторов}, изменяющих \textbf{состояние} вычислителя.

		Для конечных программ есть \textbf{начальное состояние}, \textbf{конечное состояние} и последовательность переходов:
		$$\sigma = \sigma_1 \rightarrow \sigma_2 \rightarrow ... \rightarrow \sigma_n = \sigma'$$
		
		Основные понятия:
		\begin{itemize}
			\item Переменная
			\item Присваивание
			\item Поток управления
			\begin{itemize}
				\item Последовательное исполнение
				\item Ветвления
				\item Циклы
			\end{itemize}
		\end{itemize}
	\end{frame}
	
	\begin{frame}
		\frametitle{Функциональное программирование}
		Программа как вычисление значения \textbf{выражения} в математическом смысле на некоторых входных данных.
		$$\sigma' = f(\sigma)$$
	
		\begin{itemize}
			\item Нет состояния $\Rightarrow$ нет переменных
			\item Нет переменных $\Rightarrow$ нет циклов
			\item Нет явной спецификации потока управления
		\end{itemize}
		Порядок вычислений не важен, потому что нет состояния, результат вычисления зависит только от входных данных.
	\end{frame}
	
	\begin{frame}[fragile]
		\frametitle{Сравним}
		\begin{alertblock}{C++}
			\begin{minted}{fsharp}
int factorial(int n) {
    int result = 1;
    for (int i = 1; i <= n; ++i) {
        result *= i;
    }
    return result;
}
			\end{minted}
		\end{alertblock}
		\begin{exampleblock}{F\#}
			\begin{minted}{fsharp}
let rec factorial x =
    if x = 1 then 1 else x * factorial (x - 1)
            \end{minted}
		\end{exampleblock}
\end{frame}

	\begin{frame}[fragile]
		\frametitle{Как с этим жить}
		\begin{itemize}
			\item Состояние и переменные <<эмулируются>> параметрами функций
			\item Циклы <<эмулируются>> рекурсией
			\item Последовательность вычислений --- рекурсия + параметры
		\end{itemize}
		\begin{exampleblock}{F\#}
			\begin{minted}{fsharp}
let rec sumFirst3 ls acc i =
    if i = 3 then 
         acc 
    else 
        sumFirst3 
            (List.tail ls) 
            (acc + ls.Head) 
            (i + 1)
            \end{minted}
		\end{exampleblock}
\end{frame}

	\begin{frame}
		\frametitle{Зачем}
		\begin{itemize}
			\item Строгая математическая основа
			\item Семантика программ более естественна
			\begin{itemize}
				\item Применима математическая интуиция
			\end{itemize}
			\item Программы проще для анализа
			\begin{itemize}
				\item Автоматический вывод типов
				\item Оптимизации
			\end{itemize}
			\item Более декларативно
			\begin{itemize}
				\item Ленивость
				\item Распараллеливание
			\end{itemize}
			\item Модульность и переиспользуемость
			\item Программы более выразительны
		\end{itemize}
	\end{frame}
	
	\begin{frame}[fragile]
		\frametitle{Пример: функции высших порядков}
		\begin{exampleblock}{F\#}
			\begin{minted}{fsharp}
let sumFirst3 ls = 
    Seq.fold 
        (fun x acc -> acc + x) 
        0 
        (Seq.take 3 ls)
            \end{minted}
		\end{exampleblock}
		\begin{exampleblock}{F\#}
			\begin{minted}{fsharp}
let sumFirst3 ls = ls |> Seq.take 3 |> Seq.fold (+) 0
            \end{minted}
		\end{exampleblock}
		\begin{exampleblock}{F\#}
			\begin{minted}{fsharp}
let sumFirst3 = Seq.take 3 >> Seq.fold (+) 0
            \end{minted}
		\end{exampleblock}
\end{frame}

	\begin{frame}[fragile]
		\frametitle{Ещё пример}
		\framesubtitle{Возвести в квадрат и сложить все чётные числа в списке}
		\begin{exampleblock}{F\#}
			\begin{minted}{fsharp}
let calculate = 
    Seq.filter (fun x -> x % 2 = 0) 
    >> Seq.map (fun x -> x * x) 
    >> Seq.reduce (+)
            \end{minted}
		\end{exampleblock}
\end{frame}

	\begin{frame}
		\frametitle{Почему тогда все не пишут функционально}
		\begin{itemize}
			\item Чистые функции не могут оказывать влияние на внешний мир. Ввод-вывод, работа с данными,
					вообще выполнение каких-либо действий не укладывается в функциональную модель.
			\item Сложно анализировать производительность, иногда функциональные программы проигрывают
					в производительности императивным. <<Железо>>, грубо говоря, представляет собой 
					реализацию машины Тьюринга, тогда как функциональные программы определяются над
					$\lambda$-исчислением.
			\item Требуется математический склад ума и вообще желание думать.
		\end{itemize}
	\end{frame}

\end{document}



