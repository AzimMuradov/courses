\documentclass[xetex,mathserif,serif]{beamer}
\usepackage{polyglossia}
\setdefaultlanguage[babelshorthands=true]{russian}
\usepackage{listings}
\usepackage{tabu}

\useoutertheme{infolines}

\usepackage{fontspec}
\setmainfont{FreeSans}
\newfontfamily{\russianfonttt}{FreeSans}

\setbeamertemplate{blocks}[rounded][shadow=false]
\setbeamercolor*{block title example}{fg=green!50!black,bg=green!20}
\setbeamercolor*{block body example}{fg=black,bg=green!10}

\setbeamercolor*{block title alerted}{fg=red!50!black,bg=red!20}
\setbeamercolor*{block body alerted}{fg=black,bg=red!10}

\lstset{language=Caml,basicstyle=\small\normalfont,keywordstyle=\color{red},showstringspaces=false}

\DeclareMathSymbol{\mlq}{\mathord}{operators}{``}
\DeclareMathSymbol{\mrq}{\mathord}{operators}{`'}

\title{F\#: шаблоны, типы. Базовые паттерны ФП.}
\author{Юрий Литвинов}
\date{04.03.2016г}

\begin{document}
	
	\frame{\titlepage}
	
	\section{Сопоставление шаблонов}
	
	\begin{frame}[fragile]
		\frametitle{Сопоставление шаблонов}
\begin{exampleblock}{F\#}
\begin{lstlisting}
let urlFilter url agent =
    match (url, agent) with
    | "http://www.google.com", 99 -> true
    | "http://www.yandex.ru" , _ -> false
    | _, 86 -> true
    | _ -> false
\end{lstlisting}
\end{exampleblock}

\begin{exampleblock}{F\#}
\begin{lstlisting}
let sign x =
    match x with
    | _ when x < 0 -> -1
    | _ when x > 0 -> 1
    | _ -> 0
\end{lstlisting}
\end{exampleblock}
\end{frame}

	\begin{frame}[fragile]
		\frametitle{F\# --- не Prolog}
		Не получится писать так:
\begin{exampleblock}{F\#}
\begin{lstlisting}
let isSame pair =
    match pair with
    | (a, a) -> true
    | _ -> false
\end{lstlisting}
\end{exampleblock}
Нужно так:		
\begin{exampleblock}{F\#}
\begin{lstlisting}
let isSame pair =
    match pair with
    | (a, b) when a = b -> true
    | _ -> false
\end{lstlisting}
\end{exampleblock}
\end{frame}

	\begin{frame}
		\frametitle{Какие шаблоны бывают}
		\begin{small}
			\begin{tabu} {| X[0.9 l p] | X[1 l p] | X[1 l p] |}
				\tabucline-
				Синтаксис                               & Описание                  & Пример                                \\
				\tabucline-
				\everyrow{\tabucline-}
				$(pat, \ldots, pat)$                    & Кортеж                    & $(1, 2, (``3``, x))$                    \\
				$[pat; \ldots; pat]$                       & Список                    & $[x; y; 3]$                           \\
				$pat :: pat$                            & cons                      & $h :: t$                              \\
				$pat\ |\ pat$                             & "Или"                     & $[x]\ |\ [``X``\ |\ x]$                     \\
				$pat\ \&\ pat$                            & "И"                       & $[p] \& [(x, y)]$                     \\
				$pat\ as\ id$                             & Именованный шаблон        & $[x]\ as\ inp$                          \\
				$id$                                    & Переменная                & $x$                                   \\
				$\_$                                     & Wildcard (что угодно)     & $\_$                                   \\
				литерал                               & Константа                 & $239, DayOfWeek.Monday$ \\
				$:?\ type$                               & Проверка на тип           & $:?\ string$                           \\
			\end{tabu}
		\end{small}
	\end{frame}

	\section{Последовательности}
	
	\begin{frame}[fragile]
		\frametitle{Последовательности}
		\framesubtitle{Ленивый тип данных}
		\begin{exampleblock}{F\#}
			\begin{lstlisting}
seq {0 .. 2}
seq {1I .. 1000000000000I}
\end{lstlisting}
\end{exampleblock}

		\begin{exampleblock}{F\#}
			\begin{lstlisting}
open System.IO
let rec allFiles dir =
    Seq.append
    (dir |> Directory.GetFiles)
    (dir |> Directory.GetDirectories 
         |> Seq.map allFiles 
         |> Seq.concat)
\end{lstlisting}
\end{exampleblock}
\end{frame}

	\begin{frame}
		\frametitle{Типичные операции с последовательностями}
		\begin{small}
			\begin{tabu} {| X[0.5 l p] | X[1 l p] |}
				\tabucline-
				Операция                               & Тип                    \\
				\tabucline-
				\everyrow{\tabucline-}
				Seq.append                    & $\#seq<'a> \to \#seq<'a> \to seq<'a>$ \\
				Seq.concat                    & $\#seq<\#seq<'a>> \to seq<'a>$ \\
				Seq.choose                    & $('a \to 'b option) \to \#seq<'a> \to seq<'b>$ \\
				Seq.empty                     & $seq<'a>$ \\
				Seq.map                       & $('a \to 'b) \to \#seq<'a> \to \#seq<'b>$ \\
				Seq.filter                    & $('a \to bool) \to \#seq<'a> \to seq<'a>$ \\
				Seq.fold                      & $('s \to 'a \to 's) \to 's \to seq<'a> \to 's$ \\
				Seq.initInfinite          & $(int \to 'a) \to seq<'a>$ \\
\end{tabu}
\end{small}
\end{frame}

	\begin{frame}[fragile]
		\frametitle{Задание последовательностей}
			\begin{exampleblock}{F\#}
				\begin{lstlisting}
let squares = seq { for i in 0 .. 10 -> (i, i * i) }
seq { for (i, isquared) in squares -> 
         (i, isquared, i * isquared) }
\end{lstlisting}
\end{exampleblock}
			
			\begin{exampleblock}{F\#}
				\begin{lstlisting}
let checkerboardCoordinates n =
    seq { for row in 1 .. n do
        for col in 1 .. n do
            if (row + col) % 2 = 0 then
                yield (row, col) }
\end{lstlisting}
\end{exampleblock}	
\end{frame}

	\begin{frame}[fragile]
		\frametitle{Ленивое чтение из файла}
		\begin{exampleblock}{F\#}
			\begin{lstlisting}
let rec allFiles dir =
    seq { for file in Directory.GetFiles(dir) -> file
        for subdir in Directory.GetDirectories dir ->> 
            (allFiles subdir) }
\end{lstlisting}
\end{exampleblock}
		
		\begin{exampleblock}{F\#}
			\begin{lstlisting}
let reader =
    seq { 
        use reader = new StreamReader(
            File.OpenRead("test.txt")
            )
        while not reader.EndOfStream do
            yield reader.ReadLine() }
\end{lstlisting}
\end{exampleblock}	
\end{frame}

	\section{Записи}
	
	\begin{frame}[fragile]
		\frametitle{Записи}
		\begin{exampleblock}{F\#}
			\begin{lstlisting}
type Person =
    { Name: string;
      DateOfBirth: System.DateTime; }
\end{lstlisting}
\end{exampleblock}

		\begin{exampleblock}{F\#}
			\begin{lstlisting}
{ Name = "Bill"; 
  DateOfBirth = new System.DateTime(1962, 09, 02) }

{ new Person
  with Name = "Anna"
  and DateOfBirth = new System.DateTime(1968, 07, 23) }
\end{lstlisting}
\end{exampleblock}
	
\end{frame}

	\begin{frame}[fragile]
		\frametitle{Клонирование записей}
		\begin{exampleblock}{F\#}
			\begin{lstlisting}
type Car =
    {
        Make : string
        Model : string
        Year : int
    }
\end{lstlisting}
\end{exampleblock}
		
		\begin{exampleblock}{F\#}
			\begin{lstlisting}
let thisYear's = { Make = "SomeCar"; 
                   Model = "Luxury Sedan"; 
                   Year = 2010 }
let nextYear's = { thisYear's with Year = 2011 }
\end{lstlisting}
\end{exampleblock}
	
\end{frame}

	\section{Размеченные объединения}
	
	\begin{frame}[fragile]
		\frametitle{Размеченные объединения}
		\framesubtitle{Discriminated unions}
		\begin{exampleblock}{F\#}
			\begin{lstlisting}
type Route = int
type Make = string
type Model = string

type Transport =
    | Car of Make * Model
    | Bicycle
    | Bus of Route
\end{lstlisting}
\end{exampleblock}
		
\end{frame}

	\begin{frame}[fragile]
		\frametitle{Известные примеры}
		\begin{exampleblock}{F\#}
			\begin{lstlisting}
type 'a option =
    | None
    | Some of 'a
\end{lstlisting}
\end{exampleblock}

		\begin{exampleblock}{F\#}
			\begin{lstlisting}
type 'a list =
    | ([])
    | (::) of 'a * 'a list
\end{lstlisting}
\end{exampleblock}
		
\end{frame}

	\begin{frame}[fragile]
		\frametitle{Использование размеченных объединений}
		\begin{exampleblock}{F\#}
			\begin{lstlisting}
type IntOrBool = I of int | B of bool
let i = I 99
let b = B true
\end{lstlisting}
\end{exampleblock}
		
		\begin{exampleblock}{F\#}
			\begin{lstlisting}
type C = Circle of int | Rectangle of int * int

[1..10]
|> List.map Circle

[1..10]
|> List.zip [21..30]
|> List.map Rectangle
\end{lstlisting}
\end{exampleblock}
		
\end{frame}

	\begin{frame}[fragile]
		\frametitle{Использование в match}
		\begin{exampleblock}{F\#}
			\begin{lstlisting}
type Tree<'a> =
    | Tree of 'a * Tree<'a> * Tree<'a>
    | Tip of 'a

let rec size tree =
    match tree with
    | Tree(_, l, r) -> 1 + size l + size r
    | Tip _ -> 1
\end{lstlisting}
\end{exampleblock}
		
\end{frame}

	\begin{frame}[fragile]
		\frametitle{Пример}
		\framesubtitle{Дерево разбора логического выражения}
		\begin{exampleblock}{F\#}
			\begin{lstlisting}
type Proposition =
    | True
    | And of Proposition * Proposition
    | Or of Proposition * Proposition
    | Not of Proposition

let rec eval (p: Proposition) =
    match p with
    | True -> true
    | And(p1, p2) -> eval p1 && eval p2
    | Or (p1, p2) -> eval p1 || eval p2
    | Not(p1) -> not (eval p1)

printfn "%A" <| eval (Or(True, And(True, Not True)))
\end{lstlisting}
\end{exampleblock}
		
\end{frame}

	\begin{frame}[fragile]
		\frametitle{Взаимосвязанные типы}
		\begin{exampleblock}{F\#}
			\begin{lstlisting}
type node =
    { Name : string;
      Links : link list }
and link =
    | Dangling
    | Link of node
\end{lstlisting}
\end{exampleblock}
		
\end{frame}

	\section{Паттерны ФП}

	\begin{frame}[fragile]
		\frametitle{Замена цикла рекурсией}
		\framesubtitle{Императивное разложение на множители}
		\begin{exampleblock}{F\#}
			\begin{lstlisting}
let factorizeImperative n =
    let mutable primefactor1 = 1
    let mutable primefactor2 = n
    let mutable i = 2
    let mutable fin = false
    while (i < n && not fin) do
        if (n % i = 0) then
            primefactor1 <- i
            primefactor2 <- n / i
            fin <- true
        i <- i + 1
    if (primefactor1 = 1) then None
    else Some (primefactor1, primefactor2)
\end{lstlisting}
\end{exampleblock}

\end{frame}

	\begin{frame}[fragile]
		\frametitle{Замена цикла рекурсией}
		\framesubtitle{Рекурсивное разложение на множители}
		\begin{exampleblock}{F\#}
			\begin{lstlisting}
let factorizeRecursive n =
    let rec find i =
        if i >= n then None
        elif (n % i = 0) then Some(i, n / i)
        else find (i + 1)
    find 2
\end{lstlisting}
\end{exampleblock}
		
\end{frame}

\begin{frame}[fragile]
	\frametitle{Хвостовая рекурсия, проблема}
	\framesubtitle{Императивный вариант}
	\begin{exampleblock}{F\#}
		\begin{lstlisting}
open System.Collections.Generic

let createMutableList() =
    let l = new List<int>()
    for i = 0 to 100000 do
        l.Add(i)
    l
\end{lstlisting}
\end{exampleblock}
	
\end{frame}

\begin{frame}[fragile]
	\frametitle{Хвостовая рекурсия, проблема}
	\framesubtitle{Рекурсивный вариант, казалось бы}
	\begin{exampleblock}{F\#}
		\begin{lstlisting}
let createImmutableList() =
    let rec createList i max =
        if i = max then
            []	
        else
            i :: createList (i + 1) max
    createList 0 100000
\end{lstlisting}
\end{exampleblock}
	
\end{frame}

\begin{frame}[fragile]
	\frametitle{Факториал без хвостовой рекурсии}
	\begin{exampleblock}{F\#}
		\begin{lstlisting}
let rec factorial x =
    if x <= 1
    then 1 
    else x * factorial (x - 1)
\end{lstlisting}
\end{exampleblock}
	
	\begin{exampleblock}{F\#}
		\begin{lstlisting}
let rec factorial x =
    if x <= 1
    then
        1
    else
        let resultOfRecusion = factorial (x - 1)
        let result = x * resultOfRecusion
        result
\end{lstlisting}
\end{exampleblock}
	
\end{frame}

\begin{frame}[fragile]
	\frametitle{Факториал с хвостовой рекурсией}
	\begin{exampleblock}{F\#}
		\begin{lstlisting}
let factorial x =
    let rec tailRecursiveFactorial x acc =
        if x <= 1 then
            acc
        else
            tailRecursiveFactorial (x - 1) (acc * x)
    tailRecursiveFactorial x 1
\end{lstlisting}
\end{exampleblock}

\end{frame}
	
\begin{frame}[fragile]
	\frametitle{После декомпиляции в C\#}
		\begin{alertblock}{C\#}
			\begin{lstlisting}[language=Java]
public static int tailRecursiveFactorial(int x, int acc)
{
    while (true)
    {
        if (x <= 1)
        {
            return acc;
        }
        acc *= x;
        x--;
    }
}
\end{lstlisting}
\end{alertblock}
	
\end{frame}

\begin{frame}[fragile]
	\frametitle{Паттерн ``Аккумулятор''}
	\begin{exampleblock}{F\#}
		\begin{lstlisting}
let rec map f list =
    match list with
    | [] -> []
    | hd :: tl -> (f hd) :: (map f tl)

let map f list =
    let rec mapTR f list acc =
        match list with
        | [] -> acc
        | hd :: tl -> mapTR f tl (f hd :: acc)
    mapTR f (List.rev list) []
\end{lstlisting}
\end{exampleblock}
	
\end{frame}

\begin{frame}[fragile]
	\frametitle{Continuation Passing Style}
	\frametitle{Аккумулятор --- функция}
	\begin{exampleblock}{F\#}
		\begin{lstlisting}
let printListRev list =
    let rec printListRevTR list cont =
        match list with
        | [] -> cont ()
        | hd :: tl ->
            printListRevTR tl (fun () -> 
                printf "%d " hd; cont () )
    printListRevTR list (fun () -> printfn "Done!")
\end{lstlisting}
\end{exampleblock}
	
\end{frame}

\begin{frame}[fragile]
	\frametitle{Обход дерева}
	\frametitle{Когда всё не так просто}
	\begin{exampleblock}{F\#}
		\begin{lstlisting}
type ContinuationStep<'a> =
    | Finished
    | Step of 'a * (unit -> ContinuationStep<'a>)
    
let rec linearize binTree cont =
    match binTree with
    | Empty -> cont()
    | Node(x, l, r) ->
        Step(x, (fun () -> linearize l (fun () -> 
                           linearize r cont)))
\end{lstlisting}
\end{exampleblock}
	
\end{frame}

\begin{frame}[fragile]
	\frametitle{Обход дерева}
	\frametitle{Собственно, обход}
	\begin{exampleblock}{F\#}
		\begin{lstlisting}
let iter f binTree =
    let steps = linearize binTree (fun () -> Finished)

    let rec processSteps step =
        match step with
        | Finished -> ()
        | Step(x, getNext) -> f x
            processSteps (getNext())
    
    processSteps steps
\end{lstlisting}
\end{exampleblock}
	
\end{frame}

\end{document}