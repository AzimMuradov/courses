\documentclass[xetex,mathserif,serif]{beamer}
\usepackage{polyglossia}
\setdefaultlanguage[babelshorthands=true]{russian}
\usepackage{minted}
\usepackage{tabu}
\usepackage{moresize}

\useoutertheme{infolines}

\usepackage{fontspec}
\setmainfont{FreeSans}
\newfontfamily{\russianfonttt}{FreeSans}

\definecolor{links}{HTML}{2A1B81}
\hypersetup{colorlinks,linkcolor=,urlcolor=links}

\setbeamertemplate{blocks}[rounded][shadow=false]

\setbeamercolor*{block title alerted}{fg=red!50!black,bg=red!20}
\setbeamercolor*{block body alerted}{fg=black,bg=red!10}

\tabulinesep=1.2mm

\title{Веб-программирование}
\subtitle{Часть 2}
\author[Юрий Литвинов]{Юрий Литвинов\\\small{\textcolor{gray}{yurii.litvinov@gmail.com}}}
\date{01.12.2020г}

\newcommand{\attribution}[1] {
\vspace{-5mm}\begin{flushright}\begin{scriptsize}\textcolor{gray}{\textcopyright\, #1}\end{scriptsize}\end{flushright}
}

\begin{document}

    \frame{\titlepage}

    \section{Введение}

    \begin{frame}
        \frametitle{Попробуем написать что-нибудь ``настоящее''}
        \begin{itemize}
            \item Приложение для регистрации на конференцию
            \item Титульная страница конференции со ссылкой на форму регистрации
            \item Форма регистрации
            \begin{itemize}
                \item Как слушатель или как докладчик
            \end{itemize}
            \item Страница, на которой можно просмотреть всех зарегистрировавшихся
        \end{itemize}
    \end{frame}

    \section{Заглушка проекта}

    \begin{frame}[fragile]
        \frametitle{Контроллер}
        \begin{minted}{csharp}
using Microsoft.AspNetCore.Mvc;

namespace ConferenceRegistration.Controllers
{
    public class HomeController : Controller
    {
        public IActionResult Index()
            => View("MyView");
    }
}
        \end{minted}
    \end{frame}

    \begin{frame}[fragile]
        \frametitle{Вид}
        \begin{minted}{html}
@{
    Layout = null;
}

<!DOCTYPE html>

<html>
<head>
    <meta name="viewport" content="width=device-width" />
    <title>MyView</title>
</head>
<body>
Hello, world!
</body>
</html>
        \end{minted}
    \end{frame}

    \section{Регистрация}

    \begin{frame}[fragile]
        \frametitle{Моделирование предметной области}
        \begin{minted}{csharp}
namespace ConferenceRegistration.Models
{
    public class Participant
    {
        public string Name { get; set; }
        public string Email { get; set; }
        public bool? Speaker { get; set; }
    }
}
        \end{minted}
    \end{frame}

    \begin{frame}[fragile]
        \frametitle{Страница регистрации}
        \begin{ssmall}
            \begin{minted}{html}
@model ConferenceRegistration.Models.Participant
@addTagHelper *, Microsoft.AspNetCore.Mvc.TagHelpers

<html>
<head>
    <meta name="viewport" content="width=device-width" />
    <title>Register</title>
</head>
<body>
    <form asp-action="Register" method="post">
        <p>
            <label asp-for="Name">Your name:</label>
            <input asp-for="Name" />
        </p>
        <p>
            <label asp-for="Email">Your email:</label>
            <input asp-for="Email" />
        </p>
        <p>
            <label>Are you a speaker?</label>
            <select asp-for="Speaker">
                <option value="">Choose an option</option>
                <option value="true">Yes</option>
                <option value="false">No</option>
            </select>
        </p>
        <button type="submit">Register!</button>
    </form>
</body>
</html>
            \end{minted}
        \end{ssmall}
    \end{frame}

    \begin{frame}[fragile]
        \frametitle{Титульная страница}
        \begin{minted}{html}
@addTagHelper *, Microsoft.AspNetCore.Mvc.TagHelpers

<html>
<head>
    <meta name="viewport" content="width=device-width" />
    <title>SEIM-2021 registration</title>
</head>
<body>
    <div>
        <p>SEIM-2021 conference will be held in April in St. Petersburg.</p>
        <a asp-action="Register">Register now!</a>
    </div>
</body>
</html>
        \end{minted}
    \end{frame}

    \begin{frame}[fragile]
        \frametitle{Подправим контроллер}
        \begin{minted}{csharp}
public class HomeController : Controller
{
    public IActionResult Index()
        => View("MyView");

    public IActionResult Register()
        => View();
}
        \end{minted}
    \end{frame}

    \begin{frame}[fragile]
        \frametitle{Подправим контроллер ещё раз}
        \begin{footnotesize}
            \begin{minted}{csharp}
public class HomeController : Controller
{
    public IActionResult Index()
        => View("MyView");

    [HttpGet]
    public IActionResult Register() 
        => View();

    [HttpPost]
    public IActionResult Register(Participant participant)
    {
        // TODO: Do something with registration info
        return View();
    }
}
            \end{minted}
        \end{footnotesize}
    \end{frame}

    \section{Репозиторий}

    \begin{frame}[fragile]
        \frametitle{Репозиторий}
        \begin{minted}{csharp}
namespace ConferenceRegistration.Models
{
    public static class Repository
    {
        private static readonly IList<Participant> participants 
            = new List<Participant>();

        public static IEnumerable<Participant> Participants 
            => participants;

        public static void AddParticipant(Participant participant) 
            => participants.Add(participant);
    }
}
        \end{minted}
    \end{frame}

    \begin{frame}[fragile]
        \frametitle{Контроллер с репозиторием}
        \begin{minted}{csharp}
public class HomeController : Controller
{
    public IActionResult Index()
        => View("MyView");

    [HttpGet]
    public IActionResult Register()
        => View();

    [HttpPost]
    public IActionResult Register(Participant participant)
    {
        Repository.AddParticipant(participant);
        return View();
    }
}
        \end{minted}
    \end{frame}

    \section{Подтверждение регистрации}

    \begin{frame}[fragile]
        \frametitle{Страница подтверждения регистрации}
        \begin{scriptsize}
            \begin{minted}{html}
@model ConferenceRegistration.Models.Participant

<html>
<head>
    <meta name="viewport" content="width=device-width" />
    <title>Thanks</title>
</head>
<body>
<p>
    <h1>Thank you, @Model.Name</h1>
</p>
<p>
    @if (Model.Speaker == true)
    {
        @:Please don't forget to submit your article!
    }
</p>
</body>
</html>
            \end{minted}
        \end{scriptsize}
    \end{frame}

    \section{Список участников}

    \begin{frame}[fragile]
        \frametitle{Страница со списком участников}
        \begin{ssmall}
            \begin{minted}{html}
@model IEnumerable<ConferenceRegistration.Models.Participant>

<html>
<head>
    <meta name="viewport" content="width=device-width" />
    <title>ListParticipants</title>
</head>
<body>
<h2>List of conference participants:</h2>
<table>
    <thead>
    <tr>
        <th>Name</th>
        <th>Email</th>
        <th>Is speaker</th>
    </tr>
    </thead>
    <tbody>
    @foreach (ConferenceRegistration.Models.Participant p in Model) {
        <tr>
            <td>@p.Name</td>
            <td>@p.Email</td>
            <td>@(p.Speaker == true ? "Yes" : "No")</td>
        </tr>
    }
    </tbody>
</table>
</body>
</html>
            \end{minted}
        \end{ssmall}
    \end{frame}

    \begin{frame}[fragile]
        \frametitle{Обновим контроллер}
        \begin{footnotesize}
            \begin{minted}{csharp}
public class HomeController : Controller
{
    public IActionResult Index()
        => View("MyView");

    [HttpGet]
    public IActionResult Register()
        => View();

    [HttpPost]
    public IActionResult Register(Participant participant)
    {
        Repository.AddParticipant(participant);
        return View();
    }

    public IActionResult ListParticipants()
        => View(Repository.Participants);
}
            \end{minted}
        \end{footnotesize}
    \end{frame}

    \section{Валидация}

    \begin{frame}
        \frametitle{Валидация}
        \begin{itemize}
            \item Клиентская
            \begin{itemize}
                \item Работает с помощью jquery-validation
                \item Только в браузере у клиента, без нужды связи с сервером
                \item Только если там включён JavaScript
            \end{itemize}
            \item Серверная
            \begin{itemize}
                \item Проверка модели при Model Binding-е в контроллере
            \end{itemize}
        \end{itemize}
    \end{frame}

    \begin{frame}[fragile]
        \frametitle{Добавим server-side-валидацию}
        \begin{minted}{csharp}
public class Participant
{
    [Required(ErrorMessage = "Please enter your name")]
    public string Name { get; set; }

    [Required(ErrorMessage = "Please enter your email")]
    [RegularExpression(".+\\@.+\\..+", 
        ErrorMessage = "Please enter a valid email address")]
    public string Email { get; set; }

    [Required(ErrorMessage = 
        "Please specify whether you'll be a speaker or just attending")]
    public bool? Speaker { get; set; }
}
        \end{minted}
    \end{frame}

    \begin{frame}
        \frametitle{Что ещё бывает}
        \begin{itemize}
            \item Required --- для nullable-типов требует значение, для не-nullable не имеет смысла, они обязательны всегда
            \item StringLength --- задаёт максимальную и минимальную длину строки
            \item RegularExpression
            \item Range --- ограничивает значение заданным атрибутом
        \end{itemize}
    \end{frame}

    \begin{frame}[fragile]
        \frametitle{И снова обновим контроллер}
        \begin{minted}{csharp}
public class HomeController : Controller
{
    ...
    [HttpPost]
    public IActionResult Register(Participant participant)
    {
        if (ModelState.IsValid)
        {
            Repository.AddParticipant(participant);
            return View("Thanks", participant);
        }
    
        return View();
    }
    ...
}
        \end{minted}
    \end{frame}

    \begin{frame}[fragile]
        \frametitle{А теперь и вид}
        \begin{minted}{html}
<form asp-action="Register" method="post">
    <div asp-validation-summary="All"></div>
    ...
</form>
        \end{minted}
    \end{frame}

    \section{Стили}

    \begin{frame}[fragile]
        \frametitle{Добавим немного CSS}
        \begin{scriptsize}
            \begin{minted}{css}
.field-validation-error {
    color: #f00;
}

.field-validation-valid {
    display: none;
}

.input-validation-error {
    border: 1px solid #f00;
    background-color: #fee;
}

.validation-summary-errors {
    font-weight: bold;
    color: #f00;
}

.validation-summary-valid {
    display: none;
}
            \end{minted}
        \end{scriptsize}
    \end{frame}

    \begin{frame}[fragile]
        \frametitle{И вставим .css-ку в форму}
        \begin{small}
            \begin{minted}{html}
@addTagHelper *, Microsoft.AspNetCore.Mvc.TagHelpers

<html>
<head>
    <meta name="viewport" content="width=device-width" />
    <title>Index</title>
    <link rel="stylesheet" href="/lib/bootstrap/dist/css/bootstrap.css" />
</head>
<body>
    <div class="text-center">
        <h3>SEIM-2021 conference will be held in April in St. Petersburg.</h3>
        <a class="btn btn-primary" asp-action="Register">Register now!</a>
    </div>
</body>
</html>
            \end{minted}
        \end{small}
    \end{frame}

    \begin{frame}
        \frametitle{То же с...}
        \begin{itemize}
            \item формой регистрации
            \item страницей со списком участников
            \item страницей подтверждения регистрации
            \begin{itemize}
                \item см. конспект
            \end{itemize}
        \end{itemize}
    \end{frame}

    \section{Персистентность}

    \begin{frame}[fragile]
        \frametitle{Добавим персистентность}
        \begin{itemize}
            \item Добавим Microsoft.EntityFrameworkCore, Microsoft.EntityFrameworkCore.SqlServer
        \end{itemize}
        \begin{small}
            \begin{minted}{csharp}
public class Repository : DbContext
{
    public DbSet<Participant> Participants { get; set; }

    protected override void OnConfiguring(
        DbContextOptionsBuilder optionsBuilder)
    {
        optionsBuilder.UseSqlServer(
            @"Server=(localdb)\mssqllocaldb;"
            + @"Database=ConferenceRegistration;Trusted_Connection=True;");
    }
}
            \end{minted}
        \end{small}
    \end{frame}

    \begin{frame}[fragile]
        \frametitle{И модифицируем контроллер}
        \begin{scriptsize}
            \begin{minted}{csharp}
public class HomeController : Controller
{
    ...
    [HttpPost]
    public IActionResult Register(Participant participant)
    {
        if (ModelState.IsValid)
        {
            using var repository = new Repository();
            repository.Participants.Add(participant);
            repository.SaveChanges();
            return View("Thanks", participant);
        }

        return View();
    }

    public IActionResult ListParticipants()
    {
        using var repository = new Repository();
        return View(repository.Participants.ToList());
    }
}  
            \end{minted}
        \end{scriptsize}
    \end{frame}

    \begin{frame}[fragile]
        \frametitle{И модель}
        \framesubtitle{Добавим первичный ключ}
        \begin{small}
            \begin{minted}{csharp}
public class Participant
{
    [Required(ErrorMessage = "Please enter your name")]
    public string Name { get; set; }

    [Required(ErrorMessage = "Please enter your email")]
    [RegularExpression(".+\\@.+\\..+", 
        ErrorMessage = "Please enter a valid email address")]
    [Key]
    public string Email { get; set; }

    [Required(ErrorMessage = 
        "Please specify whether you'll be a speaker or just attending")]
    public bool? Speaker { get; set; }
}
            \end{minted}
        \end{small}
    \end{frame}

    \begin{frame}
        \frametitle{А теперь магически поднимем базу данных}
        \framesubtitle{EF Migrations}
        \begin{itemize}
            \item Поставим Microsoft.EntityFrameworkCore.Tools
            \item Откроем NuGet Manager Console
            \item Напишем Add-Migration InitialCreate
            \item Напишем Update-Database
        \end{itemize}
    \end{frame}

    \begin{frame}
        \frametitle{Проверим, что всё работает}
        \begin{itemize}
            \item Запустим приложение, зарегистрируем пару участников
            \item Откроем Server Explorer
            \item Connect to Database -> Microsoft SQL Server
            \item Имя сервера -> \mintinline{text}{(localdb)\mssqllocaldb}
            \item Имя файла -> ConferenceRegistration
            \begin{itemize}
                \item Как в Connection String
            \end{itemize}
            \item Проверяем, что база имеет таблицу Participants
            \item Просматриваем данные
            \item Перезапускаем сервер, перезапускаем приложение, смотрим список участников
        \end{itemize}
    \end{frame}

\end{document}
