\documentclass[a5paper]{article}
\usepackage[a5paper, top=8mm, bottom=8mm, left=8mm, right=8mm]{geometry}

\usepackage{polyglossia}
\setdefaultlanguage[babelshorthands=true]{russian}

\usepackage{fontspec}
\setmainfont{FreeSerif}
\newfontfamily{\russianfonttt}[Scale=0.7]{DejaVuSansMono}

\usepackage[font=scriptsize]{caption}

\usepackage{amsmath}
\usepackage{amssymb,amsfonts,textcomp}
\usepackage{color}
\usepackage{array}
\usepackage{hhline}
\usepackage{cite}

\usepackage[hang,multiple]{footmisc}
\renewcommand{\footnotelayout}{\raggedright}

\PassOptionsToPackage{hyphens}{url}\usepackage[xetex,linktocpage=true,plainpages=false,pdfpagelabels=false]{hyperref}
\hypersetup{colorlinks=true, linkcolor=blue, citecolor=blue, filecolor=blue, urlcolor=blue, pdftitle=1, pdfauthor=, pdfsubject=, pdfkeywords=}

\usepackage{tabu}

\usepackage{graphicx}
\usepackage{indentfirst}
\usepackage{multirow}
\usepackage{subfig}
\usepackage{footnote}
\usepackage{minted}

\sloppy
\pagestyle{plain}

\title{Code Review}
\author{Юрий Литвинов\\\small{yurii.litvinov@gmail.com}}

\date{10.04.2018г}

\begin{document}

\maketitle
\thispagestyle{empty}

Code review (инспекция кода, ревизия кода) --- это систематическая проверка исходного кода с целью обнаружения и исправления ошибок, направленная как на повышение качества ПО, так и навыков разработчика. В процессе ревью находятся не только и не столько ошибки, связанные с функционированием программы, сколько ошибки архитектуры, оформления, уязвимости к хакерству, утечки памяти, просто некрасивый код. В процессе ревью может быть выявлено более половины труднообнаружимых ошибок. Как проводится: бывают формальные процессы ревью кода (например, в некоторых крупных компаниях и во многих проектах с открытым исходным кодом код не может быть принят в основную ветку без ревью), либо же неформальные, когда код для ревью рассылается всем участникам, после чего они собираются вместе и разбирают замеченные ошибки. Некоторые системы управления проектами, такие как Github, позволяют писать комментарии к коммитам и пуллреквестам, ``защищать'' ветки, требуя хотя бы одного положительного отзыва перед тем, как пуллреквест в принципе можно будет замерджить.

Правила проведения кодревью такие:
\begin{itemize}
    \item Код безличен, то есть ревьюится код, а не его автор. Автор может быть (и, как правило, обязательно будет) в курсе обсуждения, поэтому высказывать сомнения в его профессионализме или умственных способностях запрещается. Помните, что вы можете оказаться на его месте, и не факт, что вы пишете лучше.
    \item В общепринятых сейчас методологиях разработки пропагандируется коллективное владение кодом, при котором всем кодом, лежащим в основной ветке, владеет вся команда, несёт за него равную ответственность и имеет равные права на его исправление. Это несколько помогает процессу ревью и последующего рефакторинга.
    \item Кодревью --- это не деструктивная, а конструктивная деятельность. Мы не указываем на недостатки/ошибки и т.д., а обязательно предлагаем пути по исправлению этих ошибок.
\end{itemize}

Код для ревью, пример 1:

\begin{minted}{csharp}
using System;
using System.Collections.Generic;
using System.ComponentModel;
using System.Data;
using System.Drawing;
using System.Linq;
using System.Text;
using System.Threading.Tasks;
using System.Windows.Forms;

namespace Calculator
{
    public partial class Form1 : Form
    {
        public Form1()
        {
            InitializeComponent();
        }

        /// <summary>
        /// Procedure of adding symbol in textbox
        /// </summary>
        /// <param name="value">Value to add</param>
        public void AddSymbol(int value)
        {
            if (AlreadyPrinted)
            {
                textBox1.ResetText();
                AlreadyPrinted = false;
            }
            textBox1.Text = textBox1.Text + value;
            label2.Text = "";
        }

        /// <summary>
        /// Symbol of the operation
        /// </summary>
        private char OperationSymbol = 'x';

        /// <summary>
        /// If symbol is printed
        /// </summary>
        private bool AlreadyPrinted = false;

        /// <summary>
        /// Previous value
        /// </summary>
        private int Previous = 0;

        //Buttons with numbers clicked methods
        private void OnButton1Click(object sender, EventArgs e)
        {
            this.AddSymbol(1);
        }

        private void OnButton2Click(object sender, EventArgs e)
        {
            this.AddSymbol(2);
        }

        private void OnButton3Click(object sender, EventArgs e)
        {
            this.AddSymbol(3);
        }

        private void OnButton4Click(object sender, EventArgs e)
        {
            this.AddSymbol(4);
        }

        private void OnButton5Click(object sender, EventArgs e)
        {
            this.AddSymbol(5);
        }

        private void OnButton6Click(object sender, EventArgs e)
        {
            this.AddSymbol(6);
        }

        private void OnButton7Click(object sender, EventArgs e)
        {
            this.AddSymbol(7);
        }

        private void OnButton8Click(object sender, EventArgs e)
        {
            this.AddSymbol(8);
        }

        private void OnButton9Click(object sender, EventArgs e)
        {
            this.AddSymbol(9);
        }

        private void OnButton0Click(object sender, EventArgs e)
        {
            this.AddSymbol(0);
        }
        /// <summary>
        /// Save previous value
        /// </summary>
        public void SavePrev()
        {
            if (textBox1.Text == "")
                return;
            Previous = int.Parse(textBox1.Text);
            AlreadyPrinted = true;
            label2.Text = "";
        }
        
        //Buttons with symbols clicked methods
        private void OnButtonMinusClick(object sender, EventArgs e)
        {
            SavePrev();
            OperationSymbol = '-';
        }

        private void OnButtonPlusClick(object sender, EventArgs e)
        {
            SavePrev();
            OperationSymbol = '+';
        }

        private void OnButtonMultiplyClick(object sender, EventArgs e)
        {
            SavePrev();
            OperationSymbol = '*';
        }

        private void OnButtonDivideClick(object sender, EventArgs e)
        {
            SavePrev();
            OperationSymbol = '/';
        }

        private void buttonEquals_Click(object sender, EventArgs e)
        {
            int result = 0;
            if ((OperationSymbol == 'x') || (textBox1.Text == ""))
                return;
            if (OperationSymbol == '+')
            {
                result = Previous + int.Parse(textBox1.Text);
                textBox1.ResetText();
                textBox1.Text = textBox1.Text + result;
            }
            if (OperationSymbol == '-')
            {
                result = Previous - int.Parse(textBox1.Text);
                textBox1.ResetText();
                textBox1.Text = textBox1.Text + result;
            }
            if (OperationSymbol == '*')
            {
                result = Previous * int.Parse(textBox1.Text);
                textBox1.ResetText();
                textBox1.Text = textBox1.Text + result;
            }
            if (OperationSymbol == '/')
            {
                if (int.Parse(textBox1.Text) == 0)
                {
                    label2.Text = "ERROR";
                    textBox1.ResetText();
                    Previous = 0;
                    OperationSymbol = 'x';
                    AlreadyPrinted = false;
                    return;
                }
                result = Previous / int.Parse(textBox1.Text);
                textBox1.ResetText();
                textBox1.Text = textBox1.Text + result;
            }
            Previous = int.Parse(textBox1.Text);
            OperationSymbol = 'x';
            AlreadyPrinted = false;
        }
        /// <summary>
        /// Button C click method
        /// </summary>
        /// <param name="sender"></param>
        /// <param name="e"></param>
        private void OnButtonCClick(object sender, EventArgs e)
        {
            textBox1.ResetText();
            Previous = 0;
        }
    }
}
\end{minted}

Код для ревью, пример 2:

\begin{minted}{csharp}
using System;
using System.Collections.Generic;
using System.Linq;
using System.Text;
using System.Threading.Tasks;

namespace Queue_
{
    class Program
    {
        static void Main(string[] args)
        {
            Queue queue = new Queue();

            queue.Enqueue(2, 1);
            queue.Enqueue(8, 1);
            queue.Enqueue(5, 3);
            int tempVal = queue.Dequeue();
            tempVal = queue.Dequeue();
            tempVal = queue.Dequeue();
        }
    }

    public class Queue
    {
        public Qelement qBeg = null;
        public Qelement qEnd = null;
        public class Qelement
        {
            private int val;
            public int Value
            {
                get
                {
                    return val;
                }
                set
                {
                    val = value;
                }
            }
            public int priority;

            public Qelement Next { get; set; }
        }

        /// <summary>
        /// Проверяет очередь на пустоту.
        /// </summary>
        /// <returns></returns>
        public bool IsEmpty()
        {
            return qBeg == null;
        }

        /// <summary>
        /// Находит позицию для добавления нового элемента в очередь с приоритетами.
        /// </summary>
        /// <param name="priotity"></param>
        /// <returns></returns>
        private Qelement SearchPos(int priotity)
        {
            Qelement pos = qBeg;
            while (pos != null && priotity < pos.priority - 1)
            {
                pos = pos.Next;
            }
            return pos;
        }

        /// <summary>
        /// Добавляет новый элемент в очередь с приоритетами.
        /// </summary>
        /// <param name="val"></param>
        /// <param name="priotity"></param>
        public void Enqueue(int val, int priotity)
        {
            Qelement pos = SearchPos(priotity);
            if (pos != null)
            {
                Qelement newEl = new Qelement()
                {
                    Next = pos.Next,
                    Value = val
                };
                pos.Next = newEl;
                if (pos == qBeg && priotity > qBeg.priority)
                {
                    int temp = pos.Value;
                    pos.Value = newEl.Value;
                    newEl.Value = temp;
                }
            }
            else
            {
                Qelement newEl = new Qelement()
                {
                    Next = null,
                    Value = val
                };
                if (!IsEmpty())
                {
                    qEnd.Next = newEl;
                    qEnd = newEl;
                }
                else
                {
                    qBeg = newEl;
                    qEnd = newEl;
                }
                newEl.priority = priotity;
            }
            
        }

        /// <summary>
        /// Возвращает значение с наивысшим приоритетом и удаляет его из очереди.
        /// </summary>
        /// <returns></returns>
        public int Dequeue()
        {
            if (!IsEmpty())
            {
                int temp = qBeg.Value;
                qBeg = qBeg.Next;
                return temp;
            }
            throw new EmptyQueueExeption();
        }

    }

    /// <summary>
    /// Исключение возникает при попытке удалить элемент из пустой очереди.
    /// </summary>
    public class EmptyQueueExeption : ApplicationException
    {
        public EmptyQueueExeption()
        {
        }
        public EmptyQueueExeption(string message)
            : base(message)
        {
        }
    }

}
\end{minted}

Код для ревью, пример 3:

\begin{minted}{c++}
template <typename T>
class BST
{
public:
    struct Node
    {
        T item;
        Node* left;
        Node* right;
        Node(T value) : left(nullptr), right(nullptr) {item = value;} 
        Node() : left(nullptr), right(nullptr) {}
        ~Node()
        {
            if (!left) 
                delete left;
            if (!right)
                delete right;
        }
    };

    typedef Node* Position;

    BST()
    {
        root = new Node();
        counter = 0;
    }

    ~BST()
    {
        //call destructor of Node structure
        delete root;
    }

    //add element to the tree
    void insert(T value)
    {
        inject(root, value);
        ++counter;
    }

    //get pointer to element of tree with needed key
    Position find(T value)
    {
        Position search = root;
        while (search != nullptr && search->item != value)
        {
            if (search->item <= value)
                search = search->right;
            else 
                search = search->left;
        }
        return search;
    }

    //delete item from the tree
    void remove(T toRemove)
    {
        if (counter != 0)
        {
            detach(toRemove, root);
        }
    }

    void traverseUp(void (*action)(T))
    {
        auxTraverseUp(root, action);
    }

    void traverseDown(void (*action)(T))
    {
        auxTraverseDown(root, action);
    }

private:
    Node* root;
    size_t counter;

    //Recursively add value to tree
    void inject(Node* toAdd, T value)
    {
        if (counter == 0)
        {
            //empty-tree case
            root = new Node(value);
            return;
        }
        if (toAdd->item <= value)
            if (toAdd->right != nullptr)
                inject(toAdd->right, value);
            else 
                toAdd->right = new Node(value);
        else 
            if (toAdd->left != nullptr)
                inject(toAdd->left, value);
            else 
                toAdd->left = new Node(value);
    }

    //Recursively delete value from tree
    void detach(T value, Node* tree)
    {
        if (root->item == value)
        {
            removeNode(root);
            --counter;
            return;
        }
        if (tree->left != nullptr && tree->left->item == value)
        {
            tree->left = removeNode(tree->left);
            --counter;
            return;
        }
        if (tree->right != nullptr && tree->right->item == value)
        {
            tree->right = removeNode(tree->right);
            --counter;
            return;
        }
        if (tree->item > value)
        {
            if (tree->left != nullptr)
                detach(value, tree->left);
        }
        else
        {
            if (tree->right != nullptr)
                detach(value, tree->right);
        }
        return;
    }

    //exclude Node from tree
    Node* removeNode(Node* toDelete)
    {
        short hasRight = 5 * static_cast<short>(toDelete->right != nullptr);
        short hasLeft = 3 * static_cast<short>(toDelete->left != nullptr);
        Position temp;
        /*depends on values of hasRight and hasLeft, 
        their sum can be either 0 or 3 or 5 or 8, as far as 
        static_cast<short>(toDelete->right != nullptr) can return either 0 or 1*/
        switch (hasLeft + hasRight)
        {
        case 0:
        //has no childs
            delete toDelete;
            return nullptr;
            break;
        case 3:
        //has left child
            temp = toDelete->left;
            delete toDelete;
            return temp;
            break;
        case 5:
        //has right child
            temp = toDelete->right;
            delete toDelete;
            return temp;
            break;
        case 8:
        //has both childs
            temp = toDelete->right;
            while (temp->left != nullptr)
                temp = temp->left;
            toDelete->item = temp->item;
            toDelete->right = removeNode(temp);
            return toDelete;
            break;
        }
        return 0;
    }

    //traverse to max element with some function
    void auxTraverseUp(Node* node, void (*action)(T))
    {
        if (node->left != nullptr)
            auxTraverseUp(node->left, action);
        action(node->item);
        if (node->right != nullptr)
            auxTraverseUp(node->right, action);
        return;
    }

    //traverse to max element with some function
    void auxTraverseDown(Node* node, void (*action)(T))
    {
        if (node->right != nullptr)
            auxTraverseDown(node->right, action);
        action(node->item);
        if (node->left != nullptr)
            auxTraverseDown(node->left, action);
        return;
    }
};
\end{minted}

Код для ревью, пример 4. Файл tree.cs:

\begin{minted}{csharp}
using System;
using System.Collections.Generic;
using System.Linq;
using System.Text;
using System.IO;

namespace ParseTree
{
    public class Tree
    {
        private string[] tokens;
        private Dictionary<string, Func<double, double, double>> operation;
        private int counter;

        /// <summary>
        /// Initializes a new instance of the class
        /// </summary>
        /// <param name="path"></param>
        public Tree(string path)
        {
            string expression = "";
            using (StreamReader f = new StreamReader(path))
            {
                expression = f.ReadLine();
            }
            this.operation = new Dictionary<string, Func<double, double, double>>
            {
                { "+", (x, y) => x + y },
                { "-", (x, y) => x - y },
                { "*", (x, y) => x * y },
                { "/", (x, y) => x / y }
            };
            this.tokens = expression.Split(new char[] { ' ' });
            this.counter = 0;
        }

        public double Calculate(Node tree)
        {
            return tree.Calculate();
        }

        public void PrintTree(Node tree)
        {
            tree.Print();
        }

        /// <summary>
        /// Build tree
        /// </summary>
        public void Build(ref Node current)
        {
            string token = tokens[counter++];
            if (token == ")")
            {
                return;
            }

            if (operation.ContainsKey(token))
            {
                if (current != null)
                {
                    var currentOperation = current as NodeOperation;
                    Node newCurrent = currentOperation.AddOperand(
                        new NodeOperation(token, operation[token]));
                    Build(ref newCurrent);
                }
                else
                {
                    current = new NodeOperation(token, operation[token]);
                }
            }
            else
            {
                double value;
                if (double.TryParse(token, out value))
                {
                    if (current != null)
                    {
                        (current as NodeOperation).AddOperand(new NodeOperand(value));
                    }
                    else
                    {
                        current = new NodeOperand(value);
                    }
                }
            }
            Build(ref current);
        }
    }
}
\end{minted}

Файл node.cs:

\begin{minted}{csharp}
using System;
using System.Collections.Generic;
using System.Linq;
using System.Text;

namespace ParseTree
{
    public interface Node
    {
        double Calculate();

        void Print();
    }
}
\end{minted}

Файл nodeOperand.cs:

\begin{minted}{csharp}
using System;
using System.Collections.Generic;
using System.Linq;
using System.Text;

namespace ParseTree
{
    public class NodeOperand : Node
    {
        private double value;

        /// <summary>
        /// Initializes a new instance of the class.
        /// </summary>
        /// <param name="value"></param>
        public NodeOperand(double value)
        {
            this.value = value;
        }

        public void Print()
        {
            Console.Write(string.Format(" {0} ", this.value));
        }

        public double Calculate()
        {
            return value;
        }
    }
}
\end{minted}

Файл nodeOperation.cs:

\begin{minted}{csharp}
using System;
using System.Collections.Generic;
using System.Linq;
using System.Text;

namespace ParseTree
{
    public class NodeOperation : Node
    {
        private Node left;
        private Node right;
        private string signature;
        private Func<double, double, double> perform;

        /// <summary>
        /// Initializes a new instance of the class.
        /// </summary>
        public NodeOperation(string sign, Func<double, double, double> performOperation)
        {
            this.perform = performOperation;
            this.signature = sign;
        }

        public double Calculate()
        {
            return perform(left.Calculate(), right.Calculate());
        }

        public void Print()
        {
            Console.Write("({0} ", signature);
            left.Print();
            right.Print();
            Console.Write(")");
        }

        /// <summary>
        /// Add an operand
        /// </summary>
        public Node AddOperand(Node operand)
        {
            if (left == null)
            {
                return left = operand;
            }
            else
            {
                if (right == null)
                {
                    return right = operand;
                }
                else
                {
                    throw new ExtraNodeException();
                }
            }
        }
    }
}
\end{minted}

\end{document}
