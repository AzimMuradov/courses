\documentclass[xetex,mathserif,serif]{beamer}
\usepackage{polyglossia}
\setdefaultlanguage[babelshorthands=true]{russian}
\usepackage{minted}
\usepackage{tabu}

\useoutertheme{infolines}

\usepackage{fontspec}
\setmainfont{FreeSans}
\newfontfamily{\russianfonttt}{FreeSans}

\usepackage{forest}
\usetikzlibrary{arrows}

\definecolor{links}{HTML}{2A1B81}
\hypersetup{colorlinks,linkcolor=,urlcolor=links}

\tabulinesep=0.7mm

\title{Введение, разбор задач}
\author[Юрий Литвинов]{Юрий Литвинов \newline \textcolor{gray}{\small\texttt{y.litvinov@spbu.ru}}}

\date{07.09.2021}

\begin{document}

    \frame{\titlepage}

    \begin{frame}
        \frametitle{Формальные вопросы}
        \begin{itemize}
            \item Занятия по вторникам на второй и третьей паре в ауд. 3381
            \item Берите с собой ноуты
            \item Курс на Blackboard: \url{https://bit.ly/3yLWQnO}
            \item Условия домашек и материалы с пар будут там
            \item Решения пока присылайте на почту, как всех добавят в Blackboard --- туда
            \item Команда в Teams, код \textbf{k10tmur}
            \item Мои контакты:
            \begin{itemize}
                \item Почта: y.litvinov@spbu.ru
                \item Telegram: yurii\_litvinov
                \item Пишите по любому вопросу!
            \end{itemize}
        \end{itemize}
    \end{frame}

    \begin{frame}
        \frametitle{Критерии оценивания}
        \begin{itemize}
            \item Шкала оценивания ECTS, оценки от A до F
            \item Надо набирать баллы:
            \begin{itemize}
                \item За домашки (их будет много!)
                \item За две контрольные
                \item За зачёт, который по сути большая контрольная
            \end{itemize}
            \item Итоговый балл за домашки: $MAX(0, (\frac{n}{N} – 0.6)) * 2.5 * 100$
            \begin{itemize}
                \item Если сделано меньше 60\% --- это 0, если 80\% --- 50 баллов
            \end{itemize}
            \item Есть дедлайны и штрафы за попытки сдачи начиная с третьей (-0.5 балла)
            \item Итоговый балл за контрольные: $\frac{n}{N} * 100$, их можно переписывать
            \item Балл за зачёт считается так же, но переписывать можно только трижды
            \item В качестве итогового берётся \textbf{минимум} из этих баллов
        \end{itemize}
    \end{frame}

    \begin{frame}
        \frametitle{Шкала оценивания ECTS}
        \begin{tabu} {| X[0.9 l p] | X[1 l p] | }
            \tabucline-
            Балл                     & Оценка ECTS  \\
            \tabucline-
            \everyrow{\tabucline-}
            90-100                   & A            \\
            80-89                    & B            \\
            70-79                    & C            \\
            61-69                    & D            \\
            50-60                    & E            \\
            0-50                     & на пересдачу \\
        \end{tabu}
    \end{frame}

    \begin{frame}
        \frametitle{Что будет в курсе}
        \begin{itemize}
            \item Ликвидация безграмотности по программированию на Си
            \item Сложность алгоритмов
            \item Отладка и тестирование
            \item Классические сортировки
            \item Системы контроля версий
            \item Внутреннее представление данных, работа с файлами и т.п.
            \item Работа с указателями, стеки, очереди, списки и т.п.
            \item Деревья вообще, деревья поиска, самобалансирующиеся деревья
            \item Хеш-таблицы, графы
            \item Парадигмы программирования
            \item Автоматы, лексический анализ
        \end{itemize}
    \end{frame}

    \begin{frame}
        \frametitle{Условия задач с теста}
        \begin{enumerate}
            \item Написать алгоритм нахождения неполного частного от деления $a$ на $b$ (целые числа), используя только операции сложения, вычитания и умножения.
            \item Подсчитать число <<счастливых билетов>> (билет считается <<счастливым>>, если сумма первых трёх цифр его номера равна сумме трёх последних).
            \item Написать алгоритм проверки баланса скобок в исходной строке (т.е. число открывающих скобок равно числу закрывающих и выполняется правило вложенности скобок).
            \item Какое наименьшее количество операции умножения достаточно для вычисления значения формулы $x^4 + x^3 + x^2 + x + 1$?
        \end{enumerate}
    \end{frame}

    \begin{frame}
        \frametitle{Ещё задачи}
        \begin{enumerate}
            \item Укажите условия, при которых формулы ``$a + a - a$'' и ``$a + (a - a)$'' не эквивалентны.
            \item Поменять значения двух целочисленных переменных местами (без привлечения третьей переменной и файлов).
            \item Напишите программу, считающую количество нулевых элементов в массиве.
            \item Напишите программу, печатающую все простые числа, не превосходящие заданного числа.
            \item Заданы две строки: $S$ и $S_1$. Найдите количество вхождений $S_1$ в $S$ как подстроки.
            \item Дан массив целых чисел $x[1]...x[m+n]$, рассматриваемый как соединение двух его отрезков: начала $x[1]...x[m]$ длины $m$ и конца $x[m+1]...x[m+n]$ длины $n$. Не используя дополнительных массивов, переставить местами начало и конец.
        \end{enumerate}
    \end{frame}

\end{document}

