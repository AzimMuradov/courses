\documentclass[xetex,mathserif,serif]{beamer}
\usepackage{polyglossia}
\setdefaultlanguage[babelshorthands=true]{russian}
\usepackage{minted}
\usepackage{tabu}
\usepackage{pgfplots}

\useoutertheme{infolines}

\usepackage{fontspec}
\setmainfont{FreeSans}
\newfontfamily{\russianfonttt}{FreeSans}

\usepackage{forest}
\usetikzlibrary{arrows}

\definecolor{links}{HTML}{2A1B81}
\hypersetup{colorlinks,linkcolor=,urlcolor=links}

\pgfplotsset{compat=1.16}
\tabulinesep=0.7mm

\title{Комментарии по домашке}
\author[Юрий Литвинов]{Юрий Литвинов \newline \textcolor{gray}{\small\texttt{yurii.litvinov@gmail.com}}}

\date{15.09.2020}

\begin{document}
    
    \frame{\titlepage}
    
    \begin{frame}
        \frametitle{Стайлгайд}
        \begin{itemize}
            \item Именование: inputarr -> inputArray
            \item Пробелы: \mintinline{c}|if(a == 1)| -> \mintinline{c}|if (a == 1)|
            \item Но \mintinline{c}|balance (str)| -> \mintinline{c}|balance(str)|
            \item Булевые переменные: \mintinline{c}|if (b == false)| -> \mintinline{c}|if (!b)|
            \item Сокращённые операторы: \mintinline{c}|a = a - b| -> \mintinline{c}|a -= b|
            \item Функция, не принимающая параметров: \mintinline{c}|f(void)|
            \item Скобки: \mintinline{c}|return(x);| -> \mintinline{c}|return x;|
            \item Тернарный оператор: \mintinline{c}|printf(x == 0 ? "true" :  "false");|
        \end{itemize}
    \end{frame}

    \begin{frame}[fragile]
        \frametitle{if-else и return}
        \begin{minted}{c}
void f(int x) {
    if (x == 0) {
        ...
    } else {
        ...
    }
}
        \end{minted}
        или
        \begin{minted}{c}
void f(int x) {
    if (x == 0) {
        ...
        return;
    } 
    ...
}
        \end{minted}
    \end{frame}

    \begin{frame}[fragile]
        \frametitle{Ещё про if и return}
        \begin{minted}{c}
if (x != 0) {
    return false;
}

return true;
        \end{minted}
        это просто
        \begin{minted}{c}
return x == 0;
        \end{minted}
    \end{frame}

    \begin{frame}[fragile]
        \frametitle{Забытое значение при возврате}
        \begin{minted}{c}
bool f(int x) {
    if (x == 0) {
        return true;
    }
}

void main() {
    printf("%d", f(1));  // ??
}
        \end{minted}
    \end{frame}

\end{document}

