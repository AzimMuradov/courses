\documentclass[a5paper]{article}
\usepackage[a5paper, top=8mm, bottom=8mm, left=8mm, right=8mm]{geometry}

\usepackage{polyglossia}
\setdefaultlanguage[babelshorthands=true]{russian}

\usepackage{fontspec}
\setmainfont{FreeSerif}
\newfontfamily{\russianfonttt}[Scale=0.7]{DejaVuSansMono}

\usepackage[font=scriptsize]{caption}

\usepackage{amsmath}
\usepackage{amssymb,amsfonts,textcomp}
\usepackage{color}
\usepackage{array}
\usepackage{hhline}
\usepackage{cite}

\usepackage[hang,multiple]{footmisc}
\renewcommand{\footnotelayout}{\raggedright}

\PassOptionsToPackage{hyphens}{url}\usepackage[xetex,linktocpage=true,plainpages=false,pdfpagelabels=false]{hyperref}
\hypersetup{colorlinks=true, linkcolor=blue, citecolor=blue, filecolor=blue, urlcolor=blue, pdftitle=1, pdfauthor=, pdfsubject=, pdfkeywords=}

\usepackage{tabu}

\usepackage{graphicx}
\usepackage{indentfirst}
\usepackage{multirow}
\usepackage{subfig}
\usepackage{footnote}
\usepackage{minted}

\sloppy
\pagestyle{plain}

\title{Лекция 3: Нетипизированное \lambda-исчисление}

\date{}

\begin{document}

\maketitle
\thispagestyle{empty}

\section{Лямбда-исчисление, введение}

Лямбда-исчисление --- теоретическая основа функционального программирования, поэтому явно заслуживает отдельного обсуждения, тем более что это само по себе математически красивая теория. Обсуждение, правда, будет короче, чем обычно в курсах по ФП, всего одну пару. Так что мы успеем только нетипизированное лямбда-исчисление, тогда как реальные языки программирования базируются на типизированных лямбда-исчислениях, конечно. Но, во-первых, их несколько, во-вторых, они сложнее, и в-третьих, про них есть хорошие книги и хорошие онлайн-курсы (на первой лекции рекомендовались курсы Дениса Николаевича Москвина \url{https://www.lektorium.tv/course/22797}, \url{https://stepik.org/course/75} и \url{https://stepik.org/course/693}). Тем не менее, общая идея и связь лямбда-исчисления с F\# должны быть понйтны

Итак, лямбда-исчисление --- это формальная система, основанная на \lambda-нотации, которая была разработана как ещё один способ формализовать понятие <<вычисление>> американским логиком Алонзо Чёрчем аж в 1930-х годах (то есть задолго до появления первых ЭВМ). Лямбда-исчисление в каком-то смысле более простая альтернатива машинам Тьюринга и эквивалентно им по вычислительной мощи (то есть всё, что можно посчитать в лямбда-исчислении, можно посчитать на машине Тьюринга и наоборот), известный вам тезис Чёрча\footnote{<<любая функция, которая может быть вычислена, может быть вычислена машиной Тьюринга>>} формулируется относительно машины Тьюринга, но назван в честь Чёрча (ну, на самом деле его называют тезисом Чёрча-Тьюринга иногда). Кстати, есть ещё нормальные алгорифмы Маркова, которые ещё одна довольно известная альтернатива и машинам Тьюринга и \lambda-исчислению Чёрча, про них часто забывают, так что напоминаю тут, но речь дальше будет не про них.

Лямбда-исчисление в языках программирования появилось довольно давно, аж язык LISP (а это 1958 год!) черпал вдохновение из лямбда-исчисления (хотя только его относительно новые диалекты можно в полной мере назвать функциональными языками).

\end{document}
