\documentclass[xetex,mathserif,serif]{beamer}
\usepackage{polyglossia}
\setdefaultlanguage[babelshorthands=true]{russian}
\usepackage{minted}
\usepackage{tabu}

\useoutertheme{infolines}

\usepackage{fontspec}
\setmainfont{FreeSans}
\newfontfamily{\russianfonttt}{FreeSans}

\definecolor{links}{HTML}{2A1B81}
\hypersetup{colorlinks,linkcolor=,urlcolor=links}

\tabulinesep=0.7mm

\newcommand{\attribution}[1] {
	\vspace{-5mm}\begin{flushright}\begin{scriptsize}\textcolor{gray}{\textcopyright\, #1}\end{scriptsize}\end{flushright}
}

\title{Практика по проектированию}
\subtitle{Магазин книг}
\author[Юрий Литвинов]{Юрий Литвинов \newline \textcolor{gray}{\small\texttt{yurii.litvinov@gmail.com}}}

\date{24.11.2020г}

\begin{document}
	
	\frame{\titlepage}

	\begin{frame}
		\frametitle{Задание на пару: Магазин книг}
		В командах по 2-3 человека выполнить анализ предметной области и построить модель в виде диаграммы классов для интернет-магазина книг по следующему ТЗ:
		\begin{itemize}
			\item \url{https://goo.gl/94LyFc}
		\end{itemize}

		Обратите внимание, что это должна быть модель предметной области в соответствии с принципами DDD, детали реализации наподобие способа хранения информации в базе данных не важны.

		Будет оцениваться точность следования ТЗ, соответствие модели сущностям предметной области и, естественно, пунктуальность в следовании синтаксису UML.
	\end{frame}

\end{document}
