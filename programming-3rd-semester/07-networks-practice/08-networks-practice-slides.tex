\documentclass{../../slides-style}

\slidetitle[Практика]{Работа с сетью}{17.10.2024}

\begin{document}

    \begin{frame}[plain]
        \titlepage
    \end{frame}

    \section{Клиент ВК}

    \begin{frame}
        \frametitle{Клиент для вконтакта}
        \begin{itemize}
            \item Пишем десктопный клиент для VK через VK API
            \item Документация: \url{https://dev.vk.com/ru/api/overview}
        \end{itemize}
    \end{frame}

    \begin{frame}
        \frametitle{Часть 1: получение открытой информации о пользователе}
        \begin{itemize}
            \item Метод Users.get
            \begin{itemize}
                \item \url{https://vk.com/dev/users.get}
            \end{itemize}
            \item Авторизация по сервисному ключу
            \begin{itemize}
                \item \url{https://vk.com/dev/access_token}
            \end{itemize}
            \item Потребуется зарегистрировать своё приложение
            \item Десериализация: Json.NET
            \begin{itemize}
                \item \url{https://www.newtonsoft.com/json/help/html/DeserializeObject.htm}
            \end{itemize}
        \end{itemize}
    \end{frame}

    \begin{frame}[fragile]
        \frametitle{Часть 2: авторизация, список контактов онлайн}
        \begin{itemize}
            \item Implicit Flow
            \begin{itemize}
                \item \url{https://vk.com/dev/implicit_flow_user}
            \end{itemize}
            \item Как запустить браузер из .NET:
            \begin{minted}{csharp}
authString = authString.Replace("&", "^&");
Process.Start(new ProcessStartInfo(
        "cmd", 
        $"/c start {authString}") 
        { CreateNoWindow = true });
            \end{minted}
        \end{itemize}
    \end{frame}

    \begin{frame}[fragile]
        \frametitle{Часть 3 (продвинутая): поменяем аватарку}
        \begin{itemize}
            \item В три этапа
            \begin{itemize}
                \item Получаем адрес, по которому надо загрузить фото 
                \begin{itemize}
                    \item \url{https://vk.com/dev/photos.getOwnerPhotoUploadServer}
                \end{itemize}
                \item Загружаем фото
                \begin{itemize}
                    \item \url{https://dev.vk.com/ru/api/upload/overview}
                    \item Как сформировать multipart/form-data:
                    \begin{minted}{csharp}
var form = new MultipartFormDataContent
{
    { new ByteArrayContent(File.ReadAllBytes("cthulhu.png"))
        , "photo", "cthulhu.png" }
};
                    \end{minted}
                \end{itemize}
                \item Выставляем фото как аватар
                \begin{itemize}
                    \item \url{https://vk.com/dev/photos.saveOwnerPhoto}
                \end{itemize}
            \end{itemize}
        \end{itemize}
    \end{frame}

\end{document}
