\documentclass{../../text-style}

\texttitle{Работа с сетью, практика}

\begin{document}

\maketitle
\thispagestyle{empty}

В этот раз будет практическое занятие на использование веб-сервисов и работу с протоколом авторизации OAuth --- написание консольного клиента для ВКонтакта с использованием VK REST API. Работа предполагается в командах по 3-4 человека, где большая часть участников занимается чтением документации. Хотя бы у одного члена команды должен быть аккаунт в VK.

Собственно, вся необходимая для практики информация есть в официальной документации, \url{https://vk.com/dev/manuals}. Задача будет состоять из трёх частей, от простого к сложному: сначала получение открытой информации, затем аутентификация и получение списка контактов онлайн, затем (если успеем, что вряд ли) загрузка изображения на сервера VK и выставление его как аватара сообщества.

\section{Получение открытой информации о пользователе}

Первая задача --- научиться получать открытую информацию о пользователе по его Id. Для этого надо, во-первых, зарегистрировать своё приложение в VK API. Это делается через страницу VK, меню слева, \enquote{Управление}, \enquote{Мои приложения}, \enquote{Создать}. Выбираете при создании Standalone-приложение. Дальше в его настройках можно увидеть сервисный ключ доступа, защищённый ключ и т.д., это потом потребуется для аутентификации.

Дальше уже в среде разработки создаём консольное приложение, вспоминаем, как делать HTTP-запросы из .NET-приложений, и параллельно внимательно читаем \url{https://vk.com/dev/first_guide}. Там, чтобы было не скучно, написана неправда, запрос users.get не исполнится просто так, он требует сервисного ключа от включённого в настройках VK приложения. Сервисный ключ надо передавать как параметр access\_token. Наконец, полезно почитать про сам нужный нам метод API: \url{https://vk.com/dev/users.get} и про работу с разного сорта ключами (с сервисным ключом всё просто, но дальше нам потребуется честная аутентификация): \url{https://vk.com/dev/access_token}.

Отвечать сервер будет JSON-документами, вот справка по System.Text.Json касательно десериализации (высокоуровневой) документов: \url{https://learn.microsoft.com/en-us/dotnet/standard/serialization/system-text-json/deserialization}. Чтобы оно работало, надо в коде на C\# описать классы-данные, которые будут соответствовать полям JSON-объектов (то есть иметь свойства с правильными именами и правильными типами). См. также \url{https://learn.microsoft.com/en-us/dotnet/standard/serialization/system-text-json/character-casing}.

Если всё пойдёт хорошо, в результате у вас должно получиться консольное приложение, куда можно ввести id пользователя и получить его имя-фамилию и другие поля, которые сочтёте интересными. Сервисный ключ стоит вводить как параметр командной строки или читать из файла, только не выкладывайте его на GitHub (в том числе и в конфиге запуска).

\section{Авторизация, список контактов онлайн}

Часть вторая этого задания требует получения списка контактов текущего пользователя, которые в данный момент находятся онлайн. Это уже нельзя сделать по сервисному ключу, требуется аутентифицировать конкретного пользователя (чтобы VK знал, чьи контакты ему выдавать). 

Для этого надо почитать документацию по Implicit Flow: \url{https://vk.com/dev/implicit_flow_user}. Вам потребуется открыть страницу браузера на правильный URL, ввести свои логин и пароль, после чего браузер редиректнут на \enquote{страницу}, содержащую access\_token. Его-то и надо использовать как access\_token для каждого следующего запроса. Его можно просто просить пользователя скопипастить в окно консоли из адресной строки браузера, ничего хитрее придумывать не нужно (и на паре всё равно не успеете, скорее всего, так что проходить аутентификацию надо будет каждый раз заново)\footnote{По-хорошему, надо открыть HttpListener на какой-нибудь порт на localhost и указать \mintinline{text}{redirect_uri=http://localhost:<ваш порт>/} в параметре redirect\_uri запроса авторизации, и считать из URL входящего запроса код авторизации (обратите внимание, сервер отвечает браузеру \enquote{redirect} с указанием того URL, который вы ему передали, браузер честно делает запрос --- не сам сервер идёт на localhost, конечно).}. Нужный метод API, который, собственно, вернёт список контактов, надо найти самостоятельно.

Вот как запустить браузер из .NET под Windows:

\begin{minted}{csharp}
authString = authString.Replace("&", "^&");
Process.Start(new ProcessStartInfo(
        "cmd", 
        $"/c start {authString}") 
        { CreateNoWindow = true });
\end{minted}

\section{Поменяем аватарку}

И последняя часть задания --- поменять аватарку текущему пользователю. Тут придётся хотя бы на базовом уровне разобраться с тем, что такое MIME, и как делать POST-запросы. Тут потребуется сделать три запроса:

\begin{itemize}
    \item получить адрес, по которому надо загрузить фото: \url{https://vk.com/dev/photos.getOwnerPhotoUploadServer};
    \item собственно загрузить фото: \url{https://vk.com/dev/upload_files};
    \item выставить загруженное фото как аватар: \url{https://vk.com/dev/photos.saveOwnerPhoto}.
\end{itemize}

Тут с деталями придётся разобраться самостоятельно, но раз аутентификация уже реализована, самое сложное позади. Вот как сформировать multipart/form-data в формате, который поймёт сервер вконтакта:

\begin{minted}{csharp}
var form = new MultipartFormDataContent
{
    { new ByteArrayContent(File.ReadAllBytes("cthulhu.png"))
        , "photo", "cthulhu.png" }
};
\end{minted}

\end{document}
