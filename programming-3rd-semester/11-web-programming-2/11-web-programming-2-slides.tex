\documentclass{../../slides-style}

\slidetitle[Часть 2]{Веб-программирование}{21.11.2024}

\begin{document}

    \begin{frame}[plain]
        \titlepage
    \end{frame}

    \section{Введение}

    \begin{frame}
        \frametitle{Попробуем написать что-нибудь \enquote{настоящее}}
        \begin{itemize}
            \item Приложение для регистрации на конференцию
            \item Титульная страница конференции со ссылкой на форму регистрации
            \item Форма регистрации
            \begin{itemize}
                \item Как слушатель или как докладчик
            \end{itemize}
            \item Страница, на которой можно просмотреть всех зарегистрировавшихся
            \item Итого, многостраничное приложение на Razor Pages
            \item Create a new project -> ASP.NET Core Web App (Microsoft)
            \begin{itemize}
                \item С именем \enquote{ConferenceRegistration}
            \end{itemize}
        \end{itemize}
    \end{frame}

    \section{Hello, world}

    \begin{frame}[fragile]
        \frametitle{Вид}
        \begin{minted}{html}
@page

<!DOCTYPE html>

<html>
    <head>
        <meta name="viewport" content="width=device-width" />
        <title>Hello</title>
    </head>
    <body>
        Hello, world!
    </body>
</html>
        \end{minted}
    \end{frame}

    \section{Регистрация}

    \begin{frame}[fragile]
        \frametitle{Моделирование данных}
        \framesubtitle{Registration.cshtml/Registration.cs}
        \begin{minted}{csharp}
namespace ConferenceRegistration.Pages;

[BindProperties]
public class RegistrationModel : PageModel
{
    public string Name { get; set; } = "";

    public string Email { get; set; } = "";

    public bool IsSpeaker { get; set; }
}
        \end{minted}
    \end{frame}

    \begin{frame}[fragile]
        \frametitle{Страница регистрации}
        \begin{ssmall}
            \begin{minted}{html}
@page
@model ConferenceRegistration.Pages.RegistrationModel
@addTagHelper *, Microsoft.AspNetCore.Mvc.TagHelpers

<html>
    <head>
        <meta name="viewport" content="width=device-width" />
        <title>Register</title>
    </head>
    <body>
        <form asp-action="Register" method="post">
            <p>
                <label asp-for="Name">Your name:</label>
                <input asp-for="Name" />
            </p>
            <p>
                <label asp-for="Email">Your email:</label>
                <input asp-for="Email" />
            </p>
            <p>
                <label>Are you a speaker?</label>
                <select asp-for="IsSpeaker">
                    <option value="">Choose an option</option>
                    <option value="true">Yes</option>
                    <option value="false">No</option>
                </select>
            </p>
            <button type="submit">Register!</button>
        </form>
    </body>
</html>
            \end{minted}
        \end{ssmall}
    \end{frame}

    \begin{frame}[fragile]
        \frametitle{Титульная страница}
        \begin{footnotesize}
            \begin{minted}{html}
@page

@addTagHelper *, Microsoft.AspNetCore.Mvc.TagHelpers

<html>
    <head>
        <meta name="viewport" content="width=device-width" />
        <title>SPISOK-2025 registration</title>
    </head>
    <body>
        <div>
            <p>SPISOK-2025 conference will be (unlikely, but...) held in April in St. Petersburg.</p>
            <a asp-page="Registration">Register now!</a>
        </div>
    </body>
</html>
            \end{minted}
        \end{footnotesize}
    \end{frame}

    \begin{frame}[fragile]
        \frametitle{Вернёмся к странице регистрации}
        \begin{footnotesize}
            \begin{minted}{csharp}
namespace ConferenceRegistration.Pages;

[BindProperties]
public class RegistrationModel : PageModel
{
    public string Name { get; set; } = "";

    public string Email { get; set; } = "";

    public bool IsSpeaker { get; set; }

    public void OnPost()
    {
        // TODO: Do something with registration info.
    }
}
            \end{minted}
        \end{footnotesize}
    \end{frame}

    \section{Персистентность}

    \begin{frame}[fragile]
        \frametitle{Работа с базой}
        \framesubtitle{Модель данных, Data.Participant.cs}
        \begin{footnotesize}
            \begin{minted}{csharp}
namespace ConferenceRegistration.Data;

public class Participant
{
    public string Name { get; set; } = "";

    public string Email { get; set; } = "";

    public bool IsSpeaker { get; set; }
}
            \end{minted}
        \end{footnotesize}
    \end{frame}

    \begin{frame}[fragile]
        \frametitle{Обновим модель}
        \begin{footnotesize}
            \begin{minted}{csharp}
namespace ConferenceRegistration.Pages;

using ConferenceRegistration.Data;

[BindProperties]
public class RegistrationModel : PageModel
{
    public Participant Participant { get; set; } = new();

    public void OnPost()
    {
        // TODO: Do something with registration info.
    }
}
            \end{minted}
        \end{footnotesize}
    \end{frame}

    \begin{frame}[fragile]
        \frametitle{И представление}
        \begin{ssmall}
            \begin{minted}{html}
@page
@model ConferenceRegistration.Pages.RegistrationModel
@addTagHelper *, Microsoft.AspNetCore.Mvc.TagHelpers

<html>
    <head>
        <meta name="viewport" content="width=device-width" />
        <title>Register</title>
    </head>
    <body>
        <form asp-action="Register" method="post">
            <p>
                <label asp-for="Participant.Name">Your name:</label>
                <input asp-for="Participant.Name" />
            </p>
            <p>
                <label asp-for="Participant.Email">Your email:</label>
                <input asp-for="Participant.Email" />
            </p>
            <p>
                <label>Are you a speaker?</label>
                <select asp-for="Participant.IsSpeaker">
                    <option value="">Choose an option</option>
                    <option value="true">Yes</option>
                    <option value="false">No</option>
                </select>
            </p>
            <button type="submit">Register!</button>
        </form>
    </body>
</html>
            \end{minted}
        \end{ssmall}
    \end{frame}

    \begin{frame}[fragile]
        \frametitle{DbContext}
        \begin{footnotesize}
            \begin{minted}{csharp}
namespace ConferenceRegistration.Data;

using Microsoft.EntityFrameworkCore;

public class ConferenceRegistrationDbContext: DbContext
{
    public ConferenceRegistrationDbContext(
        DbContextOptions<ConferenceRegistrationDbContext> options)
        : base(options)
    {
    }

    public DbSet<Participant> Participants => Set<Participant>();
}
            \end{minted}
        \end{footnotesize}
    \end{frame}

    \begin{frame}[fragile]
        \frametitle{Модель с DbContext}
        \begin{scriptsize}
            \begin{minted}{csharp}
namespace ConferenceRegistration.Pages;

using ConferenceRegistration.Data;

[BindProperties]
public class RegistrationModel(ConferenceRegistrationDbContext context) : PageModel
{
    public Participant Participant { get; set; } = new();

    public async Task<IActionResult> OnPostAsync()
    {
        context.Participants.Add(Participant);
        await context.SaveChangesAsync();

        return RedirectToPage("./Index");
    }
}
            \end{minted}
        \end{scriptsize}
    \end{frame}

    \begin{frame}[fragile]
        \frametitle{Конфигурация базы}
        \begin{footnotesize}
            \begin{minted}{csharp}
global using Microsoft.AspNetCore.Mvc;
global using Microsoft.AspNetCore.Mvc.RazorPages;

using ConferenceRegistration.Data;
using Microsoft.EntityFrameworkCore;

var builder = WebApplication.CreateBuilder(args);

// Add services to the container.
builder.Services.AddRazorPages();

builder.Services.AddDbContext<ConferenceRegistrationDbContext>(options =>
    options.UseSqlite("Data Source=conferenceRegistration.db"));

var app = builder.Build();
            \end{minted}
        \end{footnotesize}
    \end{frame}

    \begin{frame}[fragile]
        \frametitle{Добавим первичный ключ в модель данных}
        \begin{footnotesize}
            \begin{minted}{csharp}
namespace ConferenceRegistration.Data;

public class Participant
{
    public int ParticipantId { get; set; }

    public string Name { get; set; } = "";

    public string Email { get; set; } = "";

    public bool IsSpeaker { get; set; }
}
            \end{minted}
        \end{footnotesize}
    \end{frame}

    \begin{frame}[fragile]
        \frametitle{Миграции}
        \begin{itemize}
            \item \emph{Миграции} --- механизм обеспечения эволюции схемы БД
            \item Генерируются автоматически по коду
            \item \mintinline{text}{dotnet tool install --global dotnet-ef}
            \item \mintinline{text}{dotnet add package Microsoft.EntityFrameworkCore.Design}
            \item \mintinline{text}{dotnet ef migrations add InitialCreate}
            \item \mintinline{text}{dotnet ef database update}
            \item Последний шаг надо применять каждый раз при создании базы
        \end{itemize}
    \end{frame}

    \section{Остальная функциональность}

    \begin{frame}[fragile]
        \frametitle{Список участников}
        \framesubtitle{ListParticipants.cshtml}
        \begin{ssmall}
            \begin{minted}{html}
@page
@model ConferenceRegistration.Pages.ListParticipantsModel

<html>
    <head>
        <meta name="viewport" content="width=device-width" />
        <title>ListParticipants</title>
    </head>
    <body>
    <h2>List of conference participants:</h2>
    <table>
        <thead>
            <tr>
                <th>Name</th>
                <th>Email</th>
                <th>Is speaker</th>
            </tr>
        </thead>
        <tbody>
            @foreach (ConferenceRegistration.Data.Participant p in Model.Participants) {
                <tr>
                    <td>@p.Name</td>
                    <td>@p.Email</td>
                    <td>@(p.IsSpeaker ? "Yes" : "No")</td>
                </tr>
            }
        </tbody>
    </table>
    </body>
</html>
            \end{minted}
        \end{ssmall}
    \end{frame}

    \begin{frame}[fragile]
        \frametitle{Модель}
        \begin{footnotesize}
            \begin{minted}{csharp}
namespace ConferenceRegistration.Pages;

using ConferenceRegistration.Data;

public class ListParticipantsModel(ConferenceRegistrationDbContext context) : PageModel
{
    public IList<Participant> Participants { get; private set; } = new List<Participant>();

    public void OnGet()
    {
        Participants = context.Participants.OrderBy(p => p.ParticipantId).ToList();
    }
}
            \end{minted}
        \end{footnotesize}
    \end{frame}

    \begin{frame}[fragile]
        \frametitle{Страница подтверждения регистрации}
        \framesubtitle{Thanks.cshtml}
        \begin{scriptsize}
            \begin{minted}{html}
@page
@model ConferenceRegistration.Pages.ThanksModel

<html>
    <head>
        <meta name="viewport" content="width=device-width" />
        <title>Thanks</title>
    </head>
    <body>
        <p>
            <h1>Thank you, @Model.Participant.Name</h1>
        </p>
        <p>
            @if (Model.Participant.IsSpeaker)
            {
                @:Please don't forget to submit your article!
            }
        </p>
    </body>
</html>
            \end{minted}
        \end{scriptsize}
    \end{frame}

    \begin{frame}[fragile]
        \frametitle{И модель для неё}
        \begin{footnotesize}
            \begin{minted}{csharp}
namespace ConferenceRegistration.Pages;
using ConferenceRegistration.Data;

public class ThanksModel : PageModel
{
    public Participant Participant { get; set; } = new();

    public void OnGet(Participant participant)
    {
        Participant = participant;
    }
}
            \end{minted}
        \end{footnotesize}
    \end{frame}

    \begin{frame}[fragile]
        \frametitle{И редирект на страницу}
        \begin{footnotesize}
            \begin{minted}{csharp}
...
public class RegistrationModel : PageModel
{
    ...
    public async Task<IActionResult> OnPostAsync()
    {
        context.Participants.Add(Participant);
        await context.SaveChangesAsync();

        return RedirectToPage("./Thanks", Participant);
    }
}
            \end{minted}
        \end{footnotesize}
    \end{frame}

    \section{Оформление}

    \begin{frame}[fragile]
        \frametitle{Оформление}
        \framesubtitle{Bootstrap}
        \begin{scriptsize}
            \begin{minted}{html}
@page
@addTagHelper *, Microsoft.AspNetCore.Mvc.TagHelpers

<html>
    <head>
        <meta name="viewport" content="width=device-width" />
        <title>SPISOK-2025 registration</title>
        <link rel="stylesheet" href="/lib/bootstrap/dist/css/bootstrap.css" />
    </head>
    <body>
        <div class="text-center">
            <h3>SPISOK-2025 conference will be (unlikely, but...) held in April in St. Petersburg.</h3>
            <a class="btn btn-primary" asp-page="Registration">Register now!</a>
        </div>
    </body>
</html>
            \end{minted}
        \end{scriptsize}
    \end{frame}

    \begin{frame}[fragile]
        \frametitle{Форма регистрации}
        \framesubtitle{Лейауты}
        \begin{ssmall}
            \begin{minted}{html}
<div class="row mb-3 text-center"><h4 class="col-sm-6">Registration form</h4></div>
<form asp-action="Register" method="post">
    <div class="row mb-3">
        <label class="col-sm-1 col-form-label col-form-label-lg" 
            asp-for="Participant.Name">Your name:</label>
        <div class="col-sm-4">
            <input class="form-control form-control-lg" 
                asp-for="Participant.Name" />
        </div>
    </div>
    <div class="row mb-3">
        <label class="col-sm-1 col-form-label col-form-label-lg" 
            asp-for="Participant.Email">Your email:</label>
        <div class="col-sm-4">
            <input class="form-control form-control-lg" 
                asp-for="Participant.Email" />
        </div>
    </div>
    ...
    <div class="row mb-3 mx-auto">
        <div class="col-sm-5 d-grid gap-2">
            <button class="btn btn-primary btn-lg" type="submit">
                Register!
            </button>
        </div>
    </div>
</form>
            \end{minted}
        \end{ssmall}
    \end{frame}

    \begin{frame}[fragile]
        \frametitle{Список участников}
        \begin{ssmall}
            \begin{minted}{html}
@page
@model ConferenceRegistration.Pages.ListParticipantsModel

<html>
    <head>
        <meta name="viewport" content="width=device-width" />
        <title>ListParticipants</title>
        <link rel="stylesheet" href="/lib/bootstrap/dist/css/bootstrap.css" />
    </head>
    <body>
        <div class="row mb-3 text-center"><h2 class="col-sm-6">List of conference participants</h2></div>
        <table class="table table-striped table-bordered">
            <thead>
                <tr>
                    <th>Name</th>
                    <th>Email</th>
                    <th>Is speaker</th>
                </tr>
            </thead>
            <tbody>
            @foreach (ConferenceRegistration.Data.Participant p in Model.Participants) {
                <tr>
                    <td>@p.Name</td>
                    <td>@p.Email</td>
                    <td>@(p.IsSpeaker ? "Yes" : "No")</td>
                </tr>
            }
            </tbody>
        </table>
    </body>
</html>
            \end{minted}
        \end{ssmall}
    \end{frame}

    \section{Валидация}

    \begin{frame}[fragile]
        \frametitle{Декларативная валидация}
        \begin{footnotesize}
            \begin{minted}{csharp}
namespace ConferenceRegistration.Data;
using System.ComponentModel.DataAnnotations;

public class Participant
{
    public int ParticipantId { get; set; }

    [Required(ErrorMessage = "Please enter your name")]
    public string Name { get; set; }

    [Required(ErrorMessage = "Please enter your email")]
    [RegularExpression(".+\\@.+\\..+", ErrorMessage = 
        "Please enter a valid email address")]
    public string Email { get; set; }

    [Required(ErrorMessage = 
        "Please specify whether you'll be a speaker or just attending")]
    public bool? IsSpeaker { get; set; }
}
            \end{minted}
        \end{footnotesize}
    \end{frame}

    \begin{frame}[fragile]
        \frametitle{Модель}
        \begin{footnotesize}
            \begin{minted}{csharp}
...
public class RegistrationModel : PageModel
{
    ...
    public async Task<IActionResult> OnPostAsync()
    {
        if (!ModelState.IsValid)
        {
            return Page();
        }

        context.Participants.Add(Participant);
        await context.SaveChangesAsync();

        return RedirectToPage("./Thanks", Participant);
    }
}
            \end{minted}
        \end{footnotesize}
    \end{frame}

    \begin{frame}[fragile]
        \frametitle{Представление}
        \begin{ssmall}
            \begin{minted}{html}
...
<div class="row mb-3 text-center"><h4 class="col-sm-6">Registration form</h4></div>
<form asp-action="Register" method="post">
    <div class="row mb-3">
        <span asp-validation-for="Participant.Name" class="text-danger"></span>
        ...
    </div>
    <div class="row mb-3">
        <span asp-validation-for="Participant.Email" class="text-danger"></span>
        ...
    </div>
    <div class="row mb-3">
        <span asp-validation-for="Participant.IsSpeaker" class="text-danger"></span>
        ...
    </div>
    <div class="row mb-3 mx-auto">
        <div class="col-sm-5 d-grid gap-2">
            <button class="btn btn-primary btn-lg" type="submit">
                Register!
            </button>
        </div>
    </div>
</form>
...
            \end{minted}
        \end{ssmall}
    \end{frame}

    \begin{frame}[fragile]
        \frametitle{Client-side-валидация}
        \begin{scriptsize}
            \begin{minted}{html}
...
<head>
    <meta name="viewport" content="width=device-width" />
    <title>Register</title>
    <link rel="stylesheet" href="/lib/bootstrap/dist/css/bootstrap.css" />
    <script src="/lib/jquery/dist/jquery.js"></script>
    <script src="/lib/jquery-validation/dist/jquery.validate.js"></script>
    <script src="/lib/jquery-validation-unobtrusive/jquery.validate.unobtrusive.js">
    </script>
</head>
...
            \end{minted}
        \end{scriptsize}
    \end{frame}

\end{document}
