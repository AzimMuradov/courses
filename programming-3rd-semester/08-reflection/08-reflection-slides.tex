\documentclass{../../slides-style}

\slidetitle{Рефлексия}{26.10.2023}

\begin{document}

    \begin{frame}[plain]
        \titlepage
    \end{frame}

    \section{Рефлексия}

    \begin{frame}
        \frametitle{Рефлексия}
        \begin{itemize}
            \item Позволяет во время выполнения получать информацию о типах
            \begin{itemize}
                \item И главное, создавать объекты этих типов и вызывать их методы
            \end{itemize}
            \item Зачем:
            \begin{itemize}
                \item Плагины
                \item Анализаторы кода
                \item Тестовые системы
                \item Библиотеки сериализации
                \item Библиотеки для работы с базами данных
                \item Inversion of Control-контейнеры
                \item ...
            \end{itemize}
            \item Проблемы:
            \begin{itemize}
                \item Медленно
                \item Нет помощи от системы типов
            \end{itemize}
        \end{itemize}
    \end{frame}

    \section{Сборки}

    \begin{frame}
        \frametitle{Рефлексия в .NET}
        \begin{itemize}
            \item Пространство имён System.Reflection
            \item Байт-код хранит всю информацию о типах 
            \begin{itemize}
                \item И даже параметрах-типах у генериков, в отличие от Java
            \end{itemize}
            \item Самая крупная штука, которой оперирует рефлексия --- \textbf{сборка}
            \begin{itemize}
                \item .dll или .exe, единица развёртывания программы
                \begin{itemize}
                    \item На самом деле это неправда, детали см. в CLR via C\#
                \end{itemize}
            \end{itemize}
            \item Сборка хранит метаинформацию:
            \begin{itemize}
                \item Таблицы модулей, типов, методов, полей, параметров, свойств и событий
            \end{itemize}
            \item На всё это можно посмотреть в ILDasm
            \item AppDomain --- логическая группа сборок во время выполнения
        \end{itemize}
    \end{frame}

    \begin{frame}[fragile]
        \frametitle{Загрузка сборки}
        \begin{small}
            \begin{minted}{csharp}
public class Assembly {
    public static Assembly Load(AssemblyName assemblyRef);
    public static Assembly Load(String assemblyString);
    public static Assembly Load(byte[] rawAssembly)
    public static Assembly LoadFrom(String path);
    public static Assembly ReflectionOnlyLoad(String assemblyString);
    public static Assembly ReflectionOnlyLoadFrom(String assemblyFile);
}
            \end{minted}
            например,
            \begin{minted}{csharp}
var a = Assembly.LoadFrom(@"http://example.com/ExampleAssembly.dll");
            \end{minted}
            Такая сборка должна быть ещё не загружена. Выгружать сборки нельзя.
        \end{small}
    \end{frame}

    \begin{frame}[fragile]
        \frametitle{Сильные и слабые имена сборок (1)}
        \begin{itemize}
            \item Сильные сборки подписаны асимметричным шифром
            \begin{itemize}
                \item Публичная часть ключа внедряется в саму сборку
                \item От сборки считается SHA-1-хеш, шифруется приватным ключом и внедряется в сборку
                \item CLR при загрузке сборки считает от неё SHA-1-хеш и проверяет, что он совпал с подписанным
                \item Позволяет проверить, что сборка именно такая, какой её собирал автор, но не позволяет проверить идентичность автора
            \end{itemize}
        \end{itemize}
    \end{frame}

    \begin{frame}[fragile]
        \frametitle{Сильные и слабые имена сборок (2)}
        \begin{itemize}
            \item Пример сильного имени:
            \begin{scriptsize}
                \begin{minted}{csharp}
"MyTypes, Version=1.0.8123.0, Culture=neutral, PublicKeyToken=b77a5c561934e089"
                \end{minted}
            \end{scriptsize}
            \begin{itemize}
                \item Однозначно идентифицирует сборку
                \item PublicKeyToken --- короткий хеш публичного ключа
                \begin{itemize}
                    \item Естественно, не криптостоек, поэтому CLR проверяет настоящий ключ при загрузке сборки
                \end{itemize}
            \end{itemize}
            \item Сборка с сильным именем может ссылаться только на сборки с сильными именами
            \item Сборка с сильным именем может быть установлена в GAC
            \begin{itemize}
                \item GACUtil --- утилита для манипуляции GAC, в стандартной поставке
            \end{itemize}
        \end{itemize}
    \end{frame}

    \begin{frame}[fragile]
        \frametitle{Пример}
        \framesubtitle{Распечатать имена всех типов в сборке}
        \begin{footnotesize}
            \begin{minted}{csharp}
using System;
using System.Reflection;

public static class Program {
    public static void Main() {
        string dataAssembly = "System.Data, version=4.0.0.0, "
                + "culture=neutral, PublicKeyToken=b77a5c561934e089";
        LoadAssemblyAndShowPublicTypes(dataAssembly);
    }

    private static void LoadAssemblyAndShowPublicTypes(string assemblyId) {
        var a = Assembly.Load(assemblyId);
        foreach (Type t in a.ExportedTypes) {
            Console.WriteLine(t.FullName);
        }
    }
}
            \end{minted}
        \end{footnotesize}
    \end{frame}

    \section{Типы}

    \begin{frame}[fragile]
        \frametitle{Создание экземпляра объекта}
        \begin{itemize}
            \item System.Activator.CreateInstance --- можно передавать тип или строку с именем типа
            \begin{itemize}
                \item Версии со строкой возвращают System.Runtime.Remoting.ObjectHandle, надо вызвать Unwrap()
            \end{itemize}
            \item System.Activator.CreateInstanceFrom --- вызывает LoadFrom для сборки
            \item System.Reflection.ConstructorInfo.Invoke --- просто вызов конструктора (несколько дольше писать, чем предыдущие варианты)
            \item Рефлексия ничего не знает о синонимах
            \begin{itemize}
                \item То есть int везде называется System.Int32
            \end{itemize}
        \end{itemize}
        Пример:
        \begin{small}
            \begin{minted}{csharp}
var zero = Activator.CreateInstance("mscorlib.dll", "System.Int32").Unwrap();
            \end{minted}
        \end{small}
    \end{frame}

    \begin{frame}[fragile]
        \frametitle{Создание экземпляра типа-генерика}
        \begin{footnotesize}
            \begin{minted}{csharp}
using System;
using System.Reflection;

internal sealed class Dictionary<TKey, TValue> { }

public static class Program {
    public static void Main() {
        Type openType = typeof(Dictionary<,>);
        Type closedType = openType.MakeGenericType(
                typeof(String), typeof(Int32));
        Object o = Activator.CreateInstance(closedType);
        Console.WriteLine(o.GetType());
    }
}
            \end{minted}
        \end{footnotesize}
    \end{frame}

    \begin{frame}
        \frametitle{Пример: как сделать свою плагинную систему}
        \begin{itemize}
            \item Сделать отдельную сборку с описанием интерфейса плагина и типов данных, которые он использует
            \begin{itemize}
                \item Менять её будет очень проблематично
            \end{itemize}
            \item Сделать ``ядро системы'' --- отдельную сборку, ссылающуюся на сборку с интерфейсом плагина
            \item Делать набор плагинов, ссылающихся на сборку с интерфейсом плагина и реализующих его
        \end{itemize}
    \end{frame}

    \begin{frame}[fragile]
        \frametitle{Пример: интерфейс плагина}
        \begin{minted}{csharp}
namespace MyCoolSystem.SDK {
    public interface IAddIn {
        string DoSomething(int x);
    }
}
        \end{minted}
    \end{frame}

    \begin{frame}[fragile]
        \frametitle{Пример: плагины}
        \begin{minted}{csharp}
using MyCoolSystem.SDK;

public sealed class AddInA : IAddIn {
    public String DoSomething(int x) {
        return "AddInA: " + x.ToString();
    }
}

public sealed class AddInB : IAddIn {
    public String DoSomething(int x) {
        return "AddInB: " + (x * 2).ToString();
    }
}
        \end{minted}
    \end{frame}

    \begin{frame}[fragile]
        \frametitle{Пример: ядро системы}
        \begin{scriptsize}
            \begin{minted}{csharp}
public static class Program
{
    public static void Main()
    {
        string addInDir = Path.GetDirectoryName(Assembly.GetEntryAssembly().Location);
        var addInAssemblies = Directory.EnumerateFiles(addInDir, "*.dll");
        var addInTypes =
            addInAssemblies.Select(Assembly.Load)
                .SelectMany(a => a.ExportedTypes)
                .Where(t => t.IsClass 
                        && typeof(IAddIn).GetTypeInfo().IsAssignableFrom(t.GetTypeInfo()));

        foreach (Type t in addInTypes)
        {
            var addIn = (IAddIn)Activator.CreateInstance(t);
            Console.WriteLine(addIn.DoSomething(5));
        }
    }
}
            \end{minted}
        \end{scriptsize}
    \end{frame}

    \section{Поля, методы и т.д.}

    \begin{frame}
        \frametitle{Информация о типах}
        \begin{center}
            \begin{tiny}
                \begin{forest}
                    for tree={rectangle,draw,l sep=1cm,s sep=3mm,edge=open triangle 60-}
                    [Object
                        [Reflection.MemberInfo
                            [Reflection.TypeInfo]
                            [Reflection.FieldInfo]
                            [Reflection.MethodBase
                                [Reflection.ConstructorInfo]
                                [Reflection.MethodInfo]
                            ]
                            [Reflection.PropertyInfo]
                            [Reflection.EventInfo]
                        ]
                    ]
                \end{forest}
            \end{tiny}
        \end{center}
    \end{frame}

    \begin{frame}[fragile]
        \frametitle{Пример: распечатать информацию о полях и методах}
        \begin{scriptsize}
            \begin{minted}{csharp}
using System;
using System.Reflection;

public static class Program {
    public static void Main() {
        Assembly[] assemblies = AppDomain.CurrentDomain.GetAssemblies();
        foreach (Assembly a in assemblies) {
            Console.WriteLine($"Assembly: {a}");
            foreach (Type t in a.ExportedTypes) {
                Console.WriteLine($"  Type: {t}");
                foreach (MemberInfo mi in t.GetTypeInfo().DeclaredMembers) {
                    var typeName = mi switch {
                        FieldInfo _ => "FieldInfo",
                        MethodInfo _ => "MethodInfo",
                        ConstructorInfo _ => "ConstructorInfo",
                        _ => ""
                    };
                    Console.WriteLine($"    {typeName}: {mi}");
                }
            }
        }
    }
}
            \end{minted}
        \end{scriptsize}
    \end{frame}

    \begin{frame}
        \frametitle{Полезные свойства MemberInfo}
        \begin{itemize}
            \item Name (string) --- имя члена класса
            \item DeclaringType (Type) --- тип
            \item Module (Module) --- модуль, в котором он объявлен
            \item CustomAttributes (IEnumerable<CustomAttributeData>) --- коллекция атрибутов, соответствующих этому члену класса
            \begin{itemize}
                \item Пример --- модульные тесты
            \end{itemize}
        \end{itemize}
    \end{frame}

    \begin{frame}
        \frametitle{Как что-нибудь сделать с MemberInfo}
        \begin{itemize}
            \item GetValue и SetValue для FieldInfo и PropertyInfo
            \item Invoke для ConstructorInfo и MethodInfo
            \item AddEventHandler и RemoveEventHandler для EventInfo
        \end{itemize}
    \end{frame}

    \begin{frame}[fragile]
        \frametitle{Пример: создать объект и вызвать его метод}
        \begin{scriptsize}
            \begin{minted}{csharp}
using System;
using System.Reflection;
using System.Linq;

internal sealed class SomeType {
    public SomeType(int test) { }
    private int DoSomething(int x) => x * 2;
}

public static class Program {
    public static void Main() {
        Type t = typeof(SomeType);
        Type ctorArgument = Type.GetType("System.Int32");
        ConstructorInfo ctor = t.GetTypeInfo().DeclaredConstructors.First(
                c => c.GetParameters()[0].ParameterType == ctorArgument);
        Object[] args = { 12 };
        Object obj = ctor.Invoke(args);
        MethodInfo mi = obj.GetType().GetTypeInfo().GetDeclaredMethod("DoSomething");
        int result = (int)mi.Invoke(obj, new object[]{3});
        Console.WriteLine($"result = {result}");
    }
}
            \end{minted}
        \end{scriptsize}
    \end{frame}

    \section{Атрибуты}

    \begin{frame}
        \frametitle{Атрибуты}
        \begin{itemize}
            \item Способ добавить произвольную информацию к коду во время компиляции
            \item Эта информация может быть использована потом во время компиляции или во время выполнения
            \begin{itemize}
                \item Типичный пример --- атрибуты юнит-тестов (Test, ExpectedException, ...)
            \end{itemize}
            \item Могут быть применены к сборке, типу, полю, методу, параметру метода, возвращаемому значению, свойству, событию, параметру-типу
            \item Могут иметь параметры
            \item На самом деле, экземпляры классов-наследников System.Attribute
        \end{itemize}
    \end{frame}

    \begin{frame}[fragile]
        \frametitle{Объявление своего атрибута}
        \begin{footnotesize}
            \begin{minted}{csharp}
public enum Animal
{
    Dog = 1,
    Cat,
    Bird,
}

public class AnimalTypeAttribute : Attribute
{
    public AnimalTypeAttribute(Animal pet)
    {
        this.Pet = pet;
    }

    public Animal Pet { get; set; }
}
            \end{minted}
        \end{footnotesize}
        \attribution{MSDN}
    \end{frame}

    \begin{frame}[fragile]
        \frametitle{Использование атрибута}
        \begin{minted}{csharp}
class AnimalTypeTestClass
{
    [AnimalType(Animal.Dog)]
    public void DogMethod() { }

    [AnimalType(Animal.Cat)]
    public void CatMethod() { }

    [AnimalType(Animal.Bird)]
    public void BirdMethod() { }
}
        \end{minted}
    \end{frame}

    \begin{frame}[fragile]
        \frametitle{Получение атрибута рефлексией}
        \begin{footnotesize}
            \begin{minted}{csharp}
static void Main(string[] args)
{
    var testClass = new AnimalTypeTestClass();
    Type type = testClass.GetType();

    foreach (MethodInfo mInfo in type.GetMethods())
    {
        foreach (Attribute attr in Attribute.GetCustomAttributes(mInfo))
        {
            // Check for the AnimalType attribute.
            if (attr.GetType() == typeof(AnimalTypeAttribute))
                Console.WriteLine(
                    "Method {0} has a pet {1} attribute.",
                    mInfo.Name, ((AnimalTypeAttribute)attr).Pet);
        }
    }
}
            \end{minted}
        \end{footnotesize}
    \end{frame}

    \begin{frame}[fragile]
        \frametitle{Ограничение области применения атрибута}
        \begin{minted}{csharp}
namespace System {
    [AttributeUsage(AttributeTargets.Enum, Inherited = false)]
    public class FlagsAttribute : System.Attribute {
        public FlagsAttribute() {
        }
    }
}
        \end{minted}
        Атрибуты у атрибутов!
    \end{frame}

    \section{Сериализация}

    \begin{frame}
        \frametitle{Библиотеки сериализации}
        \begin{itemize}
            \item Нужны для передачи объектов по сети или сохранения объектов
            \item Часто используемые форматы
            \begin{itemize}
                \item XML
                \item JSON
                \item Protobuf
            \end{itemize}
            \item Активно используют рефлексию
        \end{itemize}
    \end{frame}

    \begin{frame}[fragile]
        \frametitle{Пример, сериализация в XML, System.Xml.Serialization}
        \framesubtitle{Данные}
        \begin{footnotesize}
            \begin{minted}{csharp}
public class Point
{
    public Point() { }
    public Point(int x, int y) { this.X = x; this.Y = y; }
    public int X { get; set; }
    public int Y { get; set; }
}

public class SomeData
{
    public int Radius { get; set; }
    public Point Center { get; set; }
    public List<Point> Points { get; } = new List<Point>();
}
            \end{minted}
        \end{footnotesize}
    \end{frame}

    \begin{frame}[fragile]
        \frametitle{Сама сериализация}
        \begin{small}
            \begin{minted}{csharp}
var data = new SomeData { Radius = 10, Center = new Point(5, 5)};
data.Points.Add(new Point(1, 1));
data.Points.Add(new Point(2, 2));

var serializer = new XmlSerializer(typeof(SomeData));
using (var textWriter = new StringWriter())
{
    serializer.Serialize(textWriter, data);
    Console.WriteLine(textWriter.ToString());
}
            \end{minted}
        \end{small}
    \end{frame}

    \begin{frame}[fragile]
        \frametitle{Результат}
        \begin{small}
            \begin{minted}{xml}
<?xml version="1.0" encoding="utf-16"?>
<SomeData xmlns:xsi="..." xmlns:xsd="...">
    <Radius>10</Radius>
    <Center>
        <X>5</X>
        <Y>5</Y>
    </Center>
    <Points>
        <Point>
            <X>1</X>
            <Y>1</Y>
        </Point>
        <Point>
            <X>2</X>
            <Y>2</Y>
        </Point>
    </Points>
</SomeData>
            \end{minted}
        \end{small}
    \end{frame}

    \begin{frame}[fragile]
        \frametitle{То же с Newtonsoft.Json}
        \begin{small}
            \begin{minted}{csharp}
var data = new SomeData { Radius = 10, Center = new Point(5, 5)};
data.Points.Add(new Point(1, 1));
data.Points.Add(new Point(2, 2));

var jsonSerializer = new JsonSerializer() { Formatting = Formatting.Indented };
using (var textWriter = new StringWriter())
{
    jsonSerializer.Serialize(textWriter, data);
    Console.WriteLine(textWriter.ToString());
}
            \end{minted}
        \end{small}
    \end{frame}

    \begin{frame}[fragile]
        \frametitle{Результат}
        \begin{small}
            \begin{minted}{json}
{
  "Radius": 10,
  "Center": {
    "X": 5,
    "Y": 5
  },
  "Points": [
    {
      "X": 1,
      "Y": 1
    },
    {
      "X": 2,
      "Y": 2
    }
  ]
}
            \end{minted}
        \end{small}
    \end{frame}

    \begin{frame}[fragile]
        \frametitle{Сериализация графа объектов}
        \framesubtitle{Дьявол в деталях}
        \begin{small}
            \begin{minted}{csharp}
var center = new Point(5, 5);
var data = new SomeData { Radius = 10, Center = center };
data.Points.Add(new Point(1, 1));
data.Points.Add(center);

var jsonSerializer = new JsonSerializer() {
    Formatting = Formatting.Indented,
    PreserveReferencesHandling = PreserveReferencesHandling.All
};
using (var textWriter = new StringWriter())
{
    jsonSerializer.Serialize(textWriter, data);
    Console.WriteLine(textWriter.ToString());
}
            \end{minted}
        \end{small}
    \end{frame}

    \begin{frame}[fragile]
        \frametitle{Результат}
        \begin{footnotesize}
            \begin{minted}{json}
{
  "$id": "1",
  "Radius": 10,
  "Center": {
    "$id": "2",
    "X": 5,
    "Y": 5
  },
  "Points": [
    {
      "$id": "3",
      "X": 1,
      "Y": 1
    },
    {
      "$ref": "2"
    }
  ]
}
            \end{minted}
        \end{footnotesize}
    \end{frame}

    \begin{frame}[fragile]
        \frametitle{Десериализация}
        \begin{minted}{csharp}
var deserializedData = 
    JsonConvert.DeserializeObject<SomeData>(serializedData);
Console.WriteLine(deserializedData.Radius);
        \end{minted}
    \end{frame}

    \begin{frame}[fragile]
        \frametitle{Управление именами полей}
        \begin{minted}{csharp}
public class Videogame
{
    [JsonProperty("name")]
    public string Name { get; set; }

    [JsonProperty("release_date")]
    public DateTime ReleaseDate { get; set; }
}
        \end{minted}
        \attribution{https://www.newtonsoft.com}
    \end{frame}

    \begin{frame}[fragile]
        \frametitle{Управление тем, что надо сериализовать, а что нет}
        \begin{minted}{csharp}
[JsonObject(MemberSerialization.OptIn)]
public class File
{
    // Это поле не сериализуется,
    // потому что у него нет JsonPropertyAttribute
    public Guid Id { get; set; }

    [JsonProperty]
    public string Name { get; set; }

    [JsonProperty]
    public int Size { get; set; }
}
        \end{minted}
    \end{frame}

    \section{dynamic}

    \begin{frame}[fragile]
        \frametitle{Ключевое слово dynamic}
        \begin{scriptsize}
            \begin{minted}{csharp}
using System;

internal static class DynamicDemo
{
    public static void Main()
    {
        dynamic value;
        for (int demo = 0; demo < 2; demo++)
        {
            value = (demo == 0) ? (dynamic)5 : (dynamic)"A";
            value = value + value;
            M(value);
        }
    }

    private static void M(int n) { Console.WriteLine("M(int): " + n); }
    private static void M(string s) { Console.WriteLine("M(string): " + s); }
}
            \end{minted}
        \end{scriptsize}
    \end{frame}

    \section{ILGenerator}

    \begin{frame}[fragile]
        \frametitle{Генерация кода ``на лету''}
        \begin{scriptsize}
            \begin{minted}{csharp}
public static void Main() {
    AssemblyName assemblyName = new AssemblyName {Name = "HelloEmit"};
    AppDomain appDomain = AppDomain.CurrentDomain;
    AssemblyBuilder assemblyBuilder = appDomain.DefineDynamicAssembly(
        assemblyName, AssemblyBuilderAccess.Save);
    ModuleBuilder moduleBuilder = 
        assemblyBuilder.DefineDynamicModule(assemblyName.Name, "Hello.exe");
    TypeBuilder typeBuilder = moduleBuilder.DefineType("Test.MainClass",
        TypeAttributes.Public | TypeAttributes.Class);
    MethodBuilder methodBuilder = typeBuilder.DefineMethod("Main",
        MethodAttributes.Public | MethodAttributes.Static,
        typeof(int), new[] { typeof(string[]) });

    ILGenerator ilGenerator = methodBuilder.GetILGenerator();
    ilGenerator.Emit(OpCodes.Ldstr, "Hello, World!");
    ilGenerator.Emit(OpCodes.Call,
        typeof(Console).GetMethod("WriteLine", new[] { typeof(string) }));
    ilGenerator.Emit(OpCodes.Ldc_I4_0);
    ilGenerator.Emit(OpCodes.Ret);

    typeBuilder.CreateType();
    assemblyBuilder.SetEntryPoint(methodBuilder, PEFileKinds.ConsoleApplication);
    assemblyBuilder.Save("Hello.exe");
}
            \end{minted}
        \end{scriptsize}
    \end{frame}

\end{document}
