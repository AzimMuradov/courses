\documentclass[xetex,mathserif,serif]{beamer}
\usepackage{polyglossia}
\setdefaultlanguage[babelshorthands=true]{russian}
\usepackage{minted}
\usepackage{tabu}
\usepackage{moresize}

\useoutertheme{infolines}

\usepackage{fontspec}
\setmainfont{FreeSans}
\newfontfamily{\russianfonttt}{FreeSans}

\definecolor{links}{HTML}{2A1B81}
\hypersetup{colorlinks,linkcolor=,urlcolor=links}

\setbeamertemplate{blocks}[rounded][shadow=false]

\setbeamercolor*{block title alerted}{fg=red!50!black,bg=red!20}
\setbeamercolor*{block body alerted}{fg=black,bg=red!10}

\tabulinesep=1.2mm

\title{О презентациях семестровых}
\author[Юрий Литвинов]{Юрий Литвинов\\\small{\textcolor{gray}{yurii.litvinov@gmail.com}}}
\date{07.12.2018г}

\newcommand{\attribution}[1] {
\vspace{-5mm}\begin{flushright}\begin{scriptsize}\textcolor{gray}{\textcopyright\, #1}\end{scriptsize}\end{flushright}
}

\begin{document}

	\frame{\titlepage}

	\begin{frame}
		\frametitle{Общее}
		\begin{itemize}
			\item Доклад на 5-7 минут
			\item Возможна одна презентация на несколько человек, но у каждого должен быть свой слайд с результатами
			\item Чеклист по презентации: \url{https://docs.google.com/spreadsheets/d/1LvHveX6TdbzexuACcqGPeHIEph6cm4Hd0arCRQBqODw}
			\item Презентации прошлых лет: \url{http://se.math.spbu.ru/SE/Members/ylitvinov/semesterWorks2018_2year}
		\end{itemize}
	\end{frame}

	\begin{frame}
		\frametitle{Структура презентации}
		\begin{itemize}
			\item Титульный слайд 
			\begin{itemize}
				\item Тема, автор, научник (учёная степень если есть, должность)
			\end{itemize}
			\item Введение
			\begin{itemize}
				\item Краткий рассказ про предметную область
				\item Обосновать актуальность задачи
			\end{itemize}
			\item Постановка задачи (обязательно!)
			\begin{itemize}
				\item ``Целью работы является...''
				\item Список из 3-5 задач, которые надо было решить для достижения цели
				\item ``Сделать обзор'' (чего?), ``Разработать архитектуру'', ``Реализовать'', ``Провести эксперименты''...
			\end{itemize}
		\end{itemize}
	\end{frame}

	\begin{frame}
		\frametitle{Структура презентации (2)}
		\begin{itemize}
			\item Обзор
			\begin{itemize}
				\item Существующие решения
				\item Используемые технологии
				\item Всё, что делали не вы, но что нужно для понимания работы
			\end{itemize}
			\item Описание реализации
			\begin{itemize}
				\item Архитектура (UML-диаграммы приветствуются)
				\item Особенности реализации (то, над чем пришлось подумать)
			\end{itemize}
			\item Эксперименты
			\begin{itemize}
				\item Численные измерения (нужен матстат --- матожидание, дисперсия)
				\item Подписи к осям
				\item Примеры использования
				\item Сравнение с существующими аналогами, выводы
			\end{itemize}
		\end{itemize}
	\end{frame}

	\begin{frame}
		\frametitle{Структура презентации (3)}
		\begin{itemize}
			\item Результаты
			\begin{itemize}
				\item Список того, что выносится на защиту
				\item Должно соответствовать списку задач (лучше --- полностью повторять, с заменой ``сделать'' на ``сделано'')
				\item Всё, что перечислено в результатах, должно быть отражено ранее на слайдах
				\item Не очень приветствуются неотчуждаемые результаты (типа ``изучил'')
				\item Должно быть последним слайдом
			\end{itemize}
		\end{itemize}
	\end{frame}

\end{document}
