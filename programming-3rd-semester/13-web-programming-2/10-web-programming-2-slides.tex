\documentclass[xetex,mathserif,serif]{beamer}
\usepackage{polyglossia}
\setdefaultlanguage[babelshorthands=true]{russian}
\usepackage{minted}
\usepackage{tabu}
\usepackage{moresize}

\useoutertheme{infolines}

\usepackage{fontspec}
\setmainfont{FreeSans}
\newfontfamily{\russianfonttt}{FreeSans}

\definecolor{links}{HTML}{2A1B81}
\hypersetup{colorlinks,linkcolor=,urlcolor=links}

\setbeamertemplate{blocks}[rounded][shadow=false]

\setbeamercolor*{block title alerted}{fg=red!50!black,bg=red!20}
\setbeamercolor*{block body alerted}{fg=black,bg=red!10}

\tabulinesep=1.2mm

\title{Веб-программирование}
\subtitle{Часть 2}
\author[Юрий Литвинов]{Юрий Литвинов\\\small{\textcolor{gray}{yurii.litvinov@gmail.com}}}
\date{06.12.2019г}

\newcommand{\attribution}[1] {
\vspace{-5mm}\begin{flushright}\begin{scriptsize}\textcolor{gray}{\textcopyright\, #1}\end{scriptsize}\end{flushright}
}

\begin{document}

	\frame{\titlepage}

	\begin{frame}
		\frametitle{Попробуем написать что-нибудь ``настоящее''}
		\begin{itemize}
			\item Приложение для регистрации на конференцию
			\item Титульная страница конференции со ссылкой на форму регистрации
			\item Форма регистрации
			\begin{itemize}
				\item Как слушатель или как докладчик
			\end{itemize}
			\item Страница, на которой можно просмотреть всех зарегистрировавшихся
		\end{itemize}
	\end{frame}

	\begin{frame}[fragile]
		\frametitle{Моделирование предметной области}
		\begin{minted}{csharp}
public class Participant
{
    public int Id { get; set; }
    public string Name { get; set; }
    public string Email { get; set; }
    public bool? Speaker { get; set; }
}
		\end{minted}
	\end{frame}

	\begin{frame}[fragile]
		\frametitle{Титульная страница}
		\begin{minted}{html}
@page
@addTagHelper *, Microsoft.AspNetCore.Mvc.TagHelpers

<html>
<body>
    <h1>SEIM-2019</h1>
    <p>Coolest conference ever</p>
    <p>Please register:</p>
    <a asp-page="Register">Register</a>
</body>
</html>
		\end{minted}
	\end{frame}

	\begin{frame}[fragile]
		\frametitle{Страница регистрации}
		\begin{ssmall}
			\begin{minted}{html}
@page
@model WebApplication1.Pages.RegisterModel
@addTagHelper *, Microsoft.AspNetCore.Mvc.TagHelpers

<html>
<body>
    <form asp-action="Register" method="post">
        <p>
            <label asp-for="Participant.Name">Your name:</label>
            <input asp-for="Participant.Name" />
        </p>
        <p>
            <label asp-for="Participant.Email">Your email:</label>
            <input asp-for="Participant.Email" />
        </p>
        <p>
            <label>Are you a speaker?</label>
            <select asp-for="Participant.Speaker">
                <option value="">Choose an option</option>
                <option value="true">Yes</option>
                7
                <option value="false">No</option>
            </select>
        </p>
        <button type="submit">Register!</button>
    </form>
</body>
</html>
			\end{minted}
		\end{ssmall}
	\end{frame}

	\begin{frame}[fragile]
		\frametitle{Code behind}
		\begin{footnotesize}
			\begin{minted}{csharp}
public class RegisterModel : PageModel
{
    public void OnGet()
    {
    }

    [BindProperty]
    public Data.Participant Participant { get; set; }

    public IActionResult OnPost()
    {
        if (!ModelState.IsValid)
        {
            return Page();
        }

        return RedirectToPage($"/Thanks", new { name = Participant.Name });
    }
}
			\end{minted}
		\end{footnotesize}
	\end{frame}

	\begin{frame}[fragile]
		\frametitle{Страница благодарности}
		\begin{small}
			\begin{minted}{html}
@page "{name}"

<html>
<body>
    <h1>Thanks for registration, @RouteData.Values["name"]!</h1>
</body>
</html>
			\end{minted}
		\end{small}
	\end{frame}

	\begin{frame}[fragile]
		\frametitle{Добавим немного СУБД}
		\framesubtitle{Класс Data.AppDbContext}
		\begin{small}
			\begin{minted}{csharp}
public class AppDbContext: DbContext
{
    public AppDbContext(DbContextOptions options)
        : base(options)
    {
    }

    public DbSet<Participant> Participants { get; set; }
}
			\end{minted}
		\end{small}
	\end{frame}

	\begin{frame}[fragile]
		\frametitle{Но где же Connection String?}
		\begin{itemize}
			\item Dependency Injection --- архитектурный паттерн, отделяющий создание и конфигурирование объектов от их логики
			\item За создание и конфигурирование/переконфигурирование отвечает DI-контейнер, обычно отдельная подсистема
			\item Мы лишь описываем ограничения на то, что должно получиться
			\item Конкретно в нашем случае:
			\begin{itemize} 
				\item Подключаем пакет Microsoft.EntityFrameworkCore.Sqlite
				\item Пишем в Startup.ConfigureServices:
				\begin{minted}{csharp}
services.AddDbContext<Data.AppDbContext>(
    options => options.UseSqlite(@"Data source=Test.db;"));
				\end{minted}
			\end{itemize}
			\item DI-контейнер создаёт и регистрирует контекст, передавая ему в конструктор опции
		\end{itemize}
	\end{frame}

	\begin{frame}
		\frametitle{Инициализация базы данных}
		\begin{itemize}
			\item Идём ВНЕЗАПНО в Tools -> Nuget Package Manager -> Package Manager Console
			\item Набираем в консоли Add-Migration Initial
			\begin{itemize}
				\item Создаётся \textit{миграция}, которая создаст базу
			\end{itemize}
			\item Набираем в консоли Update-Database
			\begin{itemize}
				\item Миграция применяется, в папке с проектом создаётся Test.db
			\end{itemize}
		\end{itemize}
	\end{frame}

	\begin{frame}[fragile]
		\frametitle{Пришла пора воспользоваться СУБД}
		\begin{ssmall}
			\begin{minted}{csharp}
public class RegisterModel : PageModel
{
    private readonly AppDbContext db;

    public RegisterModel(AppDbContext db)
    {
        this.db = db;
    }

    [BindProperty]
    public Participant Participant { get; set; }

    public async Task<IActionResult> OnPostAsync()
    {
        if (!ModelState.IsValid)
        {
            return Page();
        }

        db.Participants.Add(Participant);
        await db.SaveChangesAsync();

        return RedirectToPage($"/Thanks", new { name = Participant.Name });
    }
}
			\end{minted}
		\end{ssmall}
	\end{frame}

	\begin{frame}[fragile]
		\frametitle{Ну и последняя страница, просмотр участников}
		\framesubtitle{Code Behind}
		\begin{scriptsize}
			\begin{minted}{csharp}
public class ViewModel : PageModel
{
    private readonly AppDbContext db;

    public ViewModel(AppDbContext db)
    {
        this.db = db;
    }

    public IList<Participant> Participants { get; private set; }

    public async Task OnGetAsync()
    {
        Participants = await db.Participants.AsNoTracking().ToListAsync();
    }
}
			\end{minted}
		\end{scriptsize}
	\end{frame}

	\begin{frame}[fragile]
		\frametitle{.cshtml}
		\begin{ssmall}
			\begin{minted}{html}
@page
@model WebApplication1.Pages.ViewModel

<html>
<body>
    <table class="table">
        <thead>
            <tr>
                <th>ID</th>
                <th>Name</th>
                <th>Email</th>
                <th>Is Speaker</th>
            </tr>
        </thead>
        <tbody>
            @foreach (var participant in Model.Participants)
            {
            <tr>
                <td>@participant.Id</td>
                <td>@participant.Name</td>
                <td>@participant.Email</td>
                <td>@(participant.Speaker == true ? "Yes" : "No")</td>
            </tr>
            }
        </tbody>
    </table>
</body>
</html>
			\end{minted}
		\end{ssmall}
	\end{frame}

	\begin{frame}[fragile]
		\frametitle{Но так даже в 90-е не верстали!}
		\framesubtitle{Добавим Bootstrap}
		\begin{tiny}
			\begin{minted}{html}
@page
@model WebApplication1.Pages.ViewModel

<html>
<head>
    <link rel="stylesheet" href="/lib/bootstrap/dist/css/bootstrap.css" />
</head>
<body>
    <table class="table table-striped table-bordered">
        <thead>
            <tr>
                <th scope="col">ID</th>
                <th scope="col">Name</th>
                <th scope="col">Email</th>
                <th scope="col">Is Speaker</th>
            </tr>
        </thead>
        <tbody>
            @foreach (var participant in Model.Participants)
            {
            <tr>
                <td>@participant.Id</td>
                <td>@participant.Name</td>
                <td>@participant.Email</td>
                <td>@(participant.Speaker == true ? "Yes" : "No")</td>
            </tr>
            }
        </tbody>
    </table>
</body>
</html>
			\end{minted}
		\end{tiny}
	\end{frame}

	\begin{frame}
		\frametitle{Валидация}
		\begin{itemize}
			\item Клиентская
			\begin{itemize}
				\item Работает с помощью jquery-validation
				\item Только в браузере у клиента, без нужды связи с сервером
				\item Только если там включён JavaScript
			\end{itemize}
			\item Серверная
			\begin{itemize}
				\item Проверка модели при Model Binding-е в контроллере
			\end{itemize}
		\end{itemize}
	\end{frame}

	\begin{frame}[fragile]
		\frametitle{Атрибуты валидации}
		\begin{scriptsize}
			\begin{minted}{csharp}
public class Participant
{
    public int Id { get; set; }

    [Required]
    public string Name { get; set; }

    [Required]
    [RegularExpression(@"^([a-zA-Z0-9_\-\.]+)"
        + @"@([a-zA-Z0-9_\-\.]+)\.([a-zA-Z]{2,5})$")]
    public string Email { get; set; }

    [Required(ErrorMessage = "Укажите Вашу роль")]
    public bool? Speaker { get; set; }
}
			\end{minted}
		\end{scriptsize}
	\end{frame}

	\begin{frame}
		\frametitle{Что ещё бывает}
		\begin{itemize}
			\item Required --- для nullable-типов требует значение, для не-nullable не имеет смысла, они обязательны всегда
			\item StringLength --- задаёт максимальную и минимальную длину строки
			\item RegularExpression
			\item Range --- ограничивает значение заданным атрибутом
		\end{itemize}
	\end{frame}

	\begin{frame}[fragile]
		\frametitle{Что надо сделать на клиенте}
		\begin{tiny}
			\begin{minted}{html}
<form asp-action="Register" method="post">
    <div asp-validation-summary="ModelOnly" class="text-danger"></div>
    <p>
        <label asp-for="Participant.Name">Your name:</label>
        <input asp-for="Participant.Name" />
        <span asp-validation-for="Participant.Name" class="text-danger" />
    </p>
    <p>
        <label asp-for="Participant.Email">Your email:</label>
        <input asp-for="Participant.Email" />
        <span asp-validation-for="Participant.Email" class="text-danger" />
    </p>
    <p>
        <label>Are you a speaker?</label>
        <select asp-for="Participant.Speaker">
            <option value="">Choose an option</option>
            <option value="true">Yes</option>
            <option value="false">No</option>
        </select>
        <span asp-validation-for="Participant.Speaker" class="text-danger" />
    </p>
    <button type="submit">Register!</button>
</form>
...
@section Scripts {
    @{await Html.RenderPartialAsync("_ValidationScriptsPartial"); }
}
			\end{minted}
		\end{tiny}
	\end{frame}

\end{document}
