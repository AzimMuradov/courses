\documentclass[xetex,mathserif,serif]{beamer}
\usepackage{polyglossia}
\setdefaultlanguage[babelshorthands=true]{russian}
\usepackage{minted}
\usepackage{tabu}

\useoutertheme{infolines}

\usepackage{fontspec}
\setmainfont{FreeSans}
\newfontfamily{\russianfonttt}{FreeSans}

\setbeamertemplate{blocks}[rounded][shadow=false]
\setbeamercolor*{block title example}{fg=green!50!black,bg=green!20}
\setbeamercolor*{block body example}{fg=black,bg=green!10}

\setbeamercolor*{block title alerted}{fg=red!50!black,bg=red!20}
\setbeamercolor*{block body alerted}{fg=black,bg=red!10}

\tabulinesep=0.7mm

\title{Практика по Java, часть 2}
\subtitle{Пара 1: Разминка}
\author[Юрий Литвинов]{Юрий Литвинов \newline \textcolor{gray}{\small\texttt{yurii.litvinov@gmail.com}}}

\date{16.02.2017г}

\begin{document}
	
	\frame{\titlepage}
	
	\begin{frame}
		\frametitle{Правила игры}
		\begin{itemize}
			\item Как обычно, куча домашек, две контрольные (и переписывание в конце), баллы и дедлайны, HwProj
			\begin{itemize}
				\item Будет три крупные задачи (разбитые на части) и несколько мелких
			\end{itemize}
			\item За практику будет выставляться оценка, которая потом будет учитываться при сдаче экзамена
			\begin{itemize}
				\item по какой-то хитрой формуле, учитывающей баллы за домашку и контрольные
			\end{itemize}
		\end{itemize}
	\end{frame}

	\begin{frame}
		\frametitle{Напоминание про штрафы}
		\begin{scriptsize}
			\begin{tabu} {| X[1 l p] | X[0.3 l p] |}
				\tabucline-
				\everyrow{\tabucline-}
				Пропущенный дедлайн                                                                   & -10 \\
				Задача на момент дедлайна не реализует все требования условия                         & пропорционально объёму невыполненных требований \\
				Замечания не исправлены за неделю после их получения                                  & -2 \\
				Неумение пользоваться гитом                                                           & -2 \\
				Проблемы со сборкой (в том числе, забытый org.jetbrains.annotations)                  & -2 \\
				Отсутствие JavaDoc-ов для всех классов, интерфейсов и паблик-методов                  & -2 \\
				Отсутствие описания метода в целом (в том числе, комментарии, начинающиеся с @return) & -1 \\
				Слишком широкие области видимости для полей                                           & -2 \\
				if (...) return true; else return false;                                              & -2 \\
				Комментарии для параметров с заглавной буквы                                          & -0.5 \\
			\end{tabu}
		\end{scriptsize}
		\begin{center}
			\scriptsize{Список может расширяться!}
		\end{center}
	\end{frame}

\begin{frame}[fragile]
\frametitle{Задача}
\framesubtitle{Многопоточный Lazy}

\begin{footnotesize}
Реализовать следующий интерфейс, представляющий ленивое вычисление:

\begin{exampleblock}{Java}
\begin{minted}{java}
public interface Lazy<T> {
  T get();
}
\end{minted}
\end{exampleblock}

\begin{itemize}
	\item Объект \textit{Lazy} создаётся на основе вычисления (представляемого объектом \textit{Supplier})
	\item Первый вызов \textit{get()} вызывает вычисление и возвращает результат
	\item Повторные вызовы \textit{get()} возвращают \textbf{тот же} объект, что и первый вызов
	\item В однопоточном режиме вычисление должно запускаться не более одного раза, в многопоточном --- как получится (см. далее)	
\end{itemize}
\end{footnotesize}

\end{frame}

	\begin{frame}
		\frametitle{LazyFactory}
		Создавать объекты надо не вручную, а с помощью класса \textit{LazyFactory}, который должен
		иметь три метода с сигнатурами вида \textit{public static <T> Lazy<T> 
		createLazy(Supplier<T>)}, возвращающих три разные реализации \textit{Lazy<T>}:
		\begin{itemize}
			\item Простая версия с гарантией корректной работы в однопоточном режиме 
				(без синхронизации)
			\item Гарантия корректной работы в многопоточном режиме; вычисление не должно
				производиться более одного раза
			\begin{itemize}
				\item Что-то наподобие многопоточного синглтона
			\end{itemize}
			\item То же, что и предыдущее, но lock-free; вычисление может производиться более одного
				раза, но при этом \textit{Lazy.get} всегда должен возвращать один и тот же объект
			\begin{itemize}
				\item см. \textit{AtomicReference}/\textit{AtomicReferenceFieldUpdater}
			\end{itemize}
		\end{itemize}
	\end{frame}
	
	\begin{frame}
		\frametitle{Чтобы было не скучно}
		\begin{itemize}
			\item Ограничение по памяти на каждый \textit{Lazy}-объект: не больше двух ссылок
			\item \textit{Supplier.get} вправе вернуть \textit{null}
		\end{itemize}
	\end{frame}

	\begin{frame}
		\frametitle{Дополнительные требования}
		\begin{itemize}
			\item Gradle/Maven
			\item CI, на котором проходят ваши тесты
			\item Тесты
			\begin{itemize}
				\item Однопоточные, на разные хорошие и плохие случаи
				\item Многопоточные, на наличие гонок
			\end{itemize}
			\item Дедлайн: до 23:59 02.03
			\item Пуллреквест к себе в репозиторий, отметка о сдаче на HwProj
		\end{itemize}
	\end{frame}

\end{document}

