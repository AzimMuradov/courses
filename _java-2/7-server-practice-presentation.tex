\documentclass[xetex,mathserif,serif]{beamer}
\usepackage{polyglossia}
\setdefaultlanguage[babelshorthands=true]{russian}
\usepackage{minted}

\useoutertheme{infolines}

\usepackage{fontspec}
\setmainfont{FreeSans}
\newfontfamily{\russianfonttt}{FreeSans}

\title{Практика, неблокирующий сервер}
\author[Юрий Литвинов]{Юрий Литвинов \newline \textcolor{gray}{\small\texttt{yurii.litvinov@gmail.com}}}

\date{06.04.2017г}

\begin{document}
	
	\frame{\titlepage}
	
	\section{Разбор контрольной}

	\begin{frame}[fragile]
		\frametitle{Разбор контрольной}
		\framesubtitle{Правильное (в каком-то смысле) решение, однопоточный вариант}
		\begin{scriptsize}
			\begin{minted}{java}
public @NotNull byte[] hash(@NotNull Path path) throws 
        NoSuchAlgorithmException, IOException {
    MessageDigest md = MessageDigest.getInstance("MD5");
    if (Files.isDirectory(path)) {
        md.update(path.getFileName().toString().getBytes());
        for (Path subPath : Files.walk(path).filter(Files::isRegularFile).collect(Collectors.toList())) {
            md.update(hash(subPath));
        }
        return md.digest();
    }

    try (InputStream rawStream = Files.newInputStream(path)) {
        DigestInputStream stream = new DigestInputStream(rawStream, md);
        byte[] buffer = new byte[BUFFER_SIZE];
        while (stream.read(buffer) != -1)
        {
        }
        return stream.getMessageDigest().digest();
    }
}
			\end{minted}
		\end{scriptsize}
\end{frame}

	\begin{frame}[fragile]
		\frametitle{Fork/Join-вариант}
		\begin{scriptsize}
			\begin{minted}{java}
MessageDigest md = MessageDigest.getInstance("MD5");
if (Files.isDirectory(path)) {
    md.update(path.getFileName().toString().getBytes());
    List<ForkJoinHasher> subTasks = new LinkedList<>();
    for (Path subPath : Files.walk(path).filter(Files::isRegularFile).collect(Collectors.toList())) {
        ForkJoinHasher task = new ForkJoinHasher(subPath);
        task.fork();
        subTasks.add(task);
    }
    for (ForkJoinHasher task : subTasks) {
        md.update(task.join());
    }
    return md.digest();
} else {
    try (InputStream rawStream = Files.newInputStream(path)) {
        DigestInputStream stream = new DigestInputStream(rawStream, md);
        byte[] buffer = new byte[BUFFER_SIZE];
        while (stream.read(buffer) != -1)
        {
        }
        return stream.getMessageDigest().digest();
    }
}
			\end{minted}
		\end{scriptsize}
\end{frame}

	\begin{frame}
		\frametitle{Частые проблемы}
		\begin{itemize}
			\item Невнимательное чтение условия
			\item Недостаток тестов
			\item Суровый копипаст в тестах
			\item Конечно же, комментарии
			\item Излишнее доверие пользователю в main-е
			\item Нетехнологичность
			\begin{itemize}
				\item Отсутствие аннотаций \mintinline{java}|@NotNull/@Nullable|
				\item \mintinline{java}|@Rule TemporaryFolder|
				\item try-with-resources
				\item assertArrayEquals
				\begin{itemize}
					\item Hamcrest? Только одно решение, seriously?
				\end{itemize}
				\item \textbf{Tools are everything!}
			\end{itemize}
		\end{itemize}
	\end{frame}

	\section{Домашка}

	\begin{frame}
		\frametitle{Домашка, Simple FTP}
		Требуется реализовать сервер, обрабатывающий два запроса.
		\begin{itemize}
			\item \textbf{list} --- листинг файлов в директории на сервере
			\item \textbf{get} --- скачивание файла с сервера
		\end{itemize}
		И клиент, позволяющий исполнять указанные запросы.
	\end{frame}

	\begin{frame}
		\frametitle{List}
		Формат запроса: \textit{<1: Int> <path: String>}

		\begin{itemize}
			\item path --- путь к директории
		\end{itemize}

		Формат ответа: \textit{<size: Int> (<name: String> <is\_dir: Boolean>)*}

		\begin{itemize}
			\item size --- количество файлов и папок в директории
			\item name --- название файла или папки
			\item is\_dir --- флаг, принимающий значение True для директорий
		\end{itemize}
		Если директории не существует, сервер посылает ответ с size = 0
	\end{frame}

	\begin{frame}
		\frametitle{Get}
		Формат запроса: \textit{<2: Int> <path: String>}
		\begin{itemize}
			\item path --- путь к файлу
		\end{itemize}
		
		Формат ответа: \textit{<size: Long> <content: Bytes>}
		\begin{itemize}
			\item size --- размер файла
			\item content --- его содержимое
		\end{itemize}
		Если файла не существует, сервер посылает ответ с size = 0
	\end{frame}

	\begin{frame}
		\frametitle{Примечания}
		\begin{itemize}
			\item Разрешается использовать библиотеки для упрощения ввода-вывода
			\item Рекомендуется взглянуть на DataInputStream и DataOutputStream
			\item Рекомендуется задуматься об интерфейсе сервера и клиента, возможно стоит сделать что-то подобное:
			\begin{itemize}
				\item Server: start/stop
				\item Client: connect/disconnect/executeList/executeGet
			\end{itemize}
		\end{itemize}
		Срок: \textbf{26.04.2017 23:59}.
	\end{frame}

	\section{Задача на пару}

	\begin{frame}
		\frametitle{Задача на пару, Non-blocking server}
		\begin{itemize}
			\item Сервер сразу начинает отдавать клиенту содержимое одного файла, после чего разрывает соединение
			\item Содержимое файла можно перед началом работы прочитать в память
			\item Работа с клиентскими сокетами и открытие новых должно происходить в неблокирующем режиме
			\item Сервер должен работать в одном потоке 
			\item Рекомендуется для клиентских сокетов включать TCP\_NODELAY
		\end{itemize}
	\end{frame}

\end{document}
