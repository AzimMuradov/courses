\documentclass[xetex,mathserif,serif]{beamer}
\usepackage{polyglossia}
\setdefaultlanguage[babelshorthands=true]{russian}
\usepackage{minted}

\useoutertheme{infolines}

\usepackage{fontspec}
\setmainfont{FreeSans}
\newfontfamily{\russianfonttt}{FreeSans}

\title{Контрольная}
\author[Юрий Литвинов]{Юрий Литвинов \newline \textcolor{gray}{\small\texttt{yurii.litvinov@gmail.com}}}

\date{30.03.2017г}

\begin{document}
	
	\frame{\titlepage}
	
	\section{Условие: MD5}

	\begin{frame}
		\frametitle{Задача, MD5}
		Реализовать консольное приложение, вычисляющее check-сумму директории файловой системы по такому правилу:
		\begin{itemize}
			\item \mintinline{java}|f(file) = MD5(<содержимое>)|
			\item \mintinline{java}|f(dir) = MD5(<имя папки> + f(file1) + ...)|
		\end{itemize}
		Требуется:
		\begin{itemize}
			\item Однопоточный вариант
			\item Вариант с Fork-Join
			\item Сравнить время их работы в main-е
		\end{itemize}
		Файлы могут быть большими и не помещаться в память целиком.
		
		Классы \textit{MessageDigest} и \textit{DigestInputStream} могут быть полезны. 
	\end{frame}

\end{document}
