\documentclass[xetex,mathserif,serif]{beamer}
\usepackage{polyglossia}
\setdefaultlanguage[babelshorthands=true]{russian}
\usepackage{minted}
\usepackage{tabu}

\useoutertheme{infolines}

\usepackage{fontspec}
\setmainfont{FreeSans}
\newfontfamily{\russianfonttt}{FreeSans}

\usepackage{textpos}
\setlength{\TPHorizModule}{1cm}
\setlength{\TPVertModule}{1cm}

\setbeamertemplate{blocks}[rounded][shadow=false]

\setbeamercolor*{block title alerted}{fg=red!50!black,bg=red!20}
\setbeamercolor*{block body alerted}{fg=black,bg=red!10}

\tabulinesep=1.2mm

\title{Занятие 2: работа с требованиями}
\author[Юрий Литвинов]{Юрий Литвинов\\\small{\textcolor{gray}{yurii.litvinov@gmail.com}}}
\date{20.09.2017}

\newcommand{\todo}[1] {
	\begin{center}\textcolor{red}{TODO: #1}\end{center}
}

\newcommand{\DownArrow} {
	\hspace{2cm}\begin{LARGE}$\downarrow$\end{LARGE}
}

\begin{document}

	\frame{\titlepage}

	\section{Введение}

	\begin{frame}
		\frametitle{Требования}
		\begin{itemize}
			\item \todo{Напоминание про требования вообще}
		\end{itemize}
	\end{frame}

	\section{Инструменты для работы с требованиями}

	\begin{frame}
		\frametitle{Метод случаев использования}
	\end{frame}

	\begin{frame}
		\frametitle{Диаграммы требований SysML}
	\end{frame}

	\begin{frame}
		\frametitle{Feature Diagrams}
	\end{frame}

	\section{Домашнее задание}

	\begin{frame}
		\frametitle{Домашнее задание}
		\begin{itemize}
			\item Описать формально требования к своему проекту
			\begin{itemize}
				\item Словесное описание
				\item Диаграммы в хотя бы одной из предложенных нотаций
			\end{itemize}
			\item Подготовить презентацию на 10 минут с представлением идеи проекта
			\item Дедлайн --- \textbf{20 сентября}
			\item Пары 13 сентября \textbf{не будет}!
		\end{itemize}
	\end{frame}

\end{document}
