\documentclass[xetex,mathserif,serif]{beamer}
\usepackage{polyglossia}
\setdefaultlanguage[babelshorthands=true]{russian}
\usepackage{minted}
\usepackage{tabu}

\useoutertheme{infolines}

\usepackage{fontspec}
\setmainfont{FreeSans}
\newfontfamily{\russianfonttt}{FreeSans}

\usepackage{textpos}
\setlength{\TPHorizModule}{1cm}
\setlength{\TPVertModule}{1cm}

\setbeamertemplate{blocks}[rounded][shadow=false]

\setbeamercolor*{block title alerted}{fg=red!50!black,bg=red!20}
\setbeamercolor*{block body alerted}{fg=black,bg=red!10}

\tabulinesep=1.2mm

\title{Занятие 8: Техническое задание}
\author[Юрий Литвинов]{Юрий Литвинов\\\small{\textcolor{gray}{yurii.litvinov@gmail.com}}}
\date{30.10.2019}

\newcommand{\attribution}[1] {
	\begin{flushright}\begin{scriptsize}\textcolor{gray}{\textcopyright\; #1}\end{scriptsize}\end{flushright}
}

\begin{document}

	\frame{\titlepage}

	\section{Введение}

	\begin{frame}
		\frametitle{Без хорошего ТЗ получается...}
		\begin{itemize}
			\item Техническое задание --- пожалуй, самая важная часть договора о разработке ПО
			\item Фиксирует (в юридически значимом документе) цель создания, требования к продукту, условия функционирования, сроки и этапы разработки, регламент приёмки
			\item Необходим и исполнителю, и заказчику
			\item Разрабатывается, как правило, исполнителем
			\item Стандарты:
			\begin{itemize}
				\item ГОСТ 19.201-78 Техническое задание. Требования к содержанию и оформлению.
				\item ГОСТ 34.602-89 Техническое задание на создание автоматизированной системы.
				\item IEEE 29148-2011
			\end{itemize}
			\item ``We value \textbf{working software} over comprehensive documentation''
			\begin{itemize}
				\item ``there is value in the items on the right''
			\end{itemize}
		\end{itemize}
	\end{frame}

	\section{ЕСПД}

	\begin{frame}
		\frametitle{ГОСТ 19: ЕСПД}
		\framesubtitle{Единая Система Программной Документации}
		Виды документов:
		\begin{itemize}
			\item Спецификация
			\item Ведомость держателей подлинников
			\item Текст программы
			\item Описание программы
			\item Программа и методика испытаний
			\item Техническое задание
			\item Пояснительная записка
			\item Эксплуатационные документы
		\end{itemize}
	\end{frame}

	\begin{frame}
		\frametitle{ЕСПД, эксплуатационные документы}
		\begin{itemize}
			\item Ведомость эксплуатационных документов
			\item Формуляр
			\item Описание применения
			\item Руководство системного программиста
			\item Руководство программиста
			\item Руководство оператора
			\item Описание языка
			\item Руководство по техническому обслуживанию
		\end{itemize}
	\end{frame}

	\section{ТЗ}

	\begin{frame}
		\frametitle{Техническое задание}
		\framesubtitle{ГОСТ 19.201-78}
		Разделы:
		\begin{itemize}
			\item введение
			\item основания для разработки
			\item назначение разработки
			\item требования к программе или программному изделию
			\item требования к программной документации
			\item технико-экономические показатели
			\item стадии и этапы разработки
			\item порядок контроля и приемки
			\item приложения (если есть)
		\end{itemize}
	\end{frame}

	\begin{frame}
		\frametitle{Требования, по ГОСТ}
		Разделы:
		\begin{itemize}
			\item требования к функциональным характеристикам
			\item требования к надежности
			\item условия эксплуатации
			\item требования к составу и параметрам технических средств
			\item требования к информационной и программной совместимости
			\item требования к маркировке и упаковке
			\item требования к транспортированию и хранению
			\item специальные требования
		\end{itemize}
	\end{frame}

	\section{Задание на пару}

	\begin{frame}
		\frametitle{Задание на пару}
		\begin{itemize}
			\item Оформить имеющиеся требования и сведения из устава проекта как ТЗ по ГОСТ 19.201-78
			\begin{itemize}
				\item За исключением нерелевантных разделов
			\end{itemize}
			\item Отдать на рассмотрение соседней команде, задача которой --- играть роль недобросовестного исполнителя
			\begin{itemize}
				\item Составить список ``лазеек'', которые позволят удовлетворить формальным требованиям в ТЗ, делая минимум работы
			\end{itemize}
			\item Исправить выявленные проблемы, доделать дома и выложить на вики проекта
		\end{itemize}
	\end{frame}

\end{document}
