\documentclass[a5paper]{article}
\usepackage[a5paper, top=8mm, bottom=8mm, left=8mm, right=8mm]{geometry}

\usepackage{polyglossia}
\setdefaultlanguage[babelshorthands=true]{russian}

\usepackage{fontspec}
\setmainfont{FreeSerif}
\newfontfamily{\russianfonttt}[Scale=0.7]{DejaVuSansMono}

\usepackage[font=scriptsize]{caption}

\usepackage{amsmath}
\usepackage{amssymb,amsfonts,textcomp}
\usepackage{color}
\usepackage{array}
\usepackage{hhline}
\usepackage{cite}

\usepackage[hang,multiple]{footmisc}
\renewcommand{\footnotelayout}{\raggedright}

\PassOptionsToPackage{hyphens}{url}\usepackage[xetex,linktocpage=true,plainpages=false,pdfpagelabels=false]{hyperref}
\hypersetup{colorlinks=true, linkcolor=blue, citecolor=blue, filecolor=blue, urlcolor=blue, pdftitle=1, pdfauthor=, pdfsubject=, pdfkeywords=}

\usepackage{tabu}

\usepackage{graphicx}
\usepackage{indentfirst}
\usepackage{multirow}
\usepackage{subfig}
\usepackage{footnote}
\usepackage{minted}

\sloppy
\pagestyle{plain}

\title{Jenkins, демонстрация}
\author{Юрий Литвинов\\\small{yurii.litvinov@gmail.com}}

\date{29.11.2017}

\begin{document}

\maketitle
\thispagestyle{empty}

\section{Требования}

Для работы с Jenkins потребуется:
\begin{itemize}
	\item Docker (\url{https://www.docker.com/})
	\item Сам Jenkins (\url{https://jenkins.io})
\end{itemize}

Для начала надо поставить Docker, либо как написано здесь: \url{https://docs.docker.com/engine/installation/linux/docker-ce/ubuntu/}, либо, если у вас Windows, просто скачав и поставив Docker Community Edition.
Правда, под Windows Docker включит Hyper V, после чего перестанут работать все остальные виртуальные машины типа VirtualBox, поэтому если VirtualBox нужен, лучше ставить Docker Toolbox, который то же самое, но
использует VirtualBox для виртуализации. Hyper V, впрочем, можно потом отключить. При установке компьютер захочет перезагрузиться (возможно, пару раз).

Jenkins распространяется как Java .war-файл, так что чтобы его запустить, надо выполнить \verb|java -jar jenkins.war --httpPort=8080| и, когда он проинициализируется, зайти на \url{http://localhost:8080}.
Там он предложит поставить плагины, можно выбрать ``Поставить самые популярные'', а можно и повыбирать плагины руками. Из их названий более-менее понятно, для чего они. Ставится оно под виндой в
C:/Users/<user>/.jenkins.

\section{Docker}

Прежде, чем перейти к Jenkins, поговорим о Docker. Docker --- в общем-то, промышленный стандарт в области Continuous Delivery. Идея Docker заключается в том, чтобы вообще не заниматься установкой приложения
и всех его зависимостей на целевой системе, а просто таскать всё с собой, включая все необходимые библиотеки, настройки и т.д. При этом приложение изолировано от всей остальной системы, так что конфликты версий 
библиотек или файлов конфигурации невозможны в принципе. Docker очень похож на виртуальные машины, но, в отличие от них, не эмулирует аппаратное обеспечение, а предоставляет абстракцию уровня ядра операционной
системы. Так что несколько докер-контейнеров могут использовать одно ядро, нет нужды грузить ``гостевую'' ОС (потому что нет гостевой ОС, есть одна ОС и ``гостевые'' окружения), так что докер-контейнеры работают
быстрее и весят меньше, чем виртуальные машины (зато не могут эмулировать несуществующее железо, как умеют многие виртуальные машины).

Кстати, \textit{образ} (image) --- это само приложение плюс все библиотеки, переменные окружения и конфиги, которые нужны ему для работы, а \textit{контейнер} (container) --- это результат запуска образа 
на выполнение (то есть экземпляр образа в рантайме).

Чтобы попробовать, поставьте докер и наберите в командной строке \verb|docker run hello-world|. Докер не найдёт образ hello-world на локальной машине, полезет на Docker Hub и скачает образ оттуда,
после чего запустит контейнер и выведет сообщение. Всё это практически мгновенно. На линуксе оно захочет sudo, но если добавить своего юзера в группу docker, то можно и без sudo.

Повторим туториал с официального сайта (\url{https://docs.docker.com/get-started}), чтобы посмотреть на докер своими глазами. Там созаётся приложение на питоне, которое будет работать в контейнере. Причём, магия
докера такова, что питон для этого ставить не надо.

Создадим пустую папку, создадим в ней файл Dockerfile вот такого содержания:
\begin{minted}{sh}
# Use an official Python runtime as a parent image
FROM python:2.7-slim

# Set the working directory to /app
WORKDIR /app

# Copy the current directory contents into the container at /app
ADD . /app

# Install any needed packages specified in requirements.txt
RUN pip install --trusted-host pypi.python.org -r requirements.txt

# Make port 80 available to the world outside this container
EXPOSE 80

# Define environment variable
ENV NAME World

# Run app.py when the container launches
CMD ["python", "app.py"]
\end{minted}

Создаём там же requirements.txt такого содержания:
\begin{minted}{text}
Flask
Redis
\end{minted}

Ну и, собственно, приложение, app.py:
\begin{minted}{python}
from flask import Flask
from redis import Redis, RedisError
import os
import socket

# Connect to Redis
redis = Redis(host="redis", db=0, socket_connect_timeout=2, socket_timeout=2)

app = Flask(__name__)

@app.route("/")
def hello():
    try:
        visits = redis.incr("counter")
    except RedisError:
        visits = "<i>cannot connect to Redis, counter disabled</i>"

    html = "<h3>Hello {name}!</h3>" \
            "<b>Hostname:</b> {hostname}<br/>" \
            "<b>Visits:</b> {visits}"
    return html.format(name=os.getenv("NAME", "world"), 
        hostname=socket.gethostname(), 
        visits=visits)

if __name__ == "__main__":
    app.run(host='0.0.0.0', port=80)
\end{minted}

Теперь соберём образ:

\begin{minted}{sh}
docker build -t friendlyhello .
\end{minted}

Проверим, что оно собралось:
\begin{minted}{sh}
docker images
\end{minted}

Должен вывестить список всех докер-образов на машине, в том числе и свежесобранный.

Ну и, наконец, запустим контейнер:

\begin{minted}{sh}
docker run -p 4000:80 friendlyhello
\end{minted}

Обратите внимание, что контейнер думает, что слушает порт 80, но мы замапили его на порт 4000 хостовой системы (-p --- ``publish'', опубликовать порт такой-то как такой-то).
В докерфайле же прописано EXPOSE 80, что говорит контейнеру, что его (внутренний) порт 80 должен быть доступен извне.

Чтобы остановить контейнер, надо нажать Ctrl-C, как оно просит, но если бы всё было так просто, я бы про это не рассказывал. Под виндой надо остановить контейнер явно, командами

\begin{minted}{sh}
docker container ls
...
docker container stop ...
\end{minted}
	
Где ... --- либо CONTAINER ID, либо имя (автоматически сгенерированное) контейнера. На линуксе же, чтобы контейнер работал как демон, надо его запустить с ключём -d (detached).

Теперь давайте выложим это барахло в публичный доступ с помощью команд, очень похожих на git push (и вообще, работа с образами в докере сильно напоминает гит, там есть даже тэги).
Делаем мы это для того, чтобы использовать докер-образы где-нибудь ещё, например, на Travis. Для начала надо завести аккаунт на \url{https://cloud.docker.com/}, запомнить своё
имя пользователя. Логинимся из консоли командой docker login, вводим свой логин и пароль.

Дальше надо создать тэг для своего образа (не обязательно, на самом деле, но очень рекомендуется), например, так:

\begin{minted}{sh}
docker tag friendlyhello yuriilitvinov/get-started:part2
\end{minted}

В тэге сначала идёт имя пользователя, потом имя репозитория, потом тэг текущего образа в репозитории. Можно посмотреть на то, что получилось, командой docker image ls. Теперь 
запушим вновь созданный тэг в репозиторий:

\begin{minted}{sh}
docker push yuriilitvinov/get-started:part2
\end{minted}

Теперь можно пойти на \url{https://cloud.docker.com/} и посмотреть, что там появился новый репозиторий с нашим образом. Теперь на \textbf{любой} машине достаточно набрать
\begin{minted}{sh}
docker run -p 4000:80 yuriilitvinov/get-started:part2
\end{minted}

Правда, контейнеры с виндовыми приложениями так просто на линуксе не заработают, зато скорее всего линуксовые контейнеры будут нормально работать на винде.

\section{Jenkins}

Теперь перейдём к Jenkins. Он тоже управляется с помощью конфигурационного файла, который должен лежать в корне репозитория и называться Jenkinsfile. Для того, чтобы что-то сделать, нам потребуется
гит-репозиторий, потому как Jenkins первое, что делает --- получает из указанного репозитория исходники. Достаточно пустого репозитория на гитхабе (локальный тоже пойдёт, но только если гит запущен как сервер).

Создадим Jenkinsfile, примерно такого содержания:

\begin{minted}{groovy}
pipeline {
    agent any
    stages {
        stage('build') {
            steps {
                bat 'mvn --version'
            }
        }
    }
}
\end{minted}

Да, Jenkins (а точнее, его плагин Pipeline, который, собственно, и будет управлять сборкой) использует синтаксис Groovy, знакомый по Gradle. Попрбуем создать билд на основе этого файла.
Для этого надо запустить Jenkins (командой \verb|java -jar jenkins.war --httpPort=8080|), пойти на \url{localhost:8080} и авторизоваться тем паролем, который вы указали при настройке Jenkins-а.

Дальше жмём на New Item, вводим название пайплайна (любое), выбираем Multibranch Pipeline (чтобы Jenkins следил за ветками и пуллреквестами в репозиторий и собирал их), жмём Ок. Заполняем сведения про билд, 
например, ссылку на репозиторий, жмём save --- и он уже пытается получить репозиторий и собрать. 

Так, как было выше, нам требуется установленный Maven в путях. Более модно не настраивать окружение сборки всякий раз на каждой машине, на которой мы поднимаем Jenkins, а... использовать Docker. Например, тот же
самый билд, но качающий образ с maven-ом прямо в процессе сборки:

\begin{minted}{groovy}
pipeline {
    agent { docker 'maven:3.5.2-jdk-8-slim' }
    stages {
        stage('build') {
            steps {
                bat 'mvn --version'
            }
        }
    }
}
\end{minted}

Под виндой, правда, прямо из коробки не заработает, потому что оно хочет sh и nohup в путях, но оно есть в Cygwin и даже в Git For Windows, так что по идее можно просто подредактировать PATH, указав 
на папки с нужными бинарниками. Но с большой вероятностью не заработает всё равно (потому как большинство плагинов для Jenkins ожидают линуксового окружения), поэтому идеологически правильнее
сам Jenkins запускать в Docker-контейнере, командой (для винды)

\begin{minted}{sh}
docker run ^
  --rm ^
  -u root ^
  -p 8080:8080 ^
  -v jenkins-data:/var/jenkins_home ^
  -v /var/run/docker.sock:/var/run/docker.sock ^
  -v "%HOMEPATH%":/home ^
  jenkinsci/blueocean
\end{minted}

После этого Jenkins придётся активировать ещё раз (обратите внимание, пароль для активации пишется в консоли, если его потерять, придётся освоить исполнение команд в контейнере через docker exec). И придётся 
снова скачать и обновить плагины. Зато теперь он будет работать даже под виндой, думая, что он запущен под линуксом, всякие проблемы с sh и nohup будет решать уже Docker.

Кроме того, докер-образ поставляется с плагином Blue Ocean, это новый суперудобный фронтэнд для Jenkins, который вообще всё сделает за вас. Доступен, например, по ссылке \url{http://localhost:8080/blue}, там
всё интуитивно понятно, очень рекомендую попробовать настроить билд с ним.

\end{document}
