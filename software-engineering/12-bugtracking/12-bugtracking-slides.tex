\documentclass[xetex,mathserif,serif]{beamer}
\usepackage{polyglossia}
\setdefaultlanguage[babelshorthands=true]{russian}
\usepackage{minted}
\usepackage{tabu}

\useoutertheme{infolines}

\usepackage{fontspec}
\setmainfont{FreeSans}
\newfontfamily{\russianfonttt}{FreeSans}

\usepackage{textpos}
\setlength{\TPHorizModule}{1cm}
\setlength{\TPVertModule}{1cm}

\tabulinesep=1.2mm

\newcommand{\attribution}[1] {
	\begin{flushright}\begin{scriptsize}\textcolor{gray}{\textcopyright\; #1}\end{scriptsize}\end{flushright}
}

\title{Занятие 12: багтрекинг}
\author[Юрий Литвинов]{Юрий Литвинов\\\small{\textcolor{gray}{yurii.litvinov@gmail.com}}}
\date{11.10.2018}

\begin{document}

	\frame{\titlepage}

	\section{Задача}

	\begin{frame}
		\frametitle{Задача}
		\begin{itemize}
			\item Включить Github Issues для своего проекта
			\item Завести систему меток, должны быть как минимум:
			\begin{itemize}
				\item тип
				\item серьёзность
				\item приоритет
			\end{itemize}
			\item Написать как минимум 10 воображаемых багов
			\begin{itemize}
				\item В соответствии с рекомендованными практиками
			\end{itemize}
			\item Воображаемо закрыть хотя бы три из них, полностью проведя по жизненному циклу
		\end{itemize}
	\end{frame}

\end{document}
