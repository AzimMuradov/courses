\documentclass[xetex,mathserif,serif]{beamer}
\usepackage{polyglossia}
\setdefaultlanguage[babelshorthands=true]{russian}
\usepackage{minted}
\usepackage{tabu}

\useoutertheme{infolines}

\usepackage{fontspec}
\setmainfont{FreeSans}
\newfontfamily{\russianfonttt}{FreeSans}

\usepackage{textpos}
\setlength{\TPHorizModule}{1cm}
\setlength{\TPVertModule}{1cm}

\setbeamertemplate{blocks}[rounded][shadow=false]

\setbeamercolor*{block title alerted}{fg=red!50!black,bg=red!20}
\setbeamercolor*{block body alerted}{fg=black,bg=red!10}

\tabulinesep=1.2mm

\title{Занятие 3: экстремальное программирование}
\author[Юрий Литвинов]{Юрий Литвинов\\\small{\textcolor{gray}{yurii.litvinov@gmail.com}}}
\date{18.09.2019}

\newcommand{\todo}[1] {
	\begin{center}\textcolor{red}{TODO: #1}\end{center}
}

\newcommand{\DownArrow} {
	\hspace{2cm}\begin{LARGE}$\downarrow$\end{LARGE}
}

\newcommand{\attribution}[1] {
	\begin{flushright}\begin{scriptsize}\textcolor{gray}{\textcopyright\; #1}\end{scriptsize}\end{flushright}
}

\begin{document}

	\frame{\titlepage}

	\section{Задача}

	\begin{frame}
		\frametitle{Задача}
		С использованием методов XP разработать приложение с GUI для игры в блэкджек:
		\begin{itemize}
			\item Цель игры --- набрать большую сумму очков, чем у сдающего, но не больше 21
			\begin{itemize}
				\item Карты от двойки до десятки дают 2 -- 10 очков соответственно
				\item Карты с картинками дают 10 очков
				\item Туз даёт 11 или 1 очко
			\end{itemize}
			\item Партия начинается с того, что каждому игроку сдаётся по 2 карты
			\item На каждом ходу игрок может взять ещё одну карту или пасовать
			\item Если у кого-то больше 21, он проиграл
			\item Если в результате у игроков равное количество очков, ничья
		\end{itemize}
	\end{frame}

	\begin{frame}
		\frametitle{Что делать}
		\begin{itemize}
			\item Разбиться на команды примерно по 4 человека
			\item Завести репозиторий, наладить CI
			\item Описать несколько user stories, выполнить оценку, спланировать ближайший релиз
			\item Реализовать проект на любимом языке программирования
			\begin{itemize}
				\item TDD
				\item Парное программирование
			\end{itemize}
		\end{itemize}
	\end{frame}

\end{document}
