\documentclass[xetex,mathserif,serif]{beamer}
\usepackage{polyglossia}
\setdefaultlanguage[babelshorthands=true]{russian}
\usepackage{minted}
\usepackage{tabu}

\useoutertheme{infolines}

\usepackage{fontspec}
\setmainfont{FreeSans}
\newfontfamily{\russianfonttt}{FreeSans}

\usepackage{textpos}
\setlength{\TPHorizModule}{1cm}
\setlength{\TPVertModule}{1cm}

\setbeamertemplate{blocks}[rounded][shadow=false]

\setbeamercolor*{block title alerted}{fg=red!50!black,bg=red!20}
\setbeamercolor*{block body alerted}{fg=black,bg=red!10}

\tabulinesep=1.2mm

\title{Занятие 13: Customer Relation Management}
\author[Юрий Литвинов]{Юрий Литвинов\\\small{\textcolor{gray}{yurii.litvinov@gmail.com}}}
\date{13.12.2017}

\newcommand{\todo}[1] {
	\begin{center}\textcolor{red}{TODO: #1}\end{center}
}

\newcommand{\DownArrow} {
	\hspace{2cm}\begin{LARGE}$\downarrow$\end{LARGE}
}

\newcommand{\attribution}[1] {
	\begin{flushright}\begin{scriptsize}\textcolor{gray}{\textcopyright\; #1}\end{scriptsize}\end{flushright}
}

\begin{document}

	\frame{\titlepage}

	\section{CRM}

	\begin{frame}
		\frametitle{CRM-системы}
		\textit{CRM-система} --- средство автоматизации взаимодействия с клиентами (и потенциальными клиентами)
		\begin{itemize}
			\item Централизованное хранилище контактов
			\item Отслеживание истории взаимодействия с клиентами
			\item Статистический анализ контактов с клиентами
			\item Анализ эффективности механизмов продвижения
			\item Планирование и управление работой службы продаж
		\end{itemize}
	\end{frame}

	\begin{frame}
		\frametitle{Основные понятия}
		\begin{itemize}
			\item \textit{Интерес} --- анонимный потенциальный клиент, выразивший интерес к продукту, без контактных данных
			\item \textit{Lead} --- потенциальный клиент с контактными данными
			\item \textit{Клиент} --- тот, кто уже пользуется услугами компании
			\begin{itemize}
				\item Сохранение клиентов --- одна из главных целей использования CRM
			\end{itemize}
			\item \textit{Конверсия} --- преобразование lead-а в клиента
		\end{itemize}
	\end{frame}

	\begin{frame}
		\frametitle{Что можно хотеть от CRM-системы}
		\begin{itemize}
			\item Облачная или устанавливаемая CRM
			\item Интеграция с телефонией
			\item Интеграция с электронной почтой, рассылки
			\item Интеграция с social media
			\item Поддержка контрагентов
			\item Импорт данных
			\begin{itemize}
				\item Критично для стоимости внедрения
			\end{itemize}
		\end{itemize}
	\end{frame}

	\section{На паре}

	\begin{frame}
		\frametitle{Пример: Битрикс24}
		\begin{itemize}
			\item \url{https://www.bitrix24.ru}
			\item CRM-система от 1С-Битрикс
			\item Облачная, бесплатна для компаний до 12 сотрудников
			\begin{itemize}
				\item Поэтому идеально подходит для пары
				\item Вообще же CRM-систем десятки, и каждая большая компания хочет свою
			\end{itemize}
		\end{itemize}
	\end{frame}

	\begin{frame}
		\frametitle{Задание на пару}
		\begin{itemize}
			\item Завести для своего проекта аккаунт в CRM-системе
			\item Занести всех сокомандников как сотрудников, указать воображаемую структуру организации
			\item Завести карточки воображаемых компаний-партнёров и/или конкурентов
			\item Завести карточки воображаемых lead-ов и покупателей
			\item Воображаемо заключить контракт минимум с тремя покупателями
		\end{itemize}
	\end{frame}

	\section{Домашка}

	\begin{frame}
		\frametitle{Домашнее задание}
		Подготовить презентацию своего проекта для потенциальных инвесторов
		\begin{itemize}
			\item Оценка рынка и перспектив его развития
			\item Видение и основные фичи проекта
			\begin{itemize}
				\item Позиционирование относительно конкурентов
				\item Категории потребителей
			\end{itemize}
			\item План реализации, календарные сроки, бюджет
			\item План монетизации
			\item План продвижения
			\item Команда
			\item Финансовый план
			\begin{itemize}
				\item Требуемые инвестиции
				\item Return-of-investment
				\item Выход на самоокупаемость
			\end{itemize}
		\end{itemize}
	\end{frame}

\end{document}
