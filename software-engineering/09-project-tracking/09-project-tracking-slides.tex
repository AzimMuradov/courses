\documentclass[xetex,mathserif,serif]{beamer}
\usepackage{polyglossia}
\setdefaultlanguage[babelshorthands=true]{russian}
\usepackage{minted}
\usepackage{tabu}

\useoutertheme{infolines}

\usepackage{fontspec}
\setmainfont{FreeSans}
\newfontfamily{\russianfonttt}{FreeSans}

\usepackage{textpos}
\setlength{\TPHorizModule}{1cm}
\setlength{\TPVertModule}{1cm}

\definecolor{links}{HTML}{2A1B81}
\hypersetup{colorlinks,urlcolor=links}
\hypersetup{linkcolor=}

\tabulinesep=1.2mm

\title{Занятие 9: Отслеживание прогресса}
\author[Юрий Литвинов]{Юрий Литвинов\\\small{\textcolor{gray}{yurii.litvinov@gmail.com}}}
\date{06.11.2019}

\newcommand{\attribution}[1] {
	\begin{flushright}\begin{scriptsize}\textcolor{gray}{\textcopyright\; #1}\end{scriptsize}\end{flushright}
}

\begin{document}

	\frame{\titlepage}

	\section{Задание на пару}

	\begin{frame}
		\frametitle{Задание на пару}
		\begin{itemize}
			\item Считаем, что команды отработали два месяца, столкнулись с первыми сложностями и получили первый feedback. Надо:
			\begin{itemize}
				\item Отметить выполненные задачи на Pivotal Tracker
				\item Обновить план (диаграмму Гантта) с учётом прогресса
				\item Рассчитать текущие показатели проекта:
				\begin{itemize}
					\item Budgeted cost of work performed
					\item Actual cost of work performed
					\item Cost variance
					\item Cost performance index
					\item Estimate budget at completion
				\end{itemize}
				\item Описать change request-ы от заказчика как набор новых требований (отдельным документом) и привести план в соответствие
			\end{itemize}
			\item Доделать дома
		\end{itemize}
	\end{frame}

	\begin{frame}
		\frametitle{ABaroad}
		\begin{itemize}
			\item Опросы потенциальных пользователей выявили, что было бы интересно видеть вручную проложенные маршруты
			\begin{itemize}
				\item с комментариями, по маршруту в целом и по каждому бару
				\item например, тематические маршруты в духе ``Хипстерский маршрут по барам Петербурга''
			\end{itemize}
			\item Выяснилось, что никто в команде не имел дела с iOS, так что разработка iOS-версии затянулась (с плановых 2 до 5 недель)
			\begin{itemize}
				\item и пришлось купить недорогой MacBook для комфортной разработки без проблем с лицензиями
			\end{itemize}
			\item Зато задачу по серверной части удалось завершить раньше срока, за 2 недели
			\item Команда потеряла в общей сложности три рабочих дня в середине второго месяца из-за чрезмерного усердия в тестировании ``вживую'' первого прототипа алгоритма прокладки маршрута 
		\end{itemize}
	\end{frame}

	\begin{frame}
		\frametitle{Attention-Ad!}
		\begin{itemize}
			\item Первые эксперименты с интеграцией рекламы выявили, что в случае с видео задача сильно недооценена --- автоматическое встраивание рекламы очень неточно, а ручное очень трудоёмко
			\begin{itemize}
				\item подготовка первых двух тестовых видео заняла втрое больше времени, чем планировалось
				\item на разработку нормального механизма интеграции рекламы в видео потребуется отдельный человек и минимум три месяца работы
				\begin{itemize}
					\item при этом ещё непонятно, насколько удастся это автоматизировать, возможно, придётся ограничиться просто удобным инструментом ручного редактирования
				\end{itemize}
			\end{itemize}
			\item В ходе обсуждения с потенциальными рекламодателями выяснилось, что хочется механизм управления вознаграждениями для испытуемых
			\begin{itemize}
				\item возможность при создании проекта ``закинуть денег'' и указать желаемое количество испытуемых либо размер вознаграждения для каждого испытуемого
				\item возможность в дальнейшем изменять эти параметры (в том числе, ``закинуть ещё денег'')
			\end{itemize}
		\end{itemize}
	\end{frame}

	\begin{frame}
		\frametitle{Isochrone Map}
		\begin{itemize}
			\item Команда поздно вернулась со стажировок, поэтому старт проекта был отложен на месяц
			\item Пока удалось найти только одного адекватного программиста из нужных трёх, хотя собеседования проводились каждую неделю
			\item ГИБДД планирует изменения в ПДД, вводящие термин ``средство индивидуальной мобильности'', соответственно, возникла потребность отдельно строить достижимую зону для разных типовых видов СИМ (электросамокаты, самокаты, гироскутеры, роликовые коньки и т.д.)
			\begin{itemize}
				\item с учётом максимальной скорости конкретного СИМ и максимальной разрешённой скорости в 20 км/ч при движении в одном потоке с пешеходами
				\item нужна возможность задать запас хода для СИМ с аккумулятором и необходимость обратной дороги
			\end{itemize}
		\end{itemize}
	\end{frame}

	\begin{frame}
		\frametitle{Smart Video Security}
		\begin{itemize}
			\item До сих пор не удалось найти грамотных инженера и разработчика встроенных систем, задачи, связанные с аппаратной частью, не стартовали
			\item Детекция аномалий на тестовых видео пошла лучше, чем ожидалось, удалось добиться удовлетворительной точности в полтора раза быстрее
			\item Инвестор хочет с помощью проекта повысить эффективность продаж своих колёсных роботов, поэтому выбор аппаратной платформы для дронов фиксирован
			\begin{itemize}
				\item ПО роботов инвестора поддерживает управление только в ручном режиме, поэтому необходим поиск решения для автоматической навигации по территории
				\item необходима интеграция с ПО инвестора
				\item необходима разработка и поддержка зарядных станций для роботов
			\end{itemize}
		\end{itemize}
	\end{frame}

	\begin{frame}
		\frametitle{Stolovka}
		\begin{itemize}
			\item Первые эксперименты с онлайн-оплатой показали, что задача недооценена как минимум вдвое из-за всяких организационных моментов (например, необходимость иметь юрлицо для полноценного тестирования)
			\item Однако же в ходе опроса пользователей выяснилось, что кому-то удобно платить карточкой, кому-то --- Google Pay, кому-то Qiwi
			\item Ещё пользователи хотят авторизацию через Google или VK
			\item Задача по работе со сканером оказалась недооценённой --- сканер надо было выбрать, заказать (и его привезли только через неделю), поддержать чтение QR-кодов в приложении для выдачи
		\end{itemize}
	\end{frame}

	\begin{frame}
		\frametitle{UniOn Line}
		\begin{itemize}
			\item Команде удалось быстро найти первого заказчика, который готов покрыть все расходы на разработку, но с ним появились и новые требования:
			\begin{itemize}
				\item возможность импорта материалов из существующей системы управления контентом заказчика (система онлайн-обучения Blackboard, с доступом по WebDav)
				\item разделение курса как набора материалов и прочтения курса как конкретных занятий с конкретными датами, сопровождаемых материалами из курса
				\item возможность назначать для материалов и тестов время от начала прочтения, в которое они становятся доступны (например, ``провести тест после трёх занятий от начала'')
			\end{itemize}
			\item Команда приступила к найму сотрудников, однако быстро выяснилось, что набрать аж 11 квалифицированных программистов за месяц нереально (хотя бы потому, что никто не готов выйти на работу раньше двух недель с момента получения оффера)
		\end{itemize}
	\end{frame}

	\begin{frame}
		\frametitle{ВЕГИУС ``Знание''}
		\begin{itemize}
			\item Активное обсуждение с преподавателями выявило новые требования:
			\begin{itemize}
				\item возможность назначить проверяющих на курс как отдельную роль
				\begin{itemize}
					\item могут выставлять оценки за домашки
					\item могут резервировать домашку для проверки
					\item не могут выкладывать и редактировать учебные материалы
				\end{itemize}
				\item Отдельный сервис для работы с курсовыми и дипломами
				\begin{itemize}
					\item заявка тем, matchmaking, выкладывание отчётов, рецензирование, учёт результатов защиты, публикация материалов
				\end{itemize}
			\end{itemize}
			\item Брыксин Тимофей Александрович попросил втрое большую зарплату
		\end{itemize}
	\end{frame}

	\begin{frame}
		\frametitle{Трансферный помощник}
		\begin{itemize}
			\item Быстро выяснилось, что с открытыми данными всё хорошо, поскольку существует масса сайтов с подробными логами матчей
			\begin{itemize}
				\item Задачи исследования наличия информации и сбора данных оказались переоценены вдвое
			\end{itemize}
			\item Однако первый прототип предсказывающей модели не выявил корреляции между результативностью игрока и клубом, в котором он играет
			\item Инвестор хочет работать на американском рынке, поэтому требует поддержки и баскетбола
		\end{itemize}
	\end{frame}

	\begin{frame}
		\frametitle{Ятрофобия}
		\begin{itemize}
			\item Захайрить кого-либо за две недели оказалось невозможным (как минимум, из-за two weeks notice), найм персонала затянулся втрое и найти удалось только двух программистов из четырёх
			\item Дизайн архитектуры затянулся на неделю из-за болезни архитектора
			\item Опрос потенциальных пользователей показал, что возможность заносить посещения в календарь и редактировать их не будет востребована
			\begin{itemize}
				\item К предсказанию посещений пользователи относятся скептически, а вводить посещения ``просто так'' всем действительно лень
			\end{itemize}
		\end{itemize}
	\end{frame}

\end{document}
