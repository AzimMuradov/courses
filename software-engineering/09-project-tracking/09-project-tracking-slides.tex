\documentclass[xetex,mathserif,serif]{beamer}
\usepackage{polyglossia}
\setdefaultlanguage[babelshorthands=true]{russian}
\usepackage{minted}
\usepackage{tabu}

\useoutertheme{infolines}

\usepackage{fontspec}
\setmainfont{FreeSans}
\newfontfamily{\russianfonttt}{FreeSans}

\usepackage{textpos}
\setlength{\TPHorizModule}{1cm}
\setlength{\TPVertModule}{1cm}

\definecolor{links}{HTML}{2A1B81}
\hypersetup{colorlinks,urlcolor=links}
\hypersetup{linkcolor=}

\tabulinesep=1.2mm

\title{Занятие 9: Отслеживание прогресса}
\author[Юрий Литвинов]{Юрий Литвинов\\\small{\textcolor{gray}{yurii.litvinov@gmail.com}}}
\date{02.10.2018}

\newcommand{\attribution}[1] {
	\begin{flushright}\begin{scriptsize}\textcolor{gray}{\textcopyright\; #1}\end{scriptsize}\end{flushright}
}

\begin{document}

	\frame{\titlepage}

	\section{Задание на пару}

	\begin{frame}
		\frametitle{Задание на пару}
		\begin{itemize}
			\item Считаем, что команды отработали полтора месяца, столкнулись с первыми сложностями и получили первый feedback. Надо:
			\begin{itemize}
				\item Отметить выполненные задачи на Pivotal Tracker
				\item Обновить план (диаграмму Гантта) с учётом прогресса
				\item Рассчитать текущие показатели проекта:
				\begin{itemize}
					\item Budgeted cost of work performed
					\item Actual cost of work performed
					\item Cost variance
					\item Cost performance index
					\item Estimate budget at completion
				\end{itemize}
				\item Описать change request-ы от заказчика как набор новых требований (отдельным документом) и привести план в соответствие
			\end{itemize}
			\item Доделать дома
		\end{itemize}
	\end{frame}

	\begin{frame}
		\frametitle{Крембрюле}
		\begin{itemize}
			\item Фронтенд-разработчик позднее расчётного времени приехал со стажировки в Google, прогресса по фронтенду не было две недели
			\item Общение с потенциальными пользователями показало, что приложению необходима функциональность по согласованию времени экскурсии
			\begin{itemize}
				\item желающие записываются на экскурсию и предлагают удобное им время
				\item экскурсовод видит список желающих и назначает своё время
				\item всем рассылаются нотификации, после чего те, кому не удобно, отписываются от экскурсии
			\end{itemize}
			\item Оказалось, что задачи несколько недооценены, разработка схемы БД заняла целую неделю, а выбранная frontend-библиотека не поддерживает Internet Explorer
		\end{itemize}
	\end{frame}

	\begin{frame}
		\frametitle{Trapezium}
		\begin{itemize}
			\item Демонстрация первого прототипа выявила потребность в:
			\begin{itemize}
				\item авторизации по OAuth 2 через Google, VK, GitHub
				\item некоторой переработке пользовательского интерфейса (уменьшение элементов управления, режим ``дзен'')
				\item функциональности словаря/переводчика, интегрированной в читалку
				\begin{itemize}
					\item ``как в Google Play Books'', говорят пользователи
				\end{itemize}
			\end{itemize}
			\item Не удалось найти ни одного аналитика, согласного работать над проектом две недели, а потом быть уволенным за ненадобностью
			\item Соответственно, разработка документации затянулась
		\end{itemize}
	\end{frame}

	\begin{frame}
		\frametitle{VieR Design}
		\begin{itemize}
			\item Опрошенные пользователи указали на возможность удобной интеграции с Google Tolt Brush для рисования произвольных 3d-объектов для интерьера
			\begin{itemize}
				\item Инвестор счёл это очень важной фичей, которую надо реализовать на одной из первых итераций
			\end{itemize}
			\item План, при котором первая работоспособная версия начнёт появляться только на 20-25 неделях, инвестор считает слишком рискованным
			\begin{itemize}
				\item Первый концепт-пруф должен быть готов за два месяца
			\end{itemize}
			\item Возникли сложности в поиске грамотного инженера, так и не удалось пока найти подходящего
		\end{itemize}
	\end{frame}

	\begin{frame}
		\frametitle{Quests}
		\begin{itemize}
			\item Попытки договориться с организаторами городских квестов привели к пониманию, что их мало и они не очень расположены сотрудничать
			\begin{itemize}
				\item Инвестор вспомнил про функциональность автогенерации городских квестов и хочет её к первому релизу
				\item Автогенерированный квест должен быть интересным и вести по красивым местам города
			\end{itemize}
			\item Архитектуру мобильных клиентов удалось разработать и описать быстрее, чем ожидалось --- за два дня, серверной части --- за три дня
		\end{itemize}
	\end{frame}

	\begin{frame}
		\frametitle{CryptoGlass}
		\begin{itemize}
			\item При общении с трейдерами выяснилось, что с Python многие не знакомы, им приятнее писать стратегии на JavaScript или чём-то наподобие Solidity
			\item Хаос на криптовалютных рынках привёл к осознанию того, что стратегии должны работать очень быстро, чтобы успевать реагировать на быстро меняющуюся ситуацию
			\item Один из разработчиков не выдержал и ушёл в запой на неделю
		\end{itemize}
	\end{frame}

	\begin{frame}
		\frametitle{wall-et}
		\begin{itemize}
			\item Удалось договориться с ровно одним мусороперерабатывающим заводом и одной розничной торговой сетью, причём на условиях демо-эксплуатации, такой, что до достижения определённых объёмов собираемого мусора пользование приложением для них бесплатно
			\begin{itemize}
				\item Обязательным условием сотрудничества розничная сеть назвала наличие API для актуализации данных о товарах и разработку за счёт команды модуля синхронизации для этой сети
				\item Команде дали схему БД и доступ на чтение к реальной базе товаров сети
			\end{itemize}
			\item Не удалось найти инженера, который мог бы разработать прототип урны. На собеседование приходили люди, которые могут либо промышленный дизайн, либо разводку плат, но не всё сразу. 
		\end{itemize}
	\end{frame}

	\begin{frame}
		\frametitle{Plans.net}
		\begin{itemize}
			\item Пришлось отменить запланированную для компенсации риска третьей мировой войны постройку бункера --- инвестор отказался выделить на это деньги
			\item Предварительный этап проекта --- аренду и оборудование офиса, планирование --- удалось выполнить всего за неделю (план был уже написан в качестве домашки в университете, офис удалось снять сразу со всем необходимым)
			\item Инвестор хочет провести маркетинговое исследование для снятия финансовых рисков, для чего ему нужен прототип, позволяющий просматривать планы
			\begin{itemize}
				\item Уже через две недели
				\item С контекстной рекламой
			\end{itemize}
			\item Единственный сотрудник, умеющий хорошо проектировать БД, впал в депрессию и его продуктивность сократилась вдвое
		\end{itemize}
	\end{frame}

	\begin{frame}
		\frametitle{ExCon}
		\begin{itemize}
			\item Не удалось убедить инвестора в потенциальной доходности проекта. Инвестор полагает, что монетизация за счёт контекстной рекламы и доли от платной проверки не покроет расходы на разработку в разумное (для него) время.
			\begin{itemize}
				\item Но всё-таки он готов попробовать --- выделить 3000000 руб., а не 5760000 запрошенных
				\item Инвестор хочет войти в IT-образование для школьников, поэтому ему нужна функциональность автоматической проверки задач по программированию на языках, рекомендованных для ЕГЭ
			\end{itemize}
			\item Подготовительный этап (закупка оборудования, аренда офиса, планирование, уже с учётом урезанного финансирования) успели выполнить за две недели.
		\end{itemize}
	\end{frame}

	\begin{frame}
		\frametitle{Поиск по мемам}
		\begin{itemize}
			\item По результатам экспериментов с распознаванием изображений выяснилось, что задача не так проста, как кажется --- добиться устойчивого поиска по картинкам всего, кроме котиков, не удалось. Некоторые мемы с котиками тоже не находятся.
			\item Первый прототип показал, что поиск по мемам не настолько привлекает людей, чтобы всерьёз зарабатывать на рекламе и донате
			\begin{itemize}
				\item Инвестор предполагает продать проект крупному поисковику или соцсети, так что приоритеты разработки меняются соответственно
				\item Тем не менее, мобильные приложения было решено оставить и сделать платными
			\end{itemize}
		\end{itemize}
	\end{frame}

\end{document}
