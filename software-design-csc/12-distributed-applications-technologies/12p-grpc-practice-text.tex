\documentclass[a5paper]{article}
\usepackage[a5paper, top=8mm, bottom=8mm, left=8mm, right=8mm]{geometry}

\usepackage{polyglossia}
\setdefaultlanguage[babelshorthands=true]{russian}

\usepackage{fontspec}
\setmainfont{FreeSerif}
\newfontfamily{\russianfonttt}[Scale=0.7]{DejaVuSansMono}

\usepackage[font=scriptsize]{caption}

\usepackage{amsmath}
\usepackage{amssymb,amsfonts,textcomp}
\usepackage{color}
\usepackage{array}
\usepackage{hhline}
\usepackage{cite}
\usepackage{verse}
\usepackage{xcolor}

\usepackage[hang,multiple]{footmisc}
\renewcommand{\footnotelayout}{\raggedright}

\PassOptionsToPackage{hyphens}{url}\usepackage[xetex,linktocpage=true,plainpages=false,pdfpagelabels=false]{hyperref}
\hypersetup{colorlinks=true, linkcolor=blue, citecolor=blue, filecolor=blue, urlcolor=blue, pdftitle=1, pdfauthor=, pdfsubject=, pdfkeywords=}

\usepackage{tabu}

\usepackage{graphicx}
\usepackage{indentfirst}
\usepackage{multirow}
\usepackage{subfig}
\usepackage{footnote}
\usepackage{minted}

\newcommand{\attribution}[1] {
    \vspace{-5mm}\begin{flushright}\begin{scriptsize}\textcolor{gray}{\textcopyright\, #1}\end{scriptsize}\end{flushright}
}

\sloppy
\pagestyle{plain}

\title{Практика 12: Сетевой чат на gRPC}
\author{Юрий Литвинов\\\small{yurii.litvinov@gmail.com}}
\date{25.04.2022}

\begin{document}

\maketitle
\thispagestyle{empty}

\section{Задание на практику}

В командах по два человека разработать сетевой чат (наподобие Telegram) с помощью gRPC. При этом:

\begin{itemize}
    \item оно должно работать как peer-to-peer, то есть соединение напрямую, без всяких серверов;
    \begin{itemize}
        \item и клиент, и сервер должны быть одним и тем же приложением, работающим в разных режимах;
        \item если приложение запускается с указанием только порта, оно становится сервером;
        \item если IP и порта, то клиентом;
        \item порт можно зафиксировать в коде и не просить у пользователя;
    \end{itemize}
    \item консольный пользовательский интерфейс;
    \item отображение имени отправителя, даты отправки и текста сообщения;
    \item при запуске указываются:
    \begin{itemize}
        \item адрес peer-а и порт, если хотим подключиться;
        \begin{itemize}
            \item должно быть можно не указывать, тогда работаем в режиме сервера;
        \end{itemize}
        \item своё имя пользователя.
    \end{itemize}
\end{itemize}

Реализация допустима на любом языке программирования из поддержанных gRPC.

Сдавать надо, приложив ссылку на пуллреквест из отдельной ветки в репозиторий одного из членов команды.

\end{document}