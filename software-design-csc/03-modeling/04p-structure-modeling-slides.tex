\documentclass[xetex,mathserif,serif]{beamer}
\usepackage{polyglossia}
\setdefaultlanguage[babelshorthands=true]{russian}
\usepackage{minted}
\usepackage{tabu}

\useoutertheme{infolines}

\usepackage{fontspec}
\setmainfont{FreeSans}
\newfontfamily{\russianfonttt}{FreeSans}

\setbeamertemplate{blocks}[rounded][shadow=false]

\definecolor{links}{HTML}{2A1B81}
\hypersetup{colorlinks,linkcolor=,urlcolor=links}

\tabulinesep=0.7mm

\newcommand{\attribution}[1] {
    \vspace{-5mm}\begin{flushright}\begin{scriptsize}\textcolor{gray}{\textcopyright\, #1}\end{scriptsize}\end{flushright}
}

\title{Практика 3: моделирование структуры}
\author[Юрий Литвинов]{Юрий Литвинов \newline \textcolor{gray}{\small\texttt{yurii.litvinov@gmail.com}}}

\date{21.02.2022}

\begin{document}
    
    \frame{\titlepage}

    \section{Задания}

    \begin{frame}
        \frametitle{Задание на пару}
        Проанализировать запрос \url{https://bit.ly/defects-rfp}, построить по нему:
        \begin{enumerate}
            \item диаграмму компонентов требуемой системы, как вы её видите
            \item диаграммы классов --- по одной на каждый компонент
            \begin{itemize}
                \item надо только основные классы, технические детали не интересны
                \item не забудьте про связь с другими компонентами --- рисуйте классы из соседних компонентов с квалифицированными именами и без атрибутов/операций
            \end{itemize}
        \end{enumerate}
        \vspace{3mm}
        \begin{itemize}
            \item в конце пары разберём одно-два решения
            \item решения выложите в репозиторий с домашками в виде пуллреквеста из отдельной ветки
            \begin{itemize}
                \item даже если это просто .md-файл со ссылкой на проект в облачном редакторе
            \end{itemize}
        \end{itemize}
    \end{frame}

\end{document}