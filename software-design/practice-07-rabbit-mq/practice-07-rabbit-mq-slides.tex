\documentclass{../../slides-style}

\texttitle{Распределённые приложения, RabbitMQ}
\author[Юрий Литвинов]{Юрий Литвинов \newline \textcolor{gray}{\small\texttt{yurii.litvinov@gmail.com}}}

\begin{document}

	\begin{frame}[plain]
        \titlepage
    \end{frame}

	\section{RabbitMQ}

	\begin{frame}
		\frametitle{Задача на пару}
		Переделать сетевой чат на RabbitMQ
		\begin{itemize}
			\item Консольное приложение
			\item Сервер для обмена сообщениями, о котором договариваются клиенты
			\begin{itemize}
				\item Центральный сервер, задаваемый при запуске (127.0.0.1 по умолчанию)
			\end{itemize}
			\item При запуске указываем имя пользователя и канал
			\begin{itemize}
				\item Подписка на несуществующий канал должна его создавать
			\end{itemize}
			\item Нет истории, получать только те сообщения, что были опубликованы с момента подключения
			\begin{itemize}
				\item Может помочь \url{https://www.rabbitmq.com/tutorials/tutorial-three-python.html}
			\end{itemize}
		\end{itemize}
	\end{frame}

\end{document}