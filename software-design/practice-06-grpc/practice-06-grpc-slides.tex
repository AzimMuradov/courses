\documentclass[xetex,mathserif,serif]{beamer}
\usepackage{polyglossia}
\setdefaultlanguage[babelshorthands=true]{russian}
\usepackage{minted}
\usepackage{tabu}

\useoutertheme{infolines}

\usepackage{fontspec}
\setmainfont{FreeSans}
\newfontfamily{\russianfonttt}{FreeSans}

\definecolor{links}{HTML}{2A1B81}
\hypersetup{colorlinks,linkcolor=,urlcolor=links}

\tabulinesep=0.7mm

\title{Распределённые приложения, gRPC}
\author[Юрий Литвинов]{Юрий Литвинов \newline \textcolor{gray}{\small\texttt{yurii.litvinov@gmail.com}}}

\date{06.12.2019г}

\begin{document}
	
	\frame{\titlepage}

	\section{Задача}

	\begin{frame}
		\frametitle{Задание на пару}
		В командах по два человека разработать сетевой чат (наподобие Telegram) с помощью gRPC
		\begin{itemize}
			\item peer-to-peer, то есть соединение напрямую
			\item Консольный интерфейс
			\begin{itemize}
				\item Отображение имени отправителя, времени отправки и текста сообщения
			\end{itemize}
			\item При запуске указываются:
			\begin{itemize}
				\item Адрес peer-а и порт, если хотим подключиться
				\begin{itemize}
					\item Должно быть можно не указывать, тогда работаем в режиме сервера
				\end{itemize}
				\item Своё имя пользователя
			\end{itemize}
		\end{itemize}
	\end{frame}

\end{document}