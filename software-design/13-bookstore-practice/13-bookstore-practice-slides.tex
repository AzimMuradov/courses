\documentclass{../text-style}

\texttitle{Практика по проектированию}
\subtitle{Магазин книг}
\author[Юрий Литвинов]{Юрий Литвинов\\\small{\textcolor{gray}{y.litvinov@spbu.ru}}}

\begin{document}
	
	\begin{frame}[plain]
        \titlepage
    \end{frame}

	\begin{frame}
		\frametitle{Задание на пару: Магазин книг}
		В командах по 2-3 человека выполнить анализ предметной области и построить модель в виде диаграммы классов для интернет-магазина книг по следующему ТЗ:
		\begin{itemize}
			\item \url{https://goo.gl/94LyFc}
		\end{itemize}

		Обратите внимание, что это должна быть модель предметной области в соответствии с принципами DDD, детали реализации наподобие способа хранения информации в базе данных не важны.

		Будет оцениваться точность следования ТЗ, соответствие модели сущностям предметной области и, естественно, пунктуальность в следовании синтаксису UML.
	\end{frame}

\end{document}
