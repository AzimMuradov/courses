\documentclass{../../slides-style}

\slidetitleext{Событийно-ориентированное программирование, практика}{21.03.2025}{Событийно-ориентированное программирование}

\begin{document}

    \begin{frame}[plain]
        \titlepage
    \end{frame}

    \begin{frame}{Задача}
        \begin{itemize}
            \item Реализовать консольное приложение, позволяющее управлять персонажем, перемещающимся по карте
            \item Карта состоит из свободного пространства и стен, и должна грузиться из файла
            \item Приложение должно отображать карту и персонажа (символом \enquote{\mintinline{text}{@}}) в окне консоли, и позволять персонажу перемещаться по карте, реагируя на клавиши управления курсором
            \begin{itemize}
                \item Будут полезны свойства Console.CursorLeft и Console.CursorTop 
                \item Каждый раз перерисовывать всю карту нельзя
            \end{itemize}
            \item Нужны тесты
            \item Задача командная, на команду из 2-3 человек
        \end{itemize}
    \end{frame}

\end{document}
