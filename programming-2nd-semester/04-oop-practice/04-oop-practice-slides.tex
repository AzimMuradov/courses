\documentclass{../../slides-style}

\slidetitle{Практика по ООП, стековый калькулятор}{21.02.2025}

\begin{document}

    \begin{frame}[plain]
        \titlepage
    \end{frame}

    \begin{frame}
        \frametitle{Задача}
        Реализовать стековый калькулятор (класс, реализующий выполнение операций +, -, *, / над арифметическим выражением в виде строки в постфиксной записи).
        \begin{itemize}
            \item Строка уже дана в обратной польской записи (например, 1 2 3 + *)
            \item Числа и арифметические знаки разделены пробелами, числа только целые (но могут быть знаковыми, и уж точно не только из одной цифры)
            \item Стек реализовать двумя способами (например, массивом или списком) в двух разных классах на основе одного интерфейса
            \item Стековый калькулятор должен знать только про интерфейс стека
        \end{itemize} 
        
        В результате должно получаться число — результат вычислений. 
        \begin{enumerate}
            \item Результат может быть дробным
            \item При попытке деления на 0 должна выдаваться ошибка и программа должна корректно заканчивать работу
        \end{enumerate}
    \end{frame}

\end{document}