\documentclass{../../slides-style}

\slidetitle[Windows Forms]{Пользовательский интерфейс}{04.04.2025}

\begin{document}

    \begin{frame}[plain]
        \titlepage
    \end{frame}

    \begin{frame}
        \frametitle{Windows Forms}
        \begin{itemize}
            \item Самая старая из GUI-библиотек для .NET
            \item Событийно-ориентированная
            \item Простая
            \item Давно не развивается, но поддерживается и используется до сих пор
        \end{itemize}
    \end{frame}

    \begin{frame}[fragile]
        \frametitle{Простой пример}
        \begin{scriptsize}
            \begin{minted}{csharp}
public class MyForm : Form {
    private Button button1;

    public MyForm()
    {
        button1 = new Button();
        button1.Size = new Size(40, 40);
        button1.Location = new Point(30, 30);
        button1.Text = "Click me";
        this.Controls.Add(button1);
        button1.Click += new EventHandler(Button1Click);
    }

    private void Button1Click(object sender, EventArgs e)
        => MessageBox.Show("Hello World");

    [STAThread]
    static void Main()
    {
        Application.EnableVisualStyles();
        Application.Run(new MyForm());
    }
}
            \end{minted}
        \end{scriptsize}
    \end{frame}

    \begin{frame}
        \frametitle{Control}
        \begin{itemize}
            \item Корень иерархии элементов управления
            \item Отвечает за пользовательский ввод, события, позицию на экране
            \begin{itemize}
                \item В том числе, информацию для лейаутов
            \end{itemize}
            \item Ambient property --- свойство, наследуемое от родителя
            \begin{itemize}
                \item Cursor, Font, BackColor, ForeColor и RightToLeft
            \end{itemize}
            \item Controls --- коллекция дочерних контролов
        \end{itemize}
    \end{frame}

    \begin{frame}[fragile]
        \frametitle{Обработка пользовательского ввода}
        \begin{itemize}
            \item События с клавиатуры
            \begin{itemize}
                \item KeyDown, KeyPress, KeyUp
                \item PreProcessMessage, WndProc
            \end{itemize}
            \begin{scriptsize}
                \begin{minted}{csharp}
private void textBox1_KeyDown(object sender, KeyEventArgs e)
{
    if (e.KeyCode == Keys.F1 && (e.Alt || e.Control || e.Shift))
    {
        Help.ShowPopup(textBox1, "Enter your name.", 
            new Point(textBox1.Bottom, textBox1.Right));
    }
}
                \end{minted}
            \end{scriptsize}
            \item События мыши
            \begin{itemize}
                \item MouseDown, Click, MouseClick, MouseUp
                \item Или MouseDown, Click, MouseClick, MouseUp, MouseDown, DoubleClick, MouseDoubleClick, MouseUp
            \end{itemize}
        \end{itemize}
    \end{frame}

    \begin{frame}{Демонстрация}
        \begin{Huge}\begin{center}Демонстрация\end{center}\end{Huge}
    \end{frame}

    \begin{frame}
        \frametitle{Best practices}
        \begin{itemize}
            \item Отделяйте логику работы программы от представления
            \begin{itemize}
                \item Писать логику прямо в обработчиках совсем плохо
                \item Уровневая архитектура
                \item Model-View-Presenter
                \begin{itemize}
                    \item Может быть overkill-ом для небольших проектов
                \end{itemize}
            \end{itemize}
            \item Форма должна адекватно вести себя при ресайзе
            \item Пользовательский интерфейс не должен ломать привычек пользователя
            \item Самые часто используемые элементы интерфейса должны находиться ближе всего к рабочей области
            \item Не следует думать, что пользователи будут читать документацию
        \end{itemize}
    \end{frame}

    \begin{frame}
        \frametitle{Валидация ввода}
        \begin{itemize}
            \item Простой способ --- MaskedTextBox
            \item ``Правильный'' способ --- событие \textit{Validating}
            \begin{itemize}
                \item CancelEventArgs, свойство Cancel
                \item Последовательность событий при потере фокуса: Leave, Validating, Validated, LostFocus
                \item Свойство AutoValidate
                \item Можно инициировать вручную, методом Validate или ValidateChildren
            \end{itemize}
        \end{itemize}
    \end{frame}

    \begin{frame}[fragile]
        \frametitle{Data binding}
        \begin{minted}{csharp}
public partial class Form1 : Form
{
    public string MyText { get; set; } = "Click me";

    public Form1()
    {
        InitializeComponent();
        button1.DataBindings.Add("Text", this, "MyText");
    }
}
        \end{minted}
    \end{frame}

\end{document}
