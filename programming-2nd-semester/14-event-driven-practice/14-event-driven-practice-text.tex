\documentclass{../../text-style}

\texttitle{Событийно-ориентированное программирование, практика}

\begin{document}

\maketitle
\thispagestyle{empty}

\section{Задача на практику}

На базе класса, генерирующего события по нажатию на клавиши управления курсором (EventLoop с пары), реализовать консольное приложение, позволяющее управлять персонажем, перемещающимся по карте. Карта состоит из свободного пространства и стен, и должна грузиться из файла. Приложение должно отображать карту и персонажа (символом \enquote{\mintinline{text}{@}}) в окне консоли, и позволять персонажу перемещаться по карте, реагируя на клавиши управления курсором. Будут полезны свойства \mintinline{csharp}{Console.CursorLeft} и \mintinline{csharp}{Console.CursorTop}. Каждый раз перерисовывать всю карту нельзя.

Обратите внимание, для данной задачи, как и для остальных, обязательны юнит-тесты, однако использовать функции управления устройством \enquote{Консоль} (такие как Console.CursorLeft и Console.CursorTop) из юнит-тестов не получится. Подумайте, как применить знания про лямбда-функции, чтобы тесты не пытались делать то, чего делать не могут, но в \enquote{боевом} режиме всё работало.

\end{document}
