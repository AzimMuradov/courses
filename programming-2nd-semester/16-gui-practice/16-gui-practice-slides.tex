\documentclass{../../slides-style}

\slidetitle{Пользовательский интерфейс, практика}{04.04.2025}

\begin{document}

    \begin{frame}[plain]
        \titlepage
    \end{frame}

    \section{Рисование на формах}

    \begin{frame}{Рисование на формах}
        \begin{itemize}
            \item Graphics Device Interface (GDI+) --- часть WinAPI, отвечающая за рисование на контролах
            \begin{itemize}
                \item Используется .NET на Windows и реализован отдельно на Linux
            \end{itemize}
            \item Событие Paint
            \item Graphics --- канва для рисования
            \begin{itemize}
                \item Можно получить из PaintEventArgs
                \item Можно создать самим, Control.CreateGraphics --- чтобы рисовать не по событию перерисовки, а когда надо
            \end{itemize}
        \end{itemize}
    \end{frame}

    \begin{frame}[fragile]{Пример}
        \begin{minted}{csharp}
private void ControlPaint(object sender, 
    System.Windows.Forms.PaintEventArgs e)
{
    var font = new Font("Arial", 10);

    e.Graphics.DrawString("This is a diagonal line drawn on the control",
        font, System.Drawing.Brushes.Blue, new Point(30, 30));

    e.Graphics.DrawLine(
        System.Drawing.Pens.Red,
        control.Left,
        control.Top,
        control.Right,
        control.Bottom);
}
        \end{minted}
    \end{frame}

    \begin{frame}{Что ещё бывает}
        \begin{itemize}
            \item Brush --- стиль заливки (абстрактный класс), хранит цвет
            \begin{itemize}
                \item Бывает SolidBrush, TextureBrush, LinearGradientBrush и т.п.
                \item Цвет --- структура Color в формате ARGB
            \end{itemize}
            \item Pen --- стиль линий
            \item Bitmap --- изображения и иконки
            \item Свойство DoubleBuffered --- более плавное (по идее) обновление
            \item Манипуляции с системой координат: ScaleTransform, RotateTransform, TranslateTransform
            \begin{itemize}
                \item Если у вас в коде отрисовки встречаются тригонометрические функции --- скорее всего, вы что-то делаете не так
            \end{itemize}
        \end{itemize}
    \end{frame}

    \begin{frame}[fragile]{Пример}
        \begin{minted}{csharp}
var rect = new Rectangle(0, 0, 50, 50);
var pen = new Pen(Color.FromArgb(128, 200, 0, 200), 2);
e.Graphics.ResetTransform();
e.Graphics.ScaleTransform(1.75f, 0.5f);
e.Graphics.RotateTransform(28, MatrixOrder.Append);
e.Graphics.TranslateTransform(150, 150, MatrixOrder.Append);
e.Graphics.DrawRectangle(pen, rect);
        \end{minted}
    \end{frame}

    \section{Задача}

    \begin{frame}{Задача на остаток пары}
        \begin{itemize}
            \item Вспомнить игру с прошлой практики, собраться в команды
            \item Реализовать вывод карты и персонажа на форму, в любом удобном виде
            \begin{itemize}
                \item Предпочтительно, цветными квадратами/кругами
            \end{itemize}
            \item Сделать на форме кнопки, управляющие перемещением персонажа
            \begin{itemize}
                \item Вверх-вниз-влево-вправо
                \item Используйте лейауты для позиционирования кнопок
            \end{itemize}
            \item Если успеем, поддержать и управление с клавиатуры
        \end{itemize}
    \end{frame}

\end{document}
