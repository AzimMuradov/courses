\documentclass{../../slides-style}

\slidetitle{Введение в Linux, часть 2}{07.03.2025}

\begin{document}
    
    \begin{frame}[plain]
        \titlepage
    \end{frame}

    \section{Файловая система}

    \begin{frame}{Отвлечение про файловую систему}
        \begin{itemize}
            \item Файловая система суть набор inode (индексный дескриптор)
            \item Посмотреть можно ls -i
            \item Режим доступа: Read, Write, Execute, для владельца, группы, всех
            \begin{itemize}
                \item chmod
            \end{itemize}
            \item Имена файлов живут отдельно от inode-ов
            \item Что-то вроде сборки мусора --- если ни одно имя не указывает на inode, его удаляют
            \item Символьная ссылка --- это путь к файлу, а не inode
            \item Файлы вовсе не обязательно лежат на диске
            \begin{itemize}
                \item Всё, что поддерживает потоковый ввод-вывод --- файл!
                \item /dev/null, /dev/random, /proc/...
            \end{itemize}
        \end{itemize}
    \end{frame}

    \begin{frame}{Filesystem Hierarchy Standard}
        \begin{itemize}
            \item Бывают изменяемые и статичные файлы, и разделяемые и неразделяемые
            \begin{itemize}
                \item Разные условия доступа по сети и бэкапов
                \item Следовательно, должны лежать в разных директориях
            \end{itemize}
            \item Корневая файловая система --- только самое необходимое для загрузки и восстановления системы, к ней подмонтируются другие директории
            \begin{itemize}
                \item Никто не мешает им быть на одном разделе с корневой
                \item Могут быть на других дисках или вообще в сети
            \end{itemize}
            \item Всё не критичное для системы барахло --- в /usr
            \item Запрещено создавать файлы и директории в корне
        \end{itemize}
    \end{frame}

    \begin{frame}{Стандартные директории}
        \begin{itemize}
            \item /bin --- нужные программы (в т.ч. для пользователя)
            \item /boot --- файлы загрузчика
            \item /dev --- файлы устройств (они не лежат на диске, если что)
            \item /etc --- конфиги (не разделяемые)
            \item /lib --- библиотеки и модули ядра
            \item /media --- сюда монтируются внешние носители типа флешек
            \item /mnt --- директория для временного монтирования
            \item /opt --- директория для дополнительных пакетов (обычно проприетарных)
            \item /sbin --- нужные программы (для администратора)
            \item /tmp --- временные файлы
            \item /usr --- вторичная иерархия (примерно та же иерархия, но из некритичных файлов)
            \begin{itemize}
                \item Сейчас популярен \enquote{joined root}, когда bin -> /usr/bin, а /sbin -> /usr/sbin
            \end{itemize}
            \item /var --- изменяемые данные
            \item \dots
        \end{itemize}
    \end{frame}

    \section{Работа с консолью --- 2}

    \begin{frame}{Немного более продвинутая работа с консолью}
        \begin{itemize}
            \item wildcards (globs) --- \mintinline{text}{file[1-8]}, \mintinline{text}{file*}, есть ещё \mintinline{text}{?}, \mintinline{text}{[!]}
            \begin{itemize}
                \item Это те самые глобы из .gitignore
                \item Они раскрываются \emph{до} вызова команды
            \end{itemize}
            \item Сильное и слабое квотирование
            \item Переменные окружения: \mintinline{text}{$PATH}, export
            \item which --- найти программу в путях, whereis --- найти все её файлы
            \item find: \mintinline{text}{find /usr/share/doc -name README}
            \item locate --- ищет и пути
        \end{itemize}
    \end{frame}

    \begin{frame}{Управление процессами}
        \begin{itemize}
            \item Убить процесс --- Ctrl-C
            \item Остановить --- Ctrl-Z (как бы ставит на паузу)
            \item fg и bg, оператор \mintinline{text}{&}
            \item kill, killall --- послать сигнал процессу
            \begin{itemize}
                \item SIGINT, SIGTERM, SIGKILL
            \end{itemize}
            \item ps, top
            \item Перенаправление: \mintinline{text}{>}, \mintinline{text}{>>}, \mintinline{text}{|}
            \begin{itemize}
                \item \mintinline{text}{echo lol > file.txt}
                \item \mintinline{text}{echo lol | wc}
                \item \mintinline{text}{cat file.txt | sort | uniq | wc -l}
            \end{itemize}
        \end{itemize}
    \end{frame}

    \begin{frame}{Работа с текстом}
        \begin{itemize}
            \item sort, uniq, wc
            \item head, tail
            \item more, less
            \item sed, awk
            \item vim
        \end{itemize}
    \end{frame}

    \begin{frame}{Полезные штуки}
        \begin{itemize}
            \item Табуляция
            \item .bashrc, .bash\_profile, alias
            \item Midnight Commander
            \item reverse-i-search (Ctrl-R, Ctrl-S)
            \item Ctrl-W, Ctrl-U
            \item Выделение мышью, вставка средней кнопкой
            \item Клавиша Compose
            \item Ctrl-Alt-F1, Ctrl-Alt-F2 и т.д., Ctrl-Alt-Backspace
            \item Всегда имеет смысл ставить проприетарные драйвера для видеокарты, и они почти никогда не ставятся по умолчанию
        \end{itemize}
    \end{frame}

    \section{Справка}

    \begin{frame}{Документация}
        \begin{itemize}
            \item Стандартизована, в целом лучше поддерживается, чем для типичных Windows-программ
            \item man
            \begin{itemize}
                \item RTFM!
                \item Секции man, 9 разделов, 1 --- пользовательские программы, 2 --- системные вызовы и т.д.
            \end{itemize}
            \item whatis
            \item apropos
            \item info --- более продвинутый формат документации, с гиперссылками
            \item Документация дистрибутива
        \end{itemize}
    \end{frame}

\end{document}