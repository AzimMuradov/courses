\documentclass[xetex,mathserif,serif]{beamer}
\usepackage{polyglossia}
\setdefaultlanguage[babelshorthands=true]{russian}
\usepackage{minted}
\usepackage{tabu}
\usepackage{pgfplots}

\useoutertheme{infolines}

\usepackage{fontspec}
\setmainfont{FreeSans}
\newfontfamily{\russianfonttt}{FreeSans}

\usepackage{forest}
\usetikzlibrary{arrows}

\setbeamertemplate{blocks}[rounded][shadow=false]
\setbeamercolor*{block title example}{fg=green!50!black,bg=green!20}
\setbeamercolor*{block body example}{fg=black,bg=green!10}

\setbeamercolor*{block title alerted}{fg=red!50!black,bg=red!20}
\setbeamercolor*{block body alerted}{fg=black,bg=red!10}

\tabulinesep=0.7mm

\title{Внутреннее представление данных}
\author[Юрий Литвинов]{Юрий Литвинов \newline \textcolor{gray}{\small\texttt{yurii.litvinov@gmail.com}}}

\date{05.10.2018}

\begin{document}
	
	\frame{\titlepage}
	
	\begin{frame}
		\frametitle{Побитовые операции}
		\begin{columns}
			\begin{column}{0.6\textwidth}
				\begin{itemize}
					\item \& --- побитовое ``И''
					\item | --- побитовое ``ИЛИ''
					\item $\sim$ --- побитовое ``НЕ''
					\item 1 \& 2 == false, но 1 \&\& 2 == true
					\item $<<$, $>>$ --- битовый сдвиг
					\begin{itemize}
						\item int x = 1 $<<$ 3
					\end{itemize}
					\item sizeof --- размер типа в байтах
					\begin{itemize}
						\item int s = sizeof(int) * 8
					\end{itemize}
					\item Обратите внимание, что ВСЁ хранится как набор бит
					\begin{itemize}
						\item ``3'' --- литерал, лишь удобная форма записи 00...0011 в коде
					\end{itemize}
				\end{itemize}
			\end{column}
			\begin{column}{0.4\textwidth}
				Маски
				
				\&:
				\vspace{0.1cm}
				
				\begin{tabu} {| X[1 l p] | X[1 l p] | X[1 l p] | X[1 l p] | X[1 l p] | X[1 l p] | X[1 l p] | X[1 l p] |}
					\tabucline-
					\everyrow{\tabucline-}
					1 & 1 & 0 & 1 & 1 & 0 & 1 & 0 \\
					0 & 0 & 0 & 0 & 0 & 0 & 0 & 1 \\
					0 & 0 & 0 & 0 & 0 & 0 & 0 & 0 \\
				\end{tabu}
				\vspace{0.5cm}

				\&:
				\vspace{0.1cm}

				\begin{tabu} {| X[1 l p] | X[1 l p] | X[1 l p] | X[1 l p] | X[1 l p] | X[1 l p] | X[1 l p] | X[1 l p] |}
					\tabucline-
					\everyrow{\tabucline-}
					1 & 1 & 0 & 1 & 1 & 0 & 1 & 0 \\
					0 & 0 & 0 & 0 & 0 & 0 & 1 & 0 \\
					0 & 0 & 0 & 0 & 0 & 0 & 1 & 0 \\
				\end{tabu}
			\end{column}
		\end{columns}
	\end{frame}

	\begin{frame}[fragile]
		\frametitle{Работа с масками}
		\begin{footnotesize}
			\begin{minted}{cpp}
char x = 5;

int bit = 0b10000000;
for (int j = 0; j < 8; ++j)
{
    printf((x & bit) ? "1" : "0");
    bit = bit >> 1;
}
			\end{minted}
		\end{footnotesize}
	\end{frame}

	\begin{frame}
		\frametitle{Целые числа}
		\begin{itemize}
			\item Прямой код
			\begin{itemize}
				\item 5 --- 00000101, -5 --- 10000101
			\end{itemize}
			\item Обратный код
			\begin{itemize}
				\item 5 --- 00000101, -5 --- 11111010
			\end{itemize}
			\item Дополнительный код
			\begin{itemize}
				\item 5 --- 00000101, -5 --- 11111011
				\item $-x$ представляется как $2^n - x$, поэтому и дополнительный
				\begin{itemize}
					\item $n$ --- разрядность регистра
				\end{itemize} 
			\end{itemize}
		\end{itemize}
	\end{frame}

	\begin{frame}
		\frametitle{Арифметические действия}
		\begin{itemize}
			\item В обратном коде единица переноса в старшем разряде прибавляется к младшему разряду
			\item В дополнительном коде единица переноса в старшем разряде отбрасывается
		\end{itemize}
	\end{frame}

	\begin{frame}
		\frametitle{OLOLO}
		\begin{center}
			\begin{tikzpicture}
				\begin{axis}[
						axis lines = left,
						xlabel = $n$,
						ylabel = {$f(n)$},
						ymin = 0,
						every axis plot/.append style={ultra thick}
					]
					\addplot [
							domain=0:3, 
							samples=100, 
							color=red,
						]
					{1.2*exp(x) + 2};
					\addplot [
							domain=0:3, 
							samples=100, 
							color=green,
						]
					{2*x + 10};
					\addplot [
							domain=0:3, 
							samples=100, 
							color=blue,
						]
					{1.7*x^2 + 7};
				\end{axis}
			\end{tikzpicture}
		\end{center}
	\end{frame}

	\begin{frame}
		\frametitle{TROLOLO}
		\begin{tabu} {| X[1 l p] | X[1 l p] | X[1 l p] | X[1 l p] | X[1 l p] | X[1 l p] | X[1 l p] | X[1 l p] |}
			\tabucline-
			\everyrow{\tabucline-}
			1 & 1 & 0 & 1 & 1 & 0 & 1 & 0 \\
			0 & 0 & 0 & 0 & 0 & 0 & 0 & 1 \\
			0 & 0 & 0 & 0 & 0 & 0 & 0 & 0 \\
		\end{tabu}
		\vspace{2cm}
		\begin{tabu} {| X[1 l p] | X[1 l p] | X[1 l p] | X[1 l p] | X[1 l p] | X[1 l p] | X[1 l p] | X[1 l p] |}
			\tabucline-
			\everyrow{\tabucline-}
			1 & 1 & 0 & 1 & 1 & 0 & 1 & 0 \\
			0 & 0 & 0 & 0 & 0 & 0 & 1 & 0 \\
			0 & 0 & 0 & 0 & 0 & 0 & 1 & 0 \\
		\end{tabu}
	\end{frame}

\end{document}

