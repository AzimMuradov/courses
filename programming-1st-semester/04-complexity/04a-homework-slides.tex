\documentclass{../../slides-style}

\slidetitle{Комментарии по домашке}{15.09.2023}

\begin{document}
    
    \begin{frame}[plain]
        \titlepage
    \end{frame}

    \begin{frame}[fragile]
        \frametitle{Общие замечания}
        \begin{itemize}
            \item Инициализируйте по одной переменной за раз
            \item Инициализируйте всё, даже строки и массивы
            \item camelCase --- даже если очень хочется snake\_case
            \begin{itemize}
                \item Исключение --- константы, объявляемые через \mintinline{text}{#define}, они именуются \mintinline{text}{КАПСОМ_С_ПОДЧЁРКИВАНИЕМ}
            \end{itemize}
            \item Используйте функции, не пишите полотнища!
            \item Не называйте файлы по-русски или с пробелами
            \item Две пустые строки подряд --- нехорошо
            \item Перед закрывающей фигурной скобкой пустая строка не ставится
            \item Не используйте табуляции
        \end{itemize}
    \end{frame}

    \begin{frame}[fragile]
        \frametitle{if-else и return}
        \begin{minted}{c}
void f(int x) {
    if (x == 0) {
        ...
    } else {
        ...
    }
}
        \end{minted}
        или
        \begin{minted}{c}
void f(int x) {
    if (x == 0) {
        ...
        return;
    } 
    ...
}
        \end{minted}
    \end{frame}

    \begin{frame}[fragile]
        \frametitle{Ещё рекомендации}
        \begin{itemize}
            \item Тернарный оператор: \mintinline{c}|printf(x == 0 ? "true" :  "false");|
            \item xor: \mintinline{c}|a ^ b ^ b == a|
            \item i++ vs ++i
            \item a == true равносильно a == true == true и т.д., пишите if (a) или if (!a)
            \item Отделяйте логику работы от общения с пользователем --- для переиспользования
            \item \mintinline{c}{int function()} vs \mintinline{c}{int function(void)}
            \item Именование файлов --- тоже в camelCase
            \item Сброс входного потока: \mintinline{c}|scanf("%*[^\n]");| или \mintinline{c}|while (getchar() != '\n') {|
        \end{itemize}
    \end{frame}

\end{document}

