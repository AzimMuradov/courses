\documentclass{../../slides-style}

\slidetitle{Консоль и системы сборки}{16.11.2022}

\begin{document}
    
    \begin{frame}[plain]
        \titlepage
    \end{frame}

    \section{Консоль}

    \begin{frame}
        \frametitle{Консоль, зачем}
        \begin{itemize}
            \item Программы с интерфейсом командной строки
            \begin{itemize}
                \item В Linux-подобных системах очень многие программы имеют только интерфейс командной строки
            \end{itemize}
            \item Пакетный режим и автоматизация
            \item Удалённое управление
            \item Не везде есть графический интерфейс
        \end{itemize}
    \end{frame}

    \begin{frame}
        \frametitle{Интерпретаторы командной строки, терминалы}
        Командные интерпретаторы
        \begin{itemize}
            \item Windows
            \begin{itemize}
                \item cmd
                \item Windows Power Shell
            \end{itemize}
            \item Linux
            \begin{itemize}
                \item bash, zsh, csh, …
            \end{itemize}
        \end{itemize}
        Терминалы
        \begin{itemize}
            \item Windows --- cmd, far, cmder
            \item Linux --- terminal, xterm, konsole, yaquake, ...
            \item Удалённое управление --- PuTTY, ssh
        \end{itemize}
    \end{frame}

    \begin{frame}
        \frametitle{cmd}
        \begin{itemize}
            \item Есть в любой Windows из коробки, но лучше поставить Far
            \begin{itemize}
                \item Win-R, Cmd
            \end{itemize}
            \item Команды
            \begin{itemize}
                \item dir, cd, xcopy, mkdir, del, …
                \item ключ /?
            \end{itemize}
            \item Пути
            \begin{itemize}
                \item ./ololo.exe = ololo.exe
                \item ../ololo.exe
                \item относительный путь: ../myProgram/bin/ololo.exe
                \item абсолютный путь: C:/myProgram/bin/ololo.exe
            \end{itemize}
            \item Потоки --- stdout, stderr
            \begin{itemize}
                \item echo ``test'' > someFile.txt 2> errors.txt
            \end{itemize}
        \end{itemize}
    \end{frame}

    \begin{frame}[fragile]
        \frametitle{.bat-файлы}
        \begin{itemize}
            \item echo ``Hello, world''
            \item Параметры командной строки
            \begin{itemize}
                \item  \%1, \%2, …, \%*
            \end{itemize}
            \item Циклы, условия, goto и т.д.
            \begin{itemize}
                \item 
                    \begin{footnotesize}
                        \begin{minted}{text}
for /l %x in (1, 1, 100) do echo %x
                        \end{minted}
                    \end{footnotesize}
            \end{itemize}
            \item rem Это комментарий
            \item @echo off
            \item call --- вызов другого скрипта
            \item cmd /C --- создание нового командного интерпретатора
        \end{itemize}
        Ctrl-C --- прервать выполнение скрипта (если что-то пошло не так), Ctrl-D --- конец входного потока
    \end{frame}

    \begin{frame}[fragile]
        \frametitle{.sh-файлы}
        \begin{itemize}
            \item Сильно зависят от интерпретатора (bash, zsh, csh, …)
            \begin{itemize}
                \item \#!/bin/bash --- ``shebang''
            \end{itemize}
            \item echo ``Hello, world''
            \item Параметры командной строки
            \begin{itemize}
                \item \$0, \$1, …, \$\#, \$@
            \end{itemize}
            \item Циклы, условия и т.д.
            \begin{itemize}
                \item 
                \begin{footnotesize}
                    \begin{minted}{text}
if ! [ -f "ololo.txt" ]; then
    echo "File not found"
    exit 1
fi
                    \end{minted}
                \end{footnotesize}
                \item 
                \begin{footnotesize}
                    \begin{minted}{text}
for i in $@; do
    echo $i
done
                    \end{minted}
                \end{footnotesize}
            \end{itemize}
            \item chmod +x ./test.sh
        \end{itemize}
    \end{frame}

    \begin{frame}[fragile]
        \frametitle{Код возврата}
        \framesubtitle{cmd}
        \begin{itemize}
            \item Можно проверить тот самый return -1 из main-а и в зависимости от него сделать или не сделать что-нибудь
        \end{itemize}
        \begin{minted}{text}
test.exe

if not errorlevel 0 (
     echo Everything is bad
) else (
     echo Everything is good
)
        \end{minted}
    \end{frame}

    \begin{frame}
        \frametitle{Переменные окружения (Windows)}
        \begin{itemize}
            \item \%<имя переменной>\%
            \item echo \%path\%
            \item set OLOLO=ololo
            \item Глобальный контекст
            \begin{itemize}
                \item ``Панель управления'' -> ``Система'' -> ``Дополнительные параметры системы'' -> ``Переменные среды''
                \item setx --- требует админских прав
            \end{itemize}
            \item PATH
        \end{itemize}

        У каждого процесса свой контекст
        \begin{itemize}
            \item Working Directory
            \item Своя копия переменных окружения на момент запуска
            \item Контекст наследуется от процесса-родителя
        \end{itemize}
    \end{frame}

    \begin{frame}
        \frametitle{Переменные окружения (Linux)}
        \begin{itemize}
            \item \$<имя переменной>
            \item echo \$PATH
            \item export OLOLO=ololo
            \item Глобальный контекст
            \begin{itemize}
                \item \textasciitilde/.bashrc --- скрипт, исполняющийся при старте командного интерпретатора
                \item Туда можно писать что угодно
                \begin{itemize}
                    \item И сломать себе всё
                \end{itemize}
            \end{itemize}
            \item PATH
        \end{itemize}
    \end{frame}

    \section{Системы сборки}

    \begin{frame}
        \frametitle{Системы сборки}
        \begin{itemize}
            \item Среда разработки не всегда доступна
            \begin{itemize}
                \item Continuous Integration-сервера автоматически выполняют сборку после каждого коммита, там некому открыть Visual Studio и нажать на кнопку ``запустить''
            \end{itemize}
            \item Воспроизводимость сборки
            \begin{itemize}
                \item Если чтобы собрать программу надо открыть проект, скопировать пару десятков файлов, поправить кое-какие пути и делать это в полнолуние, то возможны ошибки
            \end{itemize}
            \item Автоматизация сборки
            \begin{itemize}
                \item git clone
                \item одна консольная команда, которая всё делает за нас
                \item ...
                \item готовое к работе приложение
            \end{itemize}
        \end{itemize}
    \end{frame}

    \begin{frame}[fragile]
        \frametitle{Сборка вручную без IDE}
        \begin{itemize}
            \item Visual Studio:
            \begin{footnotesize}
                \begin{minted}{text}
cl <имя .c-файла>
                \end{minted}
            \end{footnotesize}
            или, например,
            \begin{footnotesize}
                \begin{minted}{text}
cl /W4 /EHsc file1.c file2.c file3.c /link /out:program1.exe
                \end{minted}
            \end{footnotesize}
            \begin{itemize}
                \item Работает только из Developer Command Prompt (ну, почти только)
            \end{itemize}
            \item gcc
            \begin{footnotesize}
                \begin{minted}{text}
g++ <имя .c-файла>
                \end{minted}
            \end{footnotesize}
            или, например,
            \begin{footnotesize}
                \begin{minted}{text}
g++ -Wall -o helloworld helloworld.c
                \end{minted}
            \end{footnotesize}
            \item Если проект большой, это быстро становится грустно
            \begin{itemize}
                \item Десятки тысяч файлов --- не редкость
            \end{itemize}
        \end{itemize}
    \end{frame}

    \section{make}

    \begin{frame}
        \frametitle{make}
        \begin{itemize}
            \item Стандарт де-факто по ``низкоуровневым'' правилам сборки
            \item Сама ничего не знает про языки программирования, компиляторы и прочие подобные штуки
            \item Знает про цели, зависимости, временные штампы и правила
            \begin{itemize}
                \item Смотрит на зависимости цели, если у хоть одной временной штамп свежее цели, запускается правило для цели
                \item В процессе цель может обновить свой временной штамп, что приведёт к исполнению правил для зависящих от неё целей
                \item Цели и зависимости образуют направленный ациклический граф (DAG)
                \item make выполняет топологическую сортировку графа зависимостей
                \item Правила применяются в порядке от листьев к корню
            \end{itemize}
            \item Правила сборки описываются в Makefile
        \end{itemize}
    \end{frame}

    \begin{frame}[fragile]
        \frametitle{Пример}
        \begin{footnotesize}
            \begin{minted}{make}
target [target ...]: [component ...]
    [command 1]
    .
    .
    .
    [command n]
            \end{minted}
        \end{footnotesize}
        Пример:
        \begin{footnotesize}
            \begin{minted}{text}
hello: ; @echo "hello"
            \end{minted}
        \end{footnotesize}
    \end{frame}

    \begin{frame}[fragile]
        \frametitle{Продвинутые штуки}
        \begin{itemize}
            \item Переменные
            \begin{itemize}
                \item MACRO = definition
                \item NEW\_MACRO = \$(MACRO) - \$(MACRO2)
                \item Переопределение из командной строки
                \begin{itemize}
                    \item make MACRO=ololo
                \end{itemize}
            \end{itemize}
            \item Суффиксные правила
            \begin{footnotesize}
                \begin{minted}{make}
.SUFFIXES: .txt .html
# From .html to .txt
.html.txt:
lynx -dump $<   >   $@
                \end{minted}
            \end{footnotesize}
            \item Параллельная сборка
            \begin{itemize}
                \item make -j8
            \end{itemize}
        \end{itemize}
    \end{frame}

    \begin{frame}
        \frametitle{Под Windows}
        \begin{itemize}
            \item mingw32-make
            \begin{itemize}
                \item Используется в mingw (``Minimalist GNU for Windows'')
                \begin{itemize}
                    \item Порт gcc на Windows
                    \item Можно поставить отдельно, можно в составе Qt SDK
                \end{itemize}
            \end{itemize}
            \item nmake
            \begin{itemize}
                \item Реализация от Microsoft, в комплекте с Visual Studio
                \begin{itemize}
                    \item Запускается из Developer Command Prompt
                \end{itemize}
            \end{itemize}
        \end{itemize}
        Мейкфайлы зависят от конкретной реализации make (Makefile от mingw32-make может не собраться nmake-ом)
    \end{frame}

    \section{Системы сборки}

    \begin{frame}
        \frametitle{Высокоуровневые системы сборки}
        \begin{itemize}
            \item Либо сами вызывают необходимые инструменты, либо генерируют мейкфайлы
            \item MSBuild
            \begin{itemize}
                \item Собирает из консоли .sln, .vcxproj, .csproj и т.д. -файлы
                \item Тоже запускается из Developer Command Prompt
            \end{itemize}
            \item CMake
            \begin{itemize}
                \item Кроссплатформенная система сборки, очень популярна в C++ open source-сообществе
            \end{itemize}
        \end{itemize}
        Написание скриптов сборки для большого проекта --- отдельная и довольно трудоёмкая задача
    \end{frame}

    \section{CMake}

    \begin{frame}
        \frametitle{CMake}
        \begin{itemize}
            \item С открытым исходным кодом (\url{https://gitlab.kitware.com/cmake/cmake})
            \item Прежде всего для сборки C++-проектов, но умеет много чего
            \item Разрабатывается с 1999 года
            \item Бывает версии 2 и 3
            \item Сама сборкой не занимается, генерирует конфиг для системы сборки
            \begin{itemize}
                \item Но может сама их запускать
            \end{itemize}
        \end{itemize}
    \end{frame}

    \begin{frame}
        \frametitle{Конфигурация CMake}
        \begin{itemize}
            \item CMakeLists.txt в корне проекта
            \item Также могут быть в подкаталогах (для иерархичной конфигурации)
            \item Бывают также модули (.cmake) и скрипты
            \item Коммитить, соответственно, CMakeLists.txt (все) и .cmake, если есть
        \end{itemize}
    \end{frame}

    \begin{frame}[fragile]
        \frametitle{Минимальный пример}
        \begin{minted}{text}
cmake_minimum_required(VERSION 3.20)
project(Hello)

add_executable(Hello Hello.c)
        \end{minted}
    \end{frame}

    \begin{frame}[fragile]
        \frametitle{Out-of-source-сборка}
        \begin{columns}
            \begin{column}{0.5\textwidth}
                \begin{minted}{text}
source
    build
        сгенерированные файлы
    CMakeLists.txt
    исходники
                \end{minted}
            \end{column}
            \begin{column}{0.5\textwidth}
                \begin{minted}{text}
build
    сгенерированные файлы
source
    CMakeLists.txt
    исходники
                \end{minted}
            \end{column}
        \end{columns}
    \end{frame}

    \begin{frame}
        \frametitle{Фазы сборки}
        \begin{itemize}
            \item Конфигурация
            \begin{itemize}
                \item cmake <путь до папки с CMakeLists.txt>
                \item Читает CMakeLists.txt, строит модель, сохраняет в CMakeCache.txt
                \item Пытается угадать наличествующие инструменты и их возможности (например, поддерживаемую компилятором версию стандарта)
                \item Можно вручную подредактировать параметры конфигурации (руками или в cmake-gui)
            \end{itemize}
            \item Генерация --- генерация конфигов для сборки (например, .vcxproj-файла)
            \item Сборка
            \begin{itemize}
                \item cmake --build
            \end{itemize}
        \end{itemize}
    \end{frame}

    \begin{frame}
        \frametitle{Основные понятия}
        \begin{itemize}
            \item Команды
            \begin{itemize}
                \item \mintinline{text}|command_name(список аргументов через пробел)|
                \item В CMake всё команды
                \item Всё строки
            \end{itemize}
            \item Переменные
            \begin{itemize}
                \item \mintinline{text}|set(name value)|
                \item \mintinline{text}|message(STATUS "Name = ${name}")|
                \item Переменные окружения: \mintinline{text}|$ENV{Имя}|
            \end{itemize}
            \item Функции
        \end{itemize}
    \end{frame}

    \begin{frame}[fragile]
        \frametitle{Ветвления}
        \begin{minted}{text}
if (UNIX)
    message(STATUS "Running on Unix")
else()
    message(STATUS "Running on not Unix")
endif()
        \end{minted}
        \begin{itemize}
            \item Предопределённые переменные: \url{https://cmake.org/cmake/help/latest/manual/cmake-variables.7.html}
            \item Всё это работает на этапе конфигурации!
        \end{itemize}
    \end{frame}

    \begin{frame}
        \frametitle{Цели}
        \begin{itemize}
            \item Декларативное описание того, что хочется собрать
            \item Исполняемые файлы
            \begin{itemize}
                \item \mintinline{text}|add_executable(targetname source1 ...)|
            \end{itemize}
            \item Библиотеки
            \begin{itemize}
                \item \mintinline{text}|add_library(targetname [STATIC | SHARED | ...] source1 ...)|
            \end{itemize}
            \item Линковка целей
            \begin{itemize}
                \item \mintinline{text}|target_link_libraries(myLib [PUBLIC | PRIVATE | INTERFACE] dependencyLib)|
            \end{itemize}
        \end{itemize}
    \end{frame}

    \begin{frame}
        \frametitle{Что ещё умеет}
        \begin{itemize}
            \item Конфигурации: Debug, Release, RelWithDebInfo, MinSizeRel
            \begin{itemize}
                \item \mintinline{text}|cmake ../MyProject -DCMAKE_BUILD_TYPE=Debug|
            \end{itemize}
            \item Подпапки/подпроекты
            \begin{itemize}
                \item \mintinline{text}|add_subdirectory(sourceDir ...)|
            \end{itemize}
            \item Toolchain-файлы
            \item Укачивать исходники прямо в процессе сборки!
            \begin{itemize}
                \item См., например, \url{https://google.github.io/googletest/quickstart-cmake.html}
            \end{itemize}
        \end{itemize}
    \end{frame}

    \section{CI}

    \begin{frame}
        \frametitle{Continuous Integration}
        \begin{itemize}
            \item Эталонное и единое для проекта окружение, в котором выполняется сборка
            \begin{itemize}
                \item Сборка выполняется очень часто, иногда --- после каждого коммита
            \end{itemize}
            \item Там же запускаются юнит-тесты
            \item Сборка запускается в свежей виртуальной машине или Docker-контейнере
            \item Управляется конфигурацией, хранящейся прямо в репозитории
        \end{itemize}
    \end{frame}

    \begin{frame}
        \frametitle{GitHub Actions}
        \begin{itemize}
            \item Бесплатная система облачной сборки для проектов на GitHub
            \item \url{https://docs.github.com/en/actions}
            \item Как настроить:
            \begin{itemize}
                \item В репозитории на GitHub Settings -> Actions -> Allow all actions
                \item Создаём в корне репозитория папку .github/workflows/
                \item В нём создаём файл <имя действия>.yml (например, ci.yml)
                \item Описываем процесс сборки согласно \url{https://docs.github.com/en/actions/learn-github-actions/workflow-syntax-for-github-actions}
                \begin{itemize}
                    \item Пример и описание линуксовой сборки: \url{https://www.incredibuild.com/blog/using-github-actions-with-your-c-project}
                    \item Пример .NET-сборки: \url{https://github.com/yurii-litvinov/DocUtils/blob/master/.github/workflows/ci.yml}
                \end{itemize}
                \item Коммитим-пушим
                \item Смотрим статус коммита и пуллреквеста
            \end{itemize}
        \end{itemize}
    \end{frame}

\end{document}