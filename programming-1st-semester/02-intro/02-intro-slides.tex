\documentclass[xetex,mathserif,serif]{beamer}
\usepackage{polyglossia}
\setdefaultlanguage[babelshorthands=true]{russian}
\usepackage{minted}
\usepackage{tabu}

\useoutertheme{infolines}

\usepackage{fontspec}
\setmainfont{FreeSans}
\newfontfamily{\russianfonttt}{FreeSans}

\usepackage{forest}
\usetikzlibrary{arrows}

\definecolor{links}{HTML}{2A1B81}
\hypersetup{colorlinks,linkcolor=,urlcolor=links}

\tabulinesep=0.7mm

\title{Введение, разбор задач}
\author[Юрий Литвинов]{Юрий Литвинов \newline \textcolor{gray}{\small\texttt{yurii.litvinov@gmail.com}}}

\date{10.09.2019}

\begin{document}
	
	\frame{\titlepage}
	
	\begin{frame}
		\frametitle{Формальные вопросы}
		\begin{itemize}
			\item Занятия по вторникам две пары подряд в ауд. 1416-2
			\item Берите с собой ноуты
			\item Чтобы получить зачёт, надо:
			\begin{itemize}
				\item Сдать \textbf{все} домашки
				\item Написать две контрольные
				\item Написать зачёт, который по сути большая контрольная
			\end{itemize}
			\item Условия домашек, материалы с пар и текущий статус: \url{http://hwproj.me/courses/48}
			\begin{itemize}
				\item Там надо зарегистрироваться и добавиться на курс
				\item Там же сдача домашек
			\end{itemize}
			\item Мои контакты:
			\begin{itemize}
				\item Почта: yurii.litvinov@gmail.com
				\item Telegram: yurii\_litvinov
				\item Пишите по любому вопросу!
			\end{itemize}
		\end{itemize}
	\end{frame}

	\begin{frame}
		\frametitle{Условия задач с теста}
		\begin{enumerate}
			\item Какое наименьшее количество операции умножения достаточно для вычисления значения формулы $x^4 + x^3 + x^2 + x + 1$?
			\item Укажите условия, при которых формулы ``$a + a - a$'' и ``$a + (a - a)$'' не эквивалентны.
			\item Поменять значения двух целочисленных переменных местами (без привлечения третьей переменной и файлов).
			\item Написать алгоритм нахождения неполного частного от деления $a$ на $b$ (целые числа), используя только операции сложения, вычитания и умножения.
			\item Дан массив целых чисел $x[1]...x[m+n]$, рассматриваемый как соединение двух его отрезков: начала $x[1]...x[m]$ длины $m$ и конца $x[m+1]...x[m+n]$ длины $n$. Не используя дополнительных массивов, переставить местами начало и конец.
		\end{enumerate}
	\end{frame}

	\begin{frame}
		\frametitle{Условия задач с теста}
		\begin{enumerate}
			\setcounter{enumi}{5}
			\item Подсчитать число <<счастливых билетов>> (билет считается <<счастливым>>, если сумма первых трёх цифр его номера равна сумме трёх последних).
			\item Написать алгоритм проверки баланса скобок в исходной строке (т.е. число открывающих скобок равно числу закрывающих и выполняется правило вложенности скобок).
			\item Заданы две строки: $S$ и $S_1$. Найдите количество вхождений $S_1$ в $S$ как подстроки.
			\item Напишите программу, печатающую все простые числа, не превосходящие заданного числа.
			\item Напишите программу, считающую количество нулевых элементов в массиве.
		\end{enumerate}
	\end{frame}

\end{document}

