\documentclass[a5paper]{article}
\usepackage[a5paper, top=8mm, bottom=8mm, left=8mm, right=8mm]{geometry}

\usepackage{polyglossia}
\setdefaultlanguage[babelshorthands=true]{russian}

\usepackage{fontspec}
\setmainfont{FreeSerif}
\newfontfamily{\russianfonttt}[Scale=0.7]{DejaVuSansMono}

\usepackage[font=scriptsize]{caption}

\usepackage{amsmath}
\usepackage{amssymb,amsfonts,textcomp}
\usepackage{color}
\usepackage{array}
\usepackage{hhline}
\usepackage{cite}

\usepackage[hang,multiple]{footmisc}
\renewcommand{\footnotelayout}{\raggedright}

\PassOptionsToPackage{hyphens}{url}\usepackage[xetex,linktocpage=true,plainpages=false,pdfpagelabels=false]{hyperref}
\hypersetup{colorlinks=true, linkcolor=blue, citecolor=blue, filecolor=blue, urlcolor=blue, pdftitle=1, pdfauthor=, pdfsubject=, pdfkeywords=}

\usepackage{tabu}

\usepackage{graphicx}
\usepackage{indentfirst}
\usepackage{multirow}
\usepackage{subfig}
\usepackage{footnote}
\usepackage{minted}

\sloppy
\pagestyle{plain}

\title{Введение}
\author{Юрий Литвинов\\\small{yurii.litvinov@gmail.com}}

\date{08.09.2020}

\begin{document}

\maketitle
\thispagestyle{empty}

\section{Введение}

\subsection{Организационное}


\subsection{Отчётность}


\subsection{Шкала ECTS}

Университет с этого года перешёл на европейскую шкалу оценивания ECTS, которая позволит легко перезачитывать оценки в европейских университетах, ездить на включённое обучение, поступать туда в магистратуру и т.п. Система оценивания предполагает оценки от A (отлично) до F (неудовлетворительно), то есть является шестибалльной, в отличие от традиционной четырёхбалльной, принятой у нас повсюду. ECTS предполагает <<накопительную>> систему, когда большая часть оценки зарабатывается в течение семестра, что удобно, поскольку вы всегда будете видеть свой прогресс.

\begin{itemize}
    \item Баллы за домашки пересчитываются по формуле $MAX(0, (\frac{n}{N} – 0.6)) * 2.5 * 100$, где $n$ --- сумма текущих баллов за домашки, $N$ --- сумма баллов, которую в принципе возможно было получить за курс. То есть, надо получить хотя бы 60\% баллов, иначе за домашки у вас будет 0. Сколько вы сделаете от 60\% до 100\%, как раз и определяет, насколько высокую оценку вы получите.
    \item Баллы за контрольные считаются гораздо проще: $\frac{n}{N} * 100$, где $n$ --- это сумма ваших баллов за лучшие попытки написания первой и второй контрольной, $N$ --- максимум, что можно набрать за две контрольные.
    \item Баллы за зачётную работу считаются так же, как баллы за контрольные, но учитываются отдельно, и, если контрольные можно переписывать хоть каждую неделю (после первого переписывания, которое обычно недели через две), зачётную работу переписывать можно только трижды, и дальше по одному разу на каждой пересдаче. Зато там задачи проще (чтобы если вы их не решили, то не жалко было вас отчислить).
    \item Итоговый балл за курс получается как \textbf{минимум} из этих трёх баллов. Так что если вы сделали всю домашку, но слили все контрольные --- скорее всего, домашку за вас сделал кто-то ещё и вас надо отчислить. Если вы сделали все контрольные идеально, но домашку не сделали, вы раздолбай и вас надо отчислить. Если вы не смогли в зачётную работу, то не взирая на все предыдущие заслуги вас надо отчислить, потому что вы не способны повторить свой успех. Если вы всё сделали, то вы молодец и получаете зачёт.
    \item Дальше получившийся балл преобразуется в оценку за курс по следующим правилам:
        \begin{tabu} {| X[0.9 l p] | X[1 l p] | }
            \tabucline-
            Балл                     & Оценка ECTS  \\
            \tabucline-
            \everyrow{\tabucline-}
            90-100                   & A            \\
            80-89                    & B            \\
            70-79                    & C            \\
            61-69                    & D            \\
            50-60                    & E            \\
            0-50                     & на пересдачу \\
        \end{tabu}
        Обратите внимание, меньше 80\% домашних заданий не приводит к успеху, надо сделать почти всё!
\end{itemize}

\end{document}
