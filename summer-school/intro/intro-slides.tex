\documentclass[xetex,mathserif,serif]{beamer}
\usepackage{polyglossia}
\setdefaultlanguage[babelshorthands=true]{russian}
\usepackage{minted}

\useoutertheme{infolines}

\setmainfont{FreeSans}
\newfontfamily{\russianfonttt}{FreeSans}

\title{Летняя Школа -- 2023}
\author[Юрий Литвинов]{Юрий Литвинов \newline \textcolor{gray}{\small\texttt{y.litvinov@spbu.ru}}}
\date{03.07.2023}

\begin{document}

    \frame{\titlepage}

    \begin{frame}
        \frametitle{Что это?}
        \begin{itemize}
            \item Скорее затяжной хакатон, чем школа
            \begin{itemize}
                \item Вас почти не будут собственно учить
            \end{itemize}
            \item Несколько проектов на выбор, командная работа
            \begin{itemize}
                \item Пишем код в максимально близких к реальным условиях
            \end{itemize}
            \item Научно-популярная лекционная программа
            \begin{itemize}
                \item Много разных людей рассказывают про то, что им интересно, и надеются, что это будет интересно и вам
            \end{itemize}
        \end{itemize}
    \end{frame}

    \begin{frame}
        \frametitle{Зачем?}
        \begin{itemize}
            \item Опыт почти промышленной разработки
            \item Хорошая альтернатива стажировке
            \item Возможность познакомиться с чем-то новым и расширить кругозор
            \item Возможность выбрать тему практики
            \item На самом деле, можно сделать большую часть практики за лето
            \begin{itemize}
                \item Летняя школа --- примерно как семестр работы
            \end{itemize}
        \end{itemize}
    \end{frame}

    \begin{frame}
        \frametitle{Как?}
        \begin{itemize}
            \item Сроки: с \textbf{3-го по 29-е июля}
            \item Представление проектов: \textbf{5-го июля в 12:00}
            \item До конца дня \textbf{9-го июля} определяетесь с проектом и договариваетесь с руководителем
            \item Дальше проектная работа, примерно 30 часов в неделю программирования
            \begin{itemize}
                \item Работаете в командах с помощью руководителя
                \item Перерывы на лекции
            \end{itemize}
            \item 28-го июля презентация того, что получилось
            \begin{itemize}
                \item 14-го и 21-го июля --- промежуточное демо
            \end{itemize}
            \item Связь: вообще через Тимс, но внутри проектов как договоритесь
            \begin{itemize}
                \item Контакты руководителей и организаторов легко ищутся в Тимс
            \end{itemize}
            \item Все лекции и демо --- в Тимс, проекты могут требовать очного присутствия
        \end{itemize}
    \end{frame}

\end{document}