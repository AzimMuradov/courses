\documentclass{../../text-style}

\texttitle{Демо, настройка GitHub Actions}

\begin{document}

\maketitle
\thispagestyle{empty}

\begin{itemize}
    \item Создаём пустой репозиторий на GitHub
    \item Клоним к себе
    \item Создаём там тестовый пример
    \begin{itemize}
        \item Создаём папку src
        \item Создаём файл main.cpp с кодом
            \begin{minted}{cpp}
#include <iostream>

int main() 
{
    std::cout << "Hello, world!" << std::endl;
}
            \end{minted}
        \item Создаём CMakeLists.txt такого содержания:
            \begin{minted}{cmake}
cmake_minimum_required(VERSION 3.20)

project(Hello)

add_executable(Hello main.cpp)
            \end{minted}
    \end{itemize}
    \item Проверяем, что собирается локально
    \begin{itemize}
        \item Создаём папку build в корне репозитория
        \item Делаем cmake ../src
        \item Делаем cmake --build .
        \item Идём в Debug, запускаем бинарник, проверяем, что всё ок
    \end{itemize}
    \item Коммитим-пушим
    \item Создаём папку .github/workflows
    \item Создаём workflow ci.yml вот такого содержания:
    \begin{minted}{yaml}
name: CI

on: [push]

jobs:
  build:
    runs-on: windows-latest
    steps:
      - uses: actions/checkout@v2
      - name: get-cmake
        uses: lukka/get-cmake@v3.24.3
      - run: mkdir build
      - run: |
          cd build 
          cmake ../src
          cmake --build .
      - run: ./build/Debug/Hello.exe
    \end{minted}
    \item Коммитим-пушим
    \item Идём на GitHub и смотрим, что получилось
\end{itemize}

\end{document}