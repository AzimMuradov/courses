\documentclass[xetex,mathserif,serif]{beamer}
\usepackage{polyglossia}
\setdefaultlanguage[babelshorthands=true]{russian}
\usepackage{minted}
\usepackage{tabu}
\usepackage{moresize}

\useoutertheme{infolines}

\usepackage{fontspec}
\setmainfont{FreeSans}
\newfontfamily{\russianfonttt}{FreeSans}

\definecolor{links}{HTML}{2A1B81}
\hypersetup{colorlinks,linkcolor=,urlcolor=links}

\tabulinesep=1.2mm

\title{IDEA code inspections, практика}
\author[Юрий Литвинов]{Юрий Литвинов\\\small{\textcolor{gray}{yurii.litvinov@gmail.com}}}
\date{30.01.2019г}

\newcommand{\attribution}[1] {
\vspace{-5mm}\begin{flushright}\begin{scriptsize}\textcolor{gray}{\textcopyright\, #1}\end{scriptsize}\end{flushright}
}

\begin{document}

	\frame{\titlepage}

	\section{Домашка}

	\begin{frame}
		\frametitle{Домашка}
		\begin{itemize}
			\item Надо ли удалять неиспользумые узлы дерева?
			\item Покемонов нельзя ловить в храмах, а исключения --- в тестах
			\item Модификаторы видимости надо писать самые узкие из тех, что по смыслу подходят
			\begin{itemize}
				\item Даже во вложенных классах
				\item Даже в тестах
			\end{itemize} 
			\item Перебрасывать исключения: \mintinline{java}|throw e;|
			\begin{itemize}
				\item Если хочется перебросить, подумайте, надо ли было ловить
			\end{itemize}
			\item Поля можно инициализировать прямо в месте объявления (как в C++)
			\item assertEquals принимает сначала ожидаемое, затем что получилось
			\item Комментарии к интерфейсам нужнее, чем к классам
		\end{itemize}
	\end{frame}

	\begin{frame}
		\frametitle{Новые штрафы}
		\framesubtitle{Распространяются на все коммиты после сего момента}
		\begin{itemize}
			\item catch в тестах --- -0.5 балла
			\item Слишком широкая видимость --- от -0.5 до -2 баллов (даже для вложенных классов)
			\item Неисправление замечаний к предыдущей попытке --- -0.5 балла
			\item Проект не собирается из консоли --- -1 балл
			\item Субъективная оценка --- от -2 до +2 баллов
		\end{itemize}
	\end{frame}

	\section{IDEA code inspections}

	\begin{frame}
		\frametitle{IDEA code inspections}
		\begin{itemize}
			\item Набор статических анализаторов, встроенных в IDEA
			\item Settings -> Editor -> Inspections
			\begin{itemize}
				\item Analyze -> Inspect Code...
			\end{itemize}
			\item Надо включать как можно больше
			\begin{itemize}
				\item Некоторые взаимоисключающие или противоречат стайлгайду
			\end{itemize}
		\end{itemize}
	\end{frame}

	\begin{frame}
		\frametitle{Nullability Analysis}
		\begin{itemize}
			\item \mintinline{java}|@NotNull|
			\item \mintinline{java}|@Nullable|
			\item org.jetbrains.annotations (не путать с com.sun.internal и т.п.!)
			\item Чем больше --- тем лучше!
			\item \mintinline{java}|Optional<T>| --- в каком-то смысле альтернатива
		\end{itemize}
		С этого дня \mintinline{java}|@NotNull| и \mintinline{java}|@Nullable| обязательны
	\end{frame}

	\begin{frame}
		\frametitle{Контракты}
		\begin{itemize}
			\item \mintinline{java}|@Contract("null -> null")|
			\item \mintinline{java}|@Contract("_ -> this")|
			\item \mintinline{java}|@Contract("!null,_ -> param1; null,!null -> param2; null,null -> fail")|
			\item \url{https://www.jetbrains.com/help/idea/contract-annotations.html}
		\end{itemize}
	\end{frame}

	\section{Задача}

	\begin{frame}
		\frametitle{Задача}
		\begin{itemize}
			\item Реализовать generic-класс Maybe
			\begin{itemize}
				\item \mintinline{java}|public static <T> Maybe<T> just(T t)|
				\item \mintinline{java}|public static <T> Maybe<T> nothing()|
				\item \mintinline{java}|public T get()|
				\item \mintinline{java}|public boolean isPresent()|
				\item \mintinline{java}|public <U> Maybe<U> map(Function<?, ?> mapper)|
			\end{itemize}
			\item Прочитать числа из файла:
			\begin{itemize}
				\item если на строке действительно число, возвращается ``Maybe'' с числом
				\item если на строке записано не число, возвращается ``Maybe'' с null
			\end{itemize}
			\item Возвести в квадрат прочитанные числа и сохранить их в новый файл
		\end{itemize}
	\end{frame}
	
\end{document}
