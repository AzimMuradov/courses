\documentclass[xetex,mathserif,serif]{beamer}
\usepackage{polyglossia}
\setdefaultlanguage[babelshorthands=true]{russian}
\usepackage{minted}
\usepackage{tabu}
\usepackage{moresize}

\useoutertheme{infolines}

\usepackage{fontspec}
\setmainfont{FreeSans}
\newfontfamily{\russianfonttt}{FreeSans}

\definecolor{links}{HTML}{2A1B81}
\hypersetup{colorlinks,linkcolor=,urlcolor=links}

\tabulinesep=1.2mm

\title{I/O в Java}
\author[Юрий Литвинов]{Юрий Литвинов\\\small{\textcolor{gray}{yurii.litvinov@gmail.com}}}
\date{23.01.2019г}

\newcommand{\attribution}[1] {
\vspace{-5mm}\begin{flushright}\begin{scriptsize}\textcolor{gray}{\textcopyright\, #1}\end{scriptsize}\end{flushright}
}

\begin{document}

	\frame{\titlepage}

	\section{IO Streams}

	\begin{frame}[fragile]
		\frametitle{I/O в Java}
		\begin{itemize}
			\item Пакет java.io
			\item Дескриптор файла --- класс File
			\item Интерфейсы
			\begin{itemize}
				\item InputStream, OutputStream --- работа с байтами
				\item Reader, Writer --- работа с символами
			\end{itemize}
			\item Реализации
			\begin{itemize}
				\item File* --- работа с файлами
				\item Buffered* --- оптимизация с помощью буферизации
				\item Data* --- работа с данными
				\item ByteArray* --- потоки в оперативной памяти
			\end{itemize}
			\item Пример:
				\begin{minted}{java}
new BufferedReader(new FileReader("path/to/file"));
				\end{minted}
			\item Потоки надо закрывать (close(), try-with-resource)
		\end{itemize}
	\end{frame}

	\begin{frame}
		\frametitle{Потоки ввода, основные операции}
		\begin{itemize}
			\item \mintinline{java}|int read()| --- чтение элемента
			\item \mintinline{java}|read(T[] v)|, \mintinline{java}|read(T[] v, off, len)| --- чтение элементов в массив
			\item \mintinline{java}|skip(n)| --- пропуск n элементов
			\item \mintinline{java}|available()| --- сколько элементов доступно
			\item \mintinline{java}|mark(limit)| --- пометка текущей позиции
			\item \mintinline{java}|reset()| --- возврат к помеченной позиции
			\item \mintinline{java}|close()| --- закрытие потока
			\begin{itemize}
				\item Интерфейс Closeable extends AutoCloseable
			\end{itemize}
		\end{itemize}
	\end{frame}

	\begin{frame}
		\frametitle{Потоки вывода, основные операции}
		\begin{itemize}
			\item \mintinline{java}|write(int v)| --- запись элемента
			\item \mintinline{java}|write(T[] v)| --- запись массива элементов
			\item \mintinline{java}|write(T[] v, off, len)| --- запись части массива
			\item \mintinline{java}|flush()| --- запись буфера
			\begin{itemize}
				\item Интерфейс Flushable
			\end{itemize}
			\item \mintinline{java}|close()| --- закрытие потока
		\end{itemize}
	\end{frame}

	\begin{frame}[fragile]
		\frametitle{Пример, блочное копирование}
		\begin{minted}{java}
void copy(InputStream is, OutputStream os)
    throws IOException
{
    var b = new byte[1024];
    int c = 0;
    while ((c = is.read(b)) > 0) {
        os.write(b, 0, c);
    }
}
		\end{minted}
	\end{frame}

	\begin{frame}[fragile]
		\frametitle{Пример, чтение из файла}
		\begin{minted}{java}
FileInputStream in = null;

try {
   in = new FileInputStream("test.txt");
   int c;

   while ((c = in.read()) != -1) {
       doSomething(c);
   }
} finally {
   if (in != null) {
       in.close();
   }
}
		\end{minted}
	\end{frame}

	\begin{frame}[fragile]
		\frametitle{Или (try-with-resource)}
		\begin{minted}{java}
try (var in = new FileInputStream("test.txt")) {
   int c;

   while ((c = in.read()) != -1) {
       doSomething(c);
   }
}
		\end{minted}
		\begin{itemize}
			\item Работает для классов, реализующих java.lang.AutoCloseable
		\end{itemize}
	\end{frame}

	\begin{frame}
		\frametitle{Исключения}
		\begin{itemize}
			\item IOException 
			\begin{itemize}
				\item Корень иерархии исключений ввода-вывода
				\item Бросается всеми операциями ввода/вывода
			\end{itemize}
			\item EOFException --- достигнут конец потока
			\item FileNotFoundException --- файл не найден
			\item UnsupportedEncodingException --- неизвестная кодировка
		\end{itemize}
	\end{frame}

	\begin{frame}
		\frametitle{Преобразование потоков}
		\begin{itemize}
			\item При чтении возможно преобразование байтового потока в символьный, с указанием кодировки, класс InputStreamReader
			\begin{itemize}
				\item \mintinline{java}|InputStreamReader(InputStream, encoding?)|
			\end{itemize}
			\item Наоборот, из символьного в байтовый: OutputStreamWriter
			\begin{itemize}
				\item \mintinline{java}|OutputStreamWriter(OutputStream, encoding?)|
			\end{itemize}
		\end{itemize}
	\end{frame}

	\begin{frame}[fragile]
		\frametitle{Пример, перекодирование файла}
		\begin{minted}{java}
Reader reader = new InputStreamReader(
    new FileInputStream("input.txt"), "Cp1251");
Writer writer = new OutputStreamWriter(
    new FileOutputStream("output.txt"), "Cp866");
int c = 0;
while ((c = reader.read()) >= 0) {
    writer.write(c);
}

reader.close();
writer.close();
		\end{minted}
	\end{frame}

	\begin{frame}[fragile]
		\frametitle{Ещё пример, буферизующий поток}
		\begin{minted}{java}
static String readFirstLineFromFile(String path)
       throws IOException {
   try (var br = new BufferedReader(new FileReader(path))) {
       return br.readLine();
   }
}
		\end{minted}
	\end{frame}

	\begin{frame}
		\frametitle{Потоки для работы в памяти}
		\begin{itemize}
			\item ByteArrayInputStream --- чтение из массива байт
			\item CharArrayReader --- чтение из массива символов
			\item StringReader --- чтение из строки
			\item ByteArrayOutputStream --- запись в массив байт (toByteArray())
			\item CharArrayWriter --- запись в массив символов (toString(),toCharArray())
			\item StringWriter – запись в StringBuffer (toString(), toStringBuffer())
		\end{itemize}
	\end{frame}

	\begin{frame}[fragile]
		\frametitle{Scanner}
		\begin{minted}{java}
try (var s = new Scanner(
       new BufferedReader(new FileReader("test.txt")))) {
   while (s.hasNextDouble()) {
       System.out.println(s.nextDouble());
   }
}
		\end{minted}
	\end{frame}

	\begin{frame}
		\frametitle{Работа с данными}
		\begin{itemize}
			\item Интерфейс DataOutput, класс DataOutputStream
			\begin{itemize}
				\item \mintinline{java}|writeT(T)| --- запись примитивных типов
				\item \mintinline{java}|writeUnsignedByte()| / \mintinline{java}|writeUnsignedShort()| --- запись беззнаковых целых
				\item \mintinline{java}|writeUTF()| --- запись строки в кодировке UTF-8
			\end{itemize}
			\item Интерфейс DataInput, класс DataInputStream
			\begin{itemize}
				\item \mintinline{java}|T readT()| --- чтение примитивных типов
				\item \mintinline{java}|readUnsignedByte()| / \mintinline{java}|readUnsignedShort()| --- чтение беззнаковых целых
				\item \mintinline{java}|readUTF()| --- чтение строки в кодировке UTF-8
			\end{itemize}
		\end{itemize}
	\end{frame}

	\section{Задача}

	\begin{frame}
		\frametitle{Задача на остаток пары}
		\begin{itemize}
			\item Реализовать консольную программу, работающую в трёх режимах:
			\begin{itemize}
				\item архивирует каталог (путь до него -- первый аргумент командной строки) с заданным именем (второй аргумент)
				\item разархивирует заданный архив (первый аргумент) в заданное место (второй аргумент)
				\item распечатывает содержимое архива (первый аргумент): имена файлов, размер сжатый/несжатый, метод архивации и т.п.
			\end{itemize}
			\item Могут быть полезны классы ZipFile, ZipEntry, ZipOutputStream, ZipInputStream
		\end{itemize}
	\end{frame}

\end{document}
