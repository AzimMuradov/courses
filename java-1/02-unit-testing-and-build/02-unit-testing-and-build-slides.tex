\documentclass[xetex,mathserif,serif]{beamer}
\usepackage{polyglossia}
\setdefaultlanguage[babelshorthands=true]{russian}
\usepackage{minted}
\usepackage{tabu}
\usepackage{moresize}

\useoutertheme{infolines}

\usepackage{fontspec}
\setmainfont{FreeSans}
\newfontfamily{\russianfonttt}{FreeSans}

\definecolor{links}{HTML}{2A1B81}
\hypersetup{colorlinks,linkcolor=,urlcolor=links}

\tabulinesep=1.2mm

\title{Юнит-тестирование и системы сборки}
\author[Юрий Литвинов]{Юрий Литвинов\\\small{\textcolor{gray}{yurii.litvinov@gmail.com}}}
\date{16.01.2019г}

\newcommand{\attribution}[1] {
\vspace{-5mm}\begin{flushright}\begin{scriptsize}\textcolor{gray}{\textcopyright\, #1}\end{scriptsize}\end{flushright}
}

\begin{document}

	\frame{\titlepage}

	\section{Юнит-тестирование}

	\begin{frame}
		\frametitle{Юнит-тестирование, зачем}
		\begin{itemize}
			\item Любая программа содержит ошибки
			\item Если программа не содержит ошибок, их содержит алгоритм, который реализует эта программа
			\item Если ни программа, ни алгоритм ошибок не содержат, такая программа даром никому не нужна
		\end{itemize}
		Тестирование не позволяет доказать отсутствие ошибок, оно позволяет лишь найти ошибки, которые в программе присутствуют
	\end{frame}

	\begin{frame}
		\frametitle{Информация к размышлению}
		\begin{itemize}
			\item Программа из сотни строк может иметь $10^{18}$ путей исполнения
			\begin{itemize}
				\item Времени жизни вселенной не хватило бы, чтобы их покрыть
			\end{itemize}
			\item После передачи на тестирование в программах в среднем от 1 до 3 ошибок на 100 строк кода
			\item В процессе разработки --- 1.5 ошибок на 1 строку кода (!)
			\item Если для исправления ошибки надо изменить не более 10 операторов, с первого раза это делают правильно в 50\% случаев
			\item Если для исправления ошибки надо изменить не более 50 операторов, с первого раза это делают правильно в 20\% случаев
		\end{itemize}
	\end{frame}

	\begin{frame}
		\frametitle{Модульное тестирование}
		\begin{itemize}
			\item Тест на каждый отдельный метод, функцию, иногда класс
			\item Пишутся программистами
			\item Запускаются часто (как минимум, после каждого коммита)
			\item Должны всегда проходить
			\begin{itemize}
				\item Принято не продолжать разработку, если юнит-тест не проходит
			\end{itemize}
			\item Помогают быстро искать ошибки (вы ещё помните, что исправляли), рефакторить код (``ремни безопасности''), продумывать архитектуру (мешанину невозможно оттестировать), документировать код (каждый тест -- это рабочий пример вызова)
		\end{itemize}
	\end{frame}

	\begin{frame}
		\frametitle{JUnit 5}
		
	\end{frame}

	\begin{frame}
		\frametitle{Демонстрация}
		
	\end{frame}

	\begin{frame}
		\frametitle{Best practices}
		\begin{itemize}
			\item Независимость тестов
			\begin{itemize}
				\item Желательно, чтобы поломка одного куска функциональности ломала один тест
			\end{itemize}
			\item Тесты должны работать быстро
			\begin{itemize}
				\item И запускаться после каждой сборки
				\begin{itemize}
					\item Continuous Integration
				\end{itemize}
			\end{itemize}
			\item Тестов должно быть много
			\begin{itemize}
				\item Следите за Code coverage, должно быть близко к 100\%
			\end{itemize}
			\item Каждый тест должен проверять конкретный тестовый сценарий
			\begin{itemize}
				\item Нельзя ловить исключения в тесте без крайней нужды
				\item С большой осторожностью пользоваться Random-ом
			\end{itemize}
			\item Test-driven development
		\end{itemize}
	\end{frame}

	\section{Системы сборки}

	\begin{frame}
		\frametitle{Системы сборки}
		\begin{itemize}
			\item Нужны, чтобы сборка не зависела от среды разработки
			\begin{itemize}
				\item Воспроизводимость сборки
				\item ``Пакетный режим'', CI
			\end{itemize}
			\item Умеют вычислять зависимости между компонентами, управлять внешними зависимостями
			\begin{itemize}
				\item И автоматически скачивать нужные пакеты при сборке!
			\end{itemize}
			\item Не только, собственно, сборка: тестирование, создание документации, инсталляторов, отчётов и т.д.
			\item Maven
			\item Gradle
		\end{itemize}
	\end{frame}

	\subsection{Maven}

	\begin{frame}
		\frametitle{Maven, основные принципы}
		\begin{itemize}
			\item Декларативность
			\item Convention Over Configuration
			\begin{itemize}
				\item Стандартная структура папок
			\end{itemize}
			\item Плагины
			\item Project Object Model
			\item Жизненный цикл
			\item Координаты
			\begin{itemize}
				\item groupId
				\item artifactId
				\item version
			\end{itemize}
			\item Репозиторий и on-demand-скачивание пакетов
			\begin{itemize}
				\item \url{https://mvnrepository.com/}
				\item Локальный кеш: ~/.m2
			\end{itemize}
		\end{itemize}
	\end{frame}

	\begin{frame}
		\frametitle{Maven, стандартная структура папок}
		\begin{itemize}
			\item pom.xml
			\item src
			\begin{itemize}
				\item main
				\begin{itemize}
					\item java
					\item resources
				\end{itemize}
				\item test
				\begin{itemize}
					\item java
					\item resources
				\end{itemize}
			\end{itemize}
			\item target
		\end{itemize}
	\end{frame}

	\begin{frame}
		\frametitle{Maven, демонстрация}
		\begin{center}
			\huge{Демонстрация}
		\end{center}
	\end{frame}

	\subsection{Gradle}

	\begin{frame}
		\frametitle{Gradle}
		\begin{itemize}
			\item Более легковесный, чем Maven
			\item Декларативный предметно-ориентированный язык на основе Groovy
			\item Использует Convention Over Configuration, но легко переопределить
			\begin{itemize}
				\item Та же структура папок по умолчанию
			\end{itemize}
			\item Использует репозиторий Maven для динамической загрузки зависимостей
			\item build.gradle, settings.gradle
			\item gradle-wrapper
			\begin{itemize}
				\item gradlew, gradlew.bat
				\item gradle-wrapper.jar, gradle-wrapper.properties
			\end{itemize}
			\item Для Java 11 нужна Gradle 5, с IDEA (2018.3) поставляется Gradle 4
		\end{itemize}
	\end{frame}

	\begin{frame}[fragile]
		\frametitle{Пример build.gradle}
		\begin{scriptsize}
			\begin{minted}{groovy}
plugins {
    id 'java'
}

group 'com.example'
version '1.0-SNAPSHOT'

sourceCompatibility = 11

repositories {
    mavenCentral()
}

dependencies {
    testCompile('org.junit.jupiter:junit-jupiter-api:5.3.2')
    testRuntime('org.junit.jupiter:junit-jupiter-engine:5.3.2')
}

test {
    useJUnitPlatform()
}
			\end{minted}
		\end{scriptsize}
	\end{frame}

	\begin{frame}
		\frametitle{Gradle, демонстрация}
		\begin{center}
			\huge{Демонстрация}
		\end{center}
	\end{frame}

\end{document}
