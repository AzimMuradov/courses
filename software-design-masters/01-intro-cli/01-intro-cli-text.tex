\documentclass[a5paper]{article}
\usepackage[a5paper, top=8mm, bottom=8mm, left=8mm, right=8mm]{geometry}

\usepackage{polyglossia}
\setdefaultlanguage[babelshorthands=true]{russian}

\usepackage{fontspec}
\setmainfont{FreeSerif}
\newfontfamily{\russianfonttt}[Scale=0.7]{DejaVuSansMono}

\usepackage[font=scriptsize]{caption}

\usepackage{amsmath}
\usepackage{amssymb,amsfonts,textcomp}
\usepackage{color}
\usepackage{array}
\usepackage{hhline}
\usepackage{cite}
\usepackage{ulem}

\usepackage[xetex,linktocpage=true,plainpages=false,pdfpagelabels=false]{hyperref}
\hypersetup{colorlinks=true, linkcolor=blue, citecolor=blue, filecolor=blue, urlcolor=blue, pdftitle=1, pdfauthor=, pdfsubject=, pdfkeywords=}

\usepackage{tabu}

\usepackage{graphicx}
\usepackage{indentfirst}
\usepackage{multirow}
\usepackage{subfig}
\usepackage{footnote}
\usepackage{listings}

\sloppy
\pagestyle{plain}

\title{Практика 1: Введение, задача про CLI}

\date{04.02.2019г}

\begin{document}

\maketitle
\thispagestyle{empty}

\section{Формальности}
Занятия у нас раз в неделю, в конце надо будет получить зачёт за этот курс (и сдать экзамен по соответствующему теоретическому курсу, который читает Тимофей Брыксин). На практике мы будем пытаться применять знания, полученные на теоретическом курсе, поэтому будет некоторое (довольно большое) количество домашек, которые обсуждать и начинать делать мы будем в аудитории, а доделывать по всей видимости, придётся дома. Собственно, чтобы получить зачёт, надо успешно сдать все домашки (обратите внимание, не сколько-то процентов, а все, если хоть одна задача не сдана, зачёт не ставится), причём до наступления дедлайна. За пропущенный дедлайн будет дополнительный вопрос на экзамене.

Сдавать решения надо будет, выкладывая их на GitHub и делая пуллреквест в свой же собственный репозиторий, я буду стараться этот пуллреквест комментить и принимать или не принимать. Когда пуллреквест принят (accepted, мерджить я сам не буду), задача считается зачтённой.

Есть система сдачи домашек, применяемая в СПбГУ, называется HwProj, её писали студенты, к тому же она не поддерживается сейчас, поэтому она может доставлять немного боли. Адрес у неё \url{http://hwproj.me/}, там будут публиковаться условия задач, туда же надо сабмитить решения (в виде ссылки на пуллреквест), там же будет обратная связь и табличка, где будут отмечаться сданные или несданные задачи. Там надо зарегаться и добавиться в курс с соответствующим названием, подождать, пока я подтвержу регистрацию, и начать выкладывать туда домашки (в виде либо ссылки на пуллреквест, либо прямо архив с решением). Если там что-то не работает, непонятно, всё плохо, или вдруг вы всё решили, но домашку \sout{съела собака} зажевал HwProj, пишите мне на почту. Вообще, кстати, писать мне на почту --- хорошая практика, задавать любые вопросы можно, я буду стараться на них ответить. Если потребуется более оперативная связь, есть телеграм, его можно запросить по почте. В качестве языка и среды программирования можно использовать то, что вам больше нравится, курс всё-таки больше про архитектуру, а не про реализацию, только уж совсем эзотерических языков не надо, я должен иметь возможность хотя бы собрать и запустить то, что у вас получилось. Лучше писать на Java/C\#/F\#/C++, больше шансов получить содержательные комментарии, относящиеся к технике кодирования.

Ещё обратите внимание, что просто сделать домашку может быть недостаточно. У нас тут курс по проектированию, поэтому решение, удовлетворяющее всем функциональным требованиям, вообще, строго говоря, никому не интересно. Важна архитектурная красота, обоснованность принятых решений, чистая и аккуратная реализация, наличие необходимых сопровождающих артефактов. Поскольку архитектура --- это в каком-то смысле больше искусство, чем наука, чётких критериев оценки не будет, и придираться я буду в основном к форме --- качеству кода, наличию документации, соответствию формальным правилам. 

Студентами от практики по архитектуре обычно ожидается много критики архитектуры, но тут проблема в том, что если вы достаточно владеете программированием и имеете хоть какой-то опыт, вы окажетесь вполне способны породить адекватное решение к учебным задачкам, так что, скорее всего, критика архитектуры решения будет в духе ``ок, мне нравится, но можно было ещё вот так'' или её не будет вообще. Это плохо, но делать нечего, реальные проблемы с архитектурой могут возникнуть только на реальных проектах. Кстати, если интересно и у кого-нибудь будут интересные НИР, можно устроить их архитектурный разбор. В целом же, тренировать мы будем прежде всего владение инструментами, которые позволяют создать и описать архитектуру.

\section{Что будет в курсе}

Собственно, практические занятия по проектированию ПО --- затея, изначально обречённая на провал, невозможно научиться проектированию ПО, посещая практические занятия в университете и даже делая все домашки. Лучший способ научиться проектировать ПО --- проектировать реальные проекты, причём достаточно большие. С этим может не сложиться, по крайней мере, поначалу, и уж тем более это нельзя впихнуть в формат практических занятий. Поэтому мы будем практиковаться на игрушечных примерах, но требования к ним будут предъявляться такие, будто это огромные промышленные проекты в сотни человеколет трудозатрат. Так что аргументы в духе ``да я это на питоне в двести строчек нафигачу без всякой архитектуры-шмархитектуры'' заранее признаются невалидными. 

На занятиях примерно в половине случаев я буду что-то рассказывать (больше про практическую сторону дела, какие-то примеры из реальной практики, про инструменты и методы, а иногда и просто продолжать лекции), в половине случаев прямо на паре надо будет сделать что-нибудь небольшое и, если не успели, доделать дома. Будет несколько задач на кодирование (всего штуки три крупные задачи) и много мелких задач в духе ``нарисовать диаграмму'', ``написать документ'', ``проанализировать''. Рассказывать я буду про архитектурную документацию, будет довольно много и теории и практики про UML и другие визуальные языки, которые используются при разработке архитектуры, про реализационные аспекты паттернов проектирования (что такое, как устроено и зачем надо, расскажут на лекциях, а как это писать и где можно прострелить себе ногу --- на практике, такие дела), про антипаттерны (на лекциях будут рассказывать, как писать надо, на практиках --- как не надо), примеры различных архитектур, разные подходы к созданию архитектуры и вещи, которые не так важны, чтобы включать их в курс лекций и программу экзамена, но тем не менее достойные упоминания.

Однако даже к задачам на кодирование надо относиться как к архитектурным, у нас тут всё-таки не практика по Java --- надо сначала продумать и описать архитектуру разрабатываемой системы, уже затем кидаться её кодить, и главное, при кодировании не забывать про архитектуру и её связь с кодом. Овердизайн и массированное применение знаний, полученных на лекциях, приветствуется (правда, до разумных пределов). При проверке от решений задач будет ожидаться:\nopagebreak

\begin{enumerate}
	\item работоспособность и соответствие требованиям (явным или неявным);
  \begin{enumerate}
		\item кстати, negotiation является важной частью работы архитектора, так что требования можно (и иногда нужно) уточнять, договариваться, возможно, убеждать препода, что тот или иной пункт условия ему на самом деле не нужен и т.д., но, опять-таки, до разумных пределов;
  \end{enumerate}
	\item наличие архитектурной документации;
  \begin{enumerate}
		\item её форма и количество будут меняться по ходу курса, от README и комментариев к коду в начале до UML-диаграмм и формального design document-а в конце;
		\item комментарии к каждому классу, интерфейсу и public-методу, тем не менее, всегда будут обязательны;
		\item краткое описание деталей реализации в README --- тоже;
  \end{enumerate}
	\item следование стайлгайдам и общепринятым правилам здравого смысла в программировании;
	\item наличие юнит-тестов, покрывающих все требования условия (там, где это возможно);
	\item максимально возможной кроссплатформенности и переносимости кода (писать ради этого на ANSI C не надо, но писать на Java программы, на ровном месте не работающие под виндой просто потому, что вы забыли об её существовании --- тоже);
  \begin{enumerate}
		\item проверяться задача может как под Линуксом, так и под Windows, и если у неё нет веских причин не работать под одной из этих операционных систем, а она не работает, могут попросить исправить;
  \end{enumerate}
	\item применение индустриальных практик, общепринятых при разработке production-кода: логирование, Continuous Integration (это не обязательно, но если умеете, то неплохо бы), разумные стратегии обработки исключений.
\end{enumerate}

Обратите внимание, что многое из того, что я буду рассказывать, и главное, требовать от решений, может показаться оверкиллом --- например, комментарии к \textbf{каждому} public-методу, о боже, что? Или рисование пачки диаграмм для программы в 500 строк на Питоне. Это как раз связано с тем, про что я уже говорил --- мы считаем, что речь идёт о больших реальных проектах, где это всё не выпендрёж, а единственный способ не развалиться от собственной сложности. Часто возражают, что в реальных проектах так тоже не делают --- я слышал от кого-то из магистров, что в Яндексе сколько-то гигабайт кода и ни одной UML-диаграммы --- охотно верю, \sout{поэтому у нас всё так плохо} но на этих парах мы учимся техникам, принципам и инструментам, так что некоторые требования действительно искусственны. Исключительно с учебными целями. Можно этого всего не делать и просто писать код, но тогда всю жизнь будете junior-ами, которым говорят, какой код писать.

\section{Задача про CLI}

Начнём мы с задачи на попроектировать и покодить. Задача такая: реализовать простой интерпретатор командной строки, поддерживающий следующие команды:

\begin{itemize}
	\item cat [FILE] --- вывести на экран содержимое файла;
	\item echo --- вывести на экран свой аргумент (или аргументы);
	\item wc [FILE] --- вывести количество строк, слов и байт в файле;
	\item pwd --- распечатать текущую директорию;
	\item exit --- выйти из интерпретатора.
\end{itemize}

Кроме того, должны поддерживаться одинарные и двойные кавычки (full and weak quoting, то есть одинарные кавычки передают текст как есть, двойные выполняют подстановки переменных окружения с оператором \$), собственно окружение (команды вида ``имя=значение''), оператор \$, вызов внешней программы через Process (или его аналоги) для любой команды, которую интерпретатор не знает. Должны ещё поддерживаться пайплайны (оператор ``|''), то есть перенаправление ввода и вывода. Примеры:

\begin{verbatim}
> echo "Hello, world!"
Hello, world!
> FILE=example.txt
> cat $FILE
Some example text
> cat example.txt | wc
1 3 18
> echo 123 | wc
1 1 3
> x=exit
> $x
\end{verbatim}

Тимофей ещё не рассказывал, наверное, про Architectural drivers (честно как-то не встречал адекватного русского перевода, так что будем обходиться английским термином), но ещё расскажет. Пока что --- это основные факторы, определяющие архитектуру системы. Поскольку мы пишем объектно-ориентированный код (ну, потому что), мы будем исходить из обычных для объектно-ориентированных программ качеств --- сопровождаемость, расширяемость, переиспользуемость и т.д. В нашем случае это значит, в частности, что проектировать систему надо так, чтобы новые команды было добавлять легко (ну, логично, что шелл будет постепенно расширяться новыми встроенными командами), но желательно поддерживать архитектуру достаточно простой и слабо связанной, чтобы можно было реализовать и другие требования, которые могут возникать по ходу. Может потребоваться внезапно реализовать ещё что-нибудь из того, что умеют обычные шеллы, как и в реальной жизни, желания заказчика непредсказуемы (поэтому, кстати, не надо пытаться их предугадать и заложить в архитектуру --- то, о чём вы подумали, никогда не случится, случится то, о чём вы не подумали). В итоге от задачи хочется видеть код, который делает то, что написано в условии (то есть удовлетворяет функциональным требованиям), адекватно покрыт юнит-тестами (то есть удовлетворяет требованиям качества и сопровождаемости), имеет комментарии к каждому классу и хотя бы к каждому паблик-методу (ещё один необходимый компонент сопровождаемости), имеет небольшой текстовый документ (можно Javadoc-комментарий в package-info.java), описывающий кратко архитектуру (пока design document-ы не проходили, не надо писать подробно, полстраницы ок), имеет единый стиль кодирования и написан без явных косяков кодирования на языке, который вам больше нравится. Результаты надо закоммитить к себе в репозиторий на гитхаб (можно создать один репозиторий для всех задач этого курса и раскладывать задачи по отдельным папкам) в отдельную ветку и сделать пуллреквест в мастер, про что через HwProj сообщить мне. В идеале в конце курса хочется получить репозиторий со всеми решениями в master, так что ветки с зачтёнными задачами не стесняйтесь мерджить.

Прямо сейчас задача --- выполнить анализ и определить подходы к решению задачи, выявить как можно больше потенциальных подводных камней и способов их преодоления. Даётся некоторое время подумать, после чего ожидается живое обсуждение, в ходе которого мы обсуждаем детали того, как бы вы стали писать такую задачу. И можно прямо на паре начинать писать.

Цель сейчас двоякая --- во-первых, вы, пока вам почти ничего про архитектуру не рассказали, должны продемонстрировать те навыки, что у вас уже есть (умеете UML --- рисуйте, шарите в паттернах --- используйте, чем больше, тем лучше). Во-вторых, у всех должно к концу пары сложиться понимание, как сделать домашку, потому как даже среди магистров шестого курса были люди, которые не знали даже, как подступиться к решению. Это вполне ок, чем больше окажется людей, которые сейчас без идей, как эту задачу делать, тем лучше --- в этом основной смысл архитектуры, превращать непонятные задачи в набор простых задачек на реализацию методов, с которыми справится даже новичок.

\end{document}
