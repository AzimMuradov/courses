\documentclass[xetex,mathserif,serif]{beamer}
\usepackage{polyglossia}
\setdefaultlanguage[babelshorthands=true]{russian}
\usepackage{minted}
\usepackage{tabu}

\useoutertheme{infolines}

\usepackage{fontspec}
\setmainfont{FreeSans}
\newfontfamily{\russianfonttt}{FreeSans}

\setbeamertemplate{blocks}[rounded][shadow=false]
\setbeamercolor*{block title example}{fg=green!50!black,bg=green!20}
\setbeamercolor*{block body example}{fg=black,bg=green!10}

\setbeamercolor*{block title alerted}{fg=red!50!black,bg=red!20}
\setbeamercolor*{block body alerted}{fg=black,bg=red!10}

\tabulinesep=0.7mm

\title{Продолжение про CLI, grep}
\author[Юрий Литвинов]{Юрий Литвинов \newline \textcolor{gray}{\small\texttt{yurii.litvinov@gmail.com}}}

\date{01.03.2017г}

\begin{document}
	
	\frame{\titlepage}
	
	\begin{frame}
		\frametitle{Комментарии по домашке}
		\begin{itemize}
			\item Основная проблема с ней в том, что её почти никто не сдавал
			\item Комментарии:
			\begin{itemize}
				\item К самим классам
				\item К интерфейсам
				\item Ко всем public-методам
				\begin{itemize}
					\item принцип единственности ответственности и объектно-ориентированная декомпозиция вообще
				\end{itemize}
			\end{itemize}
			\item Стайлгайд (в основном какая-то беда с пробелами)
			\begin{itemize}
				\item \url{http://checkstyle.sourceforge.net/}
				\item \url{https://www.codacy.com/}
				\item \url{https://github.com/StyleCop}
			\end{itemize}
		\end{itemize}
	\end{frame}

	\begin{frame}
		\frametitle{Grep}
		\framesubtitle{Следующая задача}
		Реализовать команду grep, 
		\begin{itemize}
			\item поддерживающую ключи
			\begin{itemize}
				\item \textit{-i} (нечувствительность к регистру)
				\item \textit{-w} (поиск только слов целиком)
				\item \textit{-A n} (распечатать n строк после строки с совпадением)
			\end{itemize}
			\item поддерживающую регулярные выражения в строке поиска
			\item использующую одну из библиотек для разбора аргументов командной строки
		\end{itemize}
	\end{frame}

	\begin{frame}[fragile]
		\frametitle{Примеры}
		\begin{minted}{bash}
> grep plugin build.gradle
    apply plugin: 'java'
    apply plugin: 'idea'
> cat build.gradle | grep plugin
    apply plugin: 'java'
    apply plugin: 'idea'
> grep -A 2 plugin build.gradle
    apply plugin: 'java'
    apply plugin: 'idea'
    group = 'ru.example'
    version = '1.0'
		\end{minted}
\end{frame}

	\begin{frame}
		\frametitle{Замечания}
		\begin{itemize}
			\item Ожидается обоснование выбора библиотеки для работы с аргументами
			\begin{itemize}
				\item Какие библиотеки были рассмотрены
				\item Почему выбрана именно та, что выбрана
				\begin{itemize}
					\item кратко описать текстом
				\end{itemize}
			\end{itemize}
			\item Сдавать как новый пуллреквест из новой ветки на базе предыдущей
		\end{itemize}
	\end{frame}

\end{document}
