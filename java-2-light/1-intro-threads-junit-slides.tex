\documentclass[xetex,mathserif,serif]{beamer}
\usepackage{polyglossia}
\setdefaultlanguage[babelshorthands=true]{russian}
\usepackage{minted}
\usepackage{tabu}

\useoutertheme{infolines}

\usepackage{fontspec}
\setmainfont{FreeSans}
\newfontfamily{\russianfonttt}{FreeSans}

\definecolor{links}{HTML}{2A1B81}
\hypersetup{colorlinks,linkcolor=,urlcolor=links}

\tabulinesep=0.7mm

\title{Практика 1: Введение, потоки, продвинутый JUnit}
\author[Юрий Литвинов]{Юрий Литвинов \newline \textcolor{gray}{\small\texttt{yurii.litvinov@gmail.com}}}

\date{16.02.2018г}

\begin{document}
	
	\frame{\titlepage}

	\begin{frame}
		\frametitle{Правила игры}
		\begin{itemize}
			\item Пара один раз в две недели
			\begin{itemize}
				\item $=>$ вдвое больше домашки за один раз
			\end{itemize}
			\item Как обычно, куча домашек, две контрольные (и переписывание в конце), баллы и дедлайны, HwProj
			\begin{itemize}
				\item Задач будет немного меньше, но они немного объёмнее
			\end{itemize}
			\item За практику будет выставляться оценка, которая потом будет учитываться при сдаче экзамена
			\begin{itemize}
				\item по какой-то хитрой формуле, учитывающей баллы за домашку и контрольные
				\item \textcolor{gray}{Максимальный итоговый балл: 1.5, складывается из балла за контрольные (макс. 0.75) и балла за домашки (макс. 0.75). Максимум за к/р --- 16 баллов, максимум за д/з пока не известен}
			\end{itemize}
		\end{itemize}
	\end{frame}

	\begin{frame}
		\frametitle{Напоминание про штрафы}
		\begin{scriptsize}
			\begin{tabu} {| X[1 l p] | X[0.3 l p] |}
				\tabucline-
				\everyrow{\tabucline-}
				Пропущенный дедлайн                                                                   & баллы делятся на два \\
				Задача на момент дедлайна не реализует все требования условия                         & пропорционально объёму невыполненных требований \\
				Неумение пользоваться гитом                                                           & -2 \\
				Проблемы со сборкой (в том числе, забытый org.jetbrains.annotations)                  & -2 \\
				Отсутствие JavaDoc-ов для всех классов, интерфейсов и паблик-методов                  & -2 \\
				Отсутствие описания метода в целом                                                    & -1 \\
				Слишком широкие области видимости для полей                                           & -2 \\
				if (...) return true; else return false;                                              & -2 \\
				Именование классов-полей-методов-... и прочие code convertions                        & -1 \\
				Неиспользование try-with-resources там, где это было бы уместно                       & -1 \\
				Комментарии для параметров с заглавной буквы                                          & -0.5 \\
			\end{tabu}
		\end{scriptsize}
		\begin{center}
			\scriptsize{Список может расширяться!}
		\end{center}
	\end{frame}


\end{document}
