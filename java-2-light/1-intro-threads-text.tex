\documentclass[a5paper]{article}
\usepackage[a5paper, top=8mm, bottom=8mm, left=8mm, right=8mm]{geometry}

\usepackage{polyglossia}
\setdefaultlanguage[babelshorthands=true]{russian}

\usepackage{fontspec}
\setmainfont{FreeSerif}
\newfontfamily{\russianfonttt}[Scale=0.7]{DejaVuSansMono}

\usepackage[font=scriptsize]{caption}

\usepackage{amsmath}
\usepackage{amssymb,amsfonts,textcomp}
\usepackage{color}
\usepackage{array}
\usepackage{hhline}
\usepackage{cite}
\usepackage{ulem}

\usepackage[xetex,linktocpage=true,plainpages=false,pdfpagelabels=false]{hyperref}
\hypersetup{colorlinks=true, linkcolor=blue, citecolor=blue, filecolor=blue, urlcolor=blue, pdftitle=1, pdfauthor=, pdfsubject=, pdfkeywords=}

\usepackage{tabu}

\usepackage{graphicx}
\usepackage{indentfirst}
\usepackage{multirow}
\usepackage{subfig}
\usepackage{footnote}
\usepackage{listings}

\sloppy
\pagestyle{plain}

\title{Практика 1: Введение, потоки, продвинутый JUnit}

\date{16.02.2018г}

\begin{document}

\maketitle
\thispagestyle{empty}

\section{Формальности}
Итак, начинается вторая часть изучения программирования на примере языка Java. Пары в этом семестре будут логическим продолжением того, что было в предыдущем семестре --- немножко больше многопоточности, программирования сетевых приложений, многопоточных И сетевых приложений, приложений с пользовательским интерфейсом наконец-то, и даже немного веб-приложений, если мы до них дойдём. Как я понял, так или иначе почти всё затрагивалось при программировании под android, но в этот раз всё будет более системным и обстоятельным.

Сначала, как водится в начале семестра, формальности. В конце будет экзамен, оценка за который получается, в том числе, и из оценки по практике, так что в этом семестре будет не просто зачёт/незачёт, а некоторое число. Чтобы получить оную оценку, надо, как обычно, качественно и вовремя делать домашки, писать контрольные, сдавать их через HwProj (\url{http://hwproj.me/courses/26}), делая пуллреквест в свой репозиторий и кидая в HwProj ссылку на пуллреквест. В HwProj надо записаться на второй семестр этого курса. Домашек будет меньше, чем в прошлом семестре, но они будут более объёмными. Списывать, как обычно, нельзя, и вообще, имеет смысл стараться решать задачи самостоятельно, курс ориентирован на сольное прохождение, к тому же, если что --- всегда можно спросить у меня.

Пары у нас будут один раз в две недели, поэтому будет меньше времени практиковаться, но домашек особо меньше не будет. На самом деле, на парах я в основном буду что-то рассказывать.

Напомню про то, за что снимались баллы в прошлом семестре:\nopagebreak

\vspace{3mm}
\begin{small}
	\begin{tabu} {| X[1 l p] | X[0.3 l p] |}
		\tabucline-
		\everyrow{\tabucline-}
		Пропущенный дедлайн                                                                   & баллы делятся на два \\
		Задача на момент дедлайна не реализует все требования условия                         & пропорционально объёму невыполненных требований \\
		Неумение пользоваться гитом                                                           & -2 \\
		Проблемы со сборкой (в том числе, забытый org.jetbrains.annotations)                  & -2 \\
		Отсутствие JavaDoc-ов для всех классов, интерфейсов и паблик-методов                  & -2 \\
		Отсутствие описания метода в целом                                                    & -1 \\
		Слишком широкие области видимости для полей                                           & -2 \\
		if (...) return true; else return false;                                              & -2 \\
		Именование классов-полей-методов-... и прочие code convertions                        & -1 \\
		Неиспользование try-with-resources там, где это было бы уместно                       & -1 \\
		Комментарии для параметров с заглавной буквы                                          & -0.5 \\
	\end{tabu}
\end{small}
\vspace{3mm}

Обнаружение ошибок из этого списка сразу влечёт снятие баллов за задачу, даже без права исправить ошибку. На остальные ошибки я буду указывать и будет возможность их поправить, но наиболее распространённые ошибки будут пополнять этот список, так что имеет смысл ходить на пары, чтобы вовремя узнать, что меня в очередной раз ужаснуло и заставило расширить список ``плохих'' ошибок. Табличку с оценками я выложу на вики.

\section{Многопоточность}

Теперь перейдём к содержательной части пары. На теории вы уже должны были начать потоки, поэтому я ещё немного про них расскажу и в домашке надо будет попрактиковаться.

\end{document}
