\documentclass[xetex,mathserif,serif]{beamer}
\usepackage{polyglossia}
\setdefaultlanguage[babelshorthands=true]{russian}
\usepackage{minted}
\usepackage{tabu}

\useoutertheme{infolines}

\usepackage{fontspec}
\setmainfont{FreeSans}
\newfontfamily{\russianfonttt}{FreeSans}

\definecolor{links}{HTML}{2A1B81}
\hypersetup{colorlinks,linkcolor=,urlcolor=links}

\tabulinesep=0.7mm

\newcommand{\attribution}[1] {
    \vspace{-5mm}\begin{flushright}\begin{scriptsize}\textcolor{gray}{\textcopyright\, #1}\end{scriptsize}\end{flushright}
}

\title{Практика 5: Порождающие шаблоны и Roguelike}
\author[Юрий Литвинов]{Юрий Литвинов \newline \textcolor{gray}{\small\texttt{yurii.litvinov@gmail.com}}}

\date{28.04.2022}

\begin{document}
    
    \frame{\titlepage}

    \begin{frame}
        \frametitle{Задачи на остаток пары}
        Уточнить модель компьютерной игры Roguelike:

        \begin{enumerate}
            \item Используя шаблон ``Строитель'' для инициализации карты
            \item Используя шаблон ``Абстрактная фабрика'' для создания мобов и предметов на карте
            \item Используя шаблон ``Прототип'' для поддержки клонирования персонажей и предметов
        \end{enumerate}

        Выложить модифицированные диаграммы классов на HwProj
    \end{frame}

    \begin{frame}
        \frametitle{Что делать}
        \begin{itemize}
            \item Заполнить форму \url{https://forms.gle/jWZZG4UQpSiT7Cv37} ссылками на проект и командный чат
            \begin{itemize}
                \item Сделать это в самом начале работы
            \end{itemize}
            \item Посматривать в общий чат в Skype
            \item За 10 минут до конца пары собираемся в общем чате и представляем результаты
        \end{itemize}
    \end{frame}

\end{document}
