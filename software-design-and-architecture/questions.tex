\documentclass[a5paper]{article}
\usepackage[a5paper, top=8mm, bottom=8mm, left=8mm, right=8mm]{geometry}

\usepackage{polyglossia}
\setdefaultlanguage[babelshorthands=true]{russian}

\usepackage{fontspec}
\setmainfont{FreeSerif}
\newfontfamily{\russianfonttt}[Scale=0.7]{DejaVuSansMono}

\usepackage[font=scriptsize]{caption}

\sloppy
\pagestyle{plain}

\title{Вопросы к зачёту <<Проектирование и архитектура ПО>>}
\author{Юрий Литвинов\\\small{y.litvinov@spbu.ru}}

\begin{document}

\thispagestyle{empty}

\section*{Вопросы к зачёту <<Проектирование и архитектура программного обеспечения>>}

\begin{flushright}\begin{small}Юрий Литвинов\\\small{y.litvinov@spbu.ru}\end{small}\end{flushright}

\begin{enumerate}
    \item Понятие архитектуры, профессия <<Архитектор>>.
    \item Архитектурные виды.
    \item Роль архитектуры в жизненном цикле программного обеспечения.
    \item Понятие декомпозиции. Модульность, связность, сопряжение, сложность.
    \item Понятия класса и объекта, абстракция, инкапсуляция, наследование.
    \item Принципы выделения абстракций предметной области.
    \item Принципы SOLID, закон Деметры.
    \item Моделирование, визуальные модели, виды моделей.
    \item Язык UML. Проектирование структуры системы, диаграммы классов.
    \item Диаграммы объектов, диаграммы пакетов UML.
    \item Диаграммы компонентов, диаграммы развёртывания UML.
    \item Диаграмма случаев использования UML.
    \item Диаграмма активностей UML.
    \item Диаграммы конечных автоматов UML.
    \item Диаграммы последовательностей UML.
    \item Диаграммы коммуникации UML.
    \item Диаграммы составных структур, коопераций, временные диаграммы.
    \item Диаграммы обзора взаимодействия, диаграммы потоков данных.
    \item Диаграммы <<Сущность-связь>>.
    \item Понятие архитектурного стиля, трёхзвенная архитектура.
    \item Model-View-Controller, Sense-Compute-Control.
    \item Слоистый стиль, <<Клиент-сервер>>.
    \item Гексагональная архитектура, луковая архитектура.
    \item Чистая архитектура.
    \item Пакетная обработка, каналы и фильтры. 
    \item Стиль Blackboard.
    \item Событийно-ориентированные стили, Publish-Subscribe, событийная шина.
    \item Понятие Domain-Driven Design, единый язык.
    \item Изоляция предметной области в DDD, антипаттерн <<Умный GUI>>.
    \item DDD, основные структурные элементы модели предметной области.
    \item DDD, паттерн <<Агрегат>>.
    \item DDD, паттерны <<Фабрика>>, <<Репозиторий>>.
    \item DDD, паттерн <<Спецификация>>.
    \item Паттерн <<Компоновщик>>.
    \item Паттерн <<Декоратор>>.
    \item Паттерн <<Стратегия>>.
    \item Паттерн <<Адаптер>>.
    \item Паттерн <<Заместитель>>.
    \item Паттерн <<Фасад>>.
    \item Паттерн <<Мост>>.
    \item Паттерн <<Приспособленец>>.
    \item Паттерн <<Фабричный метод>>.
    \item Паттерн <<Абстрактная фабрика>>.
    \item Паттерн <<Одиночка>>.
    \item Паттерн <<Прототип>>.
    \item Паттерн <<Строитель>>.
    \item Паттерн <<Наблюдатель>>.
    \item Паттерн <<Шаблонный метод>>.
    \item Паттерн <<Посредник>>.
    \item Паттерн <<Команда>>.
    \item Паттерн <<Цепочка ответственности>>.
    \item Паттерн <<Состояние>>.
    \item Паттерн <<Посетитель>>.
    \item Паттерн <<Хранитель>>.
    \item Паттерн <<Интерпретатор>>.
    \item Паттерн <<Итератор>>.
    \item Понятие распределённой системы, заблуждения при проектировании распределённых систем.
    \item RPC, RMI. Пример: gRPC.
    \item Веб-сервисы, SOAP. WCF.
    \item Веб-сервисы, REST. ASP.NET Web APIs.
    \item Архитектурные стили распределённых приложений: Big Compute, Big Data.
    \item Web-queue-worker, N-звенная архитектура.
    \item Микросервисная архитектура.
\end{enumerate}

\end{document}
