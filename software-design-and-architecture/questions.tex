\documentclass[a5paper]{article}
\usepackage[a5paper, top=8mm, bottom=8mm, left=8mm, right=8mm]{geometry}

\usepackage{polyglossia}
\setdefaultlanguage[babelshorthands=true]{russian}

\usepackage{fontspec}
\setmainfont{FreeSerif}
\newfontfamily{\russianfonttt}[Scale=0.7]{DejaVuSansMono}

\usepackage[font=scriptsize]{caption}

\usepackage{amsmath}
\usepackage{amssymb,amsfonts,textcomp}
\usepackage{color}
\usepackage{array}
\usepackage{hhline}
\usepackage{cite}
\usepackage{textcomp}

\usepackage[hang,multiple]{footmisc}
\renewcommand{\footnotelayout}{\raggedright}

\PassOptionsToPackage{hyphens}{url}\usepackage[xetex,linktocpage=true,plainpages=false,pdfpagelabels=false]{hyperref}
\hypersetup{colorlinks=true, linkcolor=blue, citecolor=blue, filecolor=blue, urlcolor=blue, pdftitle=1, pdfauthor=, pdfsubject=, pdfkeywords=}

\newlength\Colsep
\setlength\Colsep{10pt}

\usepackage{tabu}

\usepackage{graphicx}
\usepackage{indentfirst}
\usepackage{multirow}
\usepackage{subfig}
\usepackage{footnote}
\usepackage{minted}

\sloppy
\pagestyle{plain}

\title{Вопросы к зачёту ``Проектирование и архитектура ПО''}
\author{Юрий Литвинов\\\small{yurii.litvinov@gmail.com}}

\begin{document}

\thispagestyle{empty}

\section*{Вопросы к зачёту ``Проектирование и архитектура программного обеспечения''}

\begin{flushright}\begin{small}Юрий Литвинов\\\small{yurii.litvinov@gmail.com}\end{small}\end{flushright}

\begin{enumerate}
    \item Понятие архитектуры, профессия ``Архитектор''.
    \item Архитектурные виды.
    \item Роль архитектуры в жизненном цикле ПО.
    \item Понятие декомпозиции. Модульность, связность, сопряжение, сложность.
    \item Понятия класса и объекта, абстракция, инкапсуляция, наследование.
    \item Принципы выделения абстракций предметной области.
    \item Принципы SOLID.
    \item Закон Деметры. Некоторые принципы хорошего объектно-ориентированного кода.
    \item Моделирование, визуальные модели, виды моделей.
    \item Язык UML. Проектирование структуры системы, диаграммы классов.
    \item Диаграммы объектов, диаграммы пакетов UML.
    \item Диаграммы компонентов, диаграммы развёртывания UML.
    \item Диаграмма случаев использования UML.
    \item Диаграмма активностей UML.
    \item Диаграммы конечных автоматов UML.
    \item Диаграммы последовательностей UML.
    \item Диаграммы коммуникации UML.
    \item Диаграммы составных структур, коопераций, временные диаграммы.
    \item Диаграммы обзора взаимодействия, диаграммы потоков данных.
    \item Диаграммы ``Сущность-связь''.
    \item Понятие архитектурного стиля, трёхзвенная архитектура, 
    \item Model-View-Controller, Sense-Compute-Control.
    \item Структурный и объектно-ориентированный стили, слоистые архитектурные стили.
    \item Пакетная обработка, каналы и фильтры, Blackboard.
    \item Стили с неявным вызовом, Publish-Subscribe.
    \item Peer-to-peer, C2, CORBA.
    \item Понятие Domain-Driven Design, единый язык, изоляция предметной области.
    \item Основные структурные элементы модели предметной области.
    \item Паттерны ``Агрегат'', ``Фабрика'', ``Репозиторий''.
    \item Моделирование ограничений, паттерн ``Спецификация''.
    \item Паттерн ``Компоновщик''.
    \item Паттерн ``Декоратор''.
    \item Паттерн ``Стратегия''.
    \item Паттерн ``Адаптер''.
    \item Паттерн ``Прокси''.
    \item Паттерн ``Фасад''.
    \item Паттерн ``Мост''.
    \item Паттерн ``Приспособленец''.
    \item Паттерн ``Спецификация''.
    \item Паттерн ``Фабричный метод''.
    \item Паттерн ``Шаблонный метод''.
    \item Паттерн ``Абстрактная фабрика''.
    \item Паттерн ``Одиночка''.
    \item Паттерн ``Прототип''.
    \item Паттерн ``Строитель''.
    \item Паттерн ``Посредник''.
    \item Паттерн ``Команда''.
    \item Паттерн ``Цепочка ответственности''.
    \item Паттерн ``Наблюдатель''.
    \item Паттерн ``Состояние''.
    \item Паттерн ``Посетитель''.
    \item Паттерн ``Хранитель''.
    \item Архитектура распределённых систем: понятие распределённой системы, типичные архитектурные стили.
    \item Межпроцессное сетевое взаимодействие, модель OSI, стек протоколов TCP/IP, сокеты, протоколы ``запрос-ответ''.
    \item Удалённые вызовы процедур (RPC), удалённые вызовы методов (RMI). Protobuf, gRPC.
    \item Веб-сервисы, SOAP. WCF.
    \item Очереди сообщений, RabbitMQ.
    \item REST.
    \item Микросервисы, peer-to-peer.
    \item Развёртывание и балансировка нагрузки, Docker.
\end{enumerate}

\end{document}
