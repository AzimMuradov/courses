\documentclass[xetex,mathserif,serif]{beamer}
\usepackage{polyglossia}
\setdefaultlanguage[babelshorthands=true]{russian}
\usepackage{minted}
\usepackage{tabu}

\useoutertheme{infolines}

\usepackage{fontspec}
\setmainfont{FreeSans}
\newfontfamily{\russianfonttt}{FreeSans}

\definecolor{links}{HTML}{2A1B81}
\hypersetup{colorlinks,linkcolor=,urlcolor=links}

\tabulinesep=0.7mm

\newcommand{\attribution}[1] {
    \vspace{-5mm}\begin{flushright}\begin{scriptsize}\textcolor{gray}{\textcopyright\, #1}\end{scriptsize}\end{flushright}
}

\title{Практика 4: Структурные шаблоны и Roguelike}
\author[Юрий Литвинов]{Юрий Литвинов \newline \textcolor{gray}{\small\texttt{yurii.litvinov@gmail.com}}}

\date{21.04.2022}

\begin{document}
    
    \frame{\titlepage}

    \begin{frame}
        \frametitle{Задачи на остаток пары}
        Уточнить модель компьютерной игры Roguelike с предыдущего занятия:

        \begin{enumerate}
            \item Используя шаблон ``Стратегия'' для поддержки различных поведений мобов
            \begin{itemize}
                \item Агрессивное поведение, атакуют игрока, как только его видят
                \item Пассивное поведение, просто стоят на месте
                \item Трусливое поведение, стараются держаться на расстоянии от игрока
            \end{itemize}
            \item Используя шаблон ``Декоратор'' для поддержки временных эффектов, накладываемых на мобов и игрока
            \begin{itemize}
                \item Эффект конфузии, заставляющий персонажа двигаться в случайном направлении
                \item Возможность добавить другие похожие эффекты
            \end{itemize}
        \end{enumerate}

        Выложить модифицированные диаграммы как решения соответствующих заданий в Teams
    \end{frame}

\end{document}
