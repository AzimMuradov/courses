\documentclass{../../slides-style}

\slidetitle{Практика 1: Проектирование системы контроля версий}{27.02.2025}

\begin{document}
    
    \begin{frame}[plain]
        \titlepage
    \end{frame}

    \section{Задача}

    \begin{frame}
        \frametitle{Задача на практику}
        В командах по 2-3 человека спроектировать систему контроля версий:
        \begin{itemize}
            \item commit с commit message, датой коммита и автором
            \item Работу с ветками: создание и удаление
            \item checkout по имени ревизии или ветки
            \item log --- список ревизий вместе с commit message в текущей ветке
            \item merge --- сливает указанную ветку с текущей
            \item Консольное приложение
        \end{itemize}
    \end{frame}

    \begin{frame}
        \frametitle{Что надо сделать}
        \begin{itemize}
            \item Диаграмму компонентов и классов, пока как умеете
            \item Требуется некая структура, не просто клубок классов
            \item Нельзя подсматривать в Git Book
            \item В конце пары разберём пару решений
        \end{itemize}
    \end{frame}

    \begin{frame}
        \frametitle{Обратите внимание на}
        \begin{itemize}
            \item Как представляются файлы, коммиты, ветки, репозиторий?
            \item Как выполняется компрессия и выполняется ли вообще? 
            \item Насколько просто получить текущую, предыдущую, произвольную версии?
            \item Каков жизненный цикл файла?
            \item Как выполняется работа с файловой системой?
            \item Как выполняется работа с пользователем? Как представляются команды?
        \end{itemize}
        Опасайтесь архитектурной жадности
    \end{frame}

\end{document}