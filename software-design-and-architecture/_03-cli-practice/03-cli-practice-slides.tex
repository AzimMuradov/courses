\documentclass{../../slides-style}

\slidetitle{Практика 1: Command Line Interface}{28.02.2023}

\begin{document}

    \begin{frame}[plain]
        \titlepage
    \end{frame}

    \begin{frame}
        \frametitle{Задача про CLI}
        Спроектировать простой интерпретатор командной строки, поддерживающий команды:
        \begin{itemize}
            \item \textbf{cat [FILE]} --- вывести на экран содержимое файла
            \item \textbf{echo} --- вывести на экран свой аргумент (или аргументы)
            \item \textbf{wc [FILE]} --- вывести количество строк, слов и байт в файле
            \item \textbf{pwd} --- распечатать текущую директорию
            \item \textbf{exit} --- выйти из интерпретатора
        \end{itemize}
    \end{frame}
    
    \begin{frame}
        \frametitle{Задача про CLI (продолжение)}
        \begin{itemize}
            \item Должны поддерживаться одинарные и двойные кавычки (full and weak quoting)
            \item Окружение (команды вида ``имя=значение''), оператор \$
            \item Вызов внешней программы как отдельного процесса
            \begin{itemize}
                \item если введено что-то, чего интерпретатор не знает
            \end{itemize}
            \item Пайплайны (оператор ``|'')
        \end{itemize}
    \end{frame}
    
    \begin{frame}[fragile]
        \frametitle{Примеры}
\begin{minted}{sh}
>echo "Hello, world!"
Hello, world!
> FILE=example.txt
> cat $FILE
Some example text
> cat example.txt | wc
1 3 18
> echo 123 | wc
1 1 3
> x=exit
> $x
        \end{minted}
    \end{frame}

    \begin{frame}
        \frametitle{Что делать}
        Первые фазы жизненного цикла
        \begin{itemize}
            \item Выполнить анализ и определить подходы к решению
            \item Выявить подводные камни и способы их преодоления
            \item Декомпозировать задачу на подсистемы, классы и методы
            \item Нарисовать диаграмму классов
            \item Словами описать принцип работы и основные принятые решения
            \begin{itemize}
                \item Приложить к решению в Teams диаграмму классов в любом удобном формате
                \item \url{https://www.diagrams.net/}
                \item Если не защитите в конце пары, приложить также 1-2 страницы текстового описания
            \end{itemize}
        \end{itemize}
    \end{frame}

    \begin{frame}
        \frametitle{Соображения}
        \begin{itemize}
            \item Проектирование сверху вниз
            \begin{itemize}
                \item Определитесь с общей структурой системы
                \item Определитесь с компонентами, их ответственностью и связями между ними
                \item Только после этого переходите к проектированию компонентов
                \begin{itemize}
                    \item По такой же схеме
                \end{itemize}
                \item Возможно, придётся возвращаться на уровень выше
            \end{itemize}
            \item Опасайтесь архитектурной жадности, надо вовремя остановиться
        \end{itemize}
    \end{frame}

    \begin{frame}
        \frametitle{На что обратить внимание}
        \begin{itemize}
            \item Как представляются команды и пайплайны?
            \item Как создаются команды?
            \item Как они исполняются? Как взаимодействуют потоки в пайплайне?
            \item Кто и как выполняет разбор входной строки?
            \begin{itemize}
                \item Кто, как и когда выполняет подстановки?
            \end{itemize}
            \item Как представляются переменные окружения?
            \item Что с многопоточностью?
        \end{itemize}
    \end{frame}

\end{document}
