\documentclass[xetex,mathserif,serif]{beamer}
\usepackage{polyglossia}
\setdefaultlanguage[babelshorthands=true]{russian}
\usepackage{minted}
\usepackage{tabu}

\useoutertheme{infolines}

\usepackage{fontspec}
\setmainfont{FreeSans}
\newfontfamily{\russianfonttt}{FreeSans}

\definecolor{links}{HTML}{2A1B81}
\hypersetup{colorlinks,linkcolor=,urlcolor=links}

\tabulinesep=0.7mm

\title{Практика 1: Command Line Interface}
\author[Юрий Литвинов]{Юрий Литвинов \newline \textcolor{gray}{\small\texttt{yurii.litvinov@gmail.com}}}

\date{25.02.2020г}

\begin{document}

	\frame{\titlepage}

	\begin{frame}
		\frametitle{Задача про CLI}
		Спроектировать простой интерпретатор командной строки, поддерживающий команды:
		\begin{itemize}
			\item \textbf{cat [FILE]} --- вывести на экран содержимое файла
			\item \textbf{echo} --- вывести на экран свой аргумент (или аргументы)
			\item \textbf{wc [FILE]} --- вывести количество строк, слов и байт в файле
			\item \textbf{pwd} --- распечатать текущую директорию
			\item \textbf{exit} --- выйти из интерпретатора
		\end{itemize}
	\end{frame}
	
	\begin{frame}
		\frametitle{Задача про CLI (продолжение)}
		\begin{itemize}
			\item Должны поддерживаться одинарные и двойные кавычки (full and weak quoting)
			\item Окружение (команды вида ``имя=значение''), оператор \$
			\item Вызов внешней программы как отдельного процесса
			\begin{itemize}
				\item если введено что-то, чего интерпретатор не знает
			\end{itemize}
			\item Пайплайны (оператор ``|'')
		\end{itemize}
	\end{frame}
	
	\begin{frame}[fragile]
		\frametitle{Примеры}
\begin{minted}{sh}
>echo "Hello, world!"
Hello, world!
> FILE=example.txt
> cat $FILE
Some example text
> cat example.txt | wc
1 3 18
> echo 123 | wc
1 1 3
> x=exit
> $x
		\end{minted}
	\end{frame}

	\begin{frame}
		\frametitle{Нефункциональные требования}
		\begin{itemize}
			\item Легко добавлять новые команды (расширяемость)
			\item Наличие возможности реализовать что-то новое из того, что умеют другие шеллы (сопровождаемость)
			\item Архитектурное описание, как умеете (сопровождаемость)
		\end{itemize}
	\end{frame}
	
	\begin{frame}
		\frametitle{Что делать}
		Первые фазы жизненного цикла
		\begin{itemize}
			\item Выполнить анализ и определить подходы к решению
			\item Выявить подводные камни и способы их преодоления
			\item Декомпозировать задачу на подсистемы, классы и методы
			\item Нарисовать диаграмму классов
			\item Словами описать принцип работы и основные принятые решения
			\begin{itemize}
				\item Сдавать на HwProj только диаграмму классов, в любом удобном формате
			\end{itemize}
		\end{itemize}
	\end{frame}

	\begin{frame}
		\frametitle{Соображения}
		\begin{itemize}
			\item Проектирование сверху вниз
			\begin{itemize}
				\item Определитесь с общей структурой системы
				\item Определитесь с компонентами, их ответственностью и связями между ними
				\item Только после этого переходите к проектированию компонентов
				\begin{itemize}
					\item По такой же схеме
				\end{itemize}
				\item Возможно, придётся возвращаться на уровень выше
			\end{itemize}
			\item Опасайтесь архитектурной жадности, надо вовремя остановиться
		\end{itemize}
	\end{frame}

	\begin{frame}
		\frametitle{На что обратить внимание}
		\begin{itemize}
			\item Как представляются команды и пайплайны?
			\item Как создаются команды?
			\item Как они исполняются? Как взаимодействуют потоки в пайплайне?
			\item Кто и как выполняет разбор входной строки?
			\begin{itemize}
				\item Кто, как и когда выполняет подстановки?
			\end{itemize}
			\item Как представляются переменные окружения?
			\item Что с многопоточностью?
		\end{itemize}
	\end{frame}

\end{document}
