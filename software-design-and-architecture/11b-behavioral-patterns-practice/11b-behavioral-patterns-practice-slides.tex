\documentclass[xetex,mathserif,serif]{beamer}
\usepackage{polyglossia}
\setdefaultlanguage[babelshorthands=true]{russian}
\usepackage{minted}
\usepackage{tabu}

\useoutertheme{infolines}

\usepackage{fontspec}
\setmainfont{FreeSans}
\newfontfamily{\russianfonttt}{FreeSans}

\definecolor{links}{HTML}{2A1B81}
\hypersetup{colorlinks,linkcolor=,urlcolor=links}

\tabulinesep=0.7mm

\newcommand{\attribution}[1] {
	\vspace{-5mm}\begin{flushright}\begin{scriptsize}\textcolor{gray}{\textcopyright\, #1}\end{scriptsize}\end{flushright}
}

\title{Практика 7: Поведенческие шаблоны и Roguelike}
\author[Юрий Литвинов]{Юрий Литвинов \newline \textcolor{gray}{\small\texttt{yurii.litvinov@gmail.com}}}

\date{21.04.2020г}

\begin{document}
	
	\frame{\titlepage}

	\begin{frame}
		\frametitle{Задачи на остаток пары}
		Уточнить модель компьютерной игры Roguelike:

		\begin{enumerate}
			\item Используя шаблон ``Команда'' для поддержки взаимодействия с пользователем
			\item Собрать воедино все модификации и выложить итоговую архитектуру в виде набора диаграмм компонент и классов
		\end{enumerate}

		Выложить всё на HwProj
	\end{frame}

\end{document}
