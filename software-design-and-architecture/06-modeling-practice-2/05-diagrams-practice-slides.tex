\documentclass[xetex,mathserif,serif]{beamer}
\usepackage{polyglossia}
\setdefaultlanguage[babelshorthands=true]{russian}
\usepackage{minted}
\usepackage{tabu}

\useoutertheme{infolines}

\usepackage{fontspec}
\setmainfont{FreeSans}
\newfontfamily{\russianfonttt}{FreeSans}

\definecolor{links}{HTML}{2A1B81}
\hypersetup{colorlinks,linkcolor=,urlcolor=links}

\tabulinesep=0.7mm

\newcommand{\attribution}[1] {
	\vspace{-5mm}\begin{flushright}\begin{scriptsize}\textcolor{gray}{\textcopyright\, #1}\end{scriptsize}\end{flushright}
}

\title{Практика 3: Практика по рисованию диаграмм}
\author[Юрий Литвинов]{Юрий Литвинов \newline \textcolor{gray}{\small\texttt{yurii.litvinov@gmail.com}}}

\date{10.03.2020г}

\begin{document}
	
	\frame{\titlepage}

	\begin{frame}
		\frametitle{Задачи на пару}
		\begin{itemize}
			\item Вспомнить запрос \url{https://goo.gl/MiyH8c}
			\item Задача 1: Построить модель данных разрабатываемого приложения в виде диаграммы классов
			\item Задача 2: Нарисовать диаграмму компонентов разрабатываемого приложения
			\begin{itemize}
				\item По сути, первое приближение архитектуры
			\end{itemize}
			\item Задача 3: Нарисовать диаграмму развёртывания разрабатываемого приложения
			\item Отчуждаемый результат --- ссылка на проект с диаграммой на HwProj, либо пара <<исходник диаграммы --- картинка>>
		\end{itemize}
	\end{frame}

\end{document}
