\documentclass[xetex,mathserif,serif]{beamer}
\usepackage{polyglossia}
\setdefaultlanguage[babelshorthands=true]{russian}
\usepackage{minted}
\usepackage{tabu}

\useoutertheme{infolines}

\usepackage{fontspec}
\setmainfont{FreeSans}
\newfontfamily{\russianfonttt}{FreeSans}

\definecolor{links}{HTML}{2A1B81}
\hypersetup{colorlinks,linkcolor=,urlcolor=links}

\tabulinesep=0.7mm

\newcommand{\attribution}[1] {
	\vspace{-5mm}\begin{flushright}\begin{scriptsize}\textcolor{gray}{\textcopyright\, #1}\end{scriptsize}\end{flushright}
}

\title{Практика 4: Структурные шаблоны и Roguelike}
\author[Юрий Литвинов]{Юрий Литвинов \newline \textcolor{gray}{\small\texttt{yurii.litvinov@gmail.com}}}

\date{14.04.2020г}

\begin{document}
	
	\frame{\titlepage}

	\begin{frame}
		\frametitle{Задачи на остаток пары}
		Уточнить модель компьютерной игры Roguelike с предыдущего занятия:

		\begin{enumerate}
			\item Используя шаблон ``Стратегия'' для поддержки различных поведений мобов
			\item Используя шаблон ``Декоратор'' для поддержки временных эффектов, накладываемых на мобов
		\end{enumerate}

		Выложить модифицированные диаграммы классов на HwProj

		\begin{itemize}
			\item Это может быть одна диаграмма сразу с двумя паттернами или две разных
		\end{itemize}
	\end{frame}

	\begin{frame}
		\frametitle{Что делать}
		\begin{itemize}
			\item Установить легкодоступный канал связи
			\begin{itemize}
				\item Лучше комната в Zoom
			\end{itemize}
			\item Заполнить форму \url{https://forms.gle/H4gCt7QGHJRL3CFFA} ссылками на проект и командный чат
			\begin{itemize}
				\item Сделать это в самом начале работы
			\end{itemize}
			\item Посматривать в общий чат в Skype
			\item За 10 минут до конца пары собираемся в общем чате и представляем результаты
		\end{itemize}
	\end{frame}

\end{document}
