\documentclass{../../slides-style}

\slidetitle[Лекция 1: Введение в робототехнику]{Алгоритмические основы робототехники}{21.02.2024}

\begin{document}

    \begin{frame}[plain]
        \titlepage
    \end{frame}

    \section{Организационное}

    \begin{frame}
        \frametitle{Организационное}
        \begin{itemize}
            \item Семинары с одной вводной лекцией, две пары в неделю
            \item В конце устный экзамен и аттестация по докладам
            \begin{itemize}
                \item Два вопроса без подготовки
            \end{itemize}
            \item Материалы лекций, темы докладов на \url{https://hwproj.ru}
            \item Балльная система
            \begin{itemize}
                \item 5 баллов за доклад
                \item 10 баллов за экзамен
                \item Итоговая оценка: $(\textrm{<сумма оценок>} -\ 5) * 10$
            \end{itemize}
            \item Коммуникации --- в команде курса в Teams, \url{}
            \begin{itemize}
                \item Также пишите в Teams в личку
            \end{itemize}
        \end{itemize}
    \end{frame}

    \begin{frame}
        \frametitle{Зачем этот курс}
        \begin{itemize}
            \item Робототехника в общем смысле~--- очень перспективна
            \begin{itemize}
                \item Беспилотные автомобили, БПЛА, \enquote{интернет вещей}
            \end{itemize}
            \item Курс~--- краткий обзор того, что вообще бывает, какая наука за этим стоит, чем можно заниматься в магистратуре
            \item Немного общего низкоуровневого программирования, что никогда не лишне
            \item Речь в основном про наземные мобильные роботы
            \item Фокус на алгоритмике:
            \begin{itemize}
                \item Не про паяльники и резьбу по дереву, не про теорию управления, не про искусственный интеллект
            \end{itemize}
            \item Кому нужны робототехники-программисты: \enquote{Сколтех}, ресурсный центр \enquote{Робототехника и БАС} СПбГУ, \enquote{Геоскан}, \enquote{Кибертех}, \dots
        \end{itemize}
    \end{frame}

    \begin{frame}
        \frametitle{Что будет в курсе}
        \begin{itemize}
            \item Кинематика мобильного робота: виды и конфигурации колёс, \enquote{стандартная} трёхколёсная тележка, другие варианты кинематики (в т.ч. шагающие роботы)
            \item Сенсорика: типы и физические принципы работы сенсоров, работа с ошибками измерений.
            \item Отдельно видеокамеры, стереокамеры, сенсоры глубины
            \item Алгоритмы машинного зрения, сегментация
            \item Локализация, behavior-driven алгоритмы, belief representation, представление карты
            \item SLAM
            \item Планирование и \enquote{стратегическая} навигация
            \item Аппаратные робототехнические платформы, от Arduino до Kuka
            \item Программные платформы, ROS
        \end{itemize}
    \end{frame}

    \section{Введение}


\end{document}
