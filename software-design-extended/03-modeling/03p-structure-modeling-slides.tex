\documentclass{../../slides-style}

\slidetitle{Практика 3: моделирование структуры}{03.03.2025}

\begin{document}
    
    \begin{frame}[plain]
        \titlepage
    \end{frame}

    \section{В чём рисовать диаграммы}

    \begin{frame}
        \frametitle{Инструменты для рисования диаграмм}
        \begin{itemize}
            \item ``Рисовалки''
            \begin{itemize}
                \item Visio
                \item Dia
                \item SmartDraw
                \item LucidChart
                \item \url{https://creately.com/}
                \item \url{https://diagrams.net/}
                \item \url{http://plantuml.com/}
            \end{itemize}
            \item Полноценные CASE-системы
            \begin{itemize}
                \item Enterprise Architect
                \item Rational Software Architect
                \item MagicDraw
                \item Visual Paradigm
                \item \url{https://www.genmymodel.com/}
            \end{itemize}
            \item Браузерные инструменты
            \begin{itemize}
                \item \url{https://www.websequencediagrams.com/}
                \item \url{http://yuml.me/}
            \end{itemize}
        \end{itemize}
    \end{frame}

    \section{Задания}

    \begin{frame}
        \frametitle{Задание на пару}
        Проанализировать запрос \url{https://bit.ly/defects-rfp}, построить по нему:
        \begin{enumerate}
            \item диаграмму компонентов требуемой системы, как вы её видите
            \item диаграмму классов, моделирующую данные, хранимые системой
            \item диаграмму объектов, иллюстрирующую типичные данные
        \end{enumerate}
    \end{frame}

\end{document}