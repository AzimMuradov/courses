\documentclass{../../text-style}

\texttitle{Практика 3: моделирование структуры}

\begin{document}

\maketitle
\thispagestyle{empty}

\section{Инструменты для рисования диаграмм}

Наверное, самый важный для практики вопрос --- это в чём вообще можно рисовать разные диаграммы. Выбор на самом деле очень велик, поскольку в своё время визуальные языки подавали надежды на революцию в разработке программного обеспечения --- революции не произошло, но рынок подобных инструментов активно развивается с середины девяностых. Инструменты можно условно разделить на три категории:

\begin{itemize}
    \item <<рисовалки>> --- не очень умные инструменты, которые позволяют удобно рисовать диаграммы и иногда немного генерировать по ним код, но не пытаются помогать с архитектурой или отладкой программы. Используются прежде всего как графические редакторы, специально заточенные под рисование диаграмм. Иногда люди увлекаются и начинают хотеть от них большего (например, генерации исполнимого кода по модели в Visio), но для этого есть лучшие альтернативы. Примеры таких инструментов:
    \begin{itemize}
        \item Microsoft Visio --- часть пакета Microsoft Office, на самом деле редактор диаграмм вообще, UML там один из десятков разных вариантов (от диаграмм из кружочков и стрелочек до планов помещений). Причём, UML, хоть и есть в стандартной поставке, там не очень продвинутый (некоторых элементов нотации не хватает), так что лучше отдельно поставить плагин с полноценной поддержкой UML (благо в Visio есть развитая плагинная система). Visio очень популярен в бизнес-среде, но платный, и работает только под Windows.
        \item Dia --- что-то вроде Visio для Linux. Как часто бывает в Linux, бесплатна, с открытым исходным кодом, есть в репозитории любого уважающего себя дистрибутива, имеет кучу плагинов (в том числе, поддержку UML), умеет генерировать код. Больше практически ничего не умеет, поэтому как настоящая среда разработки через модели не используется.
        \item SmartDraw --- рисовалка диаграмм вообще, не только программистских. Имеет десктопную и веб-версию, но платная.
        \item LucidChart --- примерно то же самое, несколько менее платное в том смысле, что сколько-то простых диаграмм на одного пользователя можно рисовать бесплатно. Имеет только веб-версию (и вроде как мобильные версии) и очень агрессивную рекламу.
        \item Creately --- простая, но относительно удобная веб-рисовалка. Рисует страшные как моя жизнь диаграммы, но если надо быстро что-то нарисовать без установки и длительного процесса регистрации, Creately вполне подойдёт.
        \item diagrams.net --- open-source-рисовалка, изначально создававшаяся как демо для библиотеки mxGraph, но дело пошло, и теперь mxGraph не поддерживается, а diagrams.net имеет коммерческий вариант, и по сути это вполне годный веб-редактор диаграмм. Как и все редакторы диаграмм, не очень удобен в работе, и не вполне соответствует стандарту UML, но в целом вполне достойный выбор.
    \end{itemize}
    \item CASE-системы --- то самое, что должно было произвести революцию в программировании --- среды полноценной разработки программ через визуальные модели. Как правило, платные, но часто имеют бесплатные Community-версии, поэтому рекомендую пользоваться именно этими штуками, а не <<рисовалками>>. Как правило, все такие штуки десктопные, кроссплатформенные, с несколько урезанной браузерной версией. Популярные примеры:
    \begin{itemize}
        \item Enterprise Architect --- довольно популярный в IT-индустрии инструмент, более-менее всё умеет (не только UML, но и BPMN, SysML и другие полезные штуки) и не очень дорог (от 300\$ за лицензию и без бесплатной версии, так что для бедных студентов или инди-разработчиков так себе, но даже для средних стартапов это копейки).
        \item Rational Software Architect --- бывший Rational Rose (самый первый инструмент, поддерживавший UML), переписанный на платформе Eclipse. Тоже более-менее всё умеет, зато дороговат и опять-таки без бесплатной версии. Ни разу не видел, чтобы его использовали при практической разработке ПО, возможно это как-то связано с крайним неудобством его официального сайта. Достоин упоминания из-за отличной поддержки архитектурных рефакторингов.
        \item MagicDraw --- говорят, довольно хорошая и довольно популярная CASE-система, но, опять-таки, не видел её в деле.
        \item Visual Paradigm --- сам ей пользуюсь и встречал в индустрии, очень рекомендую. Умеет очень много чего, но основное её достоинство --- это usability. И наличие Community-версии.
        \item GenMyModel --- скорее, очень продвинутая браузерная <<рисовалка>>, чем настоящая CASE-система, но умеет генерировать код, умеет UML, BPMN и ещё некоторые нотации, имеет репозиторий, так что попала именно в эту категорию. В отличие от перечисленных выше настоящих CASE-систем, вообще не имеет десктопной версии, зато бесплатна для личного использования. Рекомендую как продвинутую замену Creately/diagrams.net.
    \end{itemize}
    \item Прочие инструменты. Направлены в основном на быстрое иллюстрирование документации или веб-страниц чем-то, похожим на UML-диаграммы, позволяют текстом описать, что надо нарисовать. Наверное, будет приятно хардкорным кодерам, которые мышку в руках никогда не держали. Примеры (рекомендую покликать на ссылки, чтобы хотя бы знать, что так бывает):
    \begin{itemize}
        \item \url{https://www.websequencediagrams.com/} --- как намекает название, инструмент для рисования диаграмм последовательностей UML. Диаграммы описываются на очень простом текстовом языке, например, \verb|A->B: text| позволит нарисовать диаграмму с двумя объектами, один из которых шлёт другому сообщение <<text>>.
        \item \url{http://yuml.me/} --- генерирует по параметрам в URL картинки, которые можно вставлять на любую HTML-страницу. Например, \verb|<img src="http://yuml.me/diagram/scruffy/class/[Customer]->[Billing Address]" >| вставит диаграмму классов с двумя классами и ассоциацией между ними.
        \item \url{http://plantuml.com/} --- генерирует картинки (и даже ASCII-арт) по текстовому описанию. Например, \verb&Class01 <|-- Class02& сгенерирует диаграмму классов с двумя классами, один наследник другого. Не так удобно эти картинки куда-либо встраивать, зато может рисовать практически любые UML-диаграммы, и даже очень сложные, с десятками классов и кучей разных связей.
    \end{itemize}
\end{itemize}

\section{Задание на практику}

Проанализировать запрос \url{https://bit.ly/defects-rfp}, подумать над тем, как бы вы стали делать такую систему, и построить по нему:

\begin{enumerate}
    \item диаграмму компонентов требуемой системы, как вы её видите
    \item диаграмму классов, моделирующую данные, хранимые системой
    \item диаграмму объектов, иллюстрирующую типичные данные
\end{enumerate}

Сначала даётся время на ознакомление с запросом, после чего кто-то должен будет расшарить экран и нарисовать (при помощи аудитории и преподавателя) требуемую диаграмму.

\end{document}