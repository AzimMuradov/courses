\documentclass{../../slides-style}

\slidetitle[Практика 1: задача про CLI]{Архитектура и проектирование информационных систем (практика)}{17.02.2025}

\begin{document}
    
    \begin{frame}[plain]
        \titlepage
    \end{frame}

    \section{Задача про CLI}
    
    \begin{frame}
        \frametitle{Задача про CLI}
        В командах по два-три человека реализовать простой интерпретатор командной строки, поддерживающий команды:
        \begin{itemize}
            \item \textbf{cat [FILE]} --- вывести на экран содержимое файла
            \item \textbf{echo} --- вывести на экран свой аргумент (или аргументы)
            \item \textbf{wc [FILE]} --- вывести количество строк, слов и байт в файле
            \item \textbf{pwd} --- распечатать текущую директорию
            \item \textbf{exit} --- выйти из интерпретатора
        \end{itemize}
    \end{frame}
    
    \begin{frame}
        \frametitle{Задача про CLI (продолжение)}
        \begin{itemize}
            \item Должны поддерживаться одинарные и двойные кавычки (full and weak quoting)
            \item Окружение (команды вида ``имя=значение''), оператор \$
            \item Вызов внешней программы
            \begin{itemize}
                \item если введено что-то, чего интерпретатор не знает
            \end{itemize}
            \item Пайплайны (оператор ``|'')
        \end{itemize}
    \end{frame}
    
    \begin{frame}[fragile]
        \frametitle{Примеры}
        \begin{small}
            \begin{minted}{sh}
>echo "Hello, world!"
Hello, world!

> FILE=example.txt
> cat $FILE
Some example text

> cat example.txt | wc
1 3 18

> echo 123 | wc
1 1 3

> x=ex
> y=it
> $x$y
            \end{minted}
        \end{small}
    \end{frame}

    \begin{frame}
        \frametitle{Что ожидается в качестве решения}
        \begin{itemize}
            \item Архитектурная документация, как умеете
            \begin{itemize}
                \item Структурная диаграмма (классов, компонентов, квадратиков со стрелочками)
                \item Словесное описание работы системы
                \item Достаточно подробно, чтобы не требовалось принимать важные решения при кодировании
                \begin{itemize}
                    \item Не должно быть <<ну тут мы парсим строку>>
                \end{itemize}
            \end{itemize}
            \item Реализовывать проект пока не нужно
        \end{itemize}
    \end{frame}

    \begin{frame}
        \frametitle{Что делать дома}
        \begin{itemize}
            \item Завести для этого курса репозиторий
            \item Одному из членов команды выложить решение в виде .md или .pdf-файла в отдельную ветку
            \begin{itemize}
                \item Обязательно укажите, с кем в команде вы делали
            \end{itemize}
            \item Сделать пуллреквест к себе в основную ветку
            \item Ссылку на пуллреквест приложить в качестве решения в LMS-системе
            \item Смерджить пуллреквест, когда задача зачтена
            \item \textbf{Дедлайн: 23:59 12.09.2023}
        \end{itemize}
    \end{frame}

    \begin{frame}
        \frametitle{Что делать сейчас}
        Первые фазы жизненного цикла
        \begin{itemize}
            \item Разбиться на команды по два-три человека
            \item Выполнить анализ и определить подходы к решению
            \item Выявить подводные камни и способы их преодоления
            \item Декомпозировать задачу на подсистемы, классы и методы
            \item Нарисовать первое приближение структурной диаграммы
            \item Быть готовыми в конце пары рассказать предлагаемое решение
            \item Дома это надо будет уточнить, расширить и оформить
        \end{itemize}
    \end{frame}

    \begin{frame}
        \frametitle{Соображения}
        \begin{itemize}
            \item Проектирование сверху вниз
            \begin{itemize}
                \item Определитесь с общей структурой системы
                \item Определитесь с компонентами, их ответственностью и связями между ними
                \item Только после этого переходите к проектированию компонентов
                \begin{itemize}
                    \item По такой же схеме
                \end{itemize}
                \item Возможно, придётся возвращаться на уровень выше
            \end{itemize}
            \item Опасайтесь архитектурной жадности, надо вовремя остановиться
        \end{itemize}
    \end{frame}

    \begin{frame}
        \frametitle{На что обратить внимание}
        \begin{itemize}
            \item Как представляются команды и пайплайны?
            \item Как создаются команды?
            \item Как они исполняются? Как взаимодействуют потоки в пайплайне?
            \item Кто и как выполняет разбор входной строки?
            \begin{itemize}
                \item Кто, как и когда выполняет подстановки?
            \end{itemize}
            \item Как представляются переменные окружения?
            \item Что с многопоточностью?
        \end{itemize}
    \end{frame}

\end{document}
