\documentclass{../../slides-style}

\slidetitle{Практика 15: OAuth}{05.06.2023}

\begin{document}

    \begin{frame}[plain]
        \titlepage
    \end{frame}

    \begin{frame}
        \frametitle{Задача}
        Реализовать на ``голом HTTPS'' консольный клиент Google Drive, умеющий показать список файлов и папок в корне Google Drive пользователя
        \begin{itemize}
            \item Нельзя использовать библиотеку клиента Google Drive
            \item Придётся зарегистрировать приложение на \url{https://console.developers.google.com}
            \item И включить нужный API
            \item Авторизация: \url{https://developers.google.com/drive/api/v3/about-auth}
            \begin{itemize}
                \item Наш случай: \url{https://developers.google.com/identity/protocols/oauth2/native-app}
            \end{itemize}
            \item Надо добавить OAuth Client ID и получить Client ID и Client secret
        \end{itemize}
    \end{frame}

    \begin{frame}
        \frametitle{Подсказки}
        \begin{itemize}
            \item Запрос access token должен быть обязательно POST-запросом, GET вернёт 404
            \item Копипастить access token из адресной строки браузера good enough
            \begin{itemize}
                \item Да, надо будет программно запустить браузер с правильным URL
            \end{itemize}
            \item Некоторые значения возвращаются в BASE64-кодировке, а ожидаются в plain text
            \item Не надо продлять Access Token, правильно хранить Refresh Token и всё такое
            \begin{itemize}
                \item Но если успеете, то почему нет
            \end{itemize}
            \item Значения Client Id и Client Secret можно принимать параметрами командной строки
        \end{itemize}
    \end{frame}

\end{document}