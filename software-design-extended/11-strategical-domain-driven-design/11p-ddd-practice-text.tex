\documentclass{../mcstext}

\texttitle{Практика 11: Практика по DDD}

\begin{document}

\maketitle
\thispagestyle{empty}

\section{Задача, магазин книг}

В командах по 2-3 человека надо проанализировать запрос \url{ https://goo.gl/94LyFc} и построить модель приложения согласно <<чистой архитектуре>> и канонам предметно-ориентированного проектирования.

\begin{itemize}
    \item Обратите внимание, это <<сырой>> Request For Proposals, некоторые требования там к заданию не относятся. Их надо выявить и обсудить.
    \item Основное внимание надо уделить модели предметной области, внешние сущности (такие как средства доставки --- веб-интерфейс, интерфейс веб сервиса, или средства персистентности) должны упоминаться, но без деталей.
    \item Необходима <<каноничная>> для чистой архитектуры высокоуровневая структура и явно выделенное смысловое ядро.
\end{itemize}

Что надо постараться успеть сделать за занятие:

\begin{itemize}
    \item диаграмму компонентов и диаграмму классов;
    \begin{itemize}
        \item лучше на одной диаграмме сразу;
    \end{itemize}
    \item как обычно, выложить результаты в репозиторий одного из членов команды, не забыв указать полный её состав;
    \item успеть показать на проекторе и обсудить пару решений.
\end{itemize}

Обратите внимание, что помимо аккуратного проектирования будет оцениваться и полнота покрытия требований из RFP, и пунктуальность в синтаксисе UML.

\end{document}