\documentclass{../../slides-style}

\slidetitle{Практика 11: Практика по DDD}{28.04.2025}

\begin{document}
    
    \begin{frame}[plain]
        \titlepage
    \end{frame}

    \begin{frame}
        \frametitle{Задача на пару, магазин книг}
        В командах по 2-3 человека проанализировать запрос \url{https://goo.gl/94LyFc} и построить модель приложения согласно <<чистой архитектуре>> и канонам предметно-ориентированного проектирования.
        \begin{itemize}
            \item Обратите внимание, это <<сырой>> RFP, некоторые требования там к заданию не относятся
            \begin{itemize}
                \item Выявить их и обсудить --- ваша задача
            \end{itemize}
            \item Основное внимание надо уделить модели предметной области, внешние сущности (средства доставки, средства персистентности) должны упоминаться, но без деталей
            \item Необходима <<каноничная>> высокоуровневая структура и явно выделенное смысловое ядро
        \end{itemize}
    \end{frame}

    \begin{frame}
        \frametitle{Что надо сделать за пару}
        \begin{itemize}
            \item Диаграмму компонентов + диаграмму классов
            \begin{itemize}
                \item Лучше на одной диаграмме сразу
            \end{itemize}
            \item Успеть показать на проекторе и обсудить пару решений
        \end{itemize}
    \end{frame}

\end{document}
