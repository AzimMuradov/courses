% Шаблон списка вопросов, который включает в себя вообще весь материал, который может быть в курсе.
% Под каждое конкретное прочтение из него делается список вопросов вырезанием тех, которые не успели пройти.

\documentclass[a5paper]{article}
\usepackage[a5paper, top=8mm, bottom=8mm, left=8mm, right=8mm]{geometry}

\usepackage{polyglossia}
\setdefaultlanguage[babelshorthands=true]{russian}

\usepackage{fontspec}
\setmainfont{FreeSerif}
\newfontfamily{\russianfonttt}[Scale=0.7]{DejaVuSansMono}

\usepackage[font=scriptsize]{caption}

\sloppy
\pagestyle{plain}

\begin{document}

\thispagestyle{empty}

\section*{Вопросы к экзамену <<Технология разработки программного обеспечения>>}

\begin{flushright}\begin{small}Юрий Литвинов\\\small{y.litvinov@spbu.ru}\end{small}\end{flushright}

\begin{enumerate}
    \item Программная инженерия как область знания
    \item Отличия разработки программного обеспечения от других инженерных областей
    \item Компетенции и профстандарты в области программной инженерии
    \item Понятие жизненного цикла программного обеспечения
    \item Водопадная модель жизненного цикла
    \item Итеративная, итеративно-инкрементальная и спиралевидная модели жизненного цикла
    \item Rational Unified Process
    \item Agile-подход к разработке
    \item eXtreme Programming: общий подход, достоинства и недостатки
    \item XP: практики <<Короткий цикл разработки>>
    \item XP: практики <<Непрерывность процесса>>
    \item XP: практики <<Понимание, разделяемое всеми>>
    \item Scrum, роли в команде
    \item Scrum, Backlog и спринты
    \item ScrumAnd, ScrumBut, достоинства и недостатки методологии
    \item Понятие и виды требований
    \item Требования к требованиям
    \item Работа с требованиями: выявление, анализ, проверка
    \item Навыки и трудовые функции аналитика
    \item Документы, связанные с требованиями
    \item Моделирование требований
    \item Спецификация требований к программному обеспечению (SRS)
    \item Управление требованиями
    \item Понятие User experience
    \item Стадии проектирования пользовательского интерфейса
    \item User-centered design, персонажи и сценарии
    \item Activity-centred design
    \item Data-driven design
    \item Сторителлинг и раскадровки как инструмент проектирования UI
    \item Макеты и дизайн-макеты как инструмент проектирования UI
    \item Варианты исследования UX продукта
    \item Usability-тестирование
    \item Функции менеджера проекта
    \item Матрица ответственности и план коммуникаций
    \item Основные действия по управлению рисками
    \item Декомпозиция проекта
    \item Построение графика работ: матрица зависимостей, сетевой график
    \item Построение графика работ: диаграмма Гантта
    \item Типичные ошибки при оценке проектов
    \item Треугольник равновесия проекта
    \item Приёмы балансирования равновесия на уровне проекта
    \item Приёмы балансирования равновесия на уровне бизнес-целей
    \item Отслеживание прогресса проекта
    \item Позитивная экосистема команды: базовые правила и сплочённость команды
    \item Позитивная экосистема команды: умение слушать, умение проводить совещания
    \item Совместное решение задач: анализ задач и варианты принятия решений
    \item Совместное решение задач: разрешение конфликтов и непрерывное обучение
    \item Особенности формирования команды
    \item Особенности командной разработки программного обеспечения, ревью кода
    % \item Системы контроля версий: модели ветвления <<Разработка в главной ветке>> и Gitflow
    % \item Системы контроля версий: модели ветвления <<GitHub flow>>, <<GitLab flow>>, <<OneFlow>>
    % \item Понятие качества программного обеспечения
    % \item Характеристики качества программного обеспечения по ISO 25010:2011
    % \item Метрики качества программного обеспечения, классификация
    % \item Метрики Холстеда и цикломатическая сложность
    % \item Метрики Чидамбера и Кемерера
    % \item Метрики Лоренца и Кидда
    % \item Метрики MOOD
    % \item Модель зрелости компаний CMMI
    \item Понятие и виды тестирования программного обеспечения, пирамида тестирования
    \item Тестирование требований, тестирование архитектуры
    \item Тестовые сценарии
    \item Инструменты тестирования
    \item Отслеживание ошибок, жизненный цикл ошибки
    \item Законы Лемана эволюции программного обеспечения
    \item Понятие и особенности унаследованных систем
    \item Сопровождение программного обеспечения: факторы стоимости, прогнозирование, процесс
    \item Организация технической поддержки программного обеспечения
    \item Особенности работы с унаследованным кодом
    \item Реинжиниринг программного обеспечения: варианты, факторы стоимости, основные этапы
    \item Понятие рефакторинга
    \item <<Code smells>>
    \item Рефакторинги <<Выделение метода>> и <<Перемещение метода>>
    \item Рефакторинги <<Выделение класса>> и <<Выделение подкласса>>
    \item Рефакторинги <<Сокрытие делегирование>> и <<Введение внешнего метода>>
    \item Рефакторинги <<Самоинкапсуляция поля>> и <<Введение Null-объекта>>
    \item Рефакторинги <<Замена кода типа подклассами>> и <<Замена условного оператора полиморфизмом>>
    \item Рефакторинги <<Замена конструктора фабричным методом>> и <<Замена наследования делегированием>>
    \item Проблемы при проведении рефакторинга. Причины не проводить рефакторинг
    \item Понятие Continuous Delivery
    \item Антипаттерны управления релизами
    \item Преимущества частых автоматизированных релизов
    \item Принципы непрерывного развёртывания, конфигурационное управление
    \item Continuous Integration
    \item Оценка. Используемые метрики. Конус неопределённости.
    \item Этапы процесса оценки.
    \item Частые ошибки при оценке.
\end{enumerate}

\end{document}
