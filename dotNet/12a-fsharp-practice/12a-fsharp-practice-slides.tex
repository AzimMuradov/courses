\documentclass[xetex,mathserif,serif]{beamer}
\usepackage{polyglossia}
\setdefaultlanguage[babelshorthands=true]{russian}
\usepackage{minted}
\usepackage{tabu}

\useoutertheme{infolines}

\usepackage{fontspec}
\setmainfont{FreeSans}
\newfontfamily{\russianfonttt}{FreeSans}

\setbeamertemplate{blocks}[rounded][shadow=false]

\setbeamercolor*{block title alerted}{fg=red!50!black,bg=red!20}
\setbeamercolor*{block body alerted}{fg=black,bg=red!10}

\tabulinesep=1.2mm

\title{F\#, практика}
\author{Юрий Литвинов}
\date{23.11.2017}

\begin{document}

	\frame{\titlepage}

	\begin{frame}
		\frametitle{Задачи}
		\begin{enumerate}
			\item Посчитать факториал
			\item Посчитать числа Фибоначчи (за линейное время)
			\item Проверить, что все элемента списка различны
			\item Найти такое i, что сумма i-го и (i+1)-го элементов списка максимальна
			\item Реализовать функцию, возвращающую все элементы двоичного дерева, удовлетворяющие переданному как параметр условию
			\item Описать тип <<полином>> и реализовать функцию, возводящую полином в заданную степень
			\item Описать бесконечную последовательность \newline
				1, 2, 2, 3, 3, 3, 4, 4, 4, 4, ...
		\end{enumerate}
	\end{frame}

\end{document}
