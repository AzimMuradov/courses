\documentclass[xetex,mathserif,serif]{beamer}
\usepackage{polyglossia}
\setdefaultlanguage[babelshorthands=true]{russian}
\usepackage{minted}
\usepackage{tabu}
\usepackage{forest}
\usetikzlibrary{arrows}

\useoutertheme{infolines}

\usepackage{fontspec}
\setmainfont{FreeSans}
\newfontfamily{\russianfonttt}{FreeSans}

\setbeamertemplate{blocks}[rounded][shadow=false]

\setbeamercolor*{block title alerted}{fg=red!50!black,bg=red!20}
\setbeamercolor*{block body alerted}{fg=black,bg=red!10}

\tabulinesep=1.2mm

\title{Исключения, рефлексия, модульное тестирование}
\author[Юрий Литвинов]{Юрий Литвинов \newline \textcolor{gray}{\small\texttt{yurii.litvinov@gmail.com}}}
\date{21.09.2017г}

\begin{document}

	\frame{\titlepage}

	\section{Исключения}

	\begin{frame}[fragile]
		\frametitle{Бросание исключений}
		\begin{minted}{csharp}
if (t == null)
{
    throw new NullReferenceException();
}

throw new NullReferenceException("Hello!");
		\end{minted}
	\end{frame}

	\begin{frame}[fragile]
		\frametitle{Обработка исключений}
		\begin{minted}{csharp}
try {
    // Код, который может бросать исключения
} catch (Type1 id1) {
    // Обработка исключений типа Type1
} catch (Type2 id2) {
    // Обработка исключений типа Type2
} catch {
    // Обработка всех оставшихся исключений
} finally {
    // Код, выполняющийся в любом случае
}
		\end{minted}
	\end{frame}

	\begin{frame}[fragile]
		\frametitle{Пример}
		\begin{minted}{csharp}
private void ReadData(String pathname) 
{
    FileStream fs = null;
    try {
        fs = new FileStream(pathname, FileMode.Open);
        // Делать что-то полезное
    }
    catch (IOException) {
        // Код восстановления после ошибки
    }
    finally {
        // Закрыть файл надо в любом случае
        if (fs != null) fs.Close();
    }
}
		\end{minted}
	\end{frame}

	\begin{frame}
		\frametitle{Иерархия классов-исключений}
		\begin{tiny}
			\begin{forest}
				for tree={rectangle,draw,l sep=1cm,s sep=3mm,edge=open triangle 60-}
				[Object
					[Exception
						[SystemException
							[IndexOutOfRangeException]
							[NullReferenceException]
							[StackOverflowException]
							[ArgumentException
								[ArgumentNullException]
								[ArgumentOutOfRangeException]
								[\dots]
							]
						]
						[ApplicationException]
					]
				]
			\end{forest}
		\end{tiny}
	\end{frame}

	\begin{frame}[fragile]
		\frametitle{Свойства класса Exception}
		\begin{itemize}
			\item Data
			\item HelpLink
			\item InnerException
			\item Message
			\item Source
			\item StackTrace
			\item HResult
		\end{itemize}
	\end{frame}

	\begin{frame}[fragile]
		\frametitle{Перебрасывание исключений}
		\begin{minted}{csharp}
try
{
    throw new Exception("Something is wrong");
}
catch (Exception e)
{
    Console.WriteLine("Caught Exception");
    throw;  // "throw e;" тоже работает, но сбросит stack trace
}
		\end{minted}

		Или
		\begin{minted}{csharp}
throw new Exception("Outer exception", e);
		\end{minted}
	\end{frame}

	\begin{frame}[fragile]
		\frametitle{Свои классы-исключения}
		\begin{minted}{csharp}
public class MyException : ApplicationException
{
    public MyException() 
    {
    }

    public MyException(string message)
        : base(message)
    {
    }
}
		\end{minted}
	\end{frame}

	\begin{frame}[fragile]
		\frametitle{Идеологически правильное объявление исключения}
		\begin{minted}{csharp}
[Serializable]
public class MyException : Exception
{
    public MyException() { }
    public MyException(string message) : base(message) { }
    public MyException(string message, Exception inner) 
        : base(message, inner) { }
    protected MyException(
        System.Runtime.Serialization.SerializationInfo info,
        System.Runtime.Serialization.StreamingContext context)
            : base(info, context) { }
}
		\end{minted}
	\end{frame}

	\begin{frame}
		\frametitle{``Интересные'' классы-исключения}
		\begin{itemize}
			\item Corrupted state exceptions (CSE) --- не ловятся catch-ем
			\begin{itemize}
				\item StackOverflowException
				\item AccessViolationException
				\item System.Runtime.InteropServices.SEHException
			\end{itemize}
			\item FileLoadException, BadImageFormatException, InvalidProgramException, ...
			\item ThreadAbortException
			\item TypeInitializationException
			\item TargetInvocationException
			\item OutOfMemoryException
			\item Можно кидать null. При этом рантайм сам кидает NullReferenceException.
		\end{itemize}
	\end{frame}

	\begin{frame}[fragile]
		\frametitle{Environment.FailFast}
		\begin{minted}{csharp}
try {
     // Делать что-то полезное
}
catch (OutOfMemoryException e) 
{
    Console.WriteLine("Закончилась память :(");
    Environment.FailFast(
        String.Format($"Out of Memory: {e.Message}"));
}
		\end{minted}
	\end{frame}

	\begin{frame}
		\frametitle{Исключения, best practices}
		\begin{itemize}
			\item Не бросать исключения без нужды
			\begin{itemize}
				\item В нормальном режиме работы исключения бросаться не должны вообще
			\end{itemize}
			\item Свои исключения наследовать от System.Exception
			\item Документировать все свои исключения, бросаемые методом
			\item Не ловить Exception, SystemException
			\begin{itemize}
				\item Исключения, указывающие на ошибку в коде (например, NullReferenceException) уж точно не ловить
			\end{itemize}
			\item По возможности кидать библиотечные исключения или их наследников:
			\begin{itemize}
				\item InvalidOperationException
				\item ArgumentException
			\end{itemize}
			\item Имя класса должно заканчиваться на ``Exception''
			\item Код должен быть безопасным с точки зрения исключений
			\begin{itemize}
				\item Не забывать про finally 
				\item Или using, lock, foreach, которые тоже генерят finally
			\end{itemize}
		\end{itemize}
	\end{frame}

	\section{Контракты}

	\begin{frame}
		\frametitle{Code Contracts}
		Набор инструментов для контрактно-ориентированного программирования
		\begin{itemize}
			\item Предусловия --- должны выполняться до исполнения метода
			\item Постусловия --- должны выполняться после исполнения метода
			\item Инварианты --- должны выполняться всегда, пога объект не исполняет метод
			\begin{itemize}
				\item Внутри метода инвариант объекта может нарушаться
			\end{itemize}
			\item Класс System.Diagnostics.Contracts.Contract
			\item Требует серьёзной инструментальной поддержки, чтобы делать что-то интересное
		\end{itemize}
	\end{frame}

	\begin{frame}[fragile]
		\frametitle{Пример}
		\begin{scriptsize}
			\begin{minted}{csharp}
public sealed class ShoppingCart {
    private List<Item> cart = new List<Item>();
    private Decimal totalCost = 0;
    public void AddItem(Item item) => AddItemHelper(cart, item, ref totalCost);

    private static void AddItemHelper(List<Item> cart, Item newItem, ref Decimal totalCost) {
        // Предусловия:
        Contract.Requires(newItem != null);
        Contract.Requires(Contract.ForAll(cart, s => s != newItem));
        // Постусловия:
        Contract.Ensures(Contract.Exists(cart, s => s == newItem));
        Contract.Ensures(totalCost >= Contract.OldValue(totalCost));
        Contract.EnsuresOnThrow<IOException>(totalCost == Contract.OldValue(totalCost));
        // Делаем полезную работу...
        cart.Add(newItem);
        totalCost += 1.00M;
    }

    [ContractInvariantMethod]
    private void ObjectInvariant() {
        Contract.Invariant(totalCost >= 0);
    }
}
			\end{minted}
		\end{scriptsize}
	\end{frame}

	\section{Рефлексия}

	\begin{frame}
		\frametitle{Рефлексия}
		\begin{itemize}
			\item Позволяет во время выполнения получать информацию о типах
			\begin{itemize}
				\item И главное, создавать объекты этих типов и вызывать их методы
			\end{itemize}
			\item Зачем:
			\begin{itemize}
				\item Плагины
				\item Анализаторы кода
				\item Тестовые системы
				\item ...
			\end{itemize}
			\item Проблемы:
			\begin{itemize}
				\item Медленно
				\item Нет помощи от системы типов
			\end{itemize}
		\end{itemize}
	\end{frame}

	\begin{frame}[fragile]
		\frametitle{Загрузка сборки}
		\begin{small}
			\begin{minted}{csharp}
public class Assembly {
    public static Assembly Load(AssemblyName assemblyRef);
    public static Assembly Load(String assemblyString);
    public static Assembly Load(byte[] rawAssembly)
    public static Assembly LoadFrom(String path);
    public static Assembly ReflectionOnlyLoad(String assemblyString);
    public static Assembly ReflectionOnlyLoadFrom(String assemblyFile);
}
			\end{minted}
			например,
			\begin{minted}{csharp}
var a = Assembly.LoadFrom(@"http://example.com/ExampleAssembly.dll");
			\end{minted}
			Выгружать сборки нельзя
		\end{small}
	\end{frame}

	\begin{frame}[fragile]
		\frametitle{Пример}
		\framesubtitle{Распечатать имена всех типов в сборке}
		\begin{footnotesize}
			\begin{minted}{csharp}
using System;
using System.Reflection;

public static class Program {
    public static void Main() {
        string dataAssembly = "System.Data, version=4.0.0.0, "
                + "culture=neutral, PublicKeyToken=b77a5c561934e089";
        LoadAssemblyAndShowPublicTypes(dataAssembly);
    }

    private static void LoadAssemblyAndShowPublicTypes(string assemblyId) {
        var a = Assembly.Load(assemblyId);
        foreach (Type t in a.ExportedTypes) {
            Console.WriteLine(t.FullName);
        }
    }
}
			\end{minted}
		\end{footnotesize}
	\end{frame}

	\begin{frame}
		\frametitle{Создание экземпляра объекта}
		\begin{itemize}
			\item System.Activator.CreateInstance --- можно передавать тип или строку с именем типа
			\begin{itemize}
				\item Версии со строкой возвращают System.Runtime.Remoting.ObjectHandle, надо вызвать Unwrap()
			\end{itemize}
			\item System.Activator.CreateInstanceFrom --- вызывает LoadFrom для сборки
			\item System.Reflection.ConstructorInfo.Invoke --- просто вызов конструктора (несколько дольше писать, чем предыдущие варианты)
			\item Рефлексия ничего не знает о синонимах
		\end{itemize}
	\end{frame}

	\begin{frame}[fragile]
		\frametitle{Создание экземпляра типа-генерика}
		\begin{footnotesize}
			\begin{minted}{csharp}
using System;
using System.Reflection;

internal sealed class Dictionary<TKey, TValue> { }

public static class Program {
    public static void Main() {
        Type openType = typeof(Dictionary<,>);
        Type closedType = openType.MakeGenericType(
                typeof(String), typeof(Int32));
        Object o = Activator.CreateInstance(closedType);
        Console.WriteLine(o.GetType());
    }
}
			\end{minted}
		\end{footnotesize}
	\end{frame}

	\begin{frame}
		\frametitle{Пример: как сделать свою плагинную систему}
		\begin{itemize}
			\item Сделать отдельную сборку с описанием интерфейса плагина и типов данных, которые он использует
			\begin{itemize}
				\item Менять её будет очень проблематично
			\end{itemize}
			\item Сделать ``ядро системы'' --- отдельную сборку, ссылающуюся на сборку с интерфейсом плагина
			\item Делать набор плагинов, ссылающихся на сборку с интерфейсом плагина и реализующих его
		\end{itemize}
	\end{frame}

	\begin{frame}[fragile]
		\frametitle{Пример: интерфейс плагина}
		\begin{minted}{csharp}
namespace MyCoolSystem.SDK {
    public interface IAddIn {
        string DoSomething(int x);
    }
}
		\end{minted}
	\end{frame}

	\begin{frame}[fragile]
		\frametitle{Пример: плагины}
		\begin{minted}{csharp}
using MyCoolSystem.SDK;

public sealed class AddInA : IAddIn {
    public String DoSomething(int x) {
        return "AddInA: " + x.ToString();
    }
}

public sealed class AddInB : IAddIn {
    public String DoSomething(int x) {
        return "AddInB: " + (x * 2).ToString();
    }
}
		\end{minted}
	\end{frame}

	\begin{frame}[fragile]
		\frametitle{Пример: ядро системы}
		\begin{scriptsize}
			\begin{minted}{csharp}
public static class Program
{
    public static void Main()
    {
        string addInDir = Path.GetDirectoryName(Assembly.GetEntryAssembly().Location);
        var addInAssemblies = Directory.EnumerateFiles(addInDir, "*.dll");
        var addInTypes =
            addInAssemblies.Select(Assembly.Load)
                .SelectMany(a => a.ExportedTypes)
                .Where(t => t.IsClass 
                        && typeof(IAddIn).GetTypeInfo().IsAssignableFrom(t.GetTypeInfo()));

        foreach (Type t in addInTypes)
        {
            var addIn = (IAddIn)Activator.CreateInstance(t);
            Console.WriteLine(addIn.DoSomething(5));
        }
    }
}
			\end{minted}
		\end{scriptsize}
	\end{frame}

	\begin{frame}
		\frametitle{Информация о типах}
		\begin{center}
			\begin{tiny}
				\begin{forest}
					for tree={rectangle,draw,l sep=1cm,s sep=3mm,edge=open triangle 60-}
					[Object
						[Reflection.MemberInfo
							[TypeInfo]
							[Reflection.FieldInfo]
							[Reflection.MethodBase
								[Reflection.ConstructorInfo]
								[Reflection.MethodInfo]
							]
							[Reflection.PropertyInfo]
							[Reflection.EventInfo]
						]
					]
				\end{forest}
			\end{tiny}
		\end{center}
	\end{frame}

	\begin{frame}[fragile]
		\frametitle{Пример: распечатать информацию о полях и методах}
		\begin{scriptsize}
			\begin{minted}{csharp}
using System;
using System.Reflection;

public static class Program {
    public static void Main() {
        Assembly[] assemblies = AppDomain.CurrentDomain.GetAssemblies();
        foreach (Assembly a in assemblies) {
            Console.WriteLine($"Assembly: {a}");
            foreach (Type t in a.ExportedTypes) {
                Console.WriteLine($"  Type: {t}");
                foreach (MemberInfo mi in t.GetTypeInfo().DeclaredMembers) {
                    var typeName = string.Empty;
                    if (mi is FieldInfo) typeName = "FieldInfo";
                    if (mi is MethodInfo) typeName = "MethodInfo";
                    if (mi is ConstructorInfo) typeName = "ConstructoInfo";
                    Console.WriteLine($"    {typeName}: {mi}");
                }
            }
        }
    }
}
			\end{minted}
		\end{scriptsize}
	\end{frame}

	\begin{frame}
		\frametitle{Полезные свойства MemberInfo}
		\begin{itemize}
			\item Name (string) --- имя члена класса
			\item DeclaringType (Type) --- тип
			\item Module (Module) --- модуль, в котором он объявлен
			\item CustomAttributes (IEnumerable<CustomAttributeData>) --- коллекция атрибутов, соответствующих этому члену класса
			\begin{itemize}
				\item Пример --- модульные тесты
			\end{itemize}
		\end{itemize}
	\end{frame}

	\begin{frame}
		\frametitle{Как что-нибудь сделать с MemberInfo}
		\begin{itemize}
			\item GetValue и SetValue для FieldInfo и PropertyInfo
			\item Invoke для ConstructorInfo и MethodInfo
			\item AddEventHandler и RemoveEventHandler для EventInfo
		\end{itemize}
	\end{frame}

	\begin{frame}[fragile]
		\frametitle{Пример: создать класс и вызвать его метод}
		\begin{scriptsize}
			\begin{minted}{csharp}
using System;
using System.Reflection;
using System.Linq;

internal sealed class SomeType {
    public SomeType(int test) { }
    private int DoSomething(int x) => x * 2;
}

public static class Program {
    public static void Main() {
        Type t = typeof(SomeType);
        Type ctorArgument = Type.GetType("System.Int32");
        ConstructorInfo ctor = t.GetTypeInfo().DeclaredConstructors.First(
                c => c.GetParameters()[0].ParameterType == ctorArgument);
        Object[] args = { 12 };
        Object obj = ctor.Invoke(args);
        MethodInfo mi = obj.GetType().GetTypeInfo().GetDeclaredMethod("DoSomething");
        int result = (int)mi.Invoke(obj, new object[]{3});
        Console.WriteLine($"result = {result}");
    }
}
			\end{minted}
		\end{scriptsize}
	\end{frame}

	\section{Модульное тестирование}

	\begin{frame}
		\frametitle{Модульное тестирование}
		\begin{itemize}
			\item Помогает искать ошибки
			\begin{itemize}
				\item Особо эффективно, если налажен процесс Continuous Integration
			\end{itemize}
			\item Облегчает изменение программы, рефакторинг
			\begin{itemize}
				\item Но несколько замедляет процесс, тесты тоже требуют сопровождения
			\end{itemize}
			\item Тесты --- документация к коду
			\item Тесты помогают улучшить архитектуру, спагетти-код не оттестировать
		\end{itemize}
	\end{frame}

	\begin{frame}
		\frametitle{Популярные библиотеки}
		\begin{itemize}
			\item NUnit
			\begin{itemize}
				\item Отдельный пакет
				\item Интегрируется в IDE расширениями
			\end{itemize}
			\item Microsoft Unit Test Framework
			\begin{itemize}
				\item Работает прямо из коробки в Visual Studio, но требует некоторой возни, если нет
			\end{itemize}
			\item XUnit
			\item MbUnit
		\end{itemize}
	\end{frame}

	\begin{frame}
		\frametitle{Best practices}
		\begin{itemize}
			\item Независимость тестов
			\begin{itemize}
				\item Желательно, чтобы поломка одного куска функциональности ломала один тест
			\end{itemize}
			\item Тесты должны работать быстро
			\begin{itemize}
				\item И запускаться после каждой сборки
				\begin{itemize}
					\item Continuous Integration!
				\end{itemize}
			\end{itemize}
			\item Тестов должно быть много
			\begin{itemize}
				\item Следить за Code coverage, который многие инструменты умеют считать по тестовому прогону
			\end{itemize}
			\item Каждый тест должен проверять конкретный тестовый сценарий
			\begin{itemize}
				\item Никаких try-catch внутри теста
				\item Атрибут ExpectedException для исключений
				\begin{itemize}
					\item \mintinline{csharp}|[ExpectedException(typeof(NullReferenceException))]|
				\end{itemize}
			\end{itemize}
			\item Test-driven development
		\end{itemize}
	\end{frame}

	\begin{frame}
		\frametitle{Mock-объекты}
		\begin{itemize}
			\item Объекты-заглушки, симулирующие поведение реальных объектов и контролирующие обращения к своим методам
			\begin{itemize}
				\item Как правило, такие объекты создаются с помощью библиотек
			\end{itemize}
			\item Используются, когда реальные объекты использовать
			\begin{itemize}
				\item Слишком долго
				\item Слишком опасно
				\item Слишком трудно
				\item Для добавления детерминизма в тестовый сценарий
				\item Пока реального объекта ещё нет
				\item Для изоляции тестируемого объекта
			\end{itemize}
			\item Для mock-объекта требуется, чтобы был интерфейс, который он мог бы реализовать, и какой-то механизм внедрения объекта
			\begin{itemize}
				\item Паттерны ``Dependency Injection'', ``Стратегия''
			\end{itemize}
		\end{itemize}
	\end{frame}

	\begin{frame}
		\frametitle{Популярные библиотеки}
		\begin{itemize}
			\item NSubstitute
			\item Moq
			\item Rhino Mocks
			\item ...
		\end{itemize}
	\end{frame}

	\begin{frame}[fragile]
		\frametitle{NSubstitute}
		Легковесная библиотека, очень удобно описывать поведение объектов:
		\vspace{3mm}
		\begin{minted}{csharp}
var stackStub = Substitute.For<IStack>();
stackStub.IsEmpty().Returns(false);
stackStub.Pop().Returns(1);

Assert.AreEqual(1, stackStub.Pop());

stackStub.Received().Pop();
stackStub.DidNotReceive().Push(Arg.Any<int>());
		\end{minted}
	\end{frame}

	\begin{frame}[fragile]
		\frametitle{Moq}
		Активно использует лямбда-функции и LINQ:
		\vspace{3mm}
		\begin{minted}{csharp}
var stackMock = new Mock<IStack>();
stackMock.Setup(st => st.IsEmpty()).Returns(false);
stackMock.Setup(st => st.Pop()).Returns(1);

var stack = stackMock.Object;
stack.Push(1);
stack.Pop();

stackMock.Verify(st => st.IsEmpty(), Times.Never);
stackMock.Verify(st => st.Pop(), Times.Exactly(1));
stackMock.Verify(st => st.Push(It.IsAny<int>()), Times.Exactly(1));
		\end{minted}
	\end{frame}

	\begin{frame}[fragile]
		\frametitle{Moq (2)}
		Или так (LINQ-магия):
		\vspace{3mm}
		\begin{minted}{csharp}
var stack = Mock.Of<IStack>(
        st => st.IsEmpty() == false && st.Pop() == 1);

stack.Push(1);
stack.Pop();

Mock.Get(stack).Verify(st => st.Push(It.IsAny<int>()), Times.Exactly(1));
Mock.Get(stack).Verify(st => st.Pop(), Times.Exactly(1));
Mock.Get(stack).Verify(st => st.IsEmpty(), Times.Never);
		\end{minted}
	\end{frame}

	\begin{frame}[fragile]
		\frametitle{Rhino Mocks}
		\begin{minted}{csharp}
var stack = MockRepository.Mock<IStack>();
stack.Expect(st => st.Push(1));
stack.Expect(st => st.Pop()).Returns(() => 1);

stack.Push(1);
Assert.AreEqual(1, stack.Pop());

stack.VerifyAllExpectations();
		\end{minted}
	\end{frame}

\end{document}
