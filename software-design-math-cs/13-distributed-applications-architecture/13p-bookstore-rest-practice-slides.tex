\documentclass{../mcsslides}

\slidetitle{Практика 13: REST-сервис}{29.11.2022}

\begin{document}
    
    \begin{frame}[plain]
        \titlepage
    \end{frame}

    \begin{frame}
        \frametitle{Задача на пару}
        В командах по два человека спроектировать и реализовать REST-сервис для <<открытой части>> магазина книг, спроектированного двумя неделями ранее.
        \begin{itemize}
            \item ТЗ: \url{https://goo.gl/94LyFc}
            \item Случаи использования:
            \begin{itemize}
                \item Запрос информации о продаваемых книгах, с отзывами
                \item Поиск книг по критериям, указанным в ТЗ
            \end{itemize}
            \item Авторизация не нужна, считаем, что вся информация открыта
            \item Используйте захардкоженные данные, реализация работы с БД не нужна
            \item Используйте Swagger для тестирования
            \begin{itemize}
                \item И попробуйте выполнить какой-нибудь GET-метод из браузера
            \end{itemize}
        \end{itemize}
    \end{frame}

    \begin{frame}
        \frametitle{На что обратить внимание}
        \begin{itemize}
            \item Модель данных: книга, отзывы (пользовательские и редакторские), поисковый запрос
            \begin{itemize}
                \item Подумайте над дизайном функциональности поиска
            \end{itemize}
            \item Аккуратный дизайн API: ресурсы и коллекции
            \item Гексагональная архитектура --- идеологически правильное отделение средств доставки от доменной модели
            \item (*) Пагинация
            \item (*) HATEOAS
        \end{itemize}
    \end{frame}

\end{document}