\documentclass{../mcstext}

\texttitle{Практика 15: OAuth}

\begin{document}

\maketitle
\thispagestyle{empty}

Задачей этой практики станет поработать с клиентской стороны с OAuth на примере Google Drive API. Google Drive поддержан во всех нормальных языках программирования, естественно, с удобным готовым механизмом авторизации. Поэтому, чтобы было интереснее, есть дополнительное условие --- не использовать клиентские библиотеки, реализуя работу с API на <<голом>> HTTPS. Требуется написать консольный клиент, умеющий показывать список файлов и папок в корне Google Drive аккаунта пользователя.

При этом: 
\begin{enumerate}
    \item Придётся зарегистрировать приложение на \url{https://console.developers.google.com} и включить нужный API. После этого надо создать OAuth Credentials, Google Developers Console отдаст файл credentials.json с клиентскими секретами, их надо сохранить к себе, но в систему контроля версий не выкладывать.
    \item Справка по OAuth, как он реализован в Google Drive: \url{https://developers.google.com/drive/api/v3/about-auth}. Тут поддерживаются все три <<каноничных>> режима OAuth: для веб-приложений, для серверных приложений и для настольных/мобильных клиентов, нам интересен последний случай. Вот документация конкретно про него: \url{https://developers.google.com/identity/protocols/oauth2/native-app}.
\end{enumerate}

Некоторые рекомендации:
\begin{enumerate}
    \item Запрос access token должен быть обязательно POST-запросом, GET вернёт 404. 
    \item Копипастить access token из адресной строки браузера для этой задачи вполне ок. Вообще, по-хорошему надо поднять локальный сервер и сказать странице авторизации сделать на него (на localhost) редирект с access token в URL. Сервер парсит URL и вычитывает из него access token, который потом используется в приложении. Обратите внимание, что хоть у нас и консольный клиент, в любом случае требуется запустить браузер и отправить пользователя на страницу авторизации.
    \item Некоторые значения возвращаются в BASE64-кодировке, а ожидаются в plain text.
    \item Для этой задачи не надо продлять Access Token, правильно хранить Refresh Token и всё такое. Если будет время, можно попробовать, потому что <<настоящее>> приложение должно это делать, но если у вас каждый раз при запуске приложение будет просить авторизоваться, это вполне допустимо.
    \item Значения Client Id и Client Secret (или путь до credentials.json) можно принимать параметрами командной строки.
\end{enumerate}

\end{document}
