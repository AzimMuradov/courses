\documentclass{../mcstext}

\texttitle{Практика 12: Сетевой чат на gRPC}

\begin{document}

\maketitle
\thispagestyle{empty}

\section{Задание на практику}

В командах по два человека разработать сетевой чат (наподобие Telegram) с помощью gRPC. При этом:

\begin{itemize}
    \item оно должно работать как peer-to-peer, то есть соединение напрямую, без всяких серверов;
    \begin{itemize}
        \item и клиент, и сервер должны быть одним и тем же приложением, работающим в разных режимах;
        \item если приложение запускается с указанием только порта, оно становится сервером;
        \item если IP и порта, то клиентом;
        \item порт можно зафиксировать в коде и не просить у пользователя;
    \end{itemize}
    \item консольный пользовательский интерфейс;
    \item отображение имени отправителя, даты отправки и текста сообщения;
    \item при запуске указываются:
    \begin{itemize}
        \item адрес peer-а и порт, если хотим подключиться;
        \begin{itemize}
            \item должно быть можно не указывать, тогда работаем в режиме сервера;
        \end{itemize}
        \item своё имя пользователя.
    \end{itemize}
\end{itemize}

Реализация допустима на любом языке программирования из поддержанных gRPC.

\end{document}