% Шаблон списка вопросов, который включает в себя вообще весь материал, который может быть в курсе.
% Под каждое конкретное прочтение из него делается список вопросов вырезанием тех, которые не успели пройти.

\documentclass[a5paper]{article}
\usepackage[a5paper, top=8mm, bottom=8mm, left=8mm, right=8mm]{geometry}

\usepackage{polyglossia}
\setdefaultlanguage[babelshorthands=true]{russian}

\usepackage{fontspec}
\setmainfont{FreeSerif}
\newfontfamily{\russianfonttt}[Scale=0.7]{DejaVuSansMono}

\usepackage[font=scriptsize]{caption}

\usepackage{amsmath}
\usepackage{amssymb,amsfonts,textcomp}
\usepackage{color}
\usepackage{array}
\usepackage{hhline}
\usepackage{cite}
\usepackage{textcomp}

\usepackage[hang,multiple]{footmisc}
\renewcommand{\footnotelayout}{\raggedright}

\PassOptionsToPackage{hyphens}{url}\usepackage[xetex,linktocpage=true,plainpages=false,pdfpagelabels=false]{hyperref}
\hypersetup{colorlinks=true, linkcolor=blue, citecolor=blue, filecolor=blue, urlcolor=blue, pdftitle=1, pdfauthor=, pdfsubject=, pdfkeywords=}

\newlength\Colsep
\setlength\Colsep{10pt}

\usepackage{tabu}

\usepackage{graphicx}
\usepackage{indentfirst}
\usepackage{multirow}
\usepackage{subfig}
\usepackage{footnote}
\usepackage{minted}

\newcommand{\todo}[1] {
\begin{center}\textcolor{red}{TODO: #1}\end{center}
}

\sloppy
\pagestyle{plain}

\title{Вопросы к экзамену ``Проектирование ПО''}
\author{Юрий Литвинов\\\small{yurii.litvinov@gmail.com}}

\begin{document}

\thispagestyle{empty}

\section*{Вопросы к экзамену ``Проектирование программного обеспечения''}

\begin{flushright}\begin{small}Юрий Литвинов\\\small{yurii.litvinov@gmail.com}\end{small}\end{flushright}

\begin{enumerate}
    \item Понятие архитектуры, профессия <<Архитектор>>.
    \item Архитектурные виды.
    \item Роль архитектуры в жизненном цикле программного обеспечения.
    \item Пример архитектуры: Apache Hadoop. Prescriptive и descriptive-архитектура.
    \item Понятие декомпозиции. Модульность, связность, сопряжение, сложность.
    \item Понятия класса и объекта, абстракция, инкапсуляция, наследование. 
    \item Принципы выделения абстракций предметной области.
    \item Принципы SOLID. Закон Деметры.
    \item Моделирование, визуальные модели, виды моделей, метафора визуализации.
    \item Язык UML. Диаграмма классов.
    % \item Диаграммы объектов, диаграммы пакетов UML.
    \item Диаграмма компонентов UML.
    \item Моделирование требований: диаграмма случаев использования UML, диаграмма характеристик, Feature tree.
    \item Диаграмма активностей UML, BPMN.
    % \item Моделирование данных: диаграммы <<Сущность-связь>>, ORM-диаграммы.
    \item Диаграммы конечных автоматов, последовательностей UML.
    \item Диаграммы коммуникации, составных структур UML.
    \item Диаграммы коопераций, временные диаграммы UML.
    \item Диаграммы обзора взаимодействия, диаграммы потоков данных.
    \item Диаграммы IDEF0, сети Петри. 
    \item Паттерн <<Компоновщик>>.
    \item Паттерн <<Декоратор>>.
    \item Паттерн <<Стратегия>>.
    \item Паттерн <<Адаптер>>.
    \item Паттерн <<Заместитель>>.
    \item Паттерн <<Фасад>>.
    \item Паттерн <<Приспособленец>>.
    \item Паттерн <<Мост>>.
    \item Паттерн <<Фабричный метод>>.
    \item Паттерн <<Абстрактная фабрика>>.
    \item Паттерн <<Одиночка>>.
    \item Паттерны <<Ленивая инициализация>> и <<Пул объектов>>.
    \item Паттерн <<Прототип>>.
    \item Паттерн <<Строитель>>.
    \item Паттерн <<Наблюдатель>>.
    \item Паттерн <<Шаблонный метод>>.
    \item Паттерн <<Посредник>>.
    \item Паттерн <<Команда>>.
    \item Паттерн <<Цепочка ответственности>>.
    \item Паттерн <<Состояние>>.
    \item Паттерн <<Посетитель>>.
    \item Паттерн <<Хранитель>>.
    \item Паттерн <<Интерпретатор>>.
    \item Паттерн <<Итератор>>.
    \item Понятие архитектурного стиля, трёхзвенная архитектура.
    \item Model-View-Controller, Sense-Compute-Control.
    \item Слоистый стиль, <<Клиент-сервер>>.
    \item Гексагональная архитектура, луковая архитектура.
    \item Чистая архитектура.
    \item Пакетная обработка, каналы и фильтры. 
    \item Стиль Blackboard.
    \item Событийно-ориентированные стили, Publish-Subscribe, событийная шина.
    \item Peer-to-peer.
    \item Понятие предметно-ориентированного проектирования, единый язык.
    \item Изоляция предметной области в DDD, антипаттерн <<Умный GUI>>.
    \item DDD, основные структурные элементы модели предметной области.
    \item DDD, паттерн <<Агрегат>>.
    \item DDD, паттерны <<Фабрика>>, <<Репозиторий>>.
    \item Паттерн <<Спецификация>>.
    % \item Ограниченный контекст, непрерывная интеграция, карта контекстов.
    % \item Подходы к интеграции контекстов.
    % \item Смысловое ядро, приёмы дистилляции, абстрактное ядро.
    % \item Крупномасштабная структура, метафора системы, разбиение по уровням. 
    % \item Типичные уровни в производственных и финансовых системах.
    % \item Стили <<Уровень знаний>>, <<Подключаемые компоненты>>.
    \item Понятие распределённой системы, заблуждения при проектировании распределённых систем.
    \item RPC, RMI. Пример: gRPC.
    \item Веб-сервисы, SOAP. WCF.
    \item Веб-сервисы, REST. ASP.NET Web APIs.
    % \item Очереди сообщений, RabbitMQ, Kafka.
    \item Архитектурные стили распределённых приложений: Big Compute, Big Data.
    \item Web-queue-worker, N-звенная архитектура.
    \item Микросервисная архитектура.
    \item Дизайн REST-интерфейса.
    \item Принципы дизайна распределённых приложений: самовосстановление.
    \item Паттерн <<Circuit Breaker>>.
    \item Принципы дизайна распределённых приложений: избыточность.
    \item Принципы дизайна распределённых приложений: минимизация координации.
    \item Command and Query Responsibility Segregation.
    \item CAP-теорема, модели ACID и BASE.
    \item Принципы дизайна распределённых приложений: проектирование для обслуживания.
    % \item Docker, Docker Compose.
    % \item Kubernetes.
    % \item Облачная инфраструктура, AWS, Terraform.
    % \item Архитектура командной оболочки Bash.
    % \item Архитектура системы контроля версий Git.
    % \item Архитектура компьютерной игры Battle for Wesnoth.
\end{enumerate}

\end{document}
