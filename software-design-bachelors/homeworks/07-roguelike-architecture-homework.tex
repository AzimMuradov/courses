\documentclass[a5paper]{homework}

\begin{document}

\makeHomeworkHeading{
    title = {Домашняя работа 6. Архитектура Roguelike},
    publicationDate = {28.02.2022},
    deadline = {14.03.2022},
    score = {10}
}

Roguelike --- это довольно популярный жанр компьютерных игр, назван в честь игры Rogue, 1980 года выхода. Характеризуется:

\begin{itemize}
    \item простой тайловой или консольной графикой (современные игры roguelike-игры иногда имеют вид сбоку (Dead Cells, Noita) или даже 3D (Risk of Rain 2), но (псевдо-)консольных и тайловых тоже полно (Caves of Qud, Cogmind, например));
    \item активным использованием случайной генерации;
    \item перманентной смертью персонажа и невозможностью загрузить предыдущее сохранение (не чтобы позлить игрока, а чтобы дать ему возможность попробовать разных персонажей и разные пути развития);
    \item чрезвычайно развитым набором игровых правил (чем эти игры нам и интересны, модель их предметной области может быть как очень простой, так и невероятно сложной);
    \item высокой свободой действий персонажа (так называемые <<игры-песочницы>>)
\end{itemize}

Классические примеры:
\begin{itemize}
    \item \url{https://en.wikipedia.org/wiki/NetHack}
    \item \url{https://en.wikipedia.org/wiki/Angband_(video_game)}
    \item \url{https://en.wikipedia.org/wiki/Ancient_Domains_of_Mystery}
\end{itemize}

Вашей задачей будет в командах по три-четыре человека разработать архитектуру такой компьютерной игры.

При этом должны быть выполнены следующие функциональные требования:

\begin{itemize}
    \item персонаж игрока, способный перемещаться по карте, управляемый с клавиатуры;
    \begin{itemize}
        \item карта обычно генерируется, но для некоторых уровней грузится из файла;
        \item характеристики --- здоровье, сила атаки и т.д.;
    \end{itemize}
    \item консольная графика, традиционная для этого жанра игр.
\end{itemize}

В данном задании требуется разделиться на команды (можно не так, как в CLI) и написать архитектурное описание Roguelike, как обычно пишутся design document-ы:

\begin{itemize}
    \item общие сведения о системе;
    \item ключевые требования (architectural drivers);
    \item роли и случаи использования;
    \begin{itemize}
        \item описание типичного пользователя, как, надеюсь, было в курсе по SE;
    \end{itemize}
    \item композиция (диаграмма компонентов и её текстовое описание);
    \item логическая структура (диаграмма классов и её текстовое описание; может быть по одной диаграмме классов на компонент, но все связи должны быть прослеживаемы --- то есть нужны и классы из других компонентов с полностью квалифицированными именами, но без атрибутов/операций, если на них ссылаетесь);
    \item взаимодействия и состояния (диаграммы последовательностей и конечных автоматов и их текстовое описание --- даже если для вашей архитектуры они не очень полезны, нарисуйте, потренироваться).
\end{itemize}

На что обратить внимание:

\begin{itemize}
    \item на разделение системы на компоненты --- решения вида <<большой клубок классов>> будут оценены очень низко;
    \begin{itemize}
        \item выберите и используйте какой-либо архитектурный стиль, явно укажите его в описании;
    \end{itemize}
    \item на прослеживаемость потока управления --- должно быть понятно, с какого места запускается программа, кто кому передаёт управление;
    \item что все имеют необходимые для своей работы данные --- например, пользовательский интерфейс знает про карту;
    \item на баланс детальности и читаемости диаграммы --- она должна быть достаточно детальна, чтобы при реализации не требовалось принимать серьёзных архитектурных решений, но при этом обозрима. Например, старательно выписывать все методы классов и все поля не надо, но ключевые методы и поля всё-таки нужны;
    \item на согласованность диаграмм --- на диаграммах последовательностей не должно быть классов, которых нет на диаграмме классов, например;
    \item на следование синтаксису UML.
\end{itemize}

Эту архитектуру в следующих заданиях надо будет реализовать (в составе тех же команд, в которых она проектировалась). Далее каждую неделю будут появляться новые требования, поэтому проектируйте архитектуру расширяемой и по возможности гибкой.

\end{document}
