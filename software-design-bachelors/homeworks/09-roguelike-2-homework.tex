\documentclass[a5paper]{homework}

\begin{document}

\makeHomeworkHeading{
    title = {Домашняя работа 9. Roguelike, часть 2},
    publicationDate = {14.03.2022},
    deadline = {28.03.2022},
    score = {10}
}

В команде из трёх человек продолжить работу над Roguelike. В этом задании требуется реализовать поддержку мобов по следующим требованиям:

\begin{itemize}
    \item мобов должно быть несколько разных видов, различающихся характеристиками и поведением:
    \begin{itemize}
        \item агрессивное поведение, атакуют игрока, как только его видят;
        \item пассивное поведение, просто стоят на месте;
        \item трусливое поведение, стараются держаться на расстоянии от игрока.
    \end{itemize}
    \item нужна боевая система, в которой:
    \begin{itemize}
        \item персонажи, пытающиеся занять одну клетку, наносят друг другу урон в соответствии с их параметрами (атаки и защиты, и, возможно, каких-либо ещё);
        \item урон уменьшает количество хитпойнтов, и если их становится 0 или меньше, персонаж умирает;
    \end{itemize} 
    \item нужна система экспы и уровней:
    \begin{itemize}
        \item при убийстве моба персонажу игрока начисляется некоторое количество очков опыта;
        \item при наборе достаточного количества опыта персонаж получает следующий уровень, что приводит к росту его характеристик.
    \end{itemize}
\end{itemize}

При этом надо использовать паттерны, обсуждавшиеся на теории:

\begin{itemize}
    \item паттерн <<Стратегия>> для поддержки различных поведений мобов;
    \item используя паттерн <<Декоратор>>, реализовать для игрока возможность конфузить мобов;
    \begin{itemize}
        \item моб, находящийся под эффектом конфузии, перемещается, случайно выбирая соседнюю клетку;
        \item эффект должен быть временным.
    \end{itemize}
\end{itemize}

Эту задачу надо реализовывать, отведя новую ветку от предыдущей и, соответственно, открыв новый пуллреквест. 

Также надо обновить архитектурную документацию (как диаграмму, так и словесное описание), включив туда описание новой функциональности.

Не забудьте юнит-тесты и комментарии.

\end{document}
