\documentclass[a5paper]{homework}

\begin{document}

\makeHomeworkHeading{
    title = {Домашняя работа 8. Roguelike, часть 1},
    publicationDate = {07.03.2022},
    deadline = {21.03.2022},
    score = {10}
}

В команде из трёх человек реализовать архитектуру Roguelike из предыдущего задания. При этом:

\begin{itemize}
    \item рекомендуется (хотя и не обязательно) использовать какую-либо готовую библиотеку для работы с консолью, но готовые фреймворки для разработки Roguelike-игр нельзя;
    \begin{itemize}
        \item есть много туториалов по этому делу, смотреть в них можно, списывать нет;
        \item постарайтесь не терять кроссплатформенности (у ncurses и даже python-библиотек на её основе могут быть проблемы под Windows, например);
        \item как обычно, если что-то выбираете, обоснуйте выбор в архитектурной документации;
    \end{itemize}
    \item нужны юнит-тесты, CI и комментарии ко всему, что public;
    \item надо обновить диздок, чтобы привести его в соответствие фактической реализации.
\end{itemize}

Напомним, что дальше будут появляться новые требования, которые надо будет реализовать в рамках этой кодовой базы.

\end{document}
