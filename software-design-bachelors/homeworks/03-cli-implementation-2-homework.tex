\documentclass[a5paper]{homework}

\begin{document}

\makeHomeworkHeading{
    title = {Домашняя работа 3. Реализация CLI, часть 2},
    publicationDate = {31.01.2022},
    deadline = {14.02.2022},
    score = {10}
}

Реализовать \emph{индивидуально} вторую часть архитектуры Command-Line Interface из домашней работы 1: поддержку подстановок и пайпов.

\begin{itemize}
    \item Делайте эту задачу в отдельной ветке на базе части первой CLI, сдавайте отдельным пуллреквестом.
    \item Если к первой части ещё не приступали, всё равно разделите на два пуллреквеста из двух отдельных веток.
    \item Обратите внимание на обработку ошибок --- шелл никогда не должен падать из-за пользовательского ввода, по возможности адекватно выдавать диагностику.
    \item Не забудьте про юнит-тесты на новую функциональность.
    \item При реализации допустимо отклоняться от семантики bash и других популярных шеллов, особенно если это логично в вашей архитектуре. Например, подумайте про:
    \begin{itemize}
        \item exit и его взаимодействие с пайпами;
        \item ненулевой код возврата и его взаимодействие с пайпами;
        \item что с пайпами и потоком ошибок.
    \end{itemize}
\end{itemize}

\end{document}
