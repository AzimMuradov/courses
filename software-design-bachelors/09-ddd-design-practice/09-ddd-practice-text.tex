\documentclass[a5paper]{article}
\usepackage[a5paper, top=8mm, bottom=8mm, left=8mm, right=8mm]{geometry}

\usepackage{polyglossia}
\setdefaultlanguage[babelshorthands=true]{russian}

\usepackage{fontspec}
\setmainfont{FreeSerif}
\newfontfamily{\russianfonttt}[Scale=0.7]{DejaVuSansMono}

\usepackage[hang,multiple]{footmisc}
\renewcommand{\footnotelayout}{\raggedright}

\PassOptionsToPackage{hyphens}{url}\usepackage[xetex,linktocpage=true,plainpages=false,pdfpagelabels=false]{hyperref}
\hypersetup{colorlinks=true, linkcolor=blue, citecolor=blue, filecolor=blue, urlcolor=blue, pdftitle=1, pdfauthor=, pdfsubject=, pdfkeywords=}

\usepackage{tabu}

\usepackage{indentfirst}
\usepackage{minted}
\usepackage[normalem]{ulem}
\usepackage{xcolor}
\usepackage{graphicx}

\newcommand{\attribution}[1] {
    \begin{flushright}\begin{scriptsize}\textcolor{gray}{#1}\end{scriptsize}\end{flushright}
}

\sloppy
\pagestyle{plain}

\title{Практика 8: практика по DDD}
\author{}
\date{07.03.2022}

\begin{document}

\maketitle
\thispagestyle{empty}

\section{Задание на пару}

В командах по 3 человека требуется по канонам Domain-Driven Design спроектировать систему контроля версий, представляющую из себя консольное приложение и умеющую:

\begin{itemize}
    \item commit с commit message, датой коммита и автором;
    \item работу с ветками: создание и удаление;
    \item checkout по имени ревизии или ветки;
    \item log --- список ревизий вместе с commit message в текущей ветке;
    \item merge --- сливает указанную ветку с текущей;
    \begin{itemize}
        \item должен быть предусмотрен механизм разрешения конфликтов;
    \end{itemize}
    \item работа с удалёнными репозиториями: clone, fetch/pull, push.
\end{itemize}

При этом код системы должен позволять себя использовать как библиотеку, но предполагается также наличие консольного интерфейса.

Что надо сделать:

\begin{itemize}
    \item диаграмму компонентов и диаграмму/диаграммы классов;
    \begin{itemize}
        \item лучше всего сделать одну большую диаграмму, где отображены одновременно и компоненты, и классы;
    \end{itemize}
    \item сдавать, как обычно, пуллреквестом в репозиторий одного из членов команды, либо исходника с диаграммой, либо .md-файла со ссылкой на проект в облачном редакторе;
    \begin{itemize}
        \item не забудьте расшарить;
        \item не забудьте указать, кто в команде;
    \end{itemize}
    \item текстовое описание не требуется, поясняйте непонятные моменты в комментариях на диаграмме;
    \begin{itemize}
        \item однако любая неоднозначность и непонятность будет трактована не в вашу пользу;
    \end{itemize}
    \item нельзя подсматривать в Git Book и другую архитектурную документацию систем контроля версий (там всё написано, хотя конкретно git нельзя назвать примером хорошей архитектуры, и у нас про это ещё отдельная пара будет).
\end{itemize}

Ожидается разделение на уровни, выделенная модель предметной области и выделенное смысловое ядро.

Обратите внимание на следующие вещи.

\begin{itemize}
    \item Как представляются файлы, коммиты, ветки, репозиторий?
    \item Как выполняется компрессия и выполняется ли вообще? Насколько просто получить текущую, предыдущую, произвольную версии?
    \item Каков жизненный цикл файла?
    \item Как выполняется работа с файловой системой?
    \item Как выполняется работа с пользователем? Как представляются команды?
    \item Как выполняется работа с сервером со стороны клиента?
    \item Какова архитектура серверной части?
\end{itemize}

Эту систему реализовывать будет не надо.

\end{document}
