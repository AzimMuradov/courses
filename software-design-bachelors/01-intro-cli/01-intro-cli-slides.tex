\documentclass[xetex,mathserif,serif]{beamer}
\usepackage{polyglossia}
\setdefaultlanguage[babelshorthands=true]{russian}
\usepackage{minted}
\usepackage{tabu}

\useoutertheme{infolines}

\usepackage{fontspec}
\setmainfont{FreeSans}
\newfontfamily{\russianfonttt}{FreeSans}

\definecolor{links}{HTML}{2A1B81}
\hypersetup{colorlinks,linkcolor=,urlcolor=links}

\tabulinesep=0.7mm

\title{Проектирование программного обеспечения (практика)}
\subtitle{Практика 1: Введение, задача про CLI}
\author[Юрий Литвинов]{Юрий Литвинов \newline \textcolor{gray}{\small\texttt{yurii.litvinov@gmail.com}}}

\date{16.01.2019г}

\begin{document}
	
	\frame{\titlepage}
	
	\begin{frame}
		\frametitle{Формальности}
		\begin{itemize}
			\item Чтобы получить зачёт, надо:
			\begin{itemize}
				\item Сдать все домашние работы
				\item Успеть сдать каждую задачу до дедлайна
			\end{itemize}
			\item Условия, материалы и сдача домашек через \url{http://hwproj.me/courses/38}
			\begin{itemize}
				\item Надо записаться на курс
			\end{itemize}
			\begin{itemize}
				\item Ссылка на пуллреквест в собственный репозиторий на GitHub
				\item Задач будет несколько, так что выкладывать лучше так, чтобы их все было можно смерджить
				\item Формулировки условий могут появляться много где, приоритет у тех, что на странице курса на HwProj
			\end{itemize}
		\end{itemize}
	\end{frame}

	\begin{frame}
		\frametitle{Краткое содержание курса по практике}
		\begin{itemize}
			\item Снова лекции
			\begin{itemize}
				\item Про архитектурную документацию
				\item Про UML и другие языки проектирования
				\item Про реализацию паттернов
				\item Про антипаттерны
				\item Про предметно-ориентированное проектирование
				\item Различные примеры архитектуры
			\end{itemize}
			\item Небольшие задачи прямо на паре
			\item Большие задачи на дом
		\end{itemize}
	\end{frame}
	
	\begin{frame}
		\frametitle{Что ожидается от решений}
		\begin{itemize}
			\item Работоспособность и соответствие требованиям условия
			\item Наличие архитектурной документации
			\begin{itemize}
				\item Комментарии к каждому классу, интерфейсу и public-методу
				\item Краткое описание деталей реализации в README
			\end{itemize}
			\item Следование стайлгайдам и правилам здравого смысла 
			\item Наличие юнит-тестов
			\item Применение индустриальных практик: логирование, Continuous Integration, обработка исключений
		\end{itemize}
	\end{frame}

	\begin{frame}
		\frametitle{Ещё комментарии}
		\begin{itemize}
			\item Овердизайн и активное использование знаний с лекций приветствуются
			\item Обоснованность принятых решений важнее, чем техника кодирования
			\item Некоторые требования могут показаться ненужными --- это нормально
			\begin{itemize}
				\item Мы учимся не написанию кода, а инструментам и техникам проектирования
			\end{itemize}
			\item Комментарии вида ``у вас неправильная архитектура'' будут очень редки, как ни странно
		\end{itemize}
	\end{frame}
	
	\begin{frame}
		\frametitle{Задача про CLI}
		Реализовать простой интерпретатор командной строки, поддерживающий команды:
		\begin{itemize}
			\item \textbf{cat [FILE]} --- вывести на экран содержимое файла
			\item \textbf{echo} --- вывести на экран свой аргумент (или аргументы)
			\item \textbf{wc [FILE]} --- вывести количество строк, слов и байт в файле
			\item \textbf{pwd} --- распечатать текущую директорию
			\item \textbf{exit} --- выйти из интерпретатора
		\end{itemize}
	\end{frame}
	
	\begin{frame}
		\frametitle{Задача про CLI (продолжение)}
		\begin{itemize}
			\item Должны поддерживаться одинарные и двойные кавычки (full and weak quoting)
			\item Окружение (команды вида ``имя=значение''), оператор \$
			\item Вызов внешней программы через Process (или его аналоги)
			\begin{itemize}
				\item если введено что-то, чего интерпретатор не знает
			\end{itemize}
			\item Пайплайны (оператор ``|'')
		\end{itemize}
	\end{frame}
	
	\begin{frame}[fragile]
		\frametitle{Примеры}
\begin{minted}{sh}
>echo "Hello, world!"
Hello, world!
> FILE=example.txt
> cat $FILE
Some example text
> cat example.txt | wc
1 3 18
> echo 123 | wc
1 1 3
> x=exit
> $x
\end{minted}
\end{frame}

	\begin{frame}
		\frametitle{Нефункциональные требования}
		\begin{itemize}
			\item Легко добавлять новые команды (расширяемость)
			\item Наличие возможности реализовать что-то новое из того, что умеют другие шеллы (сопровождаемость)
			\item Адекватное покрытие юнит-тестами (качество, сопровождаемость)
			\item Комментарии в коде (сопровождаемость)
			\item Архитектурное описание, как умеете (сопровождаемость)
			\item Стайлгайд (сопровождаемость)
		\end{itemize}
	\end{frame}
	
	\begin{frame}
		\frametitle{Как сдавать}
		\begin{itemize}
			\item Сделать для этого курса репозиторий
			\item Выложить решение в отдельную ветку
			\item Сделать пуллреквест к себе в мастер
			\item Записаться на курс \url{http://hwproj.me/courses/38}
			\item Приложить там ссылку на пуллреквест к решению
			\item Если я за неделю не откомментил, написать мне на почту
			\item Смерджить пуллреквест, когда задача зачтена
			\item \textbf{Дедлайн 10:00 30.01.2019}
		\end{itemize}
	\end{frame}
	
	\begin{frame}
		\frametitle{Что делать сейчас}
		Первые фазы жизненного цикла
		\begin{itemize}
			\item Выполнить анализ и определить подходы к решению
			\item Выявить подводные камни и способы их преодоления
			\item Декомпозировать задачу на подсистемы, классы и методы
			\item Нарисовать диаграмму классов
			\item Словами описать принцип работы и основные принятые решения
			\item К концу пары все должны понимать, что и как надо кодить
		\end{itemize}
	\end{frame}

\end{document}
