\documentclass[xetex,mathserif,serif]{beamer}
\usepackage{polyglossia}
\setdefaultlanguage[babelshorthands=true]{russian}
\usepackage{minted}

\useoutertheme{infolines}

\setmainfont{FreeSans}
\newfontfamily{\russianfonttt}{FreeSans}

\definecolor{links}{HTML}{2A1B81}
\hypersetup{colorlinks,linkcolor=,urlcolor=links}

\title{Про ВКР}
\author[Юрий Литвинов]{Юрий Литвинов \newline \textcolor{gray}{\small\texttt{y.litvinov@spbu.ru}}}
\date{02.09.2021}

\begin{document}

    \frame{\titlepage}

    \begin{frame}
        \frametitle{План}
        \begin{itemize}
            \item \textbf{До 20-го сентября} --- выбор научника и темы работы
            \item \textbf{Октябрь-ноябрь} --- представление темы кафедре
            \item \textbf{Конец декабря} --- зачёт по осенней практике
            \item \textbf{Конец января-начало февраля} --- утверждение рецензента
            \item \textbf{Март-апрель} --- непредпредзащита
            \item \textbf{Середина апреля} --- зачёты по весенним практикам
            \item \textbf{За две недели до защиты} --- загрузить текст в Blackboard
            \item \textbf{Примерно за две недели до защиты} --- непредзащита
            \item \textbf{С 15 мая по 15 июня} --- защиты
            \item \textbf{Начало июля} --- выдача диплома
        \end{itemize}
    \end{frame}
    
    \begin{frame}
        \frametitle{Где брать тему и научника?}
        \begin{itemize}
            \item Продолжить работать над тем, чем занимались на третьем курсе
            \item Взять тему на работе/стажировке
            \begin{itemize}
                \item Но лучше сначала обсудить с кем-то на кафедре
            \end{itemize}
            \item Из таблицы \url{https://bit.ly/se-themes-2021}
            \item \textbf{Научником может быть только сотрудник СПбГУ со степенью}
            \begin{itemize}
                \item Консультантом может быть кто угодно
                \item Надо договориться с преподавателем кафедры, кому ваша тема близка по духу
            \end{itemize}
            \item Защищаетесь на той кафедре, с которой у вас научник
            \begin{itemize}
                \item Кроме Семёна Григорьева, он на кафедре информатики, но его студенты защищаются с СП
            \end{itemize}
            \item Информацию о ВКР, как определитесь, вписать сюда: \url{https://bit.ly/vkr-2021-2022}
        \end{itemize}
    \end{frame}

    \begin{frame}
        \frametitle{Список допустимых научников СП}
        \begin{itemize}
            \item Брыксин Тимофей Александрович
            \item Булычев Дмитрий Юрьевич
            \item Граничин Олег Николаевич
            \item Григорьев Семён Вячеславович
            \item Кознов Дмитрий Владимирович
            \item Литвинов Юрий Викторович
            \item Луцив Дмитрий Вадимович
            \item Мордвинов Дмитрий Александрович
            \item Романовский Константин Юрьевич
            \item Серов Михаил Александрович
            \item Сысоев Сергей Сергеевич
            \item Терехов Андрей Николаевич
        \end{itemize}
    \end{frame}

    \begin{frame}
        \frametitle{Представление темы}
        \begin{itemize}
            \item Доклад на 3-4 минуты и примерно три слайда
            \item Рассказать введение и постановку задачи
            \item Обосновать актуальность
            \item Объяснить, почему эта тема подходит для СП
        \end{itemize}
    \end{frame}

    \begin{frame}
        \frametitle{Осенняя практика}
        \begin{itemize}
            \item Текст отчёта
            \begin{itemize}
                \item Введение, постановка задачи, обзор, намётки реализации
                \item План экспериментов/апробации
                \item Ожидаемые результаты
                \item Будет выборочное рецензирование
            \end{itemize} 
            \item Отзыв научного руководителя
            \item Отзыв консультанта
        \end{itemize}
    \end{frame}

    \begin{frame}
        \frametitle{Кто такой рецензент}
        \begin{itemize}
            \item Человек, оценивающий вашу работу и пишущий рецензию
            \item Рецензия зачитывается перед ГЭК на защите и содержит оценку
            \item Хороший рецензент читает текст, смотрит программную реализацию, воспроизводит эксперменты
            \item Может быть кто угодно, но не должен быть аффилиирован
            \begin{itemize}
                \item В случае ВКР с закрытым кодом, может быть сотрудник компании, давшей тему, но он не должен быть знаком с работой
            \end{itemize}
            \item Рецензента рекомендует научник и утверждает УМК, обычно с учётом ваших пожеланий
            \begin{itemize}
                \item Но вполне могут назначить кого-то совсем не того, кого вы хотите
            \end{itemize}
            \item Формальный рецензент никак не связан с внутрикафедральным рецензированием
        \end{itemize}
    \end{frame}

    \begin{frame}
        \frametitle{Непредпредзащита}
        \begin{itemize}
            \item Слово <<Предзащита>> использовать запрещено приказом проректора, поэтому у нас <<Непредзащита>>
            \begin{itemize}
                \item Не имеем права даже как-то намекать на оценку
            \end{itemize}
            \item Непред-предзащита --- это первая репетиция защиты
            \begin{itemize}
                \item Результаты могут быть не окончательными и эксперименты не поставленными, всё остальное должно быть как на защите
            \end{itemize}
            \item По результатам ставится зачёт по курсу <<Подготовка ВКР>>
        \end{itemize}
    \end{frame}

    \begin{frame}
        \frametitle{Весенние зачёты}
        \begin{itemize}
            \item ``Преддипломная практика'' --- отзыв научника и черновик текста ВКР
            \begin{itemize}
                \item Опять же, будет внутрикафедральное рецензирование, никак не связанное с рецензентом
            \end{itemize}
            \item ``Подготовка ВКР'' --- по факту успешного прохождения непредпредзащиты
            \item ``Практика разработки документации'' --- электив, зачёт за план и черновик текста ВКР
            \item Весенняя сессия начинается на второй-третьей неделе апреля!
        \end{itemize}
    \end{frame}

    \begin{frame}
        \frametitle{Непредзащита}
        \begin{itemize}
            \item Генеральная репетиция защиты
            \item Всё должно быть уже готово, презентация сделана, отрепетирована с научником и вылизана до блеска
            \item Если хоть один слайд надо переделать, попросят попробовать ещё раз
            \item Строго говоря, не обязательна, поскольку проходит после весенней сессии и не допустить до защиты мы вас не можем
            \begin{itemize}
                \item Но ещё никто не отказывался
                \item Кроме того, получить неуд на защите вполне реально
            \end{itemize}
        \end{itemize}
    \end{frame}

    \begin{frame}
        \frametitle{Документы на защиту}
        \begin{itemize}
            \item Текст --- строго за две недели до даты защиты
            \begin{itemize}
                \item Если нет, на защите выставляется два балла и отчисление
            \end{itemize}
            \item Отзыв научного руководителя --- за 5 дней до защиты
            \item Рецензия --- за 5 дней до защиты
            \begin{itemize}
                \item Не строго, поскольку их делаете не вы
                \item Можно защищаться и без отзыва, и без рецензии, но никто пока не пробовал
            \end{itemize}
            \item Презентация и дополнительные материалы --- за 1-2 дня до защиты
        \end{itemize}
    \end{frame}

    \begin{frame}
        \frametitle{Кто такой ГЭК}
        \begin{itemize}
            \item ГЭК --- Государственная Экзаменационная Комиссия
            \item Формируется из ведущих специалистов в отрасли (не менее 75\%) и преподавателей СПбГУ
            \begin{itemize}
                \item У нас это обычно директора или начальники отделов уважаемых компаний (JetBrains, Dell-EMC, SAP Labs, Газпром нефть, ...)
                \item Есть и молодые специалисты, понимающие в технологиях и не стесняющиеся задавать вопросы
            \end{itemize}
            \item Обычно 6-7 человек
            \item ГЭК формируется для направления, то есть бакалавры матобеса могут защищаться только в ГЭК для СВ.5006
        \end{itemize}
    \end{frame}

    \begin{frame}
        \frametitle{Как проходит защита}
        \framesubtitle{И как ставятся оценки}
        \begin{footnotesize}
            \begin{itemize}
                \item За одно заседание защищается максимум 8 человек, максимум 2 заседания в день
                \item Порядок защиты фиксируется (когда вас распределяют по датам)
                \begin{itemize}
                    \item \footnotesize{Приказ о допуске к защите --- где-то в начале-середине мая}
                \end{itemize}
                \item Выступление защищающегося, вопросы от членов ГЭК и аудитории, отзыв научника, рецензия (зачитывается научником, если рецензента нет), вопросы по отзыву/рецензии, ответное слово (если надо)
                \item ГЭК совещается (по окончании заседания)
                \item Члены ГЭК выставляют оценки по критериям, каждый независимо
                \begin{itemize}
                    \item \footnotesize{См. \url{https://edu.spbu.ru/gia.html}}
                \end{itemize}
                \item Каждый член ГЭК ставит итоговую оценку (при этом критерии --- это только рекомендации), оценки всех членов ГЭК усредняются и ставится итоговая
                \begin{itemize}
                    \item \footnotesize{Оценки научника и рецензента непосредственно не учитываются! Они имеют рекомендательное значение для ГЭК.}
                \end{itemize}
                \item Итоговая оценка заносится в протокол защиты и выставляется на Blackboard
                \item Защищающихся приглашают, оглашают результаты, поздравляют с присвоением квалификации (или нет)
            \end{itemize}
        \end{footnotesize}
    \end{frame}

    \begin{frame}
        \frametitle{Полезные ресурсы}
        \begin{itemize}
            \item Сайт кафедры --- \url{http://se.math.spbu.ru}
            \begin{itemize}
                \item Раздел <<Студентам>> --- архив работ
            \end{itemize}
            \item Титульники --- \url{https://github.com/spbu-se/matmex-diploma-template}
            \item Презентация --- \url{https://github.com/spbu-se/report_presentation_template}
            \item Онлайн-редакторы TeX --- \url{https://papeeria.com/}, \url{https://www.overleaf.com/}
            \item Чеклист по презентациям --- \url{https://goo.gl/UeDRff}
        \end{itemize}
    \end{frame}

    \begin{frame}
        \frametitle{FAQ}
        \begin{itemize}
            \item Можно ли поменять научника/тему диплома?
            \begin{itemize}
                \item До приказа об утверждении тем (где-то октябрь) --- свободно
                \item До приказа о допуске к защите --- по заявлению 
                \item Заявление --- в учебный отдел за подписью старого и нового научника 
                \item Форма заявления: \url{https://docs.google.com/document/d/1tFXwmvW60QBW2L-B5KXYYhlXBl92v_4a/edit?usp=sharing&ouid=112405309681562205852&rtpof=true&sd=true}
            \end{itemize}
            \item Можно ли менять рецензента?
            \begin{itemize}
                \item Формально рецензента вам назначают, так что студент инициировать смену рецензента не может
                \item Неформально, написать мне, и \textit{возможно}, УОП поправит приказ
            \end{itemize}
            \item Можно ли перенести защиту?
            \begin{itemize}
                \item Нет. Либо вы защищаетесь, либо отчисляетесь по незащите ВКР с правом восстановиться для защиты
                \item Теоретически можно закрыть справкой весь июнь, тогда защиту перенесут на осень
                \item Никто не помешает вам до закрытия сессии уйти в академ
            \end{itemize}
        \end{itemize}
    \end{frame}

\end{document}