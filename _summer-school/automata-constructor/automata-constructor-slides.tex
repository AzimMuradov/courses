\documentclass[xetex,mathserif,serif]{beamer}
\usepackage{polyglossia}
\setdefaultlanguage[babelshorthands=true]{russian}
\usepackage{minted}

\useoutertheme{infolines}

\setmainfont{FreeSans}
\newfontfamily{\russianfonttt}{FreeSans}

\title{Конструктор вычислителей}
\author[Юрий Литвинов]{Юрий Литвинов \newline \textcolor{gray}{\small\texttt{yurii.litvinov@gmail.com}}}
\date{12.07.2021г}

\begin{document}

    \frame{\titlepage}

    \begin{frame}
        \frametitle{Идея}
        \begin{itemize}
            \item На первом курсе часто рассказывают про автоматы, показывать не на чем
            \item Есть ещё курсы, где рисование и визуализация исполнения были бы полезны
            \item А есть и промышленные применения
            \item Поэтому давайте сделаем крутой кафедральный open source, которым все будут пользоваться
        \end{itemize}
    \end{frame}

    \begin{frame}
        \frametitle{Вычислители, которые хочется поддержать}
        \begin{itemize}
            \item Конечные автоматы (недетирминированные и детерминированные)
            \item Автоматы с эпсилон-переходами
            \item Автоматы с магазинной памятью
            \begin{itemize}
                \item Они разбирают КС-грамматики, поэтому нужны для курсов по формальным языкам
            \end{itemize}
            \item Машины Тьюринга в разных вариациях
            \begin{itemize}
                \item Которые все эквивалентны по вычислительной мощности
            \end{itemize}
            \item Автоматы с регистрами
            \item Автоматы Мили и Мура
            \item Машины состояний Харела и сети Петри
        \end{itemize}
    \end{frame}

    \begin{frame}
        \frametitle{Общая функциональность}
        \begin{itemize}
            \item Задание в виде диаграммы переходов
            \begin{itemize}
                \item Большинство аналогов поддерживают табличную форму, это не наглядно
            \end{itemize}
            \item Задание входной строки, ленты машины Тьюринга, начального состояния магазина и т.п.
            \item Пошаговое исполнение автомата на входной строке, отображение изменений
            \item Диагностика ошибок
            \item Веб- и десктопная версия, облачное хранение в обоих случаях, совместимость по формату данных
            \item Работа с пользовательскими аккаунтами
            \item Аналитические операции (преобразования автоматов, доказательства завершимости и т.п.)
        \end{itemize}
    \end{frame}

    \begin{frame}
        \frametitle{Технические детали}
        \begin{itemize}
            \item Пишется на C\# с WPF (десктопная версия) и TypeScript (веб-версия)
            \item Модульная архитектура (по идее)
            \item CI/CD, Docker
            \item Большие перспективы развития
            \begin{itemize}
                \item Текстовые представления, конвертации
                \item Генерация кода
                \item Продвинутый анализ для сетей Петри
            \end{itemize}
        \end{itemize}
    \end{frame}

    \begin{frame}
        \frametitle{Что уже есть}
        \begin{itemize}
            \item Защищено четыре учебных практики второго курса
            \item И в десктопной, и в веб-версии готовы поддержка НКА и ДКА:
            \begin{itemize}
                \item UI
                \item Симуляция работы автомата на входной строке, мгновенная и пошаговая
                \begin{itemize}
                    \item Десктопная версия умеет пакетное исполнение набора тестов
                \end{itemize}
                \item Базовая диагностика ошибок
            \end{itemize}
            \item CI/CD (более-менее)
            \item Usability-исследование
        \end{itemize}
    \end{frame}

    \begin{frame}
        \frametitle{Задачи на ЛШ}
        \begin{itemize}
            \item Улучшение UI десктопной версии
            \begin{itemize}
                \item Групповое выделение и групповые операции на сцене
                \item Отмена и повторение операций
                \item Улучшение UX
            \end{itemize}
            \item Поддержка покомпонентной загрузки автоматов
            \item Преобразование НКА в ДКА
            \item Преобразование ДКА в табличную форму
            \item Поддержка формата сохранения JFLAP
            \item Работа с аккаунтами пользователей
            \item Если успеем, ``дерево переходов'' для НКА
        \end{itemize}
    \end{frame}

    \begin{frame}
        \frametitle{Контакты}
        \begin{itemize}
            \item Десктопная версия: Плоскин Александр Евгеньевич (tg:~@bashick)
            \item Веб-версия: Усманов Артур Радикович
            \item ``Заказчик'': Баклановский Максим Викторович
            \item Научный руководитель: Литвинов Юрий Викторович
            \item Принимаем без собеседования, но нелишне хоть немного понимать, о чём речь, и владеть (хоть немного) языком разработки той версии, которой хотите заниматься
        \end{itemize}
    \end{frame}

\end{document}