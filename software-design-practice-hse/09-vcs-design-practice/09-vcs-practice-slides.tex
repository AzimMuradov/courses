\documentclass[xetex,mathserif,serif]{beamer}
\usepackage{polyglossia}
\setdefaultlanguage[babelshorthands=true]{russian}
\usepackage{minted}
\usepackage{tabu}

\useoutertheme{infolines}

\usepackage{fontspec}
\setmainfont{FreeSans}
\newfontfamily{\russianfonttt}{FreeSans}

\definecolor{links}{HTML}{2A1B81}
\hypersetup{colorlinks,linkcolor=,urlcolor=links}

\tabulinesep=0.7mm

\newcommand{\attribution}[1] {
    \vspace{-5mm}\begin{flushright}\begin{scriptsize}\textcolor{gray}{\textcopyright\, #1}\end{scriptsize}\end{flushright}
}

\title{Практика 9: проектирование VCS}
\author[Юрий Литвинов]{Юрий Литвинов \newline \textcolor{gray}{\small\texttt{yurii.litvinov@gmail.com}}}

\date{21.03.2022}

\begin{document}
    
    \frame{\titlepage}

    \begin{frame}
        \frametitle{Задача на пару, система контроля версий}
        В командах по 3 человека спроектировать систему контроля версий:
        \begin{itemize}
            \item commit с commit message, датой коммита и автором
            \item работу с ветками: создание и удаление
            \item checkout по имени ревизии или ветки
            \item log --- список ревизий вместе с commit message в текущей ветке
            \item merge --- сливает указанную ветку с текущей
            \begin{itemize}
                \item должен быть предусмотрен механизм разрешения конфликтов
            \end{itemize}
            \item работа с удалёнными репозиториями: clone, fetch/pull, push
        \end{itemize}
    \end{frame}

    \begin{frame}
        \frametitle{Что надо сделать за пару}
        \begin{itemize}
            \item диаграмму компонентов и классов
            \begin{itemize}
                \item лучше --- одну большую диаграмму, где отображены и компоненты, и классы
            \end{itemize}
            \item сдавать пуллреквестом в репозиторий одного из членов команды
            \begin{itemize}
                \item не забудьте расшарить
                \item не забудьте указать, кто в команде
            \end{itemize}
            \item текстовое описание не требуется, комментарии на диаграмме приветствуются
            \item нельзя подсматривать в Git Book
            \item уровневая структура
        \end{itemize}
    \end{frame}

    \begin{frame}
        \frametitle{О чём подумать}
        \begin{itemize}
            \item Как представляются файлы, коммиты, ветки, репозиторий?
            \item Как выполняется компрессия и выполняется ли вообще? 
            \item Насколько просто получить текущую, предыдущую, произвольную версии?
            \item Каков жизненный цикл файла?
            \item Как выполняется работа с файловой системой?
            \item Как выполняется работа с пользователем? Как представляются команды?
            \item Как выполняется работа с сервером со стороны клиента?
            \item Какова архитектура серверной части?
        \end{itemize}
    \end{frame}

\end{document}
