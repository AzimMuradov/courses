\documentclass[a5paper]{homework}

\begin{document}

\makeHomeworkHeading{
    title = {Домашняя работа 2. Реализация CLI, часть 1},
    publicationDate = {24.01.2022},
    deadline = {07.02.2022},
    score = {10}
}

Реализовать в тех же командах, в которых вы её проектировали, первую часть архитектуры Command-Line Interface из домашней работы 1, связанную с поддержкой Read-Execute-Print Loop и команд. Подстановки и пайпы пока не надо.

Должны поддерживаться:

\begin{itemize}
    \item cat [FILE] --- вывести на экран содержимое файла;
    \item echo --- вывести на экран свой аргумент (или аргументы);
    \item wc [FILE] --- вывести количество строк, слов и байт в файле;
    \item pwd --- распечатать текущую директорию;
    \item exit --- выйти из интерпретатора;
    \item если введено что-то, чего интерпретатор не знает --- вызов внешней программы.
\end{itemize}

При этом:

\begin{itemize}
    \item одинарные и двойные кавычки должны поддерживаться, хоть они ничем не отличаются без постановок: строка в кавычках --- один аргумент;
    \item переменные окружения должны поддерживаться и передаваться внешнему процессу при запуске;
    \item для команд должны поддерживаться потоки вывода, ошибок и код возврата (для встроенных команд --- на ваше усмотрение, для внешних процессов --- что они вернут).
\end{itemize}

С технической точки зрения надо:

\begin{itemize}
    \item чтобы оно собиралось и запускалось из консоли, а не только из вашей любимой IDE;
    \item очень желательно, чтобы оно работало и под Windows, и под Linux (не то чтобы это надо специально проверять, но старайтесь не использовать привязанных к ОС вещей);
    \item процесс сборки и запуска должен описан в README.md;
    \item должны быть юнит-тесты;
    \item должен быть настроенный CI, где эти юнит-тесты бы запускались (кстати, иметь CI и под Windows, и под Linux, и может даже под macOS было бы хорошей идеей);
    \item должны быть комментарии к каждому типу и каждому public-методу.
\end{itemize}

Сдавать в виде пуллреквеста с приличным названием в свой репозиторий (чтобы было понятно, какая это задача). Как сделаете, пишите в чат курса, что задача готова к проверке (не прикладывая ссылку на пуллреквест --- чтобы у коллег из других команд было меньше соблазна подсмотреть, как это всё делать).

\end{document}
