\documentclass[a5paper]{homework}

\begin{document}

\makeHomeworkHeading{
    title = {Домашняя работа 13. MyHwProj, архитектура},
    publicationDate = {18.04.2022},
    deadline = {09.05.2022},
    score = {10}
}

В командах по 2-4 человека спроектировать систему для проверки домашних заданий студентов по следующим требованиям:

\begin{itemize}
    \item считаем, что наш сервис поддерживает один курс, на который записан только один студент, поэтому авторизация, работа с группами студентов и т.п. не нужны (пока);
    \item домашняя работа имеет название, дату публикации, условие и дедлайн;
    \item попытка сдачи домашней работы помнит домашнюю работу, к которой относится, дату и время сдачи, и результат проверки, состоящий из оценки и комментариев;
    \item проверка будем считать, что выполняется полностью автоматически --- при поступлении попытки сдачи сервис запускает на решение \textit{некую программу}, которая говорит <<да>> или <<нет>> и \emph{что-то} выводит в stdout;
    \begin{itemize}
        \item предполагаем, что препод сам пишет эту программу для каждого курса, но как модельный пример можно использовать gradlew test или какой-нибудь линтер;
        \item из архитектуры должно быть понятно, как подрубить свою проверялку;
        \item проверялка должна получать всю информацию о попытке, включая информацию о домашке, дату сдачи и т.п.;
    \end{itemize}
    \item студент может:
    \begin{itemize}
        \item просматривать список домашних работ, отсортированный по близости дедлайна, причём должны показываться только работы, дата публикации которых уже наступила;
        \item сдать решение в виде ссылки на GitHub --- для этого ему надо кликнуть на элемент списка домашних работ, в результате чего он попадёт на экран с детальной информацией о работе (включая полное условие), полем для ввода ссылки на решение и кнопкой <<Submit>>;
        \item просмотреть список результатов, отсортированный по дате сдачи;
        \item просмотреть детальную информацию о попытке по клику на элемент списка результатов, включая текстовый вывод программы-проверялки;
    \end{itemize}
    \item препод может:
    \begin{itemize}
        \item добавить новую домашнюю работу;
        \item просмотреть список результатов, отсортированный по дате сдачи;
        \item просмотреть детальную информацию о попытке по клику на элемент списка результатов, включая текстовый вывод программы-проверялки;
    \end{itemize}
\end{itemize}

Авторизация пока не нужна, поэтому препод мы или студент, можно просто спрашивать при начале работы или определять по ссылке, по которой мы попадаем в приложение (например, http://localhost:8888/teacher или http://localhost:8888/student).

Нефункциональные требования:

\begin{itemize}
    \item наверное, не стоит даже говорить, что это должно быть веб-приложение;
    \item поскольку проверка может занимать значительное время, ей занимается не непосредственно веб-приложение, а \emph{раннер}, работающий как отдельное приложение и связанный с веб-частью с помощью очереди сообщений (например, RabbitMQ);
    \item раннеров может быть много, очередь должна балансировать между ними нагрузку.
\end{itemize}

Спроектированную систему через неделю надо будет реализовать.

\end{document}
