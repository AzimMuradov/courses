\documentclass[a5paper]{homework}

\begin{document}

\makeHomeworkHeading{
    title = {Домашняя работа 10. Roguelike, часть 3},
    publicationDate = {28.03.2022},
    deadline = {18.04.2022},
    score = {10}
}

В команде продолжить работу над Roguelike. В этом задании требуется улучшить генерацию карт и поддержку мобов. Разумеется, применяя паттерны с теории.

\begin{itemize}
    \item Паттерн <<Строитель>> для параметризации генератора карт. Должно быть можно сообщить строителю, грузить карту из файла или сгенерировать, указать желаемые размеры карты, и вызвать метод build(), возвращающий сгенерированную карту.
    \item Паттерн <<Абстрактная фабрика>> для генерации разных стилей мобов --- например, фэнтезийные мобы в духе <<скелет>>, <<дракон>> и т.д., или научно-фантастические, в духе <<киборг-бензопильщик>> и т.п. Фабрикой должно быть можно параметризовать строитель из предыдущего пункта.
    \item Паттерн <<Прототип>> для мобов, реплицирующихся на поле боя --- например, <<ядовитая плесень>>, которая каждый ход с вероятностью p порождает свою копию в соседней свободной клетке. Можно игру <<Жизнь>> реализовать, если хочется :)
\end{itemize}

Как обычно, задача сдаётся отдельным пуллреквестом от ветки с предыдущей задачей. Стоит также время от времени подмерждивать в неё исправления к предыдущим задачам.

Также надо обновить архитектурную документацию (как диаграмму, так и словесное описание), включив туда описание новой функциональности.

Не забудьте юнит-тесты и комментарии.

\end{document}
